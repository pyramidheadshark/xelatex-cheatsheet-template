% >>> Реструктурированный Контент для шпаргалки по Pandas v2

% ================================================
% Секция I: Загрузка и Просмотр Данных
% ================================================
\section{Загрузка и Просмотр Данных}

% --- I.A: Чтение и Запись ---
\subsection{A Чтение и Запись}

\begin{myblock}{Основные операции ввода/вывода}
    Чтение данных из файлов различных форматов и запись DataFrame в файлы.
    \begin{codebox}{python}{Чтение / Запись}
    df_csv = pd.read_csv('file.csv', sep=',')
    df_excel = pd.read_excel('file.xlsx')
    df.to_csv('output.csv', index=False)
    df.to_excel('output.xlsx', index=False)
    \end{codebox}
\end{myblock}

% --- I.B: Первичный Осмотр ---
\subsection{B Первичный Осмотр DataFrame}

\begin{myblock}{Базовые атрибуты и методы}
    Команды для получения общего представления о структуре, типах данных и статистиках DataFrame (`df`).
    \begin{codebox}{python}{Осмотр DataFrame (`df`)}
    df.head()
    df.tail(3)
    df.shape
    df.info()
    df.describe()
    df.describe(include='object')
    df.columns
    df.dtypes
    \end{codebox}
\end{myblock}

% ================================================
% Секция II: Выборка Данных
% ================================================
\section{Выборка Данных}

% --- II.A: Базовый Выбор Столбцов ---
\subsection{A Базовый Выбор Столбцов}

\begin{myblock}{Доступ к столбцам по имени}
    Выбор одного или нескольких столбцов DataFrame по их названиям.
    \begin{codebox}{python}{Выбор столбцов}
    s = df['col_name']
    df_subset = df[['col1', 'col2']]
    \end{codebox}
\end{myblock}

% --- II.B: Доступ по Меткам и Позициям: .loc vs .iloc ---
\subsection{B Доступ по Меткам и Позициям: \texttt{.loc} vs \texttt{.iloc}}

\begin{textbox}{Ключевое различие}
    Два основных метода для индексации и выборки данных в Pandas:
    \begin{itemize}[nosep, leftmargin=*]
        \item \texttt{.loc}: Выборка по \textbf{меткам} индекса и \textbf{названиям} столбцов. Правая граница среза \textbf{включается}.
        \item \texttt{.iloc}: Выборка по \textbf{целочисленным позициям} (индексам строк и столбцов, начиная с 0). Правая граница среза \textbf{не включается} (как в Python).
    \end{itemize}
\end{textbox}

\begin{myexampleblock}{Примеры использования \texttt{.loc} и \texttt{.iloc}}
    \begin{codebox}{python}{Примеры .loc и .iloc}
    df.loc['label']
    df.loc['start':'end']
    df.loc[:, 'col_name']
    df.loc['label', 'col_name']
    df.loc[['l1', 'l3'], ['c1', 'c2']]
    df.iloc[0]
    df.iloc[0:5]
    df.iloc[:, 0]
    df.iloc[0, 0]
    df.iloc[[0, 2], [0, 1]]
    \end{codebox}
\end{myexampleblock}

% ================================================
% Секция III: Фильтрация (Boolean Indexing)
% ================================================
\section{Фильтрация (Boolean Indexing)}

% --- III.A: Фильтрация по Условиям ---
\subsection{A Фильтрация по Условиям}

\begin{myexampleblock}{Выборка строк на основе логических условий}
    Используйте логические операторы \texttt{\&} (И), \texttt{|} (ИЛИ), \texttt{\textasciitilde} (НЕ) для комбинирования условий. Условия необходимо оборачивать в круглые скобки \texttt{()}. Также полезны методы \texttt{isin()} и \texttt{between()}.
    \begin{codebox}{python}{Примеры Булевой Индексации}
    df[df['score'] > 50]
    df[(df['score'] > 50) & (df['attempts'] < 3)]
    df[(df['city'] == 'Moscow') | (df['city'] == 'London')]
    df[df['city'].isin(['London', 'Paris', 'Tokyo'])]
    df[df['age'].between(18, 30)]
    df.loc[df['score'] > 90, ['name', 'score']]
    df[~(df['city'] == 'Moscow')]
    \end{codebox}
\end{myexampleblock}

% ================================================
% Секция IV: Сортировка
% ================================================
\section{Сортировка}

% --- IV.A: Сортировка DataFrame ---
\subsection{A Сортировка DataFrame}

\begin{myblock}{Упорядочивание данных}
    Сортировка строк DataFrame по значениям в одном или нескольких столбцах (\texttt{sort\_values}) или по меткам индекса (\texttt{sort\_index}).
    \begin{codebox}{python}{Примеры сортировки}
    df.sort_values(by='score')
    df.sort_values(by='score', ascending=False)
    df.sort_values(by=['city', 'score'], ascending=[True, False])
    df.sort_index()
    df.sort_index(ascending=False)
    \end{codebox}
\end{myblock}

% ================================================
% Секция V: Работа с Дубликатами
% ================================================
\section{Работа с Дубликатами}

% --- V.A: Обнаружение и Удаление Дубликатов ---
\subsection{A Обнаружение и Удаление Дубликатов}

\begin{myblock}{Идентификация и удаление повторяющихся строк}
    Методы для проверки наличия дубликатов (\texttt{duplicated}) и их удаления (\texttt{drop\_duplicates}).
    \begin{codebox}{python}{Примеры работы с дубликатами}
    df.duplicated()
    df.duplicated(subset=['user_id', 'timestamp'])
    df.drop_duplicates()
    df.drop_duplicates(keep='last')
    df.drop_duplicates(subset=['user_id'], keep='first')
    \end{codebox}
\end{myblock}

% ================================================
% Секция VI: Применение Функций и Агрегация
% ================================================
\section{Применение Функций и Агрегация}

% --- VI.A: Применение Функций к Элементам, Строкам, Столбцам ---
\subsection{A Применение Функций к Элементам, Строкам, Столбцам}

\begin{myblock}{Поэлементные и построчные/постолбцовые операции}
    Применение встроенных или пользовательских функций:
    \begin{itemize}[nosep, leftmargin=*]
        \item \textbf{Поэлементные операции:} Векторизованные операции NumPy или базовые арифметические операции.
        \item \texttt{map()}: Применение функции или словаря к каждому элементу \textbf{Series}.
        \item \texttt{apply()}: Применение функции вдоль оси DataFrame (\texttt{axis=0} к столбцам, \texttt{axis=1} к строкам).
        \item \texttt{applymap()}: Применение функции к каждому элементу \textbf{DataFrame}. (Менее рекомендуется, часто есть векторизованные альтернативы).
    \end{itemize}
    \begin{codebox}{python}{Примеры применения функций}
    df['new_col'] = df['col_A'] * 10
    df['col_B_log'] = np.log(df['col_B'])
    df['cat_code'] = df['category'].map({'A':1, 'B':2, 'C':3})
    df[['col_A', 'col_B']].apply(np.sum, axis=0)
    df['row_sum'] = df[['col_A', 'col_B']].apply(np.sum, axis=1)
    df['col_C_processed'] = df['col_C'].apply(lambda x: x**2 if x > 0 else 0)
    \end{codebox}
\end{myblock}

% --- VI.B: Встроенные Агрегирующие Функции ---
\subsection{B Встроенные Агрегирующие Функции}

\begin{myblock}{Расчет сводных статистик}
    Быстрый расчет основных статистик для Series или столбцов DataFrame. Игнорируют NaN по умолчанию.
    \begin{codebox}{python}{Примеры агрегаций}
    df['score'].mean()
    df['score'].median()
    df['score'].sum()
    df['score'].min()
    df['score'].max()
    df['score'].std()
    df['score'].var()
    df['score'].count()
    df['score'].nunique()
    df['category'].unique()
    df['category'].value_counts()
    df.mean()
    \end{codebox}
\end{myblock}

% --- VI.C: Работа со Строками (.str) ---
\subsection{C Работа со Строками (\texttt{.str})}

\begin{myexampleblock}{Векторизованные строковые операции}
    Аксессор \texttt{.str} предоставляет доступ к множеству методов для работы со строками в Series.
    \begin{codebox}{python}{Примеры .str методов}
    df['name'].str.lower()
    df['name'].str.upper()
    df['name'].str.title()
    df['address'].str.contains('Street')
    df['email'].str.startswith('info@')
    df['code'].str.endswith('Z')
    df['product_id'].str.isdigit()
    df['city'].str.replace(' ', '_')
    df['full_name'].str.split(' ')
    df['comment'].str.len()
    \end{codebox}
\end{myexampleblock}

% --- VI.D: Работа с Датами/Временем (.dt) ---
\subsection{D Работа с Датами/Временем (\texttt{.dt})}

\begin{myexampleblock}{Доступ к компонентам даты/времени}
    Аксессор \texttt{.dt} предоставляет доступ к компонентам даты и времени для Series с типом данных `datetime64[ns]`. Предварительно столбец нужно преобразовать.
    \begin{codebox}{python}{Примеры .dt атрибутов и методов}
    df['date_col'].dt.year
    df['date_col'].dt.month
    df['date_col'].dt.day
    df['date_col'].dt.hour
    df['date_col'].dt.minute
    df['date_col'].dt.second
    df['date_col'].dt.dayofweek
    df['date_col'].dt.dayofyear
    df['date_col'].dt.quarter
    df['date_col'].dt.month_name()
    df['date_col'].dt.day_name()
    df['date_col'].dt.strftime('%Y-%m-%d')
    \end{codebox}
\end{myexampleblock}

% ================================================
% Секция VII: Группировка и Сводные Таблицы
% ================================================
\section{Группировка и Сводные Таблицы}

% --- VII.A: GroupBy: Разделяй-Применяй-Объединяй ---
\subsection{A GroupBy: Разделяй-Применяй-Объединяй}

\begin{myblock}{Агрегация данных по группам}
    Механизм \texttt{groupby()} позволяет:
    \begin{enumerate}[nosep, leftmargin=*]
        \item \textbf{Разделить} данные на группы на основе значений в одном или нескольких столбцах.
        \item \textbf{Применить} агрегирующую функцию (sum, mean, count, etc.) или трансформацию к каждой группе независимо.
        \item \textbf{Объединить} результаты в новую структуру данных (Series или DataFrame).
    \end{enumerate}
    \begin{codebox}{python}{Примеры GroupBy}
    grouped = df.groupby('category')
    grouped_multi = df.groupby(['category', 'status'])
    grouped['value'].sum()
    grouped['value'].mean()
    grouped['value'].agg(['mean', 'std', 'count'])
    grouped_multi.agg(
        total_value=('value', 'sum'),
        avg_score=('score', 'mean'),
        max_score=('score', 'max'),
        unique_ids=('id', 'nunique')
    )
    grouped.size()
    \end{codebox}
\end{myblock}

% --- VII.B: Сводные Таблицы и Таблицы Сопряженности ---
\subsection{B Сводные Таблицы и Таблицы Сопряженности}

\begin{myblock}{Создание агрегированных таблиц}
    \texttt{pivot\_table}: Создает сводную таблицу в стиле Excel. Позволяет агрегировать данные по двум или более категориальным переменным. \sep
    \texttt{crosstab}: Вычисляет таблицу сопряженности (частот) для двух или более факторов.
    \begin{codebox}{python}{Pivot Table и CrossTab}
    pd.pivot_table(df,
                   values='score',
                   index='category',
                   columns='status',
                   aggfunc=np.mean,
                   fill_value=0)
    pd.crosstab(df['category'], df['status'])
    \end{codebox}
\end{myblock}

% ================================================
% Секция VIII: Работа с Пропусками (NaN)
% ================================================
\section{Работа с Пропусками (NaN)}

% --- VIII.A: Обработка Пропущенных Значений (NaN) ---
\subsection{A Обработка Пропущенных Значений (NaN)}

\begin{myblock}{{Обнаружение, удаление и заполнение NaN}}
    Стандартные подходы к работе с пропущенными данными (`NaN`, `None`, `NaT`).
    \begin{codebox}{python}{Операции с NaN}
    df.isnull()
    df.isna()
    df.notnull()
    df.isnull().sum()
    df.dropna()
    df.dropna(axis=1)
    df.dropna(subset=['col_A', 'col_B'])
    df.dropna(thresh=2)
    df.fillna(0)
    df['col_A'].fillna('Unknown')
    df['col_B'].fillna(df['col_B'].mean())
    df.fillna(method='ffill')
    df.fillna(method='bfill')
    \end{codebox}
\end{myblock}

% ================================================
% Секция IX: Объединение DataFrames
% ================================================
\section{Объединение DataFrames}

% --- IX.A: Комбинирование DataFrames ---
\subsection{A Комбинирование DataFrames}

\begin{textbox}{Основные методы объединения}
    \begin{itemize}[nosep, leftmargin=*]
        \item \texttt{pd.concat()}: "Склеивание" объектов Pandas вдоль оси (строк \texttt{axis=0} или столбцов \texttt{axis=1}). Полезно для добавления строк/столбцов.
        \item \texttt{pd.merge()}: Объединение в стиле SQL баз данных. Соединяет строки из двух DataFrame на основе одного или нескольких общих столбцов (\textbf{ключей}) или индексов.
        \item \texttt{df.join()}: Удобный метод для объединения по \textbf{индексу} (или по ключу в одном DataFrame и индексу в другом). Часто используется как обертка над \texttt{merge}.
    \end{itemize}
\end{textbox}

\begin{myexampleblock}{Примеры объединения}
    Предполагаем наличие `df1` и `df2`.
    \begin{codebox}{python}{Merge, Concat, Join}
    combined_rows = pd.concat([df1, df2], axis=0, ignore_index=True)
    combined_cols = pd.concat([df1, df2], axis=1)
    merged_inner = pd.merge(df1, df2, on='key', how='inner')
    merged_left = pd.merge(df1, df2, on='key', how='left')
    joined_index = df1.join(df2, how='inner', lsuffix='_df1', rsuffix='_df2')
    \end{codebox}
\end{myexampleblock}

% ================================================
% Секция X: Изменение Формы (Reshaping)
% ================================================
\section{Изменение Формы (Reshaping)}

% --- X.A: Изменение Структуры: \texttt{stack}, \texttt{unstack}, \texttt{melt} ---
\subsection{A Изменение Структуры: \texttt{stack}, \texttt{unstack}, \texttt{melt}}

\begin{myblock}{Преобразование между "широким" и "длинным" форматами}
    \begin{itemize}[nosep, leftmargin=*]
        \item \texttt{stack()}: "Укладывает" столбцы DataFrame в индекс, создавая Series (или DataFrame с MultiIndex). Переход от "широкого" к "длинному" формату.
        \item \texttt{unstack()}: Обратная операция к \texttt{stack()}. "Поднимает" уровень индекса в столбцы. Переход от "длинного" к "широкому" формату.
        \item \texttt{melt()}: "Расплавляет" DataFrame, преобразуя столбцы в строки. Полезно для создания "длинного" формата данных из "широкого".
    \end{itemize}
    \begin{codebox}{python}{Примеры Reshaping}
    df_melted = pd.melt(df,
                        id_vars=['id', 'name'],
                        value_vars=['Score_Q1', 'Score_Q2'],
                        var_name='Quarter',
                        value_name='Score')
    \end{codebox}
\end{myblock}

% ================================================
% Секция XI: Временные Ряды (Time Series)
% ================================================
\section{Временные Ряды (Time Series)}

% --- XI.A: Основные Операции с Временными Рядами ---
\subsection{A Основные Операции с Временными Рядами}

\begin{myblock}{Работа с датами и временем как индексом}
    Pandas предоставляет мощные инструменты для работы с временными рядами, особенно когда индекс DataFrame имеет тип `DatetimeIndex`.
    \begin{codebox}{python}{Операции с Time Series}
    date_rng = pd.date_range(start='2024-01-01', end='2024-01-10', freq='D')
    time_rng = pd.date_range('2024-01-01', periods=5, freq='H')
    df['previous_value'] = df['value'].shift(1)
    df['change'] = df['value'] - df['value'].shift(1)
    \end{codebox}
    Доступ к компонентам даты/времени через \texttt{.dt} (см. VI.D) работает и для `DatetimeIndex`.
\end{myblock}

% ================================================
% Секция XII: Преобразование Типов Данных
% ================================================
\section{Преобразование Типов Данных}

% --- XII.A: Изменение Типов Данных Столбцов ---
\subsection{A Изменение Типов Данных Столбцов}

\begin{myblock}{Явное приведение типов}
    Преобразование столбцов к нужным типам данных для корректной обработки или экономии памяти.
    \begin{codebox}{python}{Приведение типов}
    print(df.dtypes)
    df['numeric_col'] = pd.to_numeric(df['numeric_col'], errors='coerce')
    df['string_col'] = df['string_col'].astype(str)
    df['category_col'] = df['category_col'].astype('category')
    print(df.dtypes)
    \end{codebox}
\end{myblock}

% ================================================
% Секция XIII: Базовая Визуализация
% ================================================
\section{Базовая Визуализация}

% --- XIII.A: Быстрая Визуализация с .plot() ---
\subsection{A Быстрая Визуализация с \texttt{.plot()}}

\begin{myexampleblock}{Встроенные методы для графиков}
    Pandas интегрируется с Matplotlib, позволяя быстро строить базовые графики прямо из DataFrame или Series с помощью метода \texttt{.plot()}. Необходим импорт `matplotlib.pyplot`.
    \begin{codebox}{python}{Примеры `.plot()`}
    df['value_ts'].plot(kind='line', title='Trend Over Time', figsize=(10, 4))
    df['numeric_col'].plot(kind='hist', bins=30, title='Distribution')
    df.plot(kind='scatter', x='col_A', y='col_B', title='Scatter Plot A vs B')
    df['numeric_col'].plot(kind='kde', title='Density Plot')
    \end{codebox}
\end{myexampleblock}