% >>> Контент для шпаргалки по Pandas

% --- Начальные импорты (для контекста) ---
\begin{myblock}{Необходимые импорты}
    Стандартные импорты для работы с Pandas и NumPy.
    \begin{codebox}{python}{Импорты}
    import pandas as pd
    import numpy as np
    # df = pd.DataFrame(...) # Предполагаемый DataFrame
    # s = pd.Series(...) # Предполагаемый Series
    \end{codebox}
    \end{myblock}
    
    % --- Раздел 1 ---
    \section{Загрузка и Просмотр}
    
    \begin{myblock}{Чтение и Запись}
    Основные операции ввода/вывода данных.
    \begin{codebox}{python}{Чтение / Запись}
    # Чтение
    df_csv = pd.read_csv('file.csv', sep=',')
    df_excel = pd.read_excel('file.xlsx')
    # Запись
    df.to_csv('output.csv', index=False)
    df.to_excel('output.xlsx', index=False)
    \end{codebox}
    \end{myblock}
    
    \begin{myblock}{Первичный Осмотр}
    Базовые команды для понимания структуры данных.
    \begin{codebox}{python}{Осмотр DataFrame (`df`)}
    df.head()
    df.tail(3)
    df.shape
    df.info()
    df.describe()
    df.describe(include='object')
    df.columns
    df.dtypes
    \end{codebox}
    \end{myblock}
    
    % --- Раздел 2 ---
    \section{Выборка: \texttt{.loc} vs \texttt{.iloc}}
    
    \begin{textbox}{Доступ по Меткам и Позициям}
    \texttt{.loc}: по меткам индекса/названиям столбцов (включая правую границу среза). \sep
    \texttt{.iloc}: по целочисленным позициям (НЕ включая правую границу среза).
    \begin{codebox}{python}{Примеры .loc и .iloc}
    # Выбор столбца(ов)
    s = df['col_name']
    df_subset = df[['col1', 'col2']]
    
    # --- .loc (по МЕТКАМ) ---
    df.loc['label'] # Строка
    df.loc['start':'end'] # Срез строк (включительно)
    df.loc[:, 'col_name'] # Столбец
    df.loc['label', 'col_name'] # Значение
    df.loc[['l1', 'l3'], ['c1', 'c2']] # По спискам
    
    # --- .iloc (по ПОЗИЦИЯМ) ---
    df.iloc[0] # Первая строка
    df.iloc[0:5] # Первые 5 строк (не вкл. 5)
    df.iloc[:, 0] # Первый столбец
    df.iloc[0, 0] # Значение [0, 0]
    df.iloc[[0, 2], [0, 1]] # По спискам позиций
    \end{codebox}
    \end{textbox}
    
    % --- Раздел 3 ---
    \section{Фильтрация (Boolean Indexing)}
    
    \begin{myexampleblock}{Выборка по Условиям}
    Используйте \texttt{\&} (И), \texttt{|} (ИЛИ), \texttt{\textasciitilde} (НЕ), \texttt{isin()}, \texttt{between()}. Оборачивайте условия в \texttt{()}.
    \begin{codebox}{python}{Примеры фильтрации}
    df[df['score'] > 50]
    df[(df['score'] > 50) & (df['attempts'] < 3)]
    df[df['city'].isin(['London', 'Paris'])]
    df[df['age'].between(18, 30)]
    # Явное использование .loc
    df.loc[df['score'] > 90, ['name', 'score']]
    \end{codebox}
    \end{myexampleblock}
    
    % --- Раздел 4 ---
    \section{Сортировка}
    
    \begin{textbox}{Упорядочивание Данных}
    Сортировка по значениям (\texttt{sort\_values}) или индексу (\texttt{sort\_index}).
    \begin{codebox}{python}{Примеры сортировки}
    # По значениям
    df.sort_values(by='score') # Возрастание
    df.sort_values(by='score', ascending=False) # Убывание
    df.sort_values(by=['city', 'score'], ascending=[True, False])
    # По индексу
    df.sort_index(ascending=False)
    # На месте (inplace=True)
    # df.sort_values(by='score', inplace=True)
    \end{codebox}
    \end{textbox}
    
    % --- Раздел 5 ---
    \section{Удаление Дубликатов}
    
    \begin{myblock}{Работа с Дубликатами}
    Обнаружение (\texttt{duplicated}) и удаление (\texttt{drop\_duplicates}).
    \begin{codebox}{python}{Примеры работы с дубликатами}
    df.duplicated() # Проверка (bool Series)
    df.duplicated(subset=['col1', 'col2'])
    df.drop_duplicates() # Удалить (оставить первое)
    df.drop_duplicates(keep='last') # Оставить последнее
    df.drop_duplicates(subset=['user_id'], keep='first')
    # На месте (inplace=True)
    # df.drop_duplicates(inplace=True)
    \end{codebox}
    \end{myblock}
    
    % --- Раздел 6 ---
    \section{Функции и Агрегация}
    
    \begin{textbox}{Применение Функций}
    \texttt{apply} (строки/столбцы), \texttt{map} (Series), \texttt{applymap} (DataFrame). Встроенные агрегации.
    \begin{codebox}{python}{Применение функций}
    df['new_col'] = df['col_A'] * 10
    df[['col_A', 'col_B']].apply(np.sum, axis=0)
    df['cat_code'] = df['cat_name'].map({'A':1, 'B':2})
    # numeric_df.applymap(lambda x: x + 1)
    \end{codebox}
    \end{textbox}
    
    \begin{myblock}{Встроенные Агрегации}
    Расчет базовых статистик для Series или столбцов DataFrame.
    \begin{codebox}{python}{Агрегирующие функции}
    df['score'].mean()
    df['score'].sum()
    df['score'].min()
    df['score'].max()
    df['score'].count() # Не-NA значения
    df['category'].nunique() # Уникальные значения
    \end{codebox}
    \end{myblock}
    
    \begin{textbox}{Работа со Строками и Датами}
    Использование аксессоров \texttt{.str} и \texttt{.dt}.
    \begin{codebox}{python}{Аксессоры .str и .dt}
    # Строки (.str)
    df['name'].str.lower()
    df['address'].str.contains('Street')
    df['city'].str.replace(' ', '_')
    # Даты (.dt, столбец должен быть datetime)
    # df['date_col'] = pd.to_datetime(df['date_col'])
    df['date_col'].dt.year
    df['date_col'].dt.month_name()
    df['date_col'].dt.dayofweek # Пн=0, Вс=6
    \end{codebox}
    \end{textbox}
    
    % --- Раздел 7 ---
    \section{Группировка и Сводные Таблицы}
    
    \begin{myblock}{GroupBy (Разделяй-Применяй-Объединяй)}
    Группировка данных с последующей агрегацией.
    \begin{codebox}{python}{Примеры GroupBy}
    grouped = df.groupby('category')
    grouped_multi = df.groupby(['cat', 'status'])
    grouped['value'].sum()
    grouped['value'].agg(['mean', 'std', 'count'])
    grouped_multi.agg({
        'value': 'sum',
        'score': ['mean', 'max'],
        'id': 'nunique'})
    grouped.size() # Размер групп
    # df['g_mean'] = df.groupby('cat')['val'].transform('mean')
    \end{codebox}
    \end{myblock}
    
    \begin{alerttextbox}{Сводные Таблицы}
    Использование \texttt{pivot\_table} и \texttt{crosstab} для агрегации и анализа частот.
    \begin{codebox}{python}{Pivot Table и CrossTab}
    # Pivot Table
    pd.pivot_table(df, values='score', index='category',
                   columns='status', aggfunc=np.mean,
                   fill_value=0)
    # Cross Tabulation (Таблица сопряженности)
    pd.crosstab(df['category'], df['status'])
    \end{codebox}
    \end{alerttextbox}
    
    % --- Раздел 8 ---
    \section{Работа с Пропусками (NaN)}
    
    \begin{myblock}{Обработка NaN}
    Обнаружение, удаление и заполнение пропущенных данных.
    \begin{codebox}{python}{Операции с NaN}
    df.isnull().sum() # Количество NaN
    df.dropna() # Удалить строки с NaN
    df.dropna(axis=1) # Удалить столбцы с NaN
    df.dropna(subset=['col_A']) # По столбцу
    df.fillna(0) # Заполнить нулем
    df['col_A'].fillna(df['col_A'].mean()) # Средним
    df.fillna(method='ffill') # Пред. значением
    df.fillna(method='bfill') # След. значением
    \end{codebox}
    \end{myblock}
    
    % --- Раздел 9 ---
    \section{Объединение DataFrames}
    
    \begin{textbox}{{Merge, Concat, Join}}
    Комбинирование таблиц. \texttt{merge} (SQL-like), \texttt{concat} (склеивание), \texttt{join} (по индексам).
    \begin{codebox}{python}{Примеры объединения}
    # Merge (по ключу 'key')
    pd.merge(df1, df2, on='key', how='inner')
    # Concat (склеить строки)
    pd.concat([df1, df2], axis=0)
    # Concat (склеить столбцы)
    pd.concat([df1, df2], axis=1)
    # Join (по индексам)
    df1.join(df2, how='inner')
    \end{codebox}
    \end{textbox}
    
    % --- Раздел 10 ---
    \section{Изменение Формы (Reshaping)}
    
    \begin{myblock}{{Stack, Unstack, Melt}}
    Преобразование структуры DataFrame.
    \begin{codebox}{python}{Reshaping}
    # Stack (столбцы -> индекс)
    stacked = df.stack()
    # Unstack (индекс -> столбцы)
    unstacked = stacked.unstack()
    # Melt ("расплавление" столбцов)
    pd.melt(df, id_vars=['id'], value_vars=['A', 'B'],
            var_name='variable', value_name='value')
    \end{codebox}
    \end{myblock}
    
    % --- Раздел 11 ---
    \section{Временные Ряды (Time Series)}
    
    \begin{textbox}{Работа с Датами/Временем}
    Преобразование, доступ к компонентам, создание диапазонов, смещение, ресемплинг.
    \begin{codebox}{python}{Операции Time Series}
    # Преобразование и установка индекса
    df['time_col'] = pd.to_datetime(df['time_col'])
    df.set_index('time_col', inplace=True)
    # Доступ к компонентам (если индекс)
    # year = df.index.year
    # Создание диапазона
    rng = pd.date_range('2024-01-01', periods=5, freq='D')
    # Смещение
    df.shift(1)
    # Ресемплинг (агрегация)
    df['value'].resample('M').sum() # По месяцам
    df['value'].resample('W').mean() # По неделям
    \end{codebox}
    \end{textbox}
    
    % --- Раздел 12 ---
    \section{Преобразование Типов}
    
    \begin{myblock}{Изменение Типов Данных}
    Явное приведение столбцов к нужным типам (\texttt{astype}, \texttt{to\_numeric}).
    \begin{codebox}{python}{Приведение типов}
    # В числовой (с заменой ошибок на NaN)
    df['num'] = pd.to_numeric(df['num'], errors='coerce')
    # В строку
    df['col_str'] = df['col_str'].astype(str)
    # В категорию (экономия памяти)
    df['cat'] = df['cat'].astype('category')
    # В datetime (см. раздел Time Series)
    # df['date'] = pd.to_datetime(df['date'])
    df.dtypes # Проверка типов
    \end{codebox}
    \end{myblock}
    
    % --- Раздел 13 ---
    \section{Базовая Визуализация}
    
    \begin{myexampleblock}{Быстрые Графики `.plot()`}
    Создание простых графиков для EDA (требует `matplotlib`).
    \begin{codebox}{python}{Примеры `.plot()`}
    # import matplotlib.pyplot as plt
    # Линейный график
    df['value'].plot(kind='line', title='Trend')
    # Гистограмма
    df['num_col'].plot(kind='hist', bins=20)
    # Столбчатая диаграмма
    df.groupby('cat')['val'].mean().plot(kind='bar')
    # Диаграмма рассеяния
    df.plot(kind='scatter', x='col_A', y='col_B')
    # plt.show() # Показать график
    \end{codebox}
    \end{myexampleblock}
    
    % --- Конец контента ---