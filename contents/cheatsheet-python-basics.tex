% >>> Контент для шпаргалки "Python для DS - Ключевые Библиотеки" (v2)

% --- Раздел 1: NumPy ---
\section{NumPy: Математика и Массивы (\texttt{np})}

\begin{textbox}{Зачем NumPy?}
    Это фундамент для числовых вычислений в Python. Представь его как **супер-оптимизированную таблицу Excel** для работы с числами, особенно с большими наборами данных. Все операции выполняются очень быстро. Основной объект - \textbf{ndarray} (n-мерный массив).
\end{textbox}

\begin{myblock}{{Создание Массивов}}
    Основные способы "завести" себе массив.
    \begin{codebox}{python}{Создание NumPy массивов}
    import numpy as np

    # Из Python списка
    arr1 = np.array([1, 2, 3, 4, 5])

    # Массив нулей (форма: 2 строки, 3 столбца)
    zeros = np.zeros((2, 3))

    # Массив единиц
    ones = np.ones((3, 2))

    # Последовательность чисел (как range)
    seq1 = np.arange(0, 10, 2) # Старт, стоп (не вкл.), шаг

    # Заданное количество чисел в интервале
    seq2 = np.linspace(0, 1, 5) # Старт, стоп (вкл.), кол-во
    \end{codebox}
\end{myblock}

\begin{myblock}{{Базовые Операции}}
    NumPy позволяет делать математику сразу со всем массивом.
    \begin{codebox}{python}{Операции с массивами}
    a = np.array([1, 2, 3])
    b = np.array([4, 5, 6])

    # Поэлементные операции
    c = a + b  # -> array([5, 7, 9])
    d = a * 2  # -> array([2, 4, 6])
    e = a ** 2 # -> array([1, 4, 9])

    # Математические функции
    f = np.sin(a)
    g = np.exp(a)

    # Матричное умножение (для 1D - скалярное)
    dot_product = np.dot(a, b) # 1*4 + 2*5 + 3*6 = 32

    # Сравнения
    bool_arr = a > 1 # -> array([False, True, True])
    \end{codebox}
    \textit{Аналогия с широковещанием (broadcasting):} NumPy "растягивает" массивы меньшей размерности (например, число `2` в `a * 2`), чтобы их формы совпали для операции. Как если бы ты красил стену валиком (операция), а краска (число) сама распределялась по всей ширине.
\end{myblock}

\begin{myblock}{{Форма и Размер (shape, reshape)}}
    Важно понимать "габариты" массива.
    \begin{codebox}{python}{Атрибуты и методы формы}
    data = np.array([[1, 2, 3], [4, 5, 6]])

    # Форма (кортеж: строки, столбцы, ...)
    print(data.shape)  # -> (2, 3)

    # Количество измерений
    print(data.ndim)   # -> 2

    # Общее количество элементов
    print(data.size)   # -> 6

    # Изменение формы (кол-во элементов должно совпадать!)
    reshaped = data.reshape((3, 2))
    # [[1, 2],
    #  [3, 4],
    #  [5, 6]]

    # "Вытягивание" в 1D массив
    flattened = data.flatten() # или data.reshape(-1)
    # [1, 2, 3, 4, 5, 6]
    \end{codebox}
    \textbf{Важно:} \texttt{shape} - это "чертеж" массива (например, 2 строки на 3 столбца), а \texttt{size} - общее число "кирпичиков" (элементов).
\end{myblock}

\begin{myblock}{{Индексирование и Срезы}} % <<< --- НОВЫЙ БЛОК --- <<<
    Доступ к элементам массива.
    \begin{codebox}{python}{Доступ к элементам NumPy}
    arr = np.array([[10, 20, 30], [40, 50, 60]])

    # Первый элемент (первая строка, первый столбец)
    print(arr[0, 0]) # -> 10
    # или
    print(arr[0][0]) # -> 10

    # Вся первая строка
    print(arr[0, :]) # -> array([10, 20, 30])
    # или
    print(arr[0])    # -> array([10, 20, 30])

    # Весь второй столбец
    print(arr[:, 1]) # -> array([20, 50])

    # Срез: первые две колонки первой строки
    print(arr[0, 0:2]) # -> array([10, 20])

    # Boolean индексирование
    print(arr[arr > 30]) # -> array([40, 50, 60])
    \end{codebox}
\end{myblock}

% --- Раздел 2: Pandas ---
\section{Pandas: Таблицы Данных для Анализа (\texttt{pd})}

\begin{textbox}{Зачем Pandas?}
    Твой основной инструмент для работы с **табличными данными** (как в Excel или SQL). Строит свою работу поверх NumPy. Два главных объекта: \textbf{Series} (один столбец) и \textbf{DataFrame} (таблица, коллекция Series).
    \textit{Аналогия:} Если NumPy - это супер-массив чисел, то Pandas - это **умная электронная таблица**, которую можно программировать.
\end{textbox}

\begin{myblock}{{Чтение Данных (CSV)}}
    Самый частый способ загрузить данные.
    \begin{codebox}{python}{Чтение CSV}
    import pandas as pd

    # Предположим, есть файл 'your_data.csv'
    # df = pd.read_csv('your_data.csv')

    # Частые параметры:
    # sep=';' - если разделитель точка с запятой
    # header=None - если в файле нет заголовков
    # names=['col1', 'col2'] - задать имена столбцов
    # usecols=['col1', 'col3'] - прочитать только нужные

    # Создадим пример DataFrame для демонстрации
    df_data = {'col_A': [1, 2, 3, 4], 'col_B': ['x', 'y', 'x', 'z'], 'col_C': [10, 20, 30, 40]}
    df = pd.DataFrame(df_data) # Используем этот df далее

    # Посмотреть первые/последние строки
    print(df.head())    # Первые 5 строк
    print(df.tail(3))   # Последние 3 строки

    # Информация о DataFrame (типы, пропуски)
    df.info()

    # Базовые статистики для числовых столбцов
    print(df.describe())
    \end{codebox}
\end{myblock}

\begin{myblock}{{DataFrame и Series}}
    Основные структуры данных Pandas.
    \begin{codebox}{python}{Создание и доступ}
    # Создание Series (один столбец)
    s = pd.Series([10, 20, 30], index=['a', 'b', 'c'], name='MySeries')

    # Создание DataFrame (таблица) из словаря
    data = {'col_A': [1, 2, 3], 'col_B': ['x', 'y', 'z']}
    df_example = pd.DataFrame(data)

    # Доступ к столбцу (возвращает Series)
    col_a_series = df_example['col_A']
    # или (если имя без пробелов/спецсимволов)
    col_b_series = df_example.col_B

    # Доступ к нескольким столбцам (возвращает DataFrame)
    subset_df = df_example[['col_A', 'col_B']]
    \end{codebox}
\end{myblock}

\begin{alerttextbox}{{Выборка: \texttt{.loc} vs \texttt{.iloc} (Важно!)}}
    Частая путаница у новичков! Это два основных способа выбрать строки/столбцы.
    \begin{itemize}
        \item \texttt{.loc[]}: Выбирает по \textbf{МЕТКАМ} (именам) индекса и столбцов. \textbf{Включает} правую границу среза.
        \item \texttt{.iloc[]}: Выбирает по \textbf{ЦЕЛОЧИСЛЕННЫМ ПОЗИЦИЯМ} (номерам, начиная с 0). \textbf{НЕ включает} правую границу среза (как в Python).
    \end{itemize}
    \textit{Аналогия:} Представь библиотеку. \texttt{.loc} - это поиск книги по названию и автору (метки). \texttt{.iloc} - это взять "пятую книгу с третьей полки" (позиции).
    \begin{codebox}{python}{Примеры .loc и .iloc (используем df из блока "Чтение")}
    # df имеет стандартный числовой индекс [0, 1, 2, 3]

    # --- .loc (по МЕТКАМ индекса и столбцов) ---
    # Строка по метке индекса (здесь метка=число)
    print(df.loc[0])
    # Срез строк по меткам (включительно!)
    print(df.loc[0:2]) # Строки с индексами 0, 1, 2
    # Строки и столбцы по меткам
    print(df.loc[[0, 3], ['col_A', 'col_C']])
    # Конкретная ячейка
    print(df.loc[1, 'col_B'])

    # --- .iloc (по ПОЗИЦИЯМ) ---
    # Строка по номеру (первая)
    print(df.iloc[0])
    # Срез строк по номерам (НЕ включительно!)
    print(df.iloc[0:2]) # Строки 0 и 1 (т.е. с индексами 0 и 1)
    # Строки и столбцы по номерам
    print(df.iloc[[0, 3], [0, 2]]) # 1я и 4я строки, 1й и 3й столбцы
    # Конкретная ячейка
    print(df.iloc[1, 1]) # Элемент на пересечении 2й строки, 2го столбца
    \end{codebox}
\end{alerttextbox}

\begin{myblock}{{Фильтрация Данных}}
    Отбор строк по условиям (используем `df` из блока "Чтение").
    \begin{codebox}{python}{Способы фильтрации}
    # 1. Boolean Indexing (основной способ)
    # Условие: выбрать строки, где значение в 'col_A' > 2
    filter1 = df[df['col_A'] > 2]
    print("Filter 1:\n", filter1)

    # Несколько условий: & (И), | (ИЛИ), ~ (НЕ)
    # Обязательно скобки вокруг каждого условия!
    filter2 = df[(df['col_A'] > 1) & (df['col_B'] == 'x')]
    print("Filter 2:\n", filter2)

    # Использование .isin()
    filter3 = df[df['col_B'].isin(['x', 'y'])]
    print("Filter 3:\n", filter3)

    # 2. Метод .query() (удобно для сложных условий)
    # Строка запроса похожа на SQL WHERE
    filter4 = df.query('col_A > 2 and col_B == "z"')
    print("Filter 4:\n", filter4)
    # Можно использовать переменные с @
    threshold = 15
    filter5 = df.query('col_C > @threshold')
    print("Filter 5:\n", filter5)
    \end{codebox}
\end{myblock}

\begin{myblock}{{Группировка и Агрегация (\texttt{groupby().agg()})}}
    Разделяй (по группам), Властвуй (применяй функцию), Объединяй (результаты).
    \textit{Аналогия:} Разложить все фрукты по корзинам (группировка по типу фрукта), затем посчитать вес яблок, средний размер апельсинов и т.д. (агрегация), и записать результаты для каждой корзины.
    \begin{codebox}{python}{Пример GroupBy (используем df из блока "Чтение")}
    # Сгруппировать по 'col_B' (категориальный столбец),
    # посчитать сумму 'col_C' (числовой)
    # и среднее/количество для 'col_A' (числовой)
    agg_results = df.groupby('col_B').agg(
        total_C=('col_C', 'sum'),       # Новый столбец = ('старый', 'функция')
        average_A=('col_A', 'mean'),
        count_A=('col_A', 'count')
    )
    print("Aggregation Results:\n", agg_results)
    # agg_results будет DataFrame с 'col_B' в индексе

    # Можно группировать по нескольким столбцам
    # multi_group = df.groupby(['cat1', 'cat2']).size() # размер групп
    \end{codebox}
\end{myblock}

\begin{myblock}{{Объединение Таблиц (\texttt{merge}, \texttt{join}, \texttt{concat})}}
    Склеивание данных из разных источников.
    \begin{itemize}
        \item \textbf{`pd.merge(df1, df2, on='key', how='...')`}: Похоже на \textbf{SQL JOIN}. Объединяет по общим столбцам (`on='key'`) или индексам. `how` определяет тип: `'inner'` (только общие ключи), `'outer'` (все ключи), `'left'`, `'right'`.
        \item \textbf{`pd.concat([df1, df2], axis=...)`}: Простое \textbf{склеивание} таблиц. `axis=0` (по умолчанию) - добавить строки df2 под df1. `axis=1` - добавить столбцы df2 справа от df1 (требует совпадения индексов или аккуратности).
        \item \textbf{`df1.join(df2)`}: Удобный метод для объединения по \textbf{индексу} (похож на `merge` с `left\_index=True`, `right\_index=True`).
    \end{itemize}
    \textit{Аналогия:} `merge` - как найти общих друзей (ключи) в двух списках контактов. `concat` - как приклеить один список контактов под другим (`axis=0`) или положить их рядом (`axis=1`).
    \begin{codebox}{python}{Примеры Merge и Concat}
    # Создадим два DataFrame для примеров
    df1 = pd.DataFrame({'user_id': ['u1', 'u2', 'u3'], 'value1': [10, 20, 30]})
    df2 = pd.DataFrame({'user_id': ['u2', 'u3', 'u4'], 'value2': [55, 66, 77]})
    df3 = pd.DataFrame({'user_id': ['u5', 'u6'], 'value1': [40, 50]}) # Для concat

    # Merge (Inner Join по 'user_id')
    df_merged = pd.merge(df1, df2, on='user_id', how='inner')
    print("Merged DF:\n", df_merged)

    # Concat (склеить строки df1 и df3)
    df_concatenated = pd.concat([df1, df3], axis=0, ignore_index=True)
    print("Concatenated DF (rows):\n", df_concatenated)
    # ignore_index=True сбрасывает исходные индексы
    \end{codebox}
\end{myblock}

\begin{myblock}{{Применение Функций (\texttt{.apply()})}}
    Для сложных операций, которые нельзя сделать стандартными методами (используем `df` из блока "Чтение").
    \begin{codebox}{python}{Примеры .apply()}
    # Применить функцию к каждому элементу столбца (Series)
    df['C_percent'] = df['col_C'].apply(lambda x: x / df['col_C'].sum() * 100)
    print("DF with C_percent:\n", df)

    # Применить функцию к каждой строке (axis=1)
    def custom_logic(row):
        # row - это Series, представляющий строку
        return row['col_A'] * 10 + row['col_C'] if row['col_B'] == 'x' else row['col_C']

    df['result'] = df.apply(custom_logic, axis=1)
    print("DF with result:\n", df)
    \end{codebox}
    \textbf{Осторожно:} `.apply()` может быть медленным на больших данных. По возможности используй встроенные векторизованные операции NumPy/Pandas.
\end{myblock}

% --- Раздел 3: Чтение/Запись Файлов ---
\section{Чтение и Запись Файлов (Базовый Python)}

\begin{textbox}{Зачем?}
    Хотя Pandas отлично работает с CSV/Excel, иногда нужно работать с обычными текстовыми файлами (логи, конфиги, .txt). Важно делать это безопасно.
    \textit{Аналогия:} Файл - это коробка. Открытие/закрытие - это работа с крышкой. Конструкция `with open(...)` гарантирует, что ты \textbf{всегда закроешь коробку}, даже если что-то пойдет не так при упаковке/распаковке.
\end{textbox}

\begin{myblock}{{Стандартный способ Python (\texttt{with open})}}
    Рекомендуемый подход для работы с файлами.
    \begin{codebox}{python}{Чтение и запись текстовых файлов}
    # Запись в файл (перезапишет, если существует)
    lines_to_write = ["Строка 1\n", "Строка 2\n"]
    try: # Добавим try-except для случая, если файл недоступен
        with open('my_output.txt', 'w', encoding='utf-8') as f:
            f.write("Одна строка\n")
            f.writelines(lines_to_write)
        print("Файл 'my_output.txt' успешно записан.")
    except IOError as e:
        print(f"Ошибка записи файла: {e}")

    # Чтение из файла
    try:
        with open('my_output.txt', 'r', encoding='utf-8') as f:
            print("\nЧитаем файл 'my_output.txt':")
            # content_str = f.read()      # Прочитать весь файл в строку
            # content_list = f.readlines() # Прочитать все строки в список
            for line in f:             # Читать построчно (эффективно)
                print(line.strip())
    except FileNotFoundError:
        print("Файл 'my_output.txt' не найден.")
    except IOError as e:
         print(f"Ошибка чтения файла: {e}")


    # Режимы: 'r' - чтение, 'w' - запись (перезапись), 'a' - дозапись в конец
    # encoding='utf-8' - важно для русского языка!
    \end{codebox}
\end{myblock}

\begin{myblock}{{Pandas для Структурированных Файлов}}
    Для CSV, Excel, JSON, SQL и др. используй встроенные функции Pandas (используем `df` из блока "Применение Функций").
    \begin{codebox}{python}{Запись в CSV/Excel с Pandas}
    try:
        # Сохранить DataFrame в CSV без индекса
        df.to_csv('output_data.csv', index=False, encoding='utf-8')
        print("\nDataFrame успешно сохранен в 'output_data.csv'")

        # Сохранить DataFrame в Excel (раскомментируйте, если нужен Excel)
        # df.to_excel('output_data.xlsx', index=False, sheet_name='MyData')
        # print("DataFrame успешно сохранен в 'output_data.xlsx'")
    except Exception as e:
        print(f"Ошибка сохранения DataFrame: {e}")
    \end{codebox}
\end{myblock}

% --- Раздел 4: Визуализация ---
\section{Визуализация: Matplotlib и Seaborn}

\begin{textbox}{Зачем?}
    "Лучше один раз увидеть, чем сто раз услышать". Графики помогают понять данные, найти паттерны, выбросы и представить результаты.
    \begin{itemize}
        \item \textbf{Matplotlib (`plt`):} Низкоуровневая библиотека, дает полный контроль. \textit{Аналогия:} Набор инструментов художника (холст, кисти, краски).
        \item \textbf{Seaborn (`sns`):} Высокоуровневая, построена на Matplotlib. Упрощает создание красивых статистических графиков, хорошо интегрируется с Pandas. \textit{Аналогия:} Готовые шаблоны или трафареты для рисования стандартных фигур.
    \end{itemize}
\end{textbox}

\begin{myblock}{{Основы Matplotlib (\texttt{plt})}}
    Базовые команды для создания простых графиков.
    \begin{codebox}{python}{Примеры Matplotlib}
    import matplotlib.pyplot as plt
    import numpy as np
    import pandas as pd # Нужен для примера scatter

    # Данные для графиков
    x_lin = np.linspace(0, 10, 100)
    y1_lin = np.sin(x_lin)
    y2_lin = np.cos(x_lin)
    data_hist = np.random.randn(1000)
    data_box = [np.random.normal(0, std, 100) for std in range(1, 4)]
    # DataFrame для scatter
    df_scatter = pd.DataFrame({'feature1': np.random.rand(50) * 10,
                               'feature2': np.random.rand(50) * 10})

    # --- Создание фигур ---
    # Линейный график
    plt.figure(figsize=(8, 4)) # Размер фигуры (опционально)
    plt.plot(x_lin, y1_lin, label='sin(x)')
    plt.plot(x_lin, y2_lin, label='cos(x)', linestyle='--')
    plt.title('Линейный график (Matplotlib)')
    plt.xlabel('Ось X')
    plt.ylabel('Ось Y')
    plt.legend() # Показать легенду
    plt.grid(True) # Добавить сетку

    # Диаграмма рассеяния (Scatter plot)
    plt.figure()
    plt.scatter(df_scatter['feature1'], df_scatter['feature2'], alpha=0.5) # alpha - прозрачность
    plt.title('Диаграмма рассеяния (Matplotlib)')
    plt.xlabel('Feature 1')
    plt.ylabel('Feature 2')

    # Гистограмма
    plt.figure()
    plt.hist(data_hist, bins=30, color='skyblue', edgecolor='black')
    plt.title('Гистограмма (Matplotlib)')
    plt.xlabel('Значение')
    plt.ylabel('Частота')

    # Ящик с усами (Box plot)
    plt.figure()
    plt.boxplot(data_box, labels=['Gr1', 'Gr2', 'Gr3'])
    plt.title('Ящик с усами (Matplotlib)')
    plt.ylabel('Значение')

    # ВАЖНО: Отображение графиков plt.show() будет в конце секции
    \end{codebox}
\end{myblock}

\begin{myblock}{{Seaborn (\texttt{sns}) для Статистики}}
    Более красивые и статистически ориентированные графики, часто одной строкой.
    \begin{codebox}{python}{Примеры Seaborn}
    import seaborn as sns
    import matplotlib.pyplot as plt # Часто используется для донастройки sns графиков

    # Загрузка примера данных из seaborn
    tips = sns.load_dataset("tips") # DataFrame с данными о чаевых

    # --- Создание фигур ---
    # Линейный график (с доверительным интервалом по умолчанию)
    plt.figure() # Можно управлять размером через plt
    sns.lineplot(x="total_bill", y="tip", data=tips)
    plt.title('Линейный график (Seaborn)')

    # Диаграмма рассеяния (с возможностью раскраски по категории)
    plt.figure()
    sns.scatterplot(x="total_bill", y="tip", hue="time", data=tips)
    plt.title('Диаграмма рассеяния (Seaborn)')

    # Гистограмма (с оценкой плотности KDE)
    plt.figure()
    sns.histplot(data=tips, x="total_bill", kde=True)
    plt.title('Гистограмма (Seaborn)')

    # Ящик с усами (удобно для сравнения по категориям)
    plt.figure()
    sns.boxplot(x="day", y="total_bill", data=tips)
    plt.title('Ящик с усами (Seaborn)')

    # ВАЖНО: Отображение графиков plt.show() будет в конце секции
    \end{codebox}
    Seaborn часто автоматически подписывает оси и создает легенды, используя имена столбцов DataFrame.
\end{myblock}

\begin{myexampleblock}{Когда Какой График Использовать?}
    Краткий гид по выбору типа визуализации:
    \begin{itemize}
        \item \textbf{Линейный график (\texttt{plot}/\texttt{lineplot}):} Показать \textbf{тренд} или изменение показателя во времени (или по другой непрерывной оси). Сравнение трендов нескольких групп.
        \item \textbf{Диаграмма рассеяния (\texttt{scatter}/\texttt{scatterplot}):} Посмотреть \textbf{взаимосвязь} между двумя \textit{числовыми} переменными. Помогает найти корреляции, кластеры, выбросы.
        \item \textbf{Гистограмма (\texttt{hist}/\texttt{histplot}):} Понять \textbf{распределение} одной \textit{числовой} переменной. Как часто встречаются те или иные значения? Есть ли пики? Симметрично ли распределение?
        \item \textbf{Ящик с усами (\texttt{boxplot}):} Сравнить распределения \textit{числовой} переменной по нескольким \textit{категориям}. Показывает медиану, квартили, разброс и потенциальные выбросы в каждой группе.
    \end{itemize}
\end{myexampleblock}

% --- Команда для отображения всех графиков ---
% Поместите эту команду после всех блоков с графиками
\begin{center}
\textit{Для отображения всех созданных выше графиков Matplotlib/Seaborn, выполните:}
\begin{codebox}{python}{Показать все графики}
# Эта команда должна быть вызвана один раз после всех команд plt.figure()/sns.*plot()
# В средах типа Jupyter Notebook/Lab графики могут отображаться автоматически.
# В обычных Python скриптах plt.show() обязателен.
plt.show()
\end{codebox}
\end{center}


% --- Конец контента ---