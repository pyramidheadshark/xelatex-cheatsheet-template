% >>> Контент для шпаргалки по SQL для Data Science

% --- Раздел 1 ---
\section{Базовый Синтаксис и Извлечение Данных}

\begin{myblock}{Основы \texttt{SELECT} и \texttt{FROM}}
    Извлечение данных – хлеб с маслом аналитика. Начинаем с простого: выбираем нужные столбцы из нужной таблицы.
    \begin{codebox}{sql}{Выбор конкретных столбцов}
    SELECT column1, column2 -- Выбираем только нужные колонки
    FROM table_name;       -- Указываем, из какой таблицы
    \end{codebox}
    \begin{codebox}{sql}{Выбор всех столбцов}
    SELECT *               -- Выбрать ВСЕ столбцы
    FROM table_name;
    -- Внимание: Будь осторожен с '*' на больших таблицах.
    -- Это может быть медленно и неэффективно.
    -- Лучше явно перечислять нужные столбцы.
    \end{codebox}
\end{myblock}

\begin{textbox}{Псевдонимы (\texttt{AS}) и Уникальность (\texttt{DISTINCT})}
    Делаем запросы читаемыми и получаем только уникальные записи.
    \begin{codebox}{sql}{Псевдонимы (Aliases)}
    SELECT
        user_id AS id, -- Переименовываем user_id в id
        registration_date AS reg_date -- и registration_date в reg_date
    FROM
        users AS u; -- Переименовываем таблицу users в u (удобно для JOIN'ов)
    -- AS часто можно опустить: SELECT user_id id ...
    \end{codebox}
    \begin{codebox}{sql}{Уникальные значения}
    SELECT DISTINCT city -- Получить список уникальных городов
    FROM users;
    \end{codebox}
\end{textbox}

\begin{myblock}{Ограничение Вывода (\texttt{LIMIT} / \texttt{TOP})}
    Часто нужно просто взглянуть на первые несколько строк, а не тащить всю таблицу.
    \begin{codebox}{sql}{LIMIT (PostgreSQL, MySQL, SQLite)}
    SELECT user_id, name
    FROM users
    LIMIT 10; -- Показать только первые 10 строк
    \end{codebox}
    \begin{codebox}{sql}{TOP (SQL Server, MS Access)}
    SELECT TOP 10 user_id, name
    FROM users; -- Аналогично LIMIT 10
    \end{codebox}
    \textit{Примечание: Синтаксис зависит от конкретной СУБД (системы управления базами данных).}
\end{myblock}

% --- Раздел 2 ---
\section{Фильтрация Данных (\texttt{WHERE})}

\begin{myblock}{Условия отбора строк}
    \texttt{WHERE} – твой основной инструмент, чтобы отсеять ненужные строки и оставить только то, что соответствует условиям.
    \begin{codebox}{sql}{Операторы сравнения}
    SELECT product_name, price
    FROM products
    WHERE price > 100.0; -- Только товары дороже 100

    SELECT order_id, status
    FROM orders
    WHERE status != 'cancelled'; -- Все заказы, кроме отмененных (<> тоже работает)
    \end{codebox}
    \begin{codebox}{sql}{Логические операторы}
    SELECT user_id, city, registration_date
    FROM users
    WHERE city = 'Moscow' AND registration_date >= '2024-01-01'; -- Москвичи, зарег. в 2024+

    SELECT product_id, category
    FROM products
    WHERE category = 'Electronics' OR category = 'Appliances'; -- Электроника ИЛИ быт. техника
    \end{codebox}
\end{myblock}

\begin{textbox}{Более сложные условия фильтрации}
    Часто нужны диапазоны, списки или поиск по шаблону.
    \begin{codebox}{sql}{Диапазоны и Списки}
    -- BETWEEN: Включает границы диапазона
    SELECT order_id, order_date
    FROM orders
    WHERE order_date BETWEEN '2024-01-01' AND '2024-01-31';

    -- IN: Проверка на вхождение в список
    SELECT user_id, country
    FROM users
    WHERE country IN ('Russia', 'Belarus', 'Kazakhstan');
    \end{codebox}
    \begin{codebox}{sql}{Поиск по шаблону (LIKE)}
    -- '%' - любая последовательность символов (0 или больше)
    -- '_' - ровно один любой символ
    SELECT name
    FROM users
    WHERE name LIKE 'А%'; -- Имена, начинающиеся на 'А'

    SELECT email
    FROM users
    WHERE email LIKE '%@example.com'; -- Email на домене example.com

    SELECT product_code
    FROM products
    WHERE product_code LIKE 'PRD__X'; -- Коды вида PRD<2 символа>X
    \end{codebox}
    \begin{codebox}{sql}{Проверка на NULL}
    SELECT user_id, phone
    FROM users
    WHERE phone IS NULL; -- Пользователи без указанного телефона

    SELECT order_id, delivery_date
    FROM orders
    WHERE delivery_date IS NOT NULL; -- Заказы, которые были доставлены (дата есть)
    \end{codebox}
\end{textbox}

% --- Раздел 3 ---
\section{Сортировка Результатов (\texttt{ORDER BY})}

\begin{myblock}{Упорядочивание вывода}
    Чтобы результаты были представлены в нужном порядке (например, от новых к старым, от дорогих к дешевым).
    \begin{codebox}{sql}{Примеры сортировки}
    -- Сортировка по одному столбцу (по умолчанию ASC - возрастание)
    SELECT product_name, price
    FROM products
    ORDER BY price; -- От самых дешевых к дорогим (ASC)

    -- Сортировка по убыванию (DESC)
    SELECT user_id, registration_date
    FROM users
    ORDER BY registration_date DESC; -- От самых новых к старым

    -- Сортировка по нескольким столбцам
    SELECT city, name
    FROM users
    ORDER BY city ASC, name ASC; -- Сначала по городу (А-Я), потом по имени (А-Я) внутри города
    \end{codebox}
    \textit{Примечание: \texttt{ORDER BY} обычно идет в конце запроса (после \texttt{WHERE}, но до \texttt{LIMIT}).}
\end{myblock}

% --- Раздел 4 ---
\section{Агрегатные Функции и Группировка (\texttt{GROUP BY}, \texttt{HAVING})}

\begin{myblock}{Агрегация: Считаем итоги}
    Агрегатные функции выполняют вычисления над набором строк и возвращают одно значение.
    \begin{itemize}
        \item \texttt{COUNT(*)}: Общее количество строк.
        \item \texttt{COUNT(column)}: Количество не-NULL значений в столбце.
        \item \texttt{SUM(column)}: Сумма значений.
        \item \texttt{AVG(column)}: Среднее значение.
        \item \texttt{MIN(column)}: Минимальное значение.
        \item \texttt{MAX(column)}: Максимальное значение.
    \end{itemize}
    \begin{codebox}{sql}{Примеры агрегатных функций}
    -- Сколько всего пользователей?
    SELECT COUNT(*) AS total_users FROM users;

    -- Сколько пользователей указали город?
    SELECT COUNT(city) AS users_with_city FROM users;

    -- Общая сумма всех заказов
    SELECT SUM(amount) AS total_revenue FROM orders;

    -- Средняя цена товара
    SELECT AVG(price) AS avg_product_price FROM products;

    -- Самый ранний и самый поздний заказ
    SELECT MIN(order_date) AS first_order, MAX(order_date) AS last_order FROM orders;
    \end{codebox}
\end{myblock}

\begin{textbox}{Группировка (\texttt{GROUP BY})}
    \textbf{Идея:} Разделить строки на группы по значениям в одном или нескольких столбцах и применить агрегатные функции \textit{к каждой группе отдельно}.
    Представь, что ты сортируешь чеки по магазинам (\texttt{GROUP BY store}), а потом считаешь сумму покупок для каждого магазина (\texttt{SUM(amount)}).
    \begin{codebox}{sql}{Пример GROUP BY}
    -- Количество пользователей в каждом городе
    SELECT city, COUNT(*) AS user_count
    FROM users
    GROUP BY city
    ORDER BY user_count DESC; -- Сортируем по количеству

    -- Средняя сумма заказа для каждого пользователя
    SELECT user_id, AVG(amount) AS avg_order_amount
    FROM orders
    GROUP BY user_id;
    \end{codebox}
    \textbf{Важное правило:} В части \texttt{SELECT} запроса с \texttt{GROUP BY} могут быть только:
    \begin{itemize}
        \item Столбцы, перечисленные в \texttt{GROUP BY}.
        \item Агрегатные функции.
    \end{itemize}
\end{textbox}

\begin{alerttextbox}{Фильтрация После Группировки (\texttt{HAVING})}
    Что если нужно отфильтровать результаты \textit{после} агрегации? Например, показать только те города, где больше 100 пользователей? Для этого есть \texttt{HAVING}.
    \begin{codebox}{sql}{Пример HAVING}
    -- Города с более чем 100 пользователями
    SELECT city, COUNT(*) AS user_count
    FROM users
    GROUP BY city
    HAVING COUNT(*) > 100 -- Фильтруем по результату агрегации
    ORDER BY user_count DESC;

    -- Пользователи, чья средняя сумма заказа больше 500
    SELECT user_id, AVG(amount) AS avg_order_amount
    FROM orders
    GROUP BY user_id
    HAVING AVG(amount) > 500;
    \end{codebox}
    \textbf{Ключевое отличие (частый вопрос на собеседованиях):}
    \begin{itemize}
        \item \texttt{WHERE} фильтрует \textbf{строки} \textit{до} группировки и агрегации.
        \item \texttt{HAVING} фильтрует \textbf{группы} \textit{после} группировки и агрегации (работает с результатами агрегатных функций).
    \end{itemize}
    \textit{Порядок выполнения (упрощенно): \texttt{FROM} -> \texttt{WHERE} -> \texttt{GROUP BY} -> \texttt{HAVING} -> \texttt{SELECT} -> \texttt{ORDER BY} -> \texttt{LIMIT}.}
\end{alerttextbox}

% --- Раздел 5 ---
\section{Объединение Таблиц (\texttt{JOIN})}

\begin{myblock}{Соединяем данные из разных таблиц}
    Данные часто хранятся в разных таблицах (например, пользователи в одной, их заказы в другой). \texttt{JOIN} позволяет их объединить по общему ключу.
    \textbf{Аналогия:} Представь два списка – список сотрудников с их ID и список отделов с ID сотрудников. JOIN позволяет "склеить" эти списки, чтобы увидеть, кто в каком отделе работает.
\end{myblock}

\begin{textbox}{Основные типы JOIN'ов}
    \begin{codebox}{sql}{INNER JOIN (Внутреннее соединение)}
    -- Возвращает только те строки, для которых есть совпадение ключа В ОБЕИХ таблицах.
    -- Если у пользователя нет заказов, он НЕ попадет в результат.
    SELECT
        u.name, -- Имя пользователя из таблицы users
        o.order_id, -- ID заказа из таблицы orders
        o.amount -- Сумма заказа
    FROM
        users AS u
    INNER JOIN
        orders AS o ON u.user_id = o.user_id; -- Условие соединения (ключи)
    \end{codebox}

    \begin{codebox}{sql}{LEFT JOIN (Левое внешнее соединение)}
    -- Возвращает ВСЕ строки из ЛЕВОЙ таблицы (users)
    -- и совпадающие строки из ПРАВОЙ таблицы (orders).
    -- Если у пользователя нет заказов, он ВСЕ РАВНО попадет в результат,
    -- но столбцы из orders (order_id, amount) будут NULL.
    -- КРАЙНЕ ВАЖЕН, чтобы не терять данные из основной (левой) таблицы!
    SELECT
        u.name,
        o.order_id,
        o.amount
    FROM
        users AS u
    LEFT JOIN
        orders AS o ON u.user_id = o.user_id;
    \end{codebox}

    \textit{Другие типы (реже используются):}
    \begin{itemize}
        \item \texttt{RIGHT JOIN}: Аналогичен \texttt{LEFT JOIN}, но возвращает все строки из правой таблицы.
        \item \texttt{FULL OUTER JOIN}: Возвращает все строки из обеих таблиц, подставляя \texttt{NULL} там, где нет совпадений.
        \item \texttt{CROSS JOIN}: Декартово произведение (каждая строка одной таблицы с каждой строкой другой). Используй с большой осторожностью!
    \end{itemize}
\end{textbox}

% --- Раздел 6 ---
\section{Подзапросы (Subqueries)}

\begin{myblock}{Запрос внутри запроса}
    Подзапрос – это \texttt{SELECT}-запрос, вложенный внутрь другого SQL-запроса. Позволяет выполнять более сложные выборки.
    \begin{codebox}{sql}{Подзапрос в WHERE (самое частое)}
    -- Найти пользователей, сделавших заказы на сумму > 1000
    SELECT name
    FROM users
    WHERE user_id IN ( -- Выбираем user_id из подзапроса
        SELECT user_id
        FROM orders
        WHERE amount > 1000
    );

    -- Найти товары, дороже средней цены всех товаров
    SELECT product_name, price
    FROM products
    WHERE price > ( -- Сравнение со скалярным подзапросом (возвращает 1 значение)
        SELECT AVG(price) FROM products
    );
    \end{codebox}
    \begin{codebox}{sql}{Подзапрос в FROM (Производная таблица)}
    -- Сначала считаем среднее в подзапросе, потом выбираем из него
    SELECT t.avg_val
    FROM (
        SELECT category, AVG(price) AS avg_val
        FROM products
        GROUP BY category
    ) AS t -- Псевдоним для подзапроса обязателен!
    WHERE t.avg_val > 50;
    \end{codebox}
    \begin{codebox}{sql}{Подзапрос в SELECT (Скалярный подзапрос)}
    -- Добавляем столбец с максимальной ценой ко всем строкам
    SELECT
        product_name,
        price,
        (SELECT MAX(price) FROM products) AS max_overall_price
    FROM products;
    \end{codebox}
    \textit{Операторы для подзапросов в \texttt{WHERE}: \texttt{IN}, \texttt{NOT IN}, \texttt{EXISTS}, \texttt{NOT EXISTS}, операторы сравнения (\texttt{=}, \texttt{>}, \texttt{<} и т.д., если подзапрос возвращает одно значение).}
\end{myblock}

% --- Раздел 7 ---
\section{Общие Табличные Выражения (CTE - Common Table Expressions)}

\begin{myblock}{\texttt{WITH}: Улучшаем читаемость сложных запросов}
    CTE – это именованный временный результирующий набор, на который можно ссылаться в последующем \texttt{SELECT}, \texttt{INSERT}, \texttt{UPDATE} или \texttt{DELETE}.
    \textbf{Преимущества:}
    \begin{itemize}
        \item **Читаемость:** Разбивает сложный запрос на логические блоки.
        \item **Структура:** Упрощает понимание логики запроса.
        \item **Переиспользование:** На CTE можно ссылаться несколько раз внутри одного запроса (в большинстве СУБД).
        \item **Рекурсия:** Позволяют писать рекурсивные запросы (редко нужно в DS).
    \end{itemize}
    \textbf{Часто это ЛУЧШЕ, чем вложенные подзапросы!}
    \begin{codebox}{sql}{Пример CTE}
    -- Найти пользователей, сделавших > 5 заказов на сумму > 100
    WITH UserOrderStats AS ( -- Определяем CTE с именем UserOrderStats
        SELECT
            user_id,
            COUNT(*) AS order_count,
            SUM(amount) AS total_amount
        FROM orders
        WHERE amount > 100 -- Фильтр применяется здесь
        GROUP BY user_id
    )
    -- Основной запрос, использующий CTE
    SELECT
        u.name,
        uos.order_count,
        uos.total_amount
    FROM
        users AS u
    JOIN
        UserOrderStats AS uos ON u.user_id = uos.user_id
    WHERE
        uos.order_count > 5 -- Фильтруем по результатам CTE
    ORDER BY
        uos.total_amount DESC;
    \end{codebox}
\end{myblock}

% --- Раздел 8 ---
\section{Оконные Функции (Window Functions)}

\begin{alerttextbox}{Продвинутая аналитика без схлопывания строк}
    \textbf{Идея:} Оконные функции выполняют вычисления над набором строк ("окном"), которые как-то связаны с текущей строкой, но \textbf{не схлопывают строки}, как \texttt{GROUP BY}. Каждая строка сохраняется, и к ней добавляется результат вычисления оконной функции.
    \textbf{Аналогия:} Представь, что ты стоишь в очереди. \texttt{GROUP BY} может сказать тебе только, сколько всего людей в очереди. Оконная функция может сказать тебе твой номер в очереди (\texttt{ROW\_NUMBER}), или какой средний рост у людей вокруг тебя (\texttt{AVG() OVER(...)}), не убирая тебя из очереди.

    \textbf{Общий синтаксис:}
    \texttt{FUNCTION() OVER ( [PARTITION BY ...] [ORDER BY ...] [frame\_clause] )}
    \begin{itemize}
        \item \texttt{PARTITION BY column1, ...}: Делит строки на независимые группы (партиции). Функция применяется к каждой партиции отдельно. Похоже на \texttt{GROUP BY}, но строки НЕ схлопываются. Если опущено, всё окно - это все строки.
        \item \texttt{ORDER BY column2, ...}: Определяет порядок строк внутри партиции. Важно для ранжирующих функций (\texttt{RANK}, \texttt{ROW\_NUMBER}) и функций смещения (\texttt{LAG}, \texttt{LEAD}).
        \item \texttt{frame\_clause} (\texttt{ROWS/RANGE BETWEEN ...}): Определяет точное подмножество строк внутри партиции для вычисления (например, "текущая строка и 2 предыдущие"). Реже используется новичками.
    \end{itemize}
\end{alerttextbox}

\begin{myexampleblock}{Часто используемые оконные функции}
    \begin{codebox}{sql}{Ранжирующие функции}
    -- Ранжирование продуктов по цене ВНУТРИ каждой категории
    SELECT
        product_name,
        category,
        price,
        ROW_NUMBER() OVER(PARTITION BY category ORDER BY price DESC) AS rn, -- Уникальный номер 1, 2, 3...
        RANK()       OVER(PARTITION BY category ORDER BY price DESC) AS rk, -- Ранг с пропусками (1, 2, 2, 4...)
        DENSE_RANK() OVER(PARTITION BY category ORDER BY price DESC) AS drk -- Ранг без пропусков (1, 2, 2, 3...)
    FROM products;
    \end{codebox}
    \begin{codebox}{sql}{Агрегатные функции как оконные}
    -- Расчет доли цены продукта от средней цены В ЕГО КАТЕГОРИИ
    SELECT
        product_name,
        category,
        price,
        AVG(price) OVER(PARTITION BY category) AS avg_price_in_category,
        price / AVG(price) OVER(PARTITION BY category) AS price_ratio
    FROM products;

    -- Общая сумма продаж нарастающим итогом по дате
    SELECT
      order_date,
      amount,
      SUM(amount) OVER(ORDER BY order_date ASC) as cumulative_sales
    FROM orders;
    \end{codebox}
    \begin{codebox}{sql}{Функции смещения (LAG/LEAD)}
    -- Поиск предыдущей даты заказа для каждого пользователя
    SELECT
        user_id,
        order_date,
        LAG(order_date, 1) OVER(PARTITION BY user_id ORDER BY order_date ASC) AS previous_order_date
    FROM orders;
    -- LAG(столбец, смещение_назад, значение_по_умолчанию)
    -- LEAD(столбец, смещение_вперед, значение_по_умолчанию)
    \end{codebox}
    \textbf{Применение в DS:} Ранжирование (топ N клиентов/товаров в группе), расчет долей (\% от тотала по группе), поиск временных паттернов (время между событиями с \texttt{LAG}/\texttt{LEAD}), скользящие средние/суммы. \textbf{Очень популярная тема на собеседованиях!}
\end{myexampleblock}

% --- Раздел 9 ---
\section{Работа с \texttt{NULL}}

\begin{myblock}{Обработка отсутствующих значений}
    \texttt{NULL} – это специальное значение, означающее "отсутствие данных". С ним нужно работать аккуратно.
    \begin{codebox}{sql}{COALESCE: Замена NULL}
    -- Возвращает первое не-NULL значение из списка.
    -- Очень полезно для подстановки значений по умолчанию.
    SELECT
        name,
        COALESCE(phone, 'Телефон не указан') AS phone_display
    FROM users;

    -- Если first_name есть, берем его, иначе берем username
    SELECT COALESCE(first_name, username) AS display_name FROM users;
    \end{codebox}
    \begin{codebox}{sql}{NULLIF: NULL, если значения равны}
    -- Возвращает NULL, если два выражения равны, иначе возвращает первое выражение.
    -- Полезно, чтобы избежать деления на ноль или обработать "пустые" строки.
    SELECT
        product_name,
        total_sales,
        total_quantity,
        -- Избегаем деления на 0: если quantity=0, NULLIF вернет NULL, и деление даст NULL
        total_sales / NULLIF(total_quantity, 0) AS average_price
    FROM product_summary;
    \end{codebox}
    \textit{Помни: Сравнение с \texttt{NULL} через \texttt{=} или \texttt{!=} почти всегда дает \texttt{UNKNOWN} (не TRUE). Используй \texttt{IS NULL} или \texttt{IS NOT NULL}.}
\end{myblock}

% --- Раздел 10 ---
\section{Типы Данных и Преобразования}

\begin{myblock}{Основные типы и конвертация}
    У каждого столбца есть тип данных. Иногда нужно их преобразовывать.
    \textbf{Частые типы данных:}
    \begin{itemize}
        \item Числовые: \texttt{INTEGER} (INT), \texttt{FLOAT}, \texttt{REAL}, \texttt{NUMERIC}, \texttt{DECIMAL}
        \item Строковые: \texttt{VARCHAR(n)}, \texttt{TEXT}, \texttt{CHAR(n)}
        \item Даты/Время: \texttt{DATE}, \texttt{TIME}, \texttt{TIMESTAMP}, \texttt{DATETIME}
        \item Логический: \texttt{BOOLEAN} (BOOL)
    \end{itemize}
    \textit{(Точные имена и доступные типы зависят от СУБД)}

    \begin{codebox}{sql}{Явное преобразование типов (CAST/CONVERT)}
    -- Синтаксис может отличаться!

    -- CAST (стандартный SQL)
    SELECT CAST('2024-03-15' AS DATE);
    SELECT CAST(price AS INTEGER) FROM products; -- Отбросит дробную часть
    SELECT CAST(user_id AS VARCHAR) FROM users; -- Число в строку

    -- CONVERT (SQL Server)
    -- SELECT CONVERT(DATE, '2024-03-15');
    -- SELECT CONVERT(VARCHAR, user_id);

    -- Неявное преобразование тоже бывает, но лучше делать явно.
    \end{codebox}
    \textit{Будь внимателен при преобразованиях: возможна потеря данных (например, FLOAT -> INT) или ошибки, если преобразование невозможно ('abc' -> INT).}
\end{myblock}

% --- Фокус для подготовки ---
\newpage % Начать новую колонку/страницу для лучшей читаемости фокуса
\section*{Фокус для Подготовки к Собеседованию}
\begin{alerttextbox}{На что обратить особое внимание:}
    \begin{itemize}
        \item \textbf{Практика, практика, практика!} Теория важна, но умение быстро писать рабочие запросы – ключ. Используй онлайн-тренажеры (LeetCode Database, HackerRank SQL, SQL Fiddle) или скачай SQLite базу с Kaggle и экспериментируй.
        \item \textbf{JOIN'ы – твой лучший друг (особенно LEFT):} Четко понимай разницу между \texttt{INNER} и \texttt{LEFT JOIN}. Подумай, когда ты можешь потерять данные с \texttt{INNER JOIN} и почему \texttt{LEFT JOIN} часто безопаснее.
        \item \textbf{Агрегация и Фильтрация Групп:} \texttt{GROUP BY} + Агрегатные функции + \texttt{HAVING} – это классический набор для задач на собеседованиях. Убедись, что ты понимаешь, как они работают вместе и чем \texttt{WHERE} отличается от \texttt{HAVING}.
        \item \textbf{Оконные Функции – покажи свой уровень:} Даже базовое понимание \texttt{ROW\_NUMBER()}/\texttt{RANK()} и агрегатных функций с \texttt{OVER(PARTITION BY ...)} (например, для расчета доли от итога по группе) произведет хорошее впечатление. \texttt{LAG}/\texttt{LEAD} – бонусные очки.
        \item \textbf{Чистый и Читаемый Код:} Используй псевдонимы (\texttt{AS}), форматируй запросы (отступы), используй CTE (\texttt{WITH}) для сложных запросов вместо нагромождения подзапросов. Это показывает твою аккуратность и профессионализм.
    \end{itemize}
    \textit{Удачи на собеседовании! SQL – это мощный инструмент, и уверенное владение им – большой плюс для любого Data Scientist'а.}
\end{alerttextbox}

% --- Конец контента ---