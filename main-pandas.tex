\documentclass[10pt,a4paper]{article}

\usepackage[landscape,margin=0.7cm]{geometry}
\usepackage[russian,english]{babel} % Поддержка русского и английского

% >>> КОММЕНТАРИЙ: Определяем переменные для заголовка и автора
\newcommand{\cheatsheettitle}{\color{w3schools}Шпаргалка по Pandas / {\color{alert} Pandas} {\color{black} Cheatsheet (XeLaTeX)}}
\newcommand{\cheatsheetauthor}{Создано с использованием LaTeX шаблона}

% >>> КОММЕНТАРИЙ: Подключаем основной файл шаблона с настройками стилей и пакетов
% \documentclass{article} % ЗАКОММЕНТИРОВАНО: Добавлено для возможности компиляции этого фрагмента отдельно для проверки
\usepackage{fontspec}

% >>> КОММЕНТАРИЙ: Установка основных шрифтов документа (требует установленных шрифтов в системе)
\setmainfont{PT Sans} % Основной шрифт для текста
\setsansfont{PT Sans} % Шрифт без засечек (если используется отдельно)
\setmonofont{Liberation Mono} % Моноширинный шрифт (для кода)

\usepackage{fontawesome} % Для иконок (например, \faQuoteLeft)
\usepackage{hyperref} % Для создания кликабельных ссылок (в т.ч. в \resourcelink)
\usepackage{enumitem} % Для настройки списков
\usepackage{lipsum} % Для генерации текста-рыбы (не используется в финальной версии)
\usepackage{xcolor} % Для определения и использования цветов
\usepackage{minted} % Для подсветки синтаксиса кода (требует Python и Pygments, компиляция с -shell-escape)

% >>>>>> НОВОЕ: Пакет для создания графиков <<<<<<
\usepackage{pgfplots}
\pgfplotsset{compat=newest} % Устанавливаем последнюю версию совместимости
\usetikzlibrary{arrows.meta, shapes.geometric} % Загружаем полезные библиотеки TikZ

\usepackage{titlesec} % Для настройки заголовков секций
% >>> КОММЕНТАРИЙ: Настройка отступов для заголовков секций (\section)
\titlespacing*{\section}{0pt}{*1.5}{*0.8} % {отступ слева}{отступ сверху}{отступ снизу}

% >>> КОММЕНТАРИЙ: Форматирование заголовков секций (\section)
\titleformat{\section}
  {\normalfont\large\bfseries\color{mDarkTeal}} % Стиль заголовка (цвет, размер, жирность)
  {\thesection} % Номер секции
  {1em} % Отступ после номера
  {} % Код перед заголовком (здесь пусто)

% >>> КОММЕНТАРИЙ: Определение пользовательских цветов через HTML-коды
\definecolor{customcolor}{HTML}{696EA8} % Основной цвет для блоков textbox
\definecolor{alert}{HTML}{CD5C5C} % Цвет для выделения (красный)
\definecolor{w3schools}{HTML}{4CAF50} % Цвет для выделения (зеленый)
\definecolor{subbox}{gray}{0.60} % Цвет для под-блоков (не используется?)
\definecolor{codecolor}{HTML}{FFC300} % Цвет для кода (не используется?)
\definecolor{mDarkTeal}{HTML}{23373b} % Темно-бирюзовый (для заголовков секций, названий блоков)
\colorlet{beamerboxbg}{black!2} % Фон для beamerbox (не используется?)
\colorlet{beamerboxfg}{mDarkTeal} % Цвет текста для beamerbox (не используется?)
\definecolor{mDarkBrown}{HTML}{604c38} % Темно-коричневый (не используется?)
\definecolor{mLightBrown}{HTML}{EB811B} % Светло-коричневый (для заголовков myalertblock)
\definecolor{mLightGreen}{HTML}{14B03D} % Светло-зеленый (для заголовков myexampleblock)
\definecolor{alertTextBox}{HTML}{A8696E} % Цвет для блоков alerttextbox

% >>> КОММЕНТАРИЙ: Настройка пакета biblatex для библиографии
\usepackage
[citestyle=authoryear, % Стиль цитирования (Автор-Год)
sorting=nty, % Сортировка по Имени, Году, Названию
autocite=footnote, % Команда \autocite создает сноски
autolang=hyphen, % Автоматическое определение языка для переносов
mincrossrefs=1, % Минимальное количество перекрестных ссылок
backend=biber] % Используемый бэкенд (требует запуск biber)
{biblatex}

% >>> КОММЕНТАРИЙ: Настройка формата пост-заметок в цитатах (например, номера страниц)
\DeclareFieldFormat{postnote}{#1}
\DeclareFieldFormat{multipostnote}{#1}
\DeclareAutoCiteCommand{footnote}[f]{\footcite}{\footcites} % Настройка команды \autocite

% >>> КОММЕНТАРИЙ: Подключение файла с библиографическими записями
\addbibresource{literature.bib}

% >>> КОММЕНТАРИЙ: Подключение библиотеки tcolorbox для создания цветных блоков
\usepackage{tcolorbox}
\tcbuselibrary{most, listingsutf8, minted} % Загрузка необходимых библиотек tcolorbox (most включает breakable)

% >>> КОММЕНТАРИЙ: Глобальные настройки для маленьких tcbox (inline блоки)
\tcbset{tcbox width=auto,left=1mm,top=1mm,bottom=1mm,
right=1mm,boxsep=1mm,middle=1pt}

% >>> КОММЕНТАРИЙ: Определение базового окружения для цветных блоков с заголовком
% breakable: позволяет блоку разрываться между колонками/страницами.
% ВАЖНО: В multicols автоматический разрыв breakable может работать не всегда идеально.
% Если блок разрывается некрасиво, попробуйте разбить контент на несколько блоков
% или используйте \needspace{<длина>} перед блоком.
\newenvironment{mycolorbox}[2]{
\begin{tcolorbox}[capture=minipage,fonttitle=\large\bfseries, enhanced,boxsep=1mm,colback=#1!30!white,
width=\linewidth, % Ширина блока равна ширине текущей колонки
arc=2pt,outer arc=2pt, toptitle=0mm,colframe=#1,opacityback=0.7,nobeforeafter,
breakable, % Разрешить разрыв блока
title=#2] % Текст заголовка блока
}{\end{tcolorbox}}

% >>> КОММЕНТАРИЙ: Окружение subbox (не используется в текущем коде)
\newenvironment{subbox}[2]{
\begin{tcolorbox}[capture=minipage,fonttitle=\normalsize\bfseries, enhanced,boxsep=1mm,colback=#1!30!white,on line,tcbox width=auto,left=0.3em,top=1mm, toptitle=0mm,colframe=#1,opacityback=0.7,nobeforeafter,
breakable,
title=#2]\footnotesize
}{\normalsize\end{tcolorbox}\vspace{0.1em}}

% >>> КОММЕНТАРИЙ: Окружение для размещения tcolorbox'ов в несколько колонок внутри блока (не используется)
\newenvironment{multibox}[1]{
\begin{tcbraster}[raster columns=#1,raster equal height,nobeforeafter,raster column skip=1em,raster left skip=1em,raster right skip=1em]}{\end{tcbraster}}

% >>> КОММЕНТАРИЙ: Окружения-обертки для mycolorbox с предопределенными цветами и заголовками
\newenvironment{textbox}[1]{\begin{mycolorbox}{customcolor}{#1}}{\end{mycolorbox}} % Обычный текстовый блок
\newenvironment{alerttextbox}[1]{\begin{mycolorbox}{alertTextBox}{#1}}{\end{mycolorbox}} % Блок для предупреждений/важной информации

% >>> КОММЕНТАРИЙ: Определение окружения для блоков с кодом с подсветкой minted
\newtcblisting{codebox}[3][]{ % [опции minted], {язык}, {заголовок}
colback=black!10,colframe=black!20, % Цвета фона и рамки
listing only, % Показывать только листинг кода
minted options={ % Опции, передаваемые в minted
    numbers=left, % Нумерация строк слева
    style=default, % Стиль подсветки (можно выбрать другие, напр., 'friendly')
    fontsize=\scriptsize, % Размер шрифта кода
    breaklines, % Автоматический перенос длинных строк
    breaksymbolleft=, % Символ для обозначения переноса (пусто)
    autogobble, % Удаление лишних отступов слева
    linenos, % Показывать номера строк
    numbersep=0.7em, % Отступ номеров строк от кода
    #1 % Дополнительные опции minted, переданные в [1]
    },
enhanced,top=1mm, toptitle=0mm,
left=5mm, % Отступ слева внутри блока (для номеров строк)
arc=0pt,outer arc=0pt, % Прямые углы
title={#3}, % Заголовок блока кода
fonttitle=\small\bfseries\color{mDarkTeal}, % Стиль заголовка блока кода
listing engine=minted, % Движок для листинга - minted
minted language=#2, % Язык программирования для подсветки
breakable % Разрешить разрыв блока кода
}

\newcommand{\punkti}{~\lbrack\dots\rbrack~} % Команда для [...] (не используется)

% >>> КОММЕНТАРИЙ: Переопределение окружения quote для добавления иконок цитат
\renewenvironment{quote}
               {\list{\faQuoteLeft\phantom{ }}{\rightmargin\leftmargin}
                \item\relax\scriptsize\ignorespaces}
               {\unskip\unskip\phantom{xx}\faQuoteRight\endlist}

% >>> КОММЕНТАРИЙ: Вспомогательные команды для создания цветных фонов под текстом (не используются)
\newcommand{\bgupper}[3]{\colorbox{#1}{\color{#2}\huge\bfseries\MakeUppercase{#3}}}
\newcommand{\bg}[3]{\colorbox{#1}{\bfseries\color{#2}#3}}

% >>> КОММЕНТАРИЙ: Команда для форматирования описания команды/функции
% #1: Сама команда (используется \detokenize для корректного отображения спецсимволов)
% #2: Описание команды
\newcommand{\mycommand}[2]{
  {\par\noindent\ttfamily\detokenize{#1}\par} % Вывод команды моноширинным шрифтом
  \nopagebreak % Стараемся не разрывать страницу после команды
  \hangindent=1.5em \hangafter=1 \noindent % Делаем отступ для описания
  \small\textit{#2}\par\vspace{0.5ex} % Вывод описания и небольшой отступ после
}

% >>> КОММЕНТАРИЙ: Команда для форматирования ссылок на ресурсы
% #1: URL ссылки
% #2: Текст ссылки
% #3: Описание ресурса
\newcommand{\resourcelink}[3]{
  {\par\noindent\href{#1}{\ttfamily #2}\par} % Вывод кликабельной ссылки
  \nopagebreak
  \hangindent=1.5em \hangafter=1 \noindent
  \small\textit{#3}\par\vspace{0.5ex} % Вывод описания
}

% >>> КОММЕНТАРИЙ: Вспомогательные команды для стилизации
\newcommand{\sep}{{\scriptsize~\faCircle{ }~}} % Разделитель-кружок
\newcommand{\bggreen}[1]{\medskip\bgupper{w3schools}{black}{#1}\\[0.5em]} % Зеленый заголовок (не используется)
\newcommand{\green}[1]{\smallskip\bg{w3schools}{white}{#1}\\} % Текст на зеленом фоне
\newcommand{\red}[1]{\smallskip\bg{alert}{white}{#1}\\} % Текст на красном фоне
\newcommand{\alertcmd}[1]{\red{#1}\\} % Псевдоним для red (переименовано чтобы не конфликтовать с цветом alert)

\usepackage{multicol} % Пакет для создания нескольких колонок
\setlength{\columnsep}{15pt} % Расстояние между колонками

\setlength{\parindent}{0pt} % Убираем абзацный отступ по умолчанию
\usepackage{csquotes} % Для правильного отображения кавычек в разных языках (используется biblatex)
\newcommand{\loremipsum}{Lorem ipsum dolor sit amet.} % Пример текста (не используется)

% >>> КОММЕНТАРИЙ: Окружения для блоков с разными цветами заголовков (пример, предупреждение, обычный)
% Основаны на tcolorbox, похожи на mycolorbox, но с другими цветами заголовков и рамки.
% Также используют breakable. Применяются для выделения примеров, предупреждений и т.д.
% ВАЖНО: Лог может содержать предупреждения "Underfull \hbox (badness 10000)".
% Это часто случается в узких колонках multicol при наличии кода, URL или текста,
% который плохо переносится. Обычно это лишь косметическая проблема, строки выглядят
% немного не до конца заполненными. Можно игнорировать или попробовать перефразировать/
% использовать \sloppy перед проблемным абзацем.
\newenvironment{myexampleblock}[1]{ % Блок для примеров (зеленый заголовок)
    \tcolorbox[capture=minipage,fonttitle=\small\bfseries\color{mLightGreen}, enhanced,boxsep=1mm,colback=black!10,breakable,noparskip,
    on line,tcbox width=auto,left=0.3em,top=1mm, toptitle=0mm,
    colframe=black!20,arc=0pt,outer arc=0pt,
    opacityback=0.7,nobeforeafter,title=#1]}
    {\endtcolorbox}

\newenvironment{myalertblock}[1]{ % Блок для предупреждений (оранжевый заголовок)
    \tcolorbox[capture=minipage,fonttitle=\small\bfseries\color{mLightBrown}, enhanced,boxsep=1mm,colback=black!10,breakable,noparskip,
    on line,tcbox width=auto,left=0.3em,top=1mm, toptitle=0mm,
    colframe=black!20,arc=0pt,outer arc=0pt,
    opacityback=0.7,nobeforeafter,title=#1]}
    {\endtcolorbox}

\newenvironment{myblock}[1]{ % Обычный информационный блок (бирюзовый заголовок)
    \tcolorbox[capture=minipage,fonttitle=\small\bfseries\color{mDarkTeal}, enhanced,boxsep=1mm,colback=black!10,breakable,noparskip,
    on line,tcbox width=auto,left=0.3em,top=1mm, toptitle=0mm,
    colframe=black!20, arc=0pt,outer arc=0pt,
    opacityback=0.7,nobeforeafter,title=#1]}
    {\endtcolorbox}

% >>> КОММЕНТАРИЙ: Команда для вставки изображения в рамке tcolorbox с подписью
% #1: Опции (не используются)
% #2: Путь к файлу изображения (используется как подпись под изображением)
\newcommand{\mygraphics}[2][]{
\tcbox[enhanced,boxsep=0pt,top=0pt,bottom=0pt,left=0pt,
right=0pt,boxrule=0.4pt,drop fuzzy shadow,clip upper,
colback=black!75!white,toptitle=2pt,bottomtitle=2pt,nobeforeafter,
center title,fonttitle=\small\sffamily,title=\detokenize{#2}] % Используем путь как заголовок
{\includegraphics[width=\the\dimexpr(\linewidth-4mm)\relax]{#2}} % Вставляем изображение, ширина чуть меньше колонки
}

% >>> КОМЕНТАРИЙ: Команда \needspace{<длина>}
% Используйте эту команду *перед* блоком (tcolorbox, section, figure), чтобы убедиться,
% что в текущей колонке есть как минимум <длина> свободного места. Если места нет,
% LaTeX начнет новую колонку/страницу перед выполнением команды.
% Пример: \needspace{10\baselineskip} % Запросить место для примерно 10 строк текста
% Это полезно для предотвращения "висячих" заголовков или некрасивых разрывов блоков.

% >>>>>> НОВОЕ: Стили для графиков PGFPlots <<<<<<
\pgfplotsset{
    % Базовый стиль для всех графиков в шпаргалке
    cheatsheet plot style/.style={
        width=0.98\linewidth, % Ширина чуть меньше колонки для аккуратности
        height=5cm,          % Высота графика (можно изменить по необходимости)
        grid=major,          % Включаем основную сетку
        grid style={dashed, color=black!20}, % Стиль сетки: пунктирная, светло-серая
        axis lines=left,     % Линии осей только слева и снизу
        axis line style={color=black!60, thick}, % Стиль линий осей: серые, потолще
        tick style={color=black!60, thick}, % Стиль меток на осях
        ticklabel style={font=\scriptsize, color=black}, % Стиль подписей меток: маленький шрифт
        label style={font=\small, color=mDarkTeal}, % Стиль подписей осей: шрифт small, цвет как у заголовков блоков
        title style={font=\small\bfseries, color=mDarkTeal}, % Стиль заголовка графика
        legend style={        % Стиль легенды
            font=\scriptsize, % Маленький шрифт
            draw=black!30,    % Легкая рамка вокруг легенды
            fill=white,       % Белый фон
            legend cell align=left, % Выравнивание текста в ячейке легенды
            anchor=north east, % Якорь для позиционирования
            at={(rel axis cs:0.98,0.98)} % Положение в правом верхнем углу внутри области графика
        },
        cycle list={ % Список стилей для линий \addplot (цвета взяты из шаблона)
            {blue, mark=*, thick},
            {alert, mark=square, thick},
            {w3schools, mark=triangle, thick},
            {mLightBrown, mark=diamond, thick},
            {customcolor, mark=oplus, thick},
            {black!50, mark=pentagon, thick},
        },
        % Убираем рамку вокруг графика по умолчанию
        axis background/.style={fill=none},
    },
    % Дополнительный стиль для ROC-кривой (без сетки, с диагональю)
    roc curve style/.style={
        cheatsheet plot style, % Наследуем базовый стиль
        width=6cm, height=6cm, % Делаем квадратным
        grid=none, % Убираем сетку
        xlabel={False Positive Rate (FPR)},
        ylabel={True Positive Rate (TPR)},
        xmin=0, xmax=1,
        ymin=0, ymax=1,
        xtick={0, 0.5, 1},
        ytick={0, 0.5, 1},
        legend pos=south east, % Легенда внизу справа
        % Добавляем диагональную линию случайного угадывания
        extra x ticks={0.5}, extra y ticks={0.5},
        extra tick style={grid=major, grid style={dashed, color=black!40}},
        after end axis/.code={ % Добавляем линию после отрисовки осей
            \draw[dashed, color=black!40] (axis cs:0,0) -- (axis cs:1,1);
        }
    },
    % Дополнительный стиль для PR-кривой
    pr curve style/.style={
        cheatsheet plot style, % Наследуем базовый стиль
        xlabel={Recall (TPR)},
        ylabel={Precision},
        xmin=0, xmax=1,
        ymin=0, ymax=1,
        legend pos=south west, % Легенда внизу слева
    }
}

% % Добавлено для возможности компиляции этого фрагмента отдельно для проверки
% \begin{document}
% \begin{multicols}{2} % Пример использования в multicols
% \lipsum[1] % Просто текст для заполнения

% \begin{textbox}{Пример графика (Базовый стиль)}
%     \begin{center}
%     \begin{tikzpicture}
%         \begin{axis}[cheatsheet plot style, title={Пример $y=x^2$ и $y=x+1$}]
%         \addplot coordinates {(0, 0) (1, 1) (2, 4) (3, 9) (4, 16)};
%         \addlegendentry{$y=x^2$}
%         \addplot coordinates {(0, 1) (1, 2) (2, 3) (3, 4) (4, 5)};
%         \addlegendentry{$y=x+1$}
%         \end{axis}
%     \end{tikzpicture}
%     \end{center}
% \end{textbox}

% \begin{myexampleblock}{Пример ROC-кривой}
%     \begin{center}
%     \begin{tikzpicture}
%         \begin{axis}[roc curve style, title={Пример ROC Curve}]
%         % Пример данных ROC кривой
%         \addplot coordinates { (0,0) (0.1,0.3) (0.2,0.6) (0.4,0.8) (0.6,0.9) (0.8,0.95) (1,1) };
%         \addlegendentry{Модель A (AUC $\approx$ 0.8)}
%         \addplot coordinates { (0,0) (0.2,0.2) (0.4,0.4) (0.6,0.6) (0.8,0.8) (1,1) }; % Пример хуже
%         \addlegendentry{Модель B (AUC = 0.5)}
%         \end{axis}
%     \end{tikzpicture}
%     \end{center}
% \end{myexampleblock}

% \begin{myblock}{Пример PR-кривой}
%     \begin{center}
%     \begin{tikzpicture}
%         \begin{axis}[pr curve style, title={Пример Precision-Recall Curve}]
%         % Пример данных PR кривой (могут сильно зависеть от порога)
%         \addplot coordinates { (0,0.9) (0.1,0.85) (0.3,0.8) (0.5,0.7) (0.7,0.6) (0.9,0.4) (1,0.3) };
%         \addlegendentry{Модель C}
%         \end{axis}
%     \end{tikzpicture}
%     \end{center}
% \end{myblock}

% \lipsum[2-3] % Просто текст для заполнения
% \end{multicols}
% \end{document}

\begin{document}
\pagestyle{empty} % Убираем номера страниц
\small % Уменьшаем базовый размер шрифта для всего документа

% >>> КОММЕНТАРИЙ: Используем multicols для создания трех колонок.
% \raggedcolumns гарантирует, что колонки не будут искусственно растягиваться по вертикали,
% что лучше для шпаргалок, но может приводить к разной высоте колонок.
% Идеальная автоматическая балансировка сложных блоков (tcolorbox) в multicol затруднена.
% Для лучшего результата может потребоваться ручная доводка:
%  - Разбиение больших блоков контента.
%  - Использование \needspace{<длина>} перед блоками или секциями.
%  - В крайних случаях - \columnbreak для принудительного разрыва колонки.
\begin{multicols}{3}
\raggedcolumns

\noindent
\begin{minipage}{\linewidth}
    \centering
    % >>> КОММЕНТАРИЙ: Используем стандартный \bfseries для жирности заголовка.
    % Убрано избыточное \setmainfont{PT Sans Bold}, которое вызывало предупреждение в логе.
    % Основной шрифт (PT Sans) уже задан в шаблоне и должен содержать жирное начертание.
    {\bfseries\huge \cheatsheettitle \par}
    \vspace{1ex}
    {\large \cheatsheetauthor \par}
    \vspace{0.5ex}
    {\normalsize \today \par} % Вставляем текущую дату
\end{minipage}
\vspace{2ex}

\thispagestyle{empty} % Убеждаемся, что и на первой странице нет номера

\scriptsize % Уменьшаем шрифт для оглавления и основного текста
\tableofcontents % Генерируем оглавление

% >>> КОММЕНТАРИЙ: Подключаем основной контент шпаргалки
% >>> КОММЕНТАРИЙ: Начало контента шпаргалки по Pandas для EDA.
% >>> Структура следует этапам анализа данных.
% >>> Шпаргалка рассчитана на пользователя, знакомого с основами, для быстрого вспоминания команд.
% >>> Для контроля разрывов между колонками используйте \needspace{<высота>} перед \section или блоком.

\section{Основные Структуры / Core Structures}

% >>> КОММЕНТАРИЙ: Краткое описание двух фундаментальных структур данных Pandas.
\begin{textbox}{Series (1D) и DataFrame (2D)}
\textbf{Series}: Одномерный индексированный массив. Аналог столбца таблицы или словаря Python. \sep % \sep - маленький разделитель
\textbf{DataFrame}: Двумерная табличная структура с индексированными строками и столбцами. Основной объект для анализа данных. Состоит из объектов Series.

% >>> КОММЕНТАРИЙ: Базовый пример создания Series и DataFrame.
\begin{codebox}{python}{Создание Series и DataFrame}
import pandas as pd
import numpy as np

# Series (индекс по умолчанию)
s = pd.Series([10, 20, np.nan, 40])

# DataFrame из словаря Python
data = {'col_A': [1, 2, 3], 'col_B': ['X', 'Y', 'Z']}
dates_index = pd.date_range('20240101', periods=3)
df = pd.DataFrame(data, index=dates_index)
# print(s)
# print(df)
\end{codebox}
\end{textbox}

\section{Загрузка и Сохранение / IO}

% >>> КОММЕНТАРИЙ: Основные функции для чтения/записи данных из/в различные форматы.
\begin{myblock}{Чтение и запись данных}
Ключевые функции для взаимодействия с файлами.
% >>> КОММЕНТАРИЙ: \mycommand отображает команду и ее краткое описание.
\mycommand{pd.read_csv('f.csv', sep=',')}{Чтение CSV файла. Важные параметры: `sep`, `header`, `index\_col`, `usecols`, `parse\_dates`, `dtype`.}
\mycommand{df.to_csv('out.csv', index=False)}{Запись DataFrame в CSV. Важные параметры: `sep`, `index`, `header`, `columns`.}
\mycommand{pd.read_excel('f.xlsx', sheet_name=0)}{Чтение из файла Excel. Важные параметры: `sheet\_name`, `header`, `index\_col`.}
\mycommand{df.to_excel('out.xlsx', sheet_name='Data')}{Запись в Excel. Важные параметры: `sheet\_name`, `index`, `header`.}
\mycommand{pd.read_sql(query, conn)}{Чтение из SQL базы данных (требует `sqlalchemy` или аналог).}
\mycommand{pd.read_json('f.json', orient='records')}{Чтение из JSON файла/строки. `orient` важен для структуры.}
\end{myblock}

\section{Первичный Осмотр / Basic Inspection}

% >>> КОММЕНТАРИЙ: Функции для первого взгляда на данные: размер, типы, статистика, начало/конец.
\begin{textbox}{Первичный анализ DataFrame (`df`)}
Получение общего представления о данных.
\begin{codebox}{python}{Команды для осмотра `df`}
df.head(3)      # Первые N строк (по умолч. 5)
df.tail(3)      # Последние N строк (по умолч. 5)
df.shape        # Размеры (строки, столбцы) - кортеж
df.info()       # Сводка: индекс, столбцы, non-null count, типы, память
# df.info(memory_usage='deep') # Более точная оценка памяти
df.describe()   # Статистика для числовых столбцов (count, mean, std, min, max, ...)
# df.describe(include='object') # Статистика для object/string (count, unique, top, freq)
# df.describe(include='all') # Статистика для всех типов
df.columns      # Список названий столбцов
df.index        # Индекс строк
df.dtypes       # Типы данных в каждом столбце
s.value_counts() # Подсчет уникальных значений в Series (столбце)
# s.value_counts(dropna=False) # Включая NaN
df['col_A'].nunique() # Количество уникальных значений в столбце
\end{codebox}
\end{textbox}

\section{Очистка и Предобработка / Cleaning \& Preprocessing}

% >>> КОММЕНТАРИЙ: Блок посвященный подготовке данных: пропуски, дубликаты, типы, переименования.
\begin{textbox}{Подготовка данных к анализу}
Обработка аномалий и приведение данных к нужному формату.

\begin{codebox}{python}{Работа с пропусками (NaN)}
df.isnull()      # Boolean DataFrame: True где NaN
df.isnull().sum() # Количество NaN в каждом столбце
df.notnull()     # Boolean DataFrame: True где НЕ NaN
df.dropna()      # Удалить строки с любым NaN
# df.dropna(axis=1) # Удалить столбцы с любым NaN
# df.dropna(subset=['col_A']) # Удалить строки с NaN только в 'col_A'
df.fillna(0)     # Заменить все NaN на 0
# df['col_A'].fillna(df['col_A'].mean()) # Замена средним по столбцу
\end{codebox}

\begin{codebox}{python}{Работа с дубликатами}
df.duplicated() # Boolean Series: True для дублирующихся строк (кроме первого вхождения)
# df.duplicated(subset=['col_A', 'col_B']) # Проверка дубликатов по подмножеству столбцов
df.drop_duplicates() # Удалить дублирующиеся строки
# df.drop_duplicates(keep='last') # Оставить последнее вхождение
\end{codebox}

\begin{codebox}{python}{Изменение типов и переименование}
# Изменение типа столбца
# df['col_A'].astype(float)
# df['date_col'] = pd.to_datetime(df['date_col'], format='%Y-%m-%d')
# df['category_col'].astype('category') # Экономия памяти для категориальных

# Переименование столбцов/индекса
# df.rename(columns={'old_name': 'new_name'}, index={'old_idx': 'new_idx'})
\end{codebox}
\end{textbox}

\section{Выборка и Фильтрация / Selection \& Filtering}

% >>> КОММЕНТАРИЙ: Основные методы доступа к данным: по меткам, позициям, условиям.
\begin{textbox}{Доступ к данным}
\red{Индексация:} \texttt{[]} (столбцы), \texttt{.loc[]} (метки), \texttt{.iloc[]} (позиции).

\begin{codebox}{python}{Выбор столбцов и подмножеств}
df['col_A']     # Выбор одного столбца (Series)
df.col_A        # Альтернатива (если имя валидно)
df[['col_A', 'col_B']] # Выбор нескольких столбцов (DataFrame)
\end{codebox}

\begin{codebox}{python}{Выбор строк/значений по МЕТКАМ (.loc)}
# df.loc[label]         # Строка по метке индекса
# df.loc[start:end]     # Срез строк по меткам (включая end)
# df.loc[:, 'col_A']    # Все строки, столбец 'col_A'
# df.loc[label, 'col_A']# Конкретное значение
\end{codebox}

\begin{codebox}{python}{Выбор строк/значений по ПОЗИЦИЯМ (.iloc)}
# df.iloc[0]          # Первая строка (позиция 0)
# df.iloc[0:2]        # Первые две строки (срез до 2, НЕ включая 2)
# df.iloc[:, 0]       # Все строки, первый столбец (позиция 0)
# df.iloc[0, 0]       # Значение в [0, 0]
# df.iloc[[0, 2], [0, 1]] # Выбор по спискам позиций строк/столбцов
\end{codebox}
\end{textbox}

\begin{myexampleblock}{Фильтрация по условию (Boolean Indexing)}
Мощный способ выбора строк на основе логических условий.
\begin{codebox}{python}{Примеры Boolean Indexing}
df[df['col_A'] > 10] # Строки, где значение в col_A > 10
# Условия: & (И), | (ИЛИ), ~ (НЕ). Скобки обязательны!
df[(df['col_A'] > 10) & (df['col_B'] == 'X')]
df[df['col_B'].isin(['X', 'Y'])] # Строки, где col_B равно 'X' или 'Y'
# df[df['col_A'].between(10, 20)] # Значения между 10 и 20 включительно

# Использование .loc с булевым массивом (предпочтительнее для явности)
# df.loc[df['col_A'] > 10, ['col_B', 'col_C']] # Фильтр + выбор столбцов
\end{codebox}
\end{myexampleblock}

\section{Сортировка / Sorting}

% >>> КОММЕНТАРИЙ: Сортировка данных по значениям или по индексу.
\begin{textbox}{Сортировка данных}
Упорядочивание строк DataFrame.
\green{По значениям} (`sort\_values`) и \green{по индексу} (`sort\_index`).

\begin{codebox}{python}{Примеры сортировки}
# Сортировка по значениям одного или нескольких столбцов
df.sort_values(by='col_A') # По возрастанию
df.sort_values(by='col_A', ascending=False) # По убыванию
# df.sort_values(by=['col_A', 'col_B']) # По нескольким столбцам

# Сортировка по индексу
df.sort_index(ascending=False)

# Важно: сортировка возвращает НОВЫЙ DataFrame.
# Для изменения на месте: inplace=True
\end{codebox}
\end{textbox}

\section{Трансформация и Применение Функций / Transformation \& Applying Functions}

% >>> КОММЕНТАРИЙ: Изменение данных: применение функций, работа со строками/датами, создание новых столбцов.
\begin{textbox}{Изменение и создание данных}
Применение функций, работа со специальными типами, создание новых признаков.

\begin{codebox}{python}{Создание новых столбцов}
# На основе существующих
# df['new_col'] = df['col_A'] * 2
# df['col_C'] = df['col_A'] + df['col_B'] # Поэлементные операции

# С использованием np.where (аналог IF в SQL/Excel)
# df['flag'] = np.where(df['col_A'] > 10, 'High', 'Low')
\end{codebox}

\begin{codebox}{python}{Применение функций (apply, map, applymap)}
# apply: применяет функцию к строкам (axis=1) или столбцам (axis=0)
# df.apply(np.sum, axis=0) # Сумма по каждому столбцу (Series)
# df.apply(lambda row: row['col_A'] * row.name.day, axis=1) # Пример сложной функции по строкам

# map: работает поэлементно на Series (для замены или преобразования)
# s.map({'X': 1, 'Y': 2}) # Замена значений по словарю
# s.map('Value: {}'.format) # Применение строковой функции

# applymap: работает поэлементно на DataFrame (применяет функцию к каждому элементу)
# df_numeric.applymap(lambda x: x**2) # Квадрат каждого элемента
# ВАЖНО: Старайтесь использовать векторизованные операции Pandas/NumPy вместо apply/applymap, если это возможно (они быстрее!).
\end{codebox}

\begin{codebox}{python}{Работа со строками (.str)}
# Доступ к строковым методам через аксессор .str для Series
# s_text = pd.Series(['apple', ' Banana ', 'kiwi '])
# s_text.str.lower()      # -> ['apple', ' banana ', 'kiwi ']
# s_text.str.strip()      # -> ['apple', 'Banana', 'kiwi']
# s_text.str.contains('a')# -> [True, True, False]
# s_text.str.split('a')   # -> [['', 'pple'], [' B', 'n', 'n', ' '], ['kiwi ']]
# s_text.str.replace('a', 'X') # -> ['Xpple', ' BXnana ', 'kiwi ']
# s_text.str.len()        # Длина каждой строки
\end{codebox}

\begin{codebox}{python}{Работа с датами (.dt)}
# Доступ к компонентам даты/времени через аксессор .dt для Series (типа datetime)
# s_dates = pd.to_datetime(pd.Series(['2024-01-05', '2024-02-10']))
# s_dates.dt.year         # -> [2024, 2024]
# s_dates.dt.month_name() # -> ['January', 'February']
# s_dates.dt.dayofweek    # -> [4, 5] (Понедельник=0, Воскресенье=6)
# s_dates.dt.date         # Только дата (без времени)
# (s_dates - pd.Timedelta(days=1)) # Вычитание временного интервала
\end{codebox}
\end{textbox}

\section{Группировка и Агрегация / Grouping \& Aggregation}

% >>> КОММЕНТАРИЙ: Мощный инструмент анализа: разделение данных на группы, применение функций и объединение результатов.
\begin{myblock}{Split-Apply-Combine (Разделяй-Применяй-Объединяй)}
Основа `groupby`: 1. \textbf{Split}: Данные делятся на группы по ключу. 2. \textbf{Apply}: Функция применяется к каждой группе. 3. \textbf{Combine}: Результаты объединяются.
\end{myblock}

\begin{textbox}{GroupBy: Группировка и агрегация}
Расчет сводных статистик по группам.
\begin{codebox}{python}{Примеры GroupBy}
# Группировка по одному или нескольким столбцам
grouped = df.groupby('key_col')
# grouped = df.groupby(['key1', 'key2']) # Группировка по нескольким ключам

# Применение агрегирующих функций
# grouped.mean() # Среднее для всех числовых столбцов в каждой группе
# grouped['data_col'].sum() # Сумма для конкретного столбца в каждой группе
# grouped.size() # Размер каждой группы (включая NaN в ключах)
# grouped.count() # Количество НЕ-NaN значений в каждой группе/столбце

# Несколько агрегаций сразу с .agg()
# grouped['data_col'].agg(['sum', 'mean', 'std'])
# grouped.agg({
#    'data_col1': ['mean', 'min', 'max'], # Несколько функций к одному столбцу
#    'data_col2': 'nunique',             # Одна функция к другому
#    'data_col3': lambda x: x.max() - x.min() # Пользовательская lambda-функция
# })

# Применение функции к группам без агрегации (.transform)
# Часто используется для заполнения пропусков средним по группе или для нормализации
# df['group_mean'] = df.groupby('key_col')['data_col'].transform('mean')
# df['normalized'] = df['data_col'] / df.groupby('key_col')['data_col'].transform('sum')

# Фильтрация групп (.filter)
# Оставить только те группы, где среднее по data_col > 10
# filtered_groups = df.groupby('key_col').filter(lambda g: g['data_col'].mean() > 10)
\end{codebox}
\end{textbox}

% >>> КОММЕНТАРИЙ: Сводные таблицы - удобный способ переформатирования и агрегации данных.
\begin{alerttextbox}{Сводные таблицы / Pivot Tables}
Аналог сводных таблиц в Excel для агрегации и изменения формы данных.
\begin{codebox}{python}{Пример pivot\_table}
# pd.pivot_table(df,
#                values='Value_Column', # Столбец для агрегации
#                index='Row_Index_Column', # Столбец(ы) для индекса строк
#                columns='Column_Index_Column', # Столбец(ы) для названий столбцов
#                aggfunc=np.sum, # Функция агрегации (sum, mean, count, ...)
#                fill_value=0) # Значение для заполнения NaN после агрегации
#
# pd.crosstab(df['col_A'], df['col_B']) # Таблица сопряженности (частот)
\end{codebox}
\alert{Часто результатом groupby или pivot\_table является DataFrame с MultiIndex!}
\end{alerttextbox}

\section{Объединение / Merging \& Joining}

% >>> КОММЕНТАРИЙ: Соединение нескольких DataFrame вместе.
\begin{textbox}{Объединение DataFrames}
Комбинирование данных из разных источников. \red{merge()} (SQL JOIN), \red{join()} (по индексам), \red{concat()} (склеивание).

\begin{codebox}{python}{Примеры объединения}
# df1, df2 - примеры DataFrame
# merge: аналог SQL JOIN (по столбцам)
# pd.merge(df1, df2, on='key_col', how='inner') # Inner join по ключу
# pd.merge(df1, df2, left_on='key1', right_on='key2', how='left') # Left join по разным ключам
# how: 'inner' (по умолч.), 'outer', 'left', 'right'

# concat: склеивание таблиц по оси (строк или столбцов)
# pd.concat([df1, df2], axis=0) # Склеить строки (ось 0)
# pd.concat([df1, df2], axis=1) # Склеить столбцы (ось 1) - важно совпадение индексов

# join: объединение по индексам (или индекс с ключом)
# df1.join(df2.set_index('key_col'), on='key_col', how='inner')
\end{codebox}
\end{textbox}

\section{Визуализация / Visualization}

% >>> КОММЕНТАРИЙ: Краткий пример быстрой визуализации для EDA.
\begin{textbox}{Быстрая визуализация}
Простые графики для первичного анализа с помощью `.plot()`. Использует Matplotlib под капотом.
\begin{codebox}{python}{Пример простого графика}
# Требуется matplotlib: pip install matplotlib
import matplotlib.pyplot as plt

# df['numeric_col'].hist(bins=30) # Гистограмма
# df.plot(kind='scatter', x='col_A', y='col_B') # Диаграмма рассеяния
# df.groupby('category_col')['value_col'].mean().plot(kind='bar') # Столбчатая диаграмма средних по группам
# plt.show() # Отобразить график (часто не нужно в Jupyter)
# plt.savefig('img/my_plot.png') # Сохранить график
# plt.close()
\end{codebox}
% >>> КОММЕНТАРИЙ: Команда mygraphics для вставки сохраненного графика, если нужно.
% \mygraphics{img/my_plot.png}
\end{textbox}

\section{Полезные ссылки и советы / Links \& Tips}
\begin{myblock}{Полезные ресурсы}
% >>> КОММЕНТАРИЙ: Команда \resourcelink для форматирования ссылок.
\resourcelink{https://pandas.pydata.org/docs/}{Pandas Documentation}
 {Официальная документация \footcite{pandas_documentation}}
\resourcelink{https://stackoverflow.com/questions/tagged/pandas}{Stack Overflow [pandas]}
 {Вопросы и ответы сообщества}
\resourcelink{https://github.com/pandas-dev/pandas}{Pandas GitHub}
 {Исходный код библиотеки}
\end{myblock}

\begin{textbox}{Цитата / Quote}
\begin{quote}
"There should be one-- and preferably only one --obvious way to do it." \\ % "Должен быть один -- и, желательно, только один -- очевидный способ сделать это."
\emph{-- The Zen of Python (import this)}
\end{quote}
\end{textbox}

% >>> КОММЕНТАРИЙ: Важные советы по производительности и особенностям Pandas.
\begin{alerttextbox}{Советы по производительности и Best Practices}
\begin{itemize}[leftmargin=*]
    \item \textbf{Векторизация > Итерация:} Используйте встроенные функции Pandas/NumPy вместо циклов `for` или `.iterrows()`. Векторизованные операции работают на порядки быстрее.
    \item \textbf{Тип `category`:} Для столбцов с небольшим количеством уникальных строковых значений используйте `df['col'].astype('category')`. Это значительно экономит память и ускоряет операции (особенно `groupby`).
    \item \textbf{`.loc` vs `[]` для присваивания:} При изменении данных используйте `.loc` для избежания `SettingWithCopyWarning`. Например: `df.loc[mask, 'col'] = value`.
    \item \textbf{Работа с MultiIndex:} `groupby` и `pivot\_table` часто возвращают MultiIndex. Изучите методы работы с ним (`.reset\_index()`, `.unstack()`, `.stack()`, выборка через кортежи).
    \item \textbf{Оценка памяти:} Используйте `df.info(memory\_usage='deep')` для более точной оценки занимаемой памяти DataFrame, особенно при наличии строковых данных.
\end{itemize}
\end{alerttextbox}

% >>> КОММЕНТАРИЙ: Печатаем библиографию в конце документа
\AtNextBibliography{\footnotesize} % Уменьшаем шрифт для библиографии
\printbibliography

\end{multicols}

\end{document}