\documentclass[10pt,a4paper]{article}

\usepackage[landscape,margin=0.7cm]{geometry}
\usepackage[russian,english]{babel} % Поддержка русского и английского

% >>> КОММЕНТАРИЙ: Определяем переменные для заголовка и автора (можно изменить)
\newcommand{\cheatsheettitle}{\color{w3schools}Шпаргалка по Градиентному бустингу / {\color{alert} Концепции} {\color{black} Cheatsheet (XeLaTeX)}}
\newcommand{\cheatsheetauthor}{Краткий справочник}

% >>> КОММЕНТАРИЙ: Подключаем основной файл шаблона с настройками стилей и пакетов
% \documentclass{article} % ЗАКОММЕНТИРОВАНО: Добавлено для возможности компиляции этого фрагмента отдельно для проверки
\usepackage{fontspec}

% >>> КОММЕНТАРИЙ: Установка основных шрифтов документа (требует установленных шрифтов в системе)
\setmainfont{PT Sans} % Основной шрифт для текста
\setsansfont{PT Sans} % Шрифт без засечек (если используется отдельно)
\setmonofont{Liberation Mono} % Моноширинный шрифт (для кода)

\usepackage{fontawesome} % Для иконок (например, \faQuoteLeft)
\usepackage{hyperref} % Для создания кликабельных ссылок (в т.ч. в \resourcelink)
\usepackage{enumitem} % Для настройки списков
\usepackage{lipsum} % Для генерации текста-рыбы (не используется в финальной версии)
\usepackage{xcolor} % Для определения и использования цветов
\usepackage{minted} % Для подсветки синтаксиса кода (требует Python и Pygments, компиляция с -shell-escape)

% >>>>>> НОВОЕ: Пакет для создания графиков <<<<<<
\usepackage{pgfplots}
\pgfplotsset{compat=newest} % Устанавливаем последнюю версию совместимости
\usetikzlibrary{arrows.meta, shapes.geometric} % Загружаем полезные библиотеки TikZ

\usepackage{titlesec} % Для настройки заголовков секций
% >>> КОММЕНТАРИЙ: Настройка отступов для заголовков секций (\section)
\titlespacing*{\section}{0pt}{*1.5}{*0.8} % {отступ слева}{отступ сверху}{отступ снизу}

% >>> КОММЕНТАРИЙ: Форматирование заголовков секций (\section)
\titleformat{\section}
  {\normalfont\large\bfseries\color{mDarkTeal}} % Стиль заголовка (цвет, размер, жирность)
  {\thesection} % Номер секции
  {1em} % Отступ после номера
  {} % Код перед заголовком (здесь пусто)

% >>> КОММЕНТАРИЙ: Определение пользовательских цветов через HTML-коды
\definecolor{customcolor}{HTML}{696EA8} % Основной цвет для блоков textbox
\definecolor{alert}{HTML}{CD5C5C} % Цвет для выделения (красный)
\definecolor{w3schools}{HTML}{4CAF50} % Цвет для выделения (зеленый)
\definecolor{subbox}{gray}{0.60} % Цвет для под-блоков (не используется?)
\definecolor{codecolor}{HTML}{FFC300} % Цвет для кода (не используется?)
\definecolor{mDarkTeal}{HTML}{23373b} % Темно-бирюзовый (для заголовков секций, названий блоков)
\colorlet{beamerboxbg}{black!2} % Фон для beamerbox (не используется?)
\colorlet{beamerboxfg}{mDarkTeal} % Цвет текста для beamerbox (не используется?)
\definecolor{mDarkBrown}{HTML}{604c38} % Темно-коричневый (не используется?)
\definecolor{mLightBrown}{HTML}{EB811B} % Светло-коричневый (для заголовков myalertblock)
\definecolor{mLightGreen}{HTML}{14B03D} % Светло-зеленый (для заголовков myexampleblock)
\definecolor{alertTextBox}{HTML}{A8696E} % Цвет для блоков alerttextbox

% >>> КОММЕНТАРИЙ: Настройка пакета biblatex для библиографии
\usepackage
[citestyle=authoryear, % Стиль цитирования (Автор-Год)
sorting=nty, % Сортировка по Имени, Году, Названию
autocite=footnote, % Команда \autocite создает сноски
autolang=hyphen, % Автоматическое определение языка для переносов
mincrossrefs=1, % Минимальное количество перекрестных ссылок
backend=biber] % Используемый бэкенд (требует запуск biber)
{biblatex}

% >>> КОММЕНТАРИЙ: Настройка формата пост-заметок в цитатах (например, номера страниц)
\DeclareFieldFormat{postnote}{#1}
\DeclareFieldFormat{multipostnote}{#1}
\DeclareAutoCiteCommand{footnote}[f]{\footcite}{\footcites} % Настройка команды \autocite

% >>> КОММЕНТАРИЙ: Подключение файла с библиографическими записями
\addbibresource{literature.bib}

% >>> КОММЕНТАРИЙ: Подключение библиотеки tcolorbox для создания цветных блоков
\usepackage{tcolorbox}
\tcbuselibrary{most, listingsutf8, minted} % Загрузка необходимых библиотек tcolorbox (most включает breakable)

% >>> КОММЕНТАРИЙ: Глобальные настройки для маленьких tcbox (inline блоки)
\tcbset{tcbox width=auto,left=1mm,top=1mm,bottom=1mm,
right=1mm,boxsep=1mm,middle=1pt}

% >>> КОММЕНТАРИЙ: Определение базового окружения для цветных блоков с заголовком
% breakable: позволяет блоку разрываться между колонками/страницами.
% ВАЖНО: В multicols автоматический разрыв breakable может работать не всегда идеально.
% Если блок разрывается некрасиво, попробуйте разбить контент на несколько блоков
% или используйте \needspace{<длина>} перед блоком.
\newenvironment{mycolorbox}[2]{
\begin{tcolorbox}[capture=minipage,fonttitle=\large\bfseries, enhanced,boxsep=1mm,colback=#1!30!white,
width=\linewidth, % Ширина блока равна ширине текущей колонки
arc=2pt,outer arc=2pt, toptitle=0mm,colframe=#1,opacityback=0.7,nobeforeafter,
breakable, % Разрешить разрыв блока
title=#2] % Текст заголовка блока
}{\end{tcolorbox}}

% >>> КОММЕНТАРИЙ: Окружение subbox (не используется в текущем коде)
\newenvironment{subbox}[2]{
\begin{tcolorbox}[capture=minipage,fonttitle=\normalsize\bfseries, enhanced,boxsep=1mm,colback=#1!30!white,on line,tcbox width=auto,left=0.3em,top=1mm, toptitle=0mm,colframe=#1,opacityback=0.7,nobeforeafter,
breakable,
title=#2]\footnotesize
}{\normalsize\end{tcolorbox}\vspace{0.1em}}

% >>> КОММЕНТАРИЙ: Окружение для размещения tcolorbox'ов в несколько колонок внутри блока (не используется)
\newenvironment{multibox}[1]{
\begin{tcbraster}[raster columns=#1,raster equal height,nobeforeafter,raster column skip=1em,raster left skip=1em,raster right skip=1em]}{\end{tcbraster}}

% >>> КОММЕНТАРИЙ: Окружения-обертки для mycolorbox с предопределенными цветами и заголовками
\newenvironment{textbox}[1]{\begin{mycolorbox}{customcolor}{#1}}{\end{mycolorbox}} % Обычный текстовый блок
\newenvironment{alerttextbox}[1]{\begin{mycolorbox}{alertTextBox}{#1}}{\end{mycolorbox}} % Блок для предупреждений/важной информации

% >>> КОММЕНТАРИЙ: Определение окружения для блоков с кодом с подсветкой minted
\newtcblisting{codebox}[3][]{ % [опции minted], {язык}, {заголовок}
colback=black!10,colframe=black!20, % Цвета фона и рамки
listing only, % Показывать только листинг кода
minted options={ % Опции, передаваемые в minted
    numbers=left, % Нумерация строк слева
    style=default, % Стиль подсветки (можно выбрать другие, напр., 'friendly')
    fontsize=\scriptsize, % Размер шрифта кода
    breaklines, % Автоматический перенос длинных строк
    breaksymbolleft=, % Символ для обозначения переноса (пусто)
    autogobble, % Удаление лишних отступов слева
    linenos, % Показывать номера строк
    numbersep=0.7em, % Отступ номеров строк от кода
    #1 % Дополнительные опции minted, переданные в [1]
    },
enhanced,top=1mm, toptitle=0mm,
left=5mm, % Отступ слева внутри блока (для номеров строк)
arc=0pt,outer arc=0pt, % Прямые углы
title={#3}, % Заголовок блока кода
fonttitle=\small\bfseries\color{mDarkTeal}, % Стиль заголовка блока кода
listing engine=minted, % Движок для листинга - minted
minted language=#2, % Язык программирования для подсветки
breakable % Разрешить разрыв блока кода
}

\newcommand{\punkti}{~\lbrack\dots\rbrack~} % Команда для [...] (не используется)

% >>> КОММЕНТАРИЙ: Переопределение окружения quote для добавления иконок цитат
\renewenvironment{quote}
               {\list{\faQuoteLeft\phantom{ }}{\rightmargin\leftmargin}
                \item\relax\scriptsize\ignorespaces}
               {\unskip\unskip\phantom{xx}\faQuoteRight\endlist}

% >>> КОММЕНТАРИЙ: Вспомогательные команды для создания цветных фонов под текстом (не используются)
\newcommand{\bgupper}[3]{\colorbox{#1}{\color{#2}\huge\bfseries\MakeUppercase{#3}}}
\newcommand{\bg}[3]{\colorbox{#1}{\bfseries\color{#2}#3}}

% >>> КОММЕНТАРИЙ: Команда для форматирования описания команды/функции
% #1: Сама команда (используется \detokenize для корректного отображения спецсимволов)
% #2: Описание команды
\newcommand{\mycommand}[2]{
  {\par\noindent\ttfamily\detokenize{#1}\par} % Вывод команды моноширинным шрифтом
  \nopagebreak % Стараемся не разрывать страницу после команды
  \hangindent=1.5em \hangafter=1 \noindent % Делаем отступ для описания
  \small\textit{#2}\par\vspace{0.5ex} % Вывод описания и небольшой отступ после
}

% >>> КОММЕНТАРИЙ: Команда для форматирования ссылок на ресурсы
% #1: URL ссылки
% #2: Текст ссылки
% #3: Описание ресурса
\newcommand{\resourcelink}[3]{
  {\par\noindent\href{#1}{\ttfamily #2}\par} % Вывод кликабельной ссылки
  \nopagebreak
  \hangindent=1.5em \hangafter=1 \noindent
  \small\textit{#3}\par\vspace{0.5ex} % Вывод описания
}

% >>> КОММЕНТАРИЙ: Вспомогательные команды для стилизации
\newcommand{\sep}{{\scriptsize~\faCircle{ }~}} % Разделитель-кружок
\newcommand{\bggreen}[1]{\medskip\bgupper{w3schools}{black}{#1}\\[0.5em]} % Зеленый заголовок (не используется)
\newcommand{\green}[1]{\smallskip\bg{w3schools}{white}{#1}\\} % Текст на зеленом фоне
\newcommand{\red}[1]{\smallskip\bg{alert}{white}{#1}\\} % Текст на красном фоне
\newcommand{\alertcmd}[1]{\red{#1}\\} % Псевдоним для red (переименовано чтобы не конфликтовать с цветом alert)

\usepackage{multicol} % Пакет для создания нескольких колонок
\setlength{\columnsep}{15pt} % Расстояние между колонками

\setlength{\parindent}{0pt} % Убираем абзацный отступ по умолчанию
\usepackage{csquotes} % Для правильного отображения кавычек в разных языках (используется biblatex)
\newcommand{\loremipsum}{Lorem ipsum dolor sit amet.} % Пример текста (не используется)

% >>> КОММЕНТАРИЙ: Окружения для блоков с разными цветами заголовков (пример, предупреждение, обычный)
% Основаны на tcolorbox, похожи на mycolorbox, но с другими цветами заголовков и рамки.
% Также используют breakable. Применяются для выделения примеров, предупреждений и т.д.
% ВАЖНО: Лог может содержать предупреждения "Underfull \hbox (badness 10000)".
% Это часто случается в узких колонках multicol при наличии кода, URL или текста,
% который плохо переносится. Обычно это лишь косметическая проблема, строки выглядят
% немного не до конца заполненными. Можно игнорировать или попробовать перефразировать/
% использовать \sloppy перед проблемным абзацем.
\newenvironment{myexampleblock}[1]{ % Блок для примеров (зеленый заголовок)
    \tcolorbox[capture=minipage,fonttitle=\small\bfseries\color{mLightGreen}, enhanced,boxsep=1mm,colback=black!10,breakable,noparskip,
    on line,tcbox width=auto,left=0.3em,top=1mm, toptitle=0mm,
    colframe=black!20,arc=0pt,outer arc=0pt,
    opacityback=0.7,nobeforeafter,title=#1]}
    {\endtcolorbox}

\newenvironment{myalertblock}[1]{ % Блок для предупреждений (оранжевый заголовок)
    \tcolorbox[capture=minipage,fonttitle=\small\bfseries\color{mLightBrown}, enhanced,boxsep=1mm,colback=black!10,breakable,noparskip,
    on line,tcbox width=auto,left=0.3em,top=1mm, toptitle=0mm,
    colframe=black!20,arc=0pt,outer arc=0pt,
    opacityback=0.7,nobeforeafter,title=#1]}
    {\endtcolorbox}

\newenvironment{myblock}[1]{ % Обычный информационный блок (бирюзовый заголовок)
    \tcolorbox[capture=minipage,fonttitle=\small\bfseries\color{mDarkTeal}, enhanced,boxsep=1mm,colback=black!10,breakable,noparskip,
    on line,tcbox width=auto,left=0.3em,top=1mm, toptitle=0mm,
    colframe=black!20, arc=0pt,outer arc=0pt,
    opacityback=0.7,nobeforeafter,title=#1]}
    {\endtcolorbox}

% >>> КОММЕНТАРИЙ: Команда для вставки изображения в рамке tcolorbox с подписью
% #1: Опции (не используются)
% #2: Путь к файлу изображения (используется как подпись под изображением)
\newcommand{\mygraphics}[2][]{
\tcbox[enhanced,boxsep=0pt,top=0pt,bottom=0pt,left=0pt,
right=0pt,boxrule=0.4pt,drop fuzzy shadow,clip upper,
colback=black!75!white,toptitle=2pt,bottomtitle=2pt,nobeforeafter,
center title,fonttitle=\small\sffamily,title=\detokenize{#2}] % Используем путь как заголовок
{\includegraphics[width=\the\dimexpr(\linewidth-4mm)\relax]{#2}} % Вставляем изображение, ширина чуть меньше колонки
}

% >>> КОМЕНТАРИЙ: Команда \needspace{<длина>}
% Используйте эту команду *перед* блоком (tcolorbox, section, figure), чтобы убедиться,
% что в текущей колонке есть как минимум <длина> свободного места. Если места нет,
% LaTeX начнет новую колонку/страницу перед выполнением команды.
% Пример: \needspace{10\baselineskip} % Запросить место для примерно 10 строк текста
% Это полезно для предотвращения "висячих" заголовков или некрасивых разрывов блоков.

% >>>>>> НОВОЕ: Стили для графиков PGFPlots <<<<<<
\pgfplotsset{
    % Базовый стиль для всех графиков в шпаргалке
    cheatsheet plot style/.style={
        width=0.98\linewidth, % Ширина чуть меньше колонки для аккуратности
        height=5cm,          % Высота графика (можно изменить по необходимости)
        grid=major,          % Включаем основную сетку
        grid style={dashed, color=black!20}, % Стиль сетки: пунктирная, светло-серая
        axis lines=left,     % Линии осей только слева и снизу
        axis line style={color=black!60, thick}, % Стиль линий осей: серые, потолще
        tick style={color=black!60, thick}, % Стиль меток на осях
        ticklabel style={font=\scriptsize, color=black}, % Стиль подписей меток: маленький шрифт
        label style={font=\small, color=mDarkTeal}, % Стиль подписей осей: шрифт small, цвет как у заголовков блоков
        title style={font=\small\bfseries, color=mDarkTeal}, % Стиль заголовка графика
        legend style={        % Стиль легенды
            font=\scriptsize, % Маленький шрифт
            draw=black!30,    % Легкая рамка вокруг легенды
            fill=white,       % Белый фон
            legend cell align=left, % Выравнивание текста в ячейке легенды
            anchor=north east, % Якорь для позиционирования
            at={(rel axis cs:0.98,0.98)} % Положение в правом верхнем углу внутри области графика
        },
        cycle list={ % Список стилей для линий \addplot (цвета взяты из шаблона)
            {blue, mark=*, thick},
            {alert, mark=square, thick},
            {w3schools, mark=triangle, thick},
            {mLightBrown, mark=diamond, thick},
            {customcolor, mark=oplus, thick},
            {black!50, mark=pentagon, thick},
        },
        % Убираем рамку вокруг графика по умолчанию
        axis background/.style={fill=none},
    },
    % Дополнительный стиль для ROC-кривой (без сетки, с диагональю)
    roc curve style/.style={
        cheatsheet plot style, % Наследуем базовый стиль
        width=6cm, height=6cm, % Делаем квадратным
        grid=none, % Убираем сетку
        xlabel={False Positive Rate (FPR)},
        ylabel={True Positive Rate (TPR)},
        xmin=0, xmax=1,
        ymin=0, ymax=1,
        xtick={0, 0.5, 1},
        ytick={0, 0.5, 1},
        legend pos=south east, % Легенда внизу справа
        % Добавляем диагональную линию случайного угадывания
        extra x ticks={0.5}, extra y ticks={0.5},
        extra tick style={grid=major, grid style={dashed, color=black!40}},
        after end axis/.code={ % Добавляем линию после отрисовки осей
            \draw[dashed, color=black!40] (axis cs:0,0) -- (axis cs:1,1);
        }
    },
    % Дополнительный стиль для PR-кривой
    pr curve style/.style={
        cheatsheet plot style, % Наследуем базовый стиль
        xlabel={Recall (TPR)},
        ylabel={Precision},
        xmin=0, xmax=1,
        ymin=0, ymax=1,
        legend pos=south west, % Легенда внизу слева
    }
}

% % Добавлено для возможности компиляции этого фрагмента отдельно для проверки
% \begin{document}
% \begin{multicols}{2} % Пример использования в multicols
% \lipsum[1] % Просто текст для заполнения

% \begin{textbox}{Пример графика (Базовый стиль)}
%     \begin{center}
%     \begin{tikzpicture}
%         \begin{axis}[cheatsheet plot style, title={Пример $y=x^2$ и $y=x+1$}]
%         \addplot coordinates {(0, 0) (1, 1) (2, 4) (3, 9) (4, 16)};
%         \addlegendentry{$y=x^2$}
%         \addplot coordinates {(0, 1) (1, 2) (2, 3) (3, 4) (4, 5)};
%         \addlegendentry{$y=x+1$}
%         \end{axis}
%     \end{tikzpicture}
%     \end{center}
% \end{textbox}

% \begin{myexampleblock}{Пример ROC-кривой}
%     \begin{center}
%     \begin{tikzpicture}
%         \begin{axis}[roc curve style, title={Пример ROC Curve}]
%         % Пример данных ROC кривой
%         \addplot coordinates { (0,0) (0.1,0.3) (0.2,0.6) (0.4,0.8) (0.6,0.9) (0.8,0.95) (1,1) };
%         \addlegendentry{Модель A (AUC $\approx$ 0.8)}
%         \addplot coordinates { (0,0) (0.2,0.2) (0.4,0.4) (0.6,0.6) (0.8,0.8) (1,1) }; % Пример хуже
%         \addlegendentry{Модель B (AUC = 0.5)}
%         \end{axis}
%     \end{tikzpicture}
%     \end{center}
% \end{myexampleblock}

% \begin{myblock}{Пример PR-кривой}
%     \begin{center}
%     \begin{tikzpicture}
%         \begin{axis}[pr curve style, title={Пример Precision-Recall Curve}]
%         % Пример данных PR кривой (могут сильно зависеть от порога)
%         \addplot coordinates { (0,0.9) (0.1,0.85) (0.3,0.8) (0.5,0.7) (0.7,0.6) (0.9,0.4) (1,0.3) };
%         \addlegendentry{Модель C}
%         \end{axis}
%     \end{tikzpicture}
%     \end{center}
% \end{myblock}

% \lipsum[2-3] % Просто текст для заполнения
% \end{multicols}
% \end{document}

\begin{document}
\pagestyle{empty} % Убираем номера страниц
\small % Уменьшаем базовый размер шрифта для всего документа

% >>> КОММЕНТАРИЙ: Используем multicols для создания трех колонок.
\begin{multicols}{3}
\raggedcolumns % Не растягивать колонки по вертикали

\noindent
\begin{minipage}{\linewidth}
    \centering
    {\bfseries\huge \cheatsheettitle \par}
    \vspace{1ex}
    {\large \cheatsheetauthor \par}
    \vspace{0.5ex}
    {\normalsize \today \par} % Вставляем текущую дату
\end{minipage}
\vspace{2ex}

\thispagestyle{empty} % Убеждаемся, что и на первой странице нет номера

\scriptsize % Уменьшаем шрифт для оглавления и основного текста
\tableofcontents % Генерируем оглавление

% >>> КОММЕНТАРИЙ: Подключаем основной контент шпаргалки
% >>> Контент для шпаргалки по Градиентному Бустингу (GBM) - v2 (Разбитые блоки)

% ================================================
% Введение
% ================================================
\begin{myblock}{Определение Градиентного Бустинга (GBM)}
    \textbf{Градиентный Бустинг (Gradient Boosting Machine, GBM)} — это мощный ансамблевый метод машинного обучения, который строит модели \textbf{последовательно}, где каждая новая модель исправляет ошибки предыдущей. Считается одним из наиболее эффективных алгоритмов для табличных данных.
\end{myblock}

\begin{textbox}{Аналогия: Лепка Скульптуры}
    Процесс бустинга можно сравнить с лепкой скульптуры:
    \begin{itemize}[nosep, leftmargin=*]
        \item Начинаем с грубой основы (первая простая модель).
        \item Замечаем недочеты (ошибки модели).
        \item Добавляем "кусочек глины" там, где нужно (обучаем новую модель на ошибках).
        \item Повторяем добавление "кусочков" (новых моделей), пока результат не станет удовлетворительным.
    \end{itemize}
    Каждое добавление "кусочка глины" — это новая слабая модель в ансамбле бустинга.
\end{textbox}

% ================================================
% Раздел 1: Идея Бустинга (Boosting)
% ================================================
\section{Идея Бустинга (Boosting)}

\begin{myblock}{Отличие от Бэггинга}
    В отличие от \textbf{бэггинга} (например, Random Forest), где модели обучаются независимо и параллельно на разных подвыборках данных, в \textbf{бустинге} модели строятся строго \textbf{последовательно}.
\end{myblock}

\begin{textbox}{Алгоритм Последовательного Исправления Ошибок}
    Общая схема бустинга выглядит так:
    \begin{enumerate}[nosep]
        \item Обучается первая (обычно простая) модель $F_0$ на исходных данных.
        \item Вычисляются ошибки (или остатки) $e_1 = y - F_0(x)$ этой модели.
        \item Следующая модель $h_1$ обучается предсказывать эти ошибки $e_1$.
        \item Предсказание ансамбля обновляется: $F_1(x) = F_0(x) + \nu \cdot h_1(x)$ (где $\nu$ - темп обучения).
        \item Вычисляются новые ошибки $e_2 = y - F_1(x)$.
        \item Обучается следующая модель $h_2$ на ошибках $e_2$.
        \item Ансамбль обновляется: $F_2(x) = F_1(x) + \nu \cdot h_2(x)$.
        \item Шаги повторяются $M$ раз (заданное число моделей) или до остановки по критерию.
    \end{enumerate}
\end{textbox}

\begin{myblock}{Ключевая Идея}
    Ансамбль постепенно "учится" на своих ошибках. Каждая последующая модель фокусируется на тех объектах или аспектах данных, где предыдущие модели ошибались больше всего, тем самым улучшая общее предсказание.
\end{myblock}

% ================================================
% Раздел 2: Градиентный Спуск на Функциях
% ================================================
\section{Градиентный Спуск в Пространстве Функций}

\begin{myblock}{Минимизация Функции Потерь}
    GBM обобщает идею бустинга, используя \textbf{градиентный спуск} для минимизации произвольной дифференцируемой \textbf{функции потерь (Loss Function)} $L(y, F(x))$. Здесь $y$ - истинное значение, $F(x)$ - текущее предсказание ансамбля.
    \newline
    Ключевое отличие от стандартного градиентного спуска: оптимизация происходит не в пространстве параметров модели, а в \textbf{пространстве функций}.
\end{myblock}

\begin{myblock}{Шаги Градиентного Бустинга}
    Процесс обучения на шаге $m$ (для $m = 1, \dots, M$):
    \begin{enumerate}[label=\arabic*., wide, labelindent=0pt, itemsep=1ex]
        \item \textbf{Инициализация:} Начинаем с простого предсказания $F_0(x)$, обычно константы, минимизирующей loss (например, среднее $y$ для MSE, медиана для MAE, логарифм шансов для LogLoss).

        \item \textbf{Вычисление Псевдо-остатков:} Для каждого объекта $i$ вычисляется \textbf{отрицательный градиент} функции потерь по предсказанию ансамбля на предыдущем шаге $F_{m-1}(x)$:
          \[ r_{im} = - \left[ \frac{\partial L(y_i, F(x_i))}{\partial F(x_i)} \right]_{F(x) = F_{m-1}(x)} \]
          Эти $r_{im}$ называются \textbf{псевдо-остатками} и показывают, в каком "направлении" нужно изменить предсказание $F_{m-1}(x_i)$, чтобы уменьшить ошибку $L$.

        \item \textbf{Обучение Слабой Модели:} Новая слабая модель $h_m(x)$ (обычно неглубокое дерево решений) обучается аппроксимировать псевдо-остатки $\{ (x_i, r_{im}) \}_{i=1}^N$.

        \item \textbf{Поиск Оптимального Шага (опционально):} Для дерева решений часто находят оптимальные значения $\gamma_{jm}$ в листьях $j$ дерева $h_m$.

        \item \textbf{Обновление Ансамбля:} Предсказание ансамбля обновляется:
          \[ F_m(x) = F_{m-1}(x) + \nu \cdot h_m(x) \]
          где $\nu$ (nu) — это \textbf{темп обучения (learning rate)}, коэффициент $0 < \nu \le 1$ (обычно маленький, e.g., 0.01-0.1), который масштабирует вклад каждой новой модели. Он помогает предотвратить переобучение и делает сходимость более плавной.
    \end{enumerate}
\end{myblock}

\begin{textbox}{Аналогия: Спуск с Холма}
    Представьте функцию потерь как ландшафт, где высота — это ошибка.
    \begin{itemize}[nosep, leftmargin=*]
        \item Ваше текущее положение — предсказание ансамбля $F_{m-1}(x)$.
        \item Градиент $\frac{\partial L}{\partial F}$ показывает направление самого крутого подъема.
        \item Псевдо-остатки ($-\frac{\partial L}{\partial F}$) указывают в сторону спуска (анти-градиент).
        \item Обучение $h_m(x)$ на псевдо-остатках — это попытка найти "шаг" в направлении спуска.
        \item Learning rate $\nu$ — это размер этого шага. Маленькие шаги помогают не "проскочить" долину (минимум ошибки).
    \end{itemize}
\end{textbox}

% ================================================
% Раздел 3: Основные Функции Потерь (Loss Functions)
% ================================================
\section{Основные Функции Потерь (Loss Functions)}

\begin{textbox}{Зависимость от Задачи}
    Выбор функции потерь $L(y, F)$ критически важен и зависит от решаемой задачи (регрессия или классификация) и специфики данных (например, наличие выбросов).
\end{textbox}

\subsection{A Функции Потерь для Регрессии}
\begin{myexampleblock}{Регрессионные Loss-функции}
    \begin{itemize}[nosep, leftmargin=*]
        \item \textbf{MSE (Mean Squared Error) / L2 Loss:}
            \[ L(y, F) = \frac{1}{2}(y - F)^2 \]
            Псевдо-остатки: $r = y - F$. Стандартный выбор, но чувствителен к выбросам из-за квадратичной ошибки.
        \item \textbf{MAE (Mean Absolute Error) / L1 Loss:}
            \[ L(y, F) = |y - F| \]
            Псевдо-остатки: $r = \text{sign}(y - F)$. Менее чувствительна к выбросам, чем MSE.
        \item \textbf{Huber Loss:}
            \[ L(y, F) = \begin{cases} \frac{1}{2}(y - F)^2 & \text{if } |y - F| \le \delta \\ \delta (|y - F| - \frac{1}{2}\delta) & \text{if } |y - F| > \delta \end{cases} \]
            Комбинирует свойства MSE (для малых ошибок) и MAE (для больших ошибок), что делает ее робастной к выбросам. Параметр $\delta$ контролирует порог переключения.
        \item \textbf{Quantile Loss:} Используется для предсказания квантилей распределения целевой переменной.
    \end{itemize}
\end{myexampleblock}

\subsection{B Функции Потерь для Бинарной Классификации}
\begin{myexampleblock}{Бинарные Классификационные Loss-функции}
    Здесь $y \in \{0, 1\}$ или $y \in \{-1, 1\}$, а $F$ обычно представляет логит вероятности $p$, т.е. $F = \log(\frac{p}{1-p})$.
    \begin{itemize}[nosep, leftmargin=*]
        \item \textbf{LogLoss (Логистическая / Бинарная Кросс-Энтропия):}
            \[ L(y, F) = \log(1 + e^{-F}) \quad \text{(для } y=1 \text{)} \quad \text{или} \quad L(y, F) = \log(1 + e^{F}) \quad \text{(для } y=0 \text{)} \]
            Или общая форма для $y \in \{0, 1\}$: $L(y, p) = -[y \log(p) + (1-y)\log(1-p)]$, где $p = \sigma(F) = \frac{1}{1+e^{-F}}$.
            Стандартный и наиболее распространенный выбор для задач классификации.
            Псевдо-остатки: $r = y - p$.
        \item \textbf{Exponential Loss (Экспоненциальная):}
            \[ L(y, F) = e^{-yF} \quad \text{(для } y \in \{-1, 1\}\text{)} \]
            Используется в алгоритме AdaBoost. Сильнее штрафует за неверные предсказания, может быть менее робастна к шуму/выбросам, чем LogLoss.
    \end{itemize}
\end{myexampleblock}

\subsection{C Функции Потерь для Многоклассовой Классификации}
\begin{myexampleblock}{Многоклассовые Классификационные Loss-функции}
    \begin{itemize}[nosep, leftmargin=*]
        \item \textbf{Multinomial LogLoss (Категориальная Кросс-Энтропия):} Обобщение бинарной LogLoss на случай $K > 2$ классов. Ансамбль предсказывает вектор логитов $F = (F_1, \dots, F_K)$, вероятности получаются через Softmax: $p_k = \frac{e^{F_k}}{\sum_{j=1}^K e^{F_j}}$.
            \[ L(y, p) = - \sum_{k=1}^K y_k \log(p_k) \]
            где $y$ - one-hot вектор истинного класса.
    \end{itemize}
\end{myexampleblock}

% ================================================
% Раздел 4: Популярные Библиотеки: XGBoost, LightGBM, CatBoost
% ================================================
\section{Популярные Библиотеки GBM}

\begin{textbox}{Зачем Нужны Продвинутые Реализации?}
    Стандартный алгоритм GBM имеет ряд ограничений (например, склонность к переобучению, не самая высокая скорость). На практике почти всегда используют его улучшенные реализации, такие как XGBoost, LightGBM и CatBoost, которые включают множество оптимизаций и дополнительных возможностей.
\end{textbox}

\subsection{A XGBoost (eXtreme Gradient Boosting)}
\begin{myblock}{Ключевые Особенности XGBoost}
    \begin{itemize}[nosep, leftmargin=*]
        \item \textbf{Регуляризация:} В функцию потерь при построении дерева добавляются штрафы L1 (Lasso) и L2 (Ridge) на веса листьев. Это контролирует сложность моделей и эффективно борется с переобучением.
        \item \textbf{Улучшенный Поиск Сплитов:} Использует информацию о второй производной функции потерь (Гессиан) для более точного построения деревьев.
        \item \textbf{Обработка Пропусков (NaN):} Имеет встроенный механизм для работы с пропущенными значениями: при построении дерева алгоритм "учится", в какую ветку (левую или правую) лучше направлять объекты с NaN для каждого признака.
        \item \textbf{Оптимизации Скорости:}
            \begin{itemize}[label=\textbullet, nosep, leftmargin=*]
                 \item Параллельные вычисления на уровне построения дерева (по признакам).
                 \item Приближенные алгоритмы поиска сплитов (quantile approximation) для больших данных.
                 \item Кэширование градиентов и гессианов.
                 \item Блочная структура данных для эффективного доступа к памяти.
            \end{itemize}
        \item \textbf{Кросс-валидация:} Встроенная функция для кросс-валидации при подборе числа деревьев.
    \end{itemize}
    \textit{Позиционирование:} XGBoost долгое время был "золотым стандартом" и де-факто выбором №1 для соревнований и промышленных задач на табличных данных.
\end{myblock}

\subsection{B LightGBM (Light Gradient Boosting Machine)}
\begin{myblock}{Ключевые Особенности LightGBM}
    \begin{itemize}[nosep, leftmargin=*]
        \item \textbf{Высокая Скорость и Низкое Потребление Памяти:} Основное преимущество, особенно на больших датасетах. Достигается за счет нескольких техник.
        \item \textbf{Leaf-wise Рост Деревьев:} Вместо роста по уровням (level-wise, как в XGBoost), LightGBM выбирает для расщепления тот лист, который даст максимальное уменьшение функции потерь (gain). Это позволяет строить более глубокие и асимметричные деревья, что часто эффективнее, но требует контроля глубины (max\_depth) для предотвращения переобучения на малых данных.
        \item \textbf{GOSS (Gradient-based One-Side Sampling):} Для ускорения обучения на каждой итерации используются не все данные. Алгоритм сохраняет все объекты с большими градиентами (на которых модель сильно ошибается) и случайно отбирает долю объектов с малыми градиентами. Это позволяет сфокусироваться на "сложных" объектах без большого смещения оценки градиента.
        \item \textbf{EFB (Exclusive Feature Bundling):} Техника для уменьшения числа признаков путем объединения "взаимоисключающих" признаков (тех, которые редко принимают ненулевые значения одновременно, например, one-hot кодированные).
        \item \textbf{Оптимизированная Обработка Категориальных Признаков:} Поддерживает передачу категориальных признаков напрямую (без OHE), используя специальные алгоритмы сплита (Fisher).
    \end{itemize}
    \textit{Позиционирование:} Отличный выбор для очень больших датасетов, где скорость обучения и потребление памяти критичны.
\end{myblock}

\subsection{C CatBoost (Categorical Boosting)}
\begin{myblock}{Ключевые Особенности CatBoost}
    \begin{itemize}[nosep, leftmargin=*]
        \item \textbf{Лучшая Обработка Категориальных Признаков:} Главная "фишка". Использует продвинутые методы кодирования категорий "на лету" во время обучения:
            \begin{itemize}[label=\textbullet, nosep, leftmargin=*]
                 \item \textbf{Ordered Target Statistics (TS):} Вычисляет статистики целевой переменной (например, среднее значение y) для каждой категории, но делает это хитро, используя "исторические" данные (только объекты, идущие до текущего в некоторой случайной перестановке), чтобы избежать утечки целевой переменной (target leakage) и переобучения.
                 \item Комбинации категориальных признаков генерируются автоматически.
                 \item Не требует предварительной обработки категорий (как OHE), что упрощает пайплайн и часто дает лучшее качество.
            \end{itemize}
        \item \textbf{Ordered Boosting:} Модификация градиентного бустинга, которая также борется с target leakage и prediction shift, обучая модель на остатках, полученных на данных, не включающих текущий объект (для TS).
        \item \textbf{Симметричные (Oblivious) Деревья:} Все узлы на одном уровне дерева используют одно и то же условие (признак и порог) для сплита. Это:
            \begin{itemize}[label=\textbullet, nosep, leftmargin=*]
                 \item Действует как неявная регуляризация.
                 \item Значительно ускоряет предсказание модели (особенно на CPU).
                 \item Упрощает структуру модели.
            \end{itemize}
        \item \textbf{Меньше Настройки Гиперпараметров:} Часто показывает хорошие результаты "из коробки" с параметрами по умолчанию.
        \item \textbf{Хорошая Визуализация:} Встроенные инструменты для анализа модели и процесса обучения.
    \end{itemize}
    \textit{Позиционирование:} Идеален для задач с большим количеством категориальных признаков. Часто дает высокое качество с минимальной настройкой и имеет очень быстрое время предсказания.
\end{myblock}

% ================================================
% Раздел 5: Важность Признаков (Feature Importance)
% ================================================
\section{Важность Признаков (Feature Importance) в GBM}

\begin{myexampleblock}{Методы Оценки Важности}
    GBM, как и другие ансамбли деревьев, позволяет оценить вклад каждого признака в итоговое предсказание. Основные методы:
    \begin{itemize}[nosep, leftmargin=*]
        \item \textbf{Gain (Прирост / Feature Importance):} Среднее уменьшение функции потерь (или другого критерия, например, Gini impurity/Variance reduction) при использовании признака для сплита во всех деревьях ансамбля. Суммарный gain по всем сплитам признака делится на общее число сплитов по этому признаку (или просто суммируется, зависит от реализации). \textbf{Считается наиболее надежным методом.}
        \item \textbf{Split Count / Frequency (Частота Использования / Weight):} Просто подсчитывает, сколько раз признак был выбран для разделения узла во всех деревьях. Простой метод, но не учитывает, насколько *полезным* был каждый сплит. Может переоценивать важность числовых признаков с большим количеством потенциальных порогов.
        \item \textbf{Coverage (Покрытие):} Среднее количество объектов в обучающей выборке, которые проходят через сплиты по данному признаку (иногда взвешенное по gain или другим метрикам). Показывает, какую долю данных "затрагивает" признак. Предоставляется не всеми библиотеками (есть в XGBoost, LightGBM).
    \end{itemize}
\end{myexampleblock}

\begin{textbox}{Использование Информации о Важности Признаков}
    Знание важности признаков полезно для:
    \begin{itemize}[nosep, leftmargin=*]
        \item \textbf{Интерпретации модели:} Понять, какие факторы наиболее сильно влияют на предсказание (хотя GBM остается моделью "черного ящика" по сравнению с линейными моделями).
        \item \textbf{Отбора признаков (Feature Selection):} Исключить неважные или малозначимые признаки, что может упростить модель, ускорить обучение/предсказание и иногда даже улучшить качество за счет уменьшения шума.
        \item \textbf{Генерации новых признаков (Feature Engineering):} Сосредоточить усилия на создании признаков на основе наиболее важных существующих.
    \end{itemize}
\end{textbox}

\begin{alerttextbox}{Предостережение}
    Следует с осторожностью относиться к результатам оценки важности:
    \begin{itemize}[nosep, leftmargin=*]
        \item Методы (особенно 'Split Count') могут быть \textbf{смещены} в сторону числовых признаков с большим количеством уникальных значений (high cardinality numerical features) или категориальных признаков с большим числом категорий (если используется OHE или простые методы кодирования), так как у них больше потенциальных точек для сплита. CatBoost с его обработкой категорий менее подвержен этой проблеме для категориальных данных.
        \item \textbf{Коррелирующие признаки:} Если два признака сильно коррелируют и оба полезны, модель может использовать для сплитов то один, то другой. В результате их важность может быть "размазана" между ними, и каждый по отдельности будет выглядеть менее важным, чем он есть на самом деле.
    \end{itemize}
\end{alerttextbox}

% --- Конец контента ---

% >>> КОММЕНТАРИЙ: Печатаем библиографию (если используется \cite)
\AtNextBibliography{\footnotesize} % Уменьшаем шрифт для библиографии
%\printbibliography % Раскомментируйте, если добавляли цитаты через \cite или \footcite

\end{multicols}

\end{document}