\documentclass[10pt,a4paper]{article}

\usepackage[landscape,margin=0.7cm]{geometry}
\usepackage[russian,english]{babel} % Поддержка русского и английского

% >>> КОММЕНТАРИЙ: Определяем переменные для заголовка и автора (можно изменить)
\newcommand{\cheatsheettitle}{\color{w3schools}Шпаргалка по предобработке данных и feature engineering / {\color{alert} Концепции} {\color{black} Cheatsheet (XeLaTeX)}}
\newcommand{\cheatsheetauthor}{Краткий справочник}

% >>> КОММЕНТАРИЙ: Подключаем основной файл шаблона с настройками стилей и пакетов
% \documentclass{article} % ЗАКОММЕНТИРОВАНО: Добавлено для возможности компиляции этого фрагмента отдельно для проверки
\usepackage{fontspec}

% >>> КОММЕНТАРИЙ: Установка основных шрифтов документа (требует установленных шрифтов в системе)
\setmainfont{PT Sans} % Основной шрифт для текста
\setsansfont{PT Sans} % Шрифт без засечек (если используется отдельно)
\setmonofont{Liberation Mono} % Моноширинный шрифт (для кода)

\usepackage{fontawesome} % Для иконок (например, \faQuoteLeft)
\usepackage{hyperref} % Для создания кликабельных ссылок (в т.ч. в \resourcelink)
\usepackage{enumitem} % Для настройки списков
\usepackage{lipsum} % Для генерации текста-рыбы (не используется в финальной версии)
\usepackage{xcolor} % Для определения и использования цветов
\usepackage{minted} % Для подсветки синтаксиса кода (требует Python и Pygments, компиляция с -shell-escape)

% >>>>>> НОВОЕ: Пакет для создания графиков <<<<<<
\usepackage{pgfplots}
\pgfplotsset{compat=newest} % Устанавливаем последнюю версию совместимости
\usetikzlibrary{arrows.meta, shapes.geometric} % Загружаем полезные библиотеки TikZ

\usepackage{titlesec} % Для настройки заголовков секций
% >>> КОММЕНТАРИЙ: Настройка отступов для заголовков секций (\section)
\titlespacing*{\section}{0pt}{*1.5}{*0.8} % {отступ слева}{отступ сверху}{отступ снизу}

% >>> КОММЕНТАРИЙ: Форматирование заголовков секций (\section)
\titleformat{\section}
  {\normalfont\large\bfseries\color{mDarkTeal}} % Стиль заголовка (цвет, размер, жирность)
  {\thesection} % Номер секции
  {1em} % Отступ после номера
  {} % Код перед заголовком (здесь пусто)

% >>> КОММЕНТАРИЙ: Определение пользовательских цветов через HTML-коды
\definecolor{customcolor}{HTML}{696EA8} % Основной цвет для блоков textbox
\definecolor{alert}{HTML}{CD5C5C} % Цвет для выделения (красный)
\definecolor{w3schools}{HTML}{4CAF50} % Цвет для выделения (зеленый)
\definecolor{subbox}{gray}{0.60} % Цвет для под-блоков (не используется?)
\definecolor{codecolor}{HTML}{FFC300} % Цвет для кода (не используется?)
\definecolor{mDarkTeal}{HTML}{23373b} % Темно-бирюзовый (для заголовков секций, названий блоков)
\colorlet{beamerboxbg}{black!2} % Фон для beamerbox (не используется?)
\colorlet{beamerboxfg}{mDarkTeal} % Цвет текста для beamerbox (не используется?)
\definecolor{mDarkBrown}{HTML}{604c38} % Темно-коричневый (не используется?)
\definecolor{mLightBrown}{HTML}{EB811B} % Светло-коричневый (для заголовков myalertblock)
\definecolor{mLightGreen}{HTML}{14B03D} % Светло-зеленый (для заголовков myexampleblock)
\definecolor{alertTextBox}{HTML}{A8696E} % Цвет для блоков alerttextbox

% >>> КОММЕНТАРИЙ: Настройка пакета biblatex для библиографии
\usepackage
[citestyle=authoryear, % Стиль цитирования (Автор-Год)
sorting=nty, % Сортировка по Имени, Году, Названию
autocite=footnote, % Команда \autocite создает сноски
autolang=hyphen, % Автоматическое определение языка для переносов
mincrossrefs=1, % Минимальное количество перекрестных ссылок
backend=biber] % Используемый бэкенд (требует запуск biber)
{biblatex}

% >>> КОММЕНТАРИЙ: Настройка формата пост-заметок в цитатах (например, номера страниц)
\DeclareFieldFormat{postnote}{#1}
\DeclareFieldFormat{multipostnote}{#1}
\DeclareAutoCiteCommand{footnote}[f]{\footcite}{\footcites} % Настройка команды \autocite

% >>> КОММЕНТАРИЙ: Подключение файла с библиографическими записями
\addbibresource{literature.bib}

% >>> КОММЕНТАРИЙ: Подключение библиотеки tcolorbox для создания цветных блоков
\usepackage{tcolorbox}
\tcbuselibrary{most, listingsutf8, minted} % Загрузка необходимых библиотек tcolorbox (most включает breakable)

% >>> КОММЕНТАРИЙ: Глобальные настройки для маленьких tcbox (inline блоки)
\tcbset{tcbox width=auto,left=1mm,top=1mm,bottom=1mm,
right=1mm,boxsep=1mm,middle=1pt}

% >>> КОММЕНТАРИЙ: Определение базового окружения для цветных блоков с заголовком
% breakable: позволяет блоку разрываться между колонками/страницами.
% ВАЖНО: В multicols автоматический разрыв breakable может работать не всегда идеально.
% Если блок разрывается некрасиво, попробуйте разбить контент на несколько блоков
% или используйте \needspace{<длина>} перед блоком.
\newenvironment{mycolorbox}[2]{
\begin{tcolorbox}[capture=minipage,fonttitle=\large\bfseries, enhanced,boxsep=1mm,colback=#1!30!white,
width=\linewidth, % Ширина блока равна ширине текущей колонки
arc=2pt,outer arc=2pt, toptitle=0mm,colframe=#1,opacityback=0.7,nobeforeafter,
breakable, % Разрешить разрыв блока
title=#2] % Текст заголовка блока
}{\end{tcolorbox}}

% >>> КОММЕНТАРИЙ: Окружение subbox (не используется в текущем коде)
\newenvironment{subbox}[2]{
\begin{tcolorbox}[capture=minipage,fonttitle=\normalsize\bfseries, enhanced,boxsep=1mm,colback=#1!30!white,on line,tcbox width=auto,left=0.3em,top=1mm, toptitle=0mm,colframe=#1,opacityback=0.7,nobeforeafter,
breakable,
title=#2]\footnotesize
}{\normalsize\end{tcolorbox}\vspace{0.1em}}

% >>> КОММЕНТАРИЙ: Окружение для размещения tcolorbox'ов в несколько колонок внутри блока (не используется)
\newenvironment{multibox}[1]{
\begin{tcbraster}[raster columns=#1,raster equal height,nobeforeafter,raster column skip=1em,raster left skip=1em,raster right skip=1em]}{\end{tcbraster}}

% >>> КОММЕНТАРИЙ: Окружения-обертки для mycolorbox с предопределенными цветами и заголовками
\newenvironment{textbox}[1]{\begin{mycolorbox}{customcolor}{#1}}{\end{mycolorbox}} % Обычный текстовый блок
\newenvironment{alerttextbox}[1]{\begin{mycolorbox}{alertTextBox}{#1}}{\end{mycolorbox}} % Блок для предупреждений/важной информации

% >>> КОММЕНТАРИЙ: Определение окружения для блоков с кодом с подсветкой minted
\newtcblisting{codebox}[3][]{ % [опции minted], {язык}, {заголовок}
colback=black!10,colframe=black!20, % Цвета фона и рамки
listing only, % Показывать только листинг кода
minted options={ % Опции, передаваемые в minted
    numbers=left, % Нумерация строк слева
    style=default, % Стиль подсветки (можно выбрать другие, напр., 'friendly')
    fontsize=\scriptsize, % Размер шрифта кода
    breaklines, % Автоматический перенос длинных строк
    breaksymbolleft=, % Символ для обозначения переноса (пусто)
    autogobble, % Удаление лишних отступов слева
    linenos, % Показывать номера строк
    numbersep=0.7em, % Отступ номеров строк от кода
    #1 % Дополнительные опции minted, переданные в [1]
    },
enhanced,top=1mm, toptitle=0mm,
left=5mm, % Отступ слева внутри блока (для номеров строк)
arc=0pt,outer arc=0pt, % Прямые углы
title={#3}, % Заголовок блока кода
fonttitle=\small\bfseries\color{mDarkTeal}, % Стиль заголовка блока кода
listing engine=minted, % Движок для листинга - minted
minted language=#2, % Язык программирования для подсветки
breakable % Разрешить разрыв блока кода
}

\newcommand{\punkti}{~\lbrack\dots\rbrack~} % Команда для [...] (не используется)

% >>> КОММЕНТАРИЙ: Переопределение окружения quote для добавления иконок цитат
\renewenvironment{quote}
               {\list{\faQuoteLeft\phantom{ }}{\rightmargin\leftmargin}
                \item\relax\scriptsize\ignorespaces}
               {\unskip\unskip\phantom{xx}\faQuoteRight\endlist}

% >>> КОММЕНТАРИЙ: Вспомогательные команды для создания цветных фонов под текстом (не используются)
\newcommand{\bgupper}[3]{\colorbox{#1}{\color{#2}\huge\bfseries\MakeUppercase{#3}}}
\newcommand{\bg}[3]{\colorbox{#1}{\bfseries\color{#2}#3}}

% >>> КОММЕНТАРИЙ: Команда для форматирования описания команды/функции
% #1: Сама команда (используется \detokenize для корректного отображения спецсимволов)
% #2: Описание команды
\newcommand{\mycommand}[2]{
  {\par\noindent\ttfamily\detokenize{#1}\par} % Вывод команды моноширинным шрифтом
  \nopagebreak % Стараемся не разрывать страницу после команды
  \hangindent=1.5em \hangafter=1 \noindent % Делаем отступ для описания
  \small\textit{#2}\par\vspace{0.5ex} % Вывод описания и небольшой отступ после
}

% >>> КОММЕНТАРИЙ: Команда для форматирования ссылок на ресурсы
% #1: URL ссылки
% #2: Текст ссылки
% #3: Описание ресурса
\newcommand{\resourcelink}[3]{
  {\par\noindent\href{#1}{\ttfamily #2}\par} % Вывод кликабельной ссылки
  \nopagebreak
  \hangindent=1.5em \hangafter=1 \noindent
  \small\textit{#3}\par\vspace{0.5ex} % Вывод описания
}

% >>> КОММЕНТАРИЙ: Вспомогательные команды для стилизации
\newcommand{\sep}{{\scriptsize~\faCircle{ }~}} % Разделитель-кружок
\newcommand{\bggreen}[1]{\medskip\bgupper{w3schools}{black}{#1}\\[0.5em]} % Зеленый заголовок (не используется)
\newcommand{\green}[1]{\smallskip\bg{w3schools}{white}{#1}\\} % Текст на зеленом фоне
\newcommand{\red}[1]{\smallskip\bg{alert}{white}{#1}\\} % Текст на красном фоне
\newcommand{\alertcmd}[1]{\red{#1}\\} % Псевдоним для red (переименовано чтобы не конфликтовать с цветом alert)

\usepackage{multicol} % Пакет для создания нескольких колонок
\setlength{\columnsep}{15pt} % Расстояние между колонками

\setlength{\parindent}{0pt} % Убираем абзацный отступ по умолчанию
\usepackage{csquotes} % Для правильного отображения кавычек в разных языках (используется biblatex)
\newcommand{\loremipsum}{Lorem ipsum dolor sit amet.} % Пример текста (не используется)

% >>> КОММЕНТАРИЙ: Окружения для блоков с разными цветами заголовков (пример, предупреждение, обычный)
% Основаны на tcolorbox, похожи на mycolorbox, но с другими цветами заголовков и рамки.
% Также используют breakable. Применяются для выделения примеров, предупреждений и т.д.
% ВАЖНО: Лог может содержать предупреждения "Underfull \hbox (badness 10000)".
% Это часто случается в узких колонках multicol при наличии кода, URL или текста,
% который плохо переносится. Обычно это лишь косметическая проблема, строки выглядят
% немного не до конца заполненными. Можно игнорировать или попробовать перефразировать/
% использовать \sloppy перед проблемным абзацем.
\newenvironment{myexampleblock}[1]{ % Блок для примеров (зеленый заголовок)
    \tcolorbox[capture=minipage,fonttitle=\small\bfseries\color{mLightGreen}, enhanced,boxsep=1mm,colback=black!10,breakable,noparskip,
    on line,tcbox width=auto,left=0.3em,top=1mm, toptitle=0mm,
    colframe=black!20,arc=0pt,outer arc=0pt,
    opacityback=0.7,nobeforeafter,title=#1]}
    {\endtcolorbox}

\newenvironment{myalertblock}[1]{ % Блок для предупреждений (оранжевый заголовок)
    \tcolorbox[capture=minipage,fonttitle=\small\bfseries\color{mLightBrown}, enhanced,boxsep=1mm,colback=black!10,breakable,noparskip,
    on line,tcbox width=auto,left=0.3em,top=1mm, toptitle=0mm,
    colframe=black!20,arc=0pt,outer arc=0pt,
    opacityback=0.7,nobeforeafter,title=#1]}
    {\endtcolorbox}

\newenvironment{myblock}[1]{ % Обычный информационный блок (бирюзовый заголовок)
    \tcolorbox[capture=minipage,fonttitle=\small\bfseries\color{mDarkTeal}, enhanced,boxsep=1mm,colback=black!10,breakable,noparskip,
    on line,tcbox width=auto,left=0.3em,top=1mm, toptitle=0mm,
    colframe=black!20, arc=0pt,outer arc=0pt,
    opacityback=0.7,nobeforeafter,title=#1]}
    {\endtcolorbox}

% >>> КОММЕНТАРИЙ: Команда для вставки изображения в рамке tcolorbox с подписью
% #1: Опции (не используются)
% #2: Путь к файлу изображения (используется как подпись под изображением)
\newcommand{\mygraphics}[2][]{
\tcbox[enhanced,boxsep=0pt,top=0pt,bottom=0pt,left=0pt,
right=0pt,boxrule=0.4pt,drop fuzzy shadow,clip upper,
colback=black!75!white,toptitle=2pt,bottomtitle=2pt,nobeforeafter,
center title,fonttitle=\small\sffamily,title=\detokenize{#2}] % Используем путь как заголовок
{\includegraphics[width=\the\dimexpr(\linewidth-4mm)\relax]{#2}} % Вставляем изображение, ширина чуть меньше колонки
}

% >>> КОМЕНТАРИЙ: Команда \needspace{<длина>}
% Используйте эту команду *перед* блоком (tcolorbox, section, figure), чтобы убедиться,
% что в текущей колонке есть как минимум <длина> свободного места. Если места нет,
% LaTeX начнет новую колонку/страницу перед выполнением команды.
% Пример: \needspace{10\baselineskip} % Запросить место для примерно 10 строк текста
% Это полезно для предотвращения "висячих" заголовков или некрасивых разрывов блоков.

% >>>>>> НОВОЕ: Стили для графиков PGFPlots <<<<<<
\pgfplotsset{
    % Базовый стиль для всех графиков в шпаргалке
    cheatsheet plot style/.style={
        width=0.98\linewidth, % Ширина чуть меньше колонки для аккуратности
        height=5cm,          % Высота графика (можно изменить по необходимости)
        grid=major,          % Включаем основную сетку
        grid style={dashed, color=black!20}, % Стиль сетки: пунктирная, светло-серая
        axis lines=left,     % Линии осей только слева и снизу
        axis line style={color=black!60, thick}, % Стиль линий осей: серые, потолще
        tick style={color=black!60, thick}, % Стиль меток на осях
        ticklabel style={font=\scriptsize, color=black}, % Стиль подписей меток: маленький шрифт
        label style={font=\small, color=mDarkTeal}, % Стиль подписей осей: шрифт small, цвет как у заголовков блоков
        title style={font=\small\bfseries, color=mDarkTeal}, % Стиль заголовка графика
        legend style={        % Стиль легенды
            font=\scriptsize, % Маленький шрифт
            draw=black!30,    % Легкая рамка вокруг легенды
            fill=white,       % Белый фон
            legend cell align=left, % Выравнивание текста в ячейке легенды
            anchor=north east, % Якорь для позиционирования
            at={(rel axis cs:0.98,0.98)} % Положение в правом верхнем углу внутри области графика
        },
        cycle list={ % Список стилей для линий \addplot (цвета взяты из шаблона)
            {blue, mark=*, thick},
            {alert, mark=square, thick},
            {w3schools, mark=triangle, thick},
            {mLightBrown, mark=diamond, thick},
            {customcolor, mark=oplus, thick},
            {black!50, mark=pentagon, thick},
        },
        % Убираем рамку вокруг графика по умолчанию
        axis background/.style={fill=none},
    },
    % Дополнительный стиль для ROC-кривой (без сетки, с диагональю)
    roc curve style/.style={
        cheatsheet plot style, % Наследуем базовый стиль
        width=6cm, height=6cm, % Делаем квадратным
        grid=none, % Убираем сетку
        xlabel={False Positive Rate (FPR)},
        ylabel={True Positive Rate (TPR)},
        xmin=0, xmax=1,
        ymin=0, ymax=1,
        xtick={0, 0.5, 1},
        ytick={0, 0.5, 1},
        legend pos=south east, % Легенда внизу справа
        % Добавляем диагональную линию случайного угадывания
        extra x ticks={0.5}, extra y ticks={0.5},
        extra tick style={grid=major, grid style={dashed, color=black!40}},
        after end axis/.code={ % Добавляем линию после отрисовки осей
            \draw[dashed, color=black!40] (axis cs:0,0) -- (axis cs:1,1);
        }
    },
    % Дополнительный стиль для PR-кривой
    pr curve style/.style={
        cheatsheet plot style, % Наследуем базовый стиль
        xlabel={Recall (TPR)},
        ylabel={Precision},
        xmin=0, xmax=1,
        ymin=0, ymax=1,
        legend pos=south west, % Легенда внизу слева
    }
}

% % Добавлено для возможности компиляции этого фрагмента отдельно для проверки
% \begin{document}
% \begin{multicols}{2} % Пример использования в multicols
% \lipsum[1] % Просто текст для заполнения

% \begin{textbox}{Пример графика (Базовый стиль)}
%     \begin{center}
%     \begin{tikzpicture}
%         \begin{axis}[cheatsheet plot style, title={Пример $y=x^2$ и $y=x+1$}]
%         \addplot coordinates {(0, 0) (1, 1) (2, 4) (3, 9) (4, 16)};
%         \addlegendentry{$y=x^2$}
%         \addplot coordinates {(0, 1) (1, 2) (2, 3) (3, 4) (4, 5)};
%         \addlegendentry{$y=x+1$}
%         \end{axis}
%     \end{tikzpicture}
%     \end{center}
% \end{textbox}

% \begin{myexampleblock}{Пример ROC-кривой}
%     \begin{center}
%     \begin{tikzpicture}
%         \begin{axis}[roc curve style, title={Пример ROC Curve}]
%         % Пример данных ROC кривой
%         \addplot coordinates { (0,0) (0.1,0.3) (0.2,0.6) (0.4,0.8) (0.6,0.9) (0.8,0.95) (1,1) };
%         \addlegendentry{Модель A (AUC $\approx$ 0.8)}
%         \addplot coordinates { (0,0) (0.2,0.2) (0.4,0.4) (0.6,0.6) (0.8,0.8) (1,1) }; % Пример хуже
%         \addlegendentry{Модель B (AUC = 0.5)}
%         \end{axis}
%     \end{tikzpicture}
%     \end{center}
% \end{myexampleblock}

% \begin{myblock}{Пример PR-кривой}
%     \begin{center}
%     \begin{tikzpicture}
%         \begin{axis}[pr curve style, title={Пример Precision-Recall Curve}]
%         % Пример данных PR кривой (могут сильно зависеть от порога)
%         \addplot coordinates { (0,0.9) (0.1,0.85) (0.3,0.8) (0.5,0.7) (0.7,0.6) (0.9,0.4) (1,0.3) };
%         \addlegendentry{Модель C}
%         \end{axis}
%     \end{tikzpicture}
%     \end{center}
% \end{myblock}

% \lipsum[2-3] % Просто текст для заполнения
% \end{multicols}
% \end{document}

\begin{document}
\pagestyle{empty} % Убираем номера страниц
\small % Уменьшаем базовый размер шрифта для всего документа

% >>> КОММЕНТАРИЙ: Используем multicols для создания трех колонок.
\begin{multicols}{3}
\raggedcolumns % Не растягивать колонки по вертикали

\noindent
\begin{minipage}{\linewidth}
    \centering
    {\bfseries\huge \cheatsheettitle \par}
    \vspace{1ex}
    {\large \cheatsheetauthor \par}
    \vspace{0.5ex}
    {\normalsize \today \par} % Вставляем текущую дату
\end{minipage}
\vspace{2ex}

\thispagestyle{empty} % Убеждаемся, что и на первой странице нет номера

\scriptsize % Уменьшаем шрифт для оглавления и основного текста
\tableofcontents % Генерируем оглавление

% >>> КОММЕНТАРИЙ: Подключаем основной контент шпаргалки
% >>> Контент для шпаргалки по Data Preprocessing и Feature Engineering

% --- Раздел 1 ---
\section{Обработка пропусков (Missing Values)}

\begin{myblock}{Почему пропуски - это проблема?}
    Многие ML-алгоритмы не могут работать с пропущенными значениями (\texttt{NaN}, \texttt{None}). Пропуски могут исказить результаты анализа и снизить производительность модели.
    \begin{itemize}
        \item \textbf{Причины пропусков:} Ошибки ввода, сбои сенсоров, отказ пользователя отвечать, данные недоступны.
    \end{itemize}
\end{myblock}

\begin{textbox}{Стратегии обработки пропусков}
    Выбор стратегии зависит от количества пропусков, типа данных и специфики задачи. (Указанные проценты - это эвристики, а не строгие правила).
    \begin{enumerate}
        \item \textbf{Удаление (Deletion):}
            \begin{itemize}
                \item \textbf{Удаление строк (Listwise deletion):} Удаляем весь объект (строку), если в нем есть хотя бы один пропуск.
                    \textit{Когда?} Мало пропусков (< 5-10\%), данные пропущены случайно (\textbf{MCAR} - Missing Completely At Random, т.е. сам факт пропуска не зависит ни от других признаков, ни от самого пропущенного значения), большой датасет.
                    \textit{Минус:} Потеря данных.
                \item \textbf{Удаление столбцов (Dropping features):} Удаляем весь признак (столбец), если в нем слишком много пропусков (> 50-70\%).
                    \textit{Когда?} Признак не очень важен, или пропусков слишком много для заполнения.
                    \textit{Минус:} Потеря потенциально полезной информации.
            \end{itemize}
        \item \textbf{Заполнение (Imputation):} Заменяем пропуски оценочными значениями.
            \begin{itemize}
                \item \textbf{Простые методы:}
                    \begin{itemize}
                        \item \textbf{Среднее (Mean):} Для числовых признаков без сильных выбросов. Чувствительно к выбросам.
                        \item \textbf{Медиана (Median):} Для числовых признаков с выбросами. Более робастно.
                        \item \textbf{Мода (Mode):} Для категориальных признаков.
                    \end{itemize}
                \item \textbf{Более сложные методы (кратко):}
                    \begin{itemize}
                        \item Заполнение с помощью \textbf{k-ближайших соседей (k-NN Imputer):} Использует значения соседей для оценки пропуска.
                        \item Заполнение с помощью \textbf{моделей (Regression Imputation):} Предсказывает пропуск с помощью регрессии на основе других признаков.
                        \item Создание \textbf{индикаторного признака:} Добавляем новый бинарный столбец, указывающий, было ли значение пропущено (1 - да, 0 - нет), а сам пропуск заполняем (например, нулем или средним). Позволяет модели учесть сам факт пропуска, \textbf{так как иногда сам факт отсутствия данных несет полезную информацию}.
                    \end{itemize}
            \end{itemize}
            \textit{Плюс:} Сохраняет данные. \textit{Минус:} Может внести смещение (bias) в данные.
    \end{enumerate}
\end{textbox}

\begin{myexampleblock}{Аналогия: Детектив и недостающие улики}
    Представьте, что вы детектив, расследующий дело.
    \begin{itemize}
        \item \textbf{Удаление строки:} Если по свидетелю совсем нет информации (много пропусков), вы можете решить его не учитывать (удалить строку). Риск: потеряете ключевого свидетеля.
        \item \textbf{Удаление столбца:} Если какой-то тип улик (например, отпечатки пальцев) не удалось собрать почти нигде на месте преступления (много пропусков в столбце), вы можете перестать на него полагаться (удалить столбец). Риск: упустите важный тип улик.
        \item \textbf{Заполнение (Imputation):} Если у вас нет точного времени события, вы можете предположить его на основе других фактов: среднее время похожих преступлений (\textbf{mean}), наиболее вероятное время (\textbf{mode}), или предсказать его с помощью сложной модели (\textbf{k-NN/Regression}). Риск: ваше предположение может быть неверным.
        \item \textbf{Индикаторный признак:} Вы отмечаете в деле, что точное время \textit{неизвестно} (индикатор=1), но записываете предполагаемое время (например, медиану). Теперь факт неизвестности времени - это тоже улика!
    \end{itemize}
\end{myexampleblock}

% --- Раздел 2 ---
\section{Кодирование категориальных признаков}

\begin{myblock}{Зачем кодировать?}
    Большинство ML-моделей понимают только числа. Категориальные признаки (например, "цвет": "красный", "синий"; "город": "Москва", "СПб") нужно преобразовать в числовой формат.
\end{myblock}

\begin{textbox}{One-Hot Encoding (OHE) vs Label Encoding}
    Два самых популярных метода:

    \textbf{1. One-Hot Encoding (OHE):}
    \begin{itemize}
        \item \textbf{Как работает:} Создает новый бинарный (0 или 1) столбец для каждой уникальной категории признака. В каждой строке только один из этих новых столбцов равен 1, остальные — 0.
        \item \textbf{Когда использовать:} Для \textbf{номинальных} признаков (где нет естественного порядка категорий, например, "цвет", "страна"). Подходит для большинства моделей, особенно линейных, SVM, нейронных сетей.
        \item \textbf{Плюсы:} Не вносит ложного порядка.
        \item \textbf{Минусы:} Сильно увеличивает количество признаков ("проклятие размерности"), если категорий много. Может привести к мультиколлинеарности (часто удаляют один из OHE-столбцов — `drop='first'`).
    \end{itemize}

    \textbf{2. Label Encoding (LE):}
    \begin{itemize}
        \item \textbf{Как работает:} Присваивает каждой уникальной категории целое число (0, 1, 2, ...).
        \item \textbf{Когда использовать:}
            \begin{itemize}
                \item Для \textbf{порядковых} признаков (где есть естественный порядок, например, "размер": "S" < "M" < "L" -> 0, 1, 2).
                \item Иногда для \textbf{древовидных моделей} (Решающее дерево, Случайный лес, Градиентный бустинг), так как они могут разбивать признак по значениям (например, `< 1.5`). \textit{Но будьте осторожны: OHE часто все равно дает лучший результат даже для деревьев.}
            \end{itemize}
        \item \textbf{Плюсы:} Не увеличивает количество признаков.
        \item \textbf{Минусы:} Вносит \textbf{ложный порядок} для номинальных признаков (например, "Москва" = 0, "СПб" = 1, "Казань" = 2 подразумевает, что "СПб" "больше" "Москвы", что неверно). Это может запутать модели, чувствительные к значениям (линейные, SVM).
    \end{itemize}

    \textbf{Итог (Частый вопрос на собеседовании):}
    \begin{itemize}
        \item Есть порядок (ordinal)? $\rightarrow$ \textbf{Label Encoding}.
        \item Нет порядка (nominal)?
            \begin{itemize}
                \item Модель - дерево? $\rightarrow$ Можно попробовать \textbf{Label Encoding} (но OHE часто лучше).
                \item Модель - НЕ дерево (линейная, SVM, NN)? $\rightarrow$ \textbf{One-Hot Encoding}.
                \item Очень много категорий? $\rightarrow$ Рассмотреть другие методы (Target Encoding, Embedding) или Feature Engineering.
            \end{itemize}
    \end{itemize}
\end{textbox}

\begin{myexampleblock}{Аналогия: Маркировка одежды на складе}
    \begin{itemize}
        \item \textbf{Label Encoding (для порядковых):} Размеры футболок "S", "M", "L". Можно просто пронумеровать полки 0, 1, 2. Порядок сохранен, все понятно.
        \item \textbf{Label Encoding (для номинальных - ПЛОХО):} Цвета футболок "Красный", "Синий", "Зеленый". Если пронумеровать полки 0, 1, 2, то это подразумевает, что "Синий" (1) чем-то "больше" "Красного" (0), что бессмысленно. Алгоритм может сделать неверные выводы.
        \item \textbf{One-Hot Encoding (для номинальных):} Для цветов "Красный", "Синий", "Зеленый" создаем три отдельные секции на складе. Если футболка красная, кладем ее в секцию "Красный" (1), а в секциях "Синий" и "Зеленый" ее нет (0). Порядка нет, но признаков (секций) стало больше.
    \end{itemize}
\end{myexampleblock}

% --- Раздел 3 ---
\section{Масштабирование признаков (Feature Scaling)}

\begin{myblock}{Зачем масштабировать?}
    Если числовые признаки имеют разные диапазоны значений (например, возраст [0-100] и зарплата [10k-1M]), то модели, использующие \textbf{меры расстояния} (KNN, SVM, K-Means) или \textbf{градиентный спуск} (Линейная/Логистическая регрессия, Нейронные сети), могут придать неоправданно больший "вес" признакам с бОльшими значениями. Масштабирование приводит все признаки к сопоставимому диапазону.
\end{myblock}

\begin{textbox}{Нормализация vs Стандартизация}
    Два основных метода:

    \textbf{1. Нормализация (Normalization / Min-Max Scaling):}
    \begin{itemize}
        \item \textbf{Как работает:} Сжимает данные в диапазон [0, 1] (или [-1, 1]).
        \item \textbf{Формула:} \[ X_{norm} = \frac{X - X_{min}}{X_{max} - X_{min}} \]
        \item \textbf{Когда использовать:} Когда нужен строгий диапазон [0, 1] (например, для обработки изображений, где пиксели 0-255). Если данные не имеют гауссова распределения. Когда известно, что выбросов мало или они обработаны.
        \item \textbf{Минусы:} Чувствительна к \textbf{выбросам}, так как $X_{min}$ и $X_{max}$ могут сильно сдвинуться из-за одного аномального значения.
    \end{itemize}

    \textbf{2. Стандартизация (Standardization / Z-score Scaling):}
    \begin{itemize}
        \item \textbf{Как работает:} Преобразует данные так, чтобы они имели среднее значение 0 и стандартное отклонение 1.
        \item \textbf{Формула:} \[ X_{std} = \frac{X - \mu}{\sigma} \] где $\mu$ - среднее, $\sigma$ - стандартное отклонение.
        \item \textbf{Когда использовать:} \textbf{Чаще всего рекомендуется}. Особенно для алгоритмов, которые предполагают нормальное (гауссово) распределение данных (хотя работает и для других распределений). Менее чувствительна к выбросам, чем нормализация. Подходит для PCA, линейных моделей, SVM, KNN.
        \item \textbf{Минусы:} Не приводит данные к строгому диапазону (значения могут быть > 1 или < -1).
    \end{itemize}

    \textbf{Когда масштабирование НЕ обязательно (или менее критично):}
    \begin{itemize}
        \item \textbf{Древовидные модели} (Решающее Дерево, Случайный Лес, Градиентный Бустинг): Они смотрят на пороги разбиения для каждого признака независимо, поэтому масштаб не так важен. \textit{(Но иногда масштабирование все же может немного улучшить сходимость/стабильность даже в деревьях, хотя и не так критично, как для других моделей).}
    \end{itemize}
\end{textbox}

\begin{myexampleblock}{Аналогия: Сравнение оценок из разных школ}
    Ученик А получил 90 баллов из 100 (шкала 0-100). Ученик Б получил 4 балла из 5 (шкала 0-5). Кто лучше?
    \begin{itemize}
        \item \textbf{Без масштабирования:} 90 > 4, кажется, что А лучше. Но это некорректно.
        \item \textbf{Нормализация (в шкалу [0, 1]):}
            Ученик А: (90 - 0) / (100 - 0) = 0.9
            Ученик Б: (4 - 0) / (5 - 0) = 0.8
            Теперь видно, что результат А немного выше.
        \item \textbf{Стандартизация (относительно среднего по школе):} Допустим, средний балл в школе А был 70 (std=10), а в школе Б - 3 (std=0.5).
            Ученик А: (90 - 70) / 10 = +2 сигмы (значительно выше среднего по своей школе).
            Ученик Б: (4 - 3) / 0.5 = +2 сигмы (тоже значительно выше среднего по своей школе).
            Стандартизация показала, что оба ученика выступили одинаково хорошо \textit{относительно своей группы}.
    \end{itemize}
\end{myexampleblock}

% --- Раздел 4 ---
\section{Feature Engineering (Очень Кратко)}

\begin{myblock}{Идея: Создание лучших признаков для модели}
    Это процесс использования знаний о данных и предметной области для создания признаков, которые делают работу ML-моделей более эффективной. Включает в себя как \textbf{создание} новых признаков, так и \textbf{отбор} или \textbf{извлечение} существующих. \textbf{Часто это самый важный шаг для улучшения модели!}
\end{myblock}

\begin{textbox}{Основные направления Feature Engineering}
    \textbf{1. Отбор признаков (Feature Selection):}
    \begin{itemize}
        \item \textbf{Идея:} Выбрать подмножество \textit{наиболее важных} исходных признаков.
        \item \textbf{Цель:} Уменьшить сложность, ускорить обучение, бороться с переобучением, улучшить интерпретируемость.
        \item \textbf{Примеры методов (названия):} Фильтры (корреляция, Хи-квадрат, ANOVA F-value), Обертки (Recursive Feature Elimination - RFE), Встроенные (Lasso-регрессия L1, Feature Importance из деревьев).
    \end{itemize}

    \textbf{2. Извлечение признаков (Feature Extraction):}
    \begin{itemize}
        \item \textbf{Идея:} Создать \textit{новые} признаки путем комбинации или трансформации исходных, часто с понижением размерности. Новые признаки обычно не имеют прямого физического смысла исходных.
        \item \textbf{Цель:} Уменьшить размерность данных, сохранив при этом максимум полезной информации, бороться с шумом.
        \item \textbf{Примеры методов (названия):} Метод главных компонент (\textbf{PCA} - Principal Component Analysis), Линейный дискриминантный анализ (\textbf{LDA}), разложение матриц (SVD).
    \end{itemize}

    \textbf{3. Создание признаков (Feature Creation):}
    \begin{itemize}
        \item \textbf{Идея:} Генерация новых признаков из существующих на основе знаний о предметной области или анализа данных.
        \item \textbf{Цель:} Добавить в модель полезную информацию, которой не было в исходных признаках в явном виде.
        \item \textbf{Примеры:}
            \begin{itemize}
                \item Полиномиальные признаки ($x_1^2$, $x_1 \times x_2$) - часто для линейных моделей.
                \item Взаимодействия признаков ($x_1 / x_2$, $x_1 + x_2$, разница между датами).
                \item Признаки из дат (день недели, месяц, час, является ли день праздником/выходным).
                \item Агрегированные признаки (например, для клиента: средняя сумма покупки за месяц, количество покупок за неделю).
                \item Биннинг (группировка) непрерывных признаков (например, возраст $\rightarrow$ возрастные группы: "молодой", "средний", "пожилой").
            \end{itemize}
    \end{itemize}
\end{textbox}

\begin{myexampleblock}{Аналогия: Подготовка к походу}
    \begin{itemize}
        \item \textbf{Feature Selection (Отбор):} У вас много снаряжения (признаков). Вы выбираете только самое необходимое для конкретного похода (самые важные признаки): нож, палатку, спальник. Отбрасываете ненужное (менее важные признаки): утюг, фен. Вы используете \textit{исходные} предметы.
        \item \textbf{Feature Extraction (Извлечение):} У вас есть много разных продуктов (исходные признаки). Вместо того чтобы нести их все, вы готовите из них концентрированную, калорийную походную еду (новые признаки), например, сублиматы. Эта еда содержит энергию из исходных продуктов, но сама по себе является чем-то новым и занимает меньше места (понижение размерности).
        \item \textbf{Feature Creation (Создание):} У вас есть орехи и сухофрукты (исходные признаки). Вы смешиваете их и делаете питательный энергетический батончик (новый признак). Батончик - это комбинация исходных, но он удобнее и дает энергию по-другому. Или вы смотрите на карту (данные) и понимаете, что нужно пересечь реку (знание о данных) - вы создаете признак "нужен ли плот" (новый признак).
    \end{itemize}
\end{myexampleblock}

% --- Конец контента ---

% >>> КОММЕНТАРИЙ: Печатаем библиографию (если используется \cite)
\AtNextBibliography{\footnotesize} % Уменьшаем шрифт для библиографии
%\printbibliography % Раскомментируйте, если добавляли цитаты через \cite или \footcite

\end{multicols}

\end{document}