\documentclass[10pt,a4paper]{article}

\usepackage[landscape,margin=0.7cm]{geometry}
\usepackage[russian,english]{babel} % Поддержка русского и английского

% >>> КОММЕНТАРИЙ: Определяем переменные для заголовка и автора (можно изменить)
\newcommand{\cheatsheettitle}{\color{w3schools}Шпаргалка по частым задачам на leetcode / {\color{alert} Паттерны и примеры} {\color{black} Cheatsheet (XeLaTeX)}}
\newcommand{\cheatsheetauthor}{Краткий справочник}

% >>> КОММЕНТАРИЙ: Подключаем основной файл шаблона с настройками стилей и пакетов
% \documentclass{article} % ЗАКОММЕНТИРОВАНО: Добавлено для возможности компиляции этого фрагмента отдельно для проверки
\usepackage{fontspec}

% >>> КОММЕНТАРИЙ: Установка основных шрифтов документа (требует установленных шрифтов в системе)
\setmainfont{PT Sans} % Основной шрифт для текста
\setsansfont{PT Sans} % Шрифт без засечек (если используется отдельно)
\setmonofont{Liberation Mono} % Моноширинный шрифт (для кода)

\usepackage{fontawesome} % Для иконок (например, \faQuoteLeft)
\usepackage{hyperref} % Для создания кликабельных ссылок (в т.ч. в \resourcelink)
\usepackage{enumitem} % Для настройки списков
\usepackage{lipsum} % Для генерации текста-рыбы (не используется в финальной версии)
\usepackage{xcolor} % Для определения и использования цветов
\usepackage{minted} % Для подсветки синтаксиса кода (требует Python и Pygments, компиляция с -shell-escape)

% >>>>>> НОВОЕ: Пакет для создания графиков <<<<<<
\usepackage{pgfplots}
\pgfplotsset{compat=newest} % Устанавливаем последнюю версию совместимости
\usetikzlibrary{arrows.meta, shapes.geometric} % Загружаем полезные библиотеки TikZ

\usepackage{titlesec} % Для настройки заголовков секций
% >>> КОММЕНТАРИЙ: Настройка отступов для заголовков секций (\section)
\titlespacing*{\section}{0pt}{*1.5}{*0.8} % {отступ слева}{отступ сверху}{отступ снизу}

% >>> КОММЕНТАРИЙ: Форматирование заголовков секций (\section)
\titleformat{\section}
  {\normalfont\large\bfseries\color{mDarkTeal}} % Стиль заголовка (цвет, размер, жирность)
  {\thesection} % Номер секции
  {1em} % Отступ после номера
  {} % Код перед заголовком (здесь пусто)

% >>> КОММЕНТАРИЙ: Определение пользовательских цветов через HTML-коды
\definecolor{customcolor}{HTML}{696EA8} % Основной цвет для блоков textbox
\definecolor{alert}{HTML}{CD5C5C} % Цвет для выделения (красный)
\definecolor{w3schools}{HTML}{4CAF50} % Цвет для выделения (зеленый)
\definecolor{subbox}{gray}{0.60} % Цвет для под-блоков (не используется?)
\definecolor{codecolor}{HTML}{FFC300} % Цвет для кода (не используется?)
\definecolor{mDarkTeal}{HTML}{23373b} % Темно-бирюзовый (для заголовков секций, названий блоков)
\colorlet{beamerboxbg}{black!2} % Фон для beamerbox (не используется?)
\colorlet{beamerboxfg}{mDarkTeal} % Цвет текста для beamerbox (не используется?)
\definecolor{mDarkBrown}{HTML}{604c38} % Темно-коричневый (не используется?)
\definecolor{mLightBrown}{HTML}{EB811B} % Светло-коричневый (для заголовков myalertblock)
\definecolor{mLightGreen}{HTML}{14B03D} % Светло-зеленый (для заголовков myexampleblock)
\definecolor{alertTextBox}{HTML}{A8696E} % Цвет для блоков alerttextbox

% >>> КОММЕНТАРИЙ: Настройка пакета biblatex для библиографии
\usepackage
[citestyle=authoryear, % Стиль цитирования (Автор-Год)
sorting=nty, % Сортировка по Имени, Году, Названию
autocite=footnote, % Команда \autocite создает сноски
autolang=hyphen, % Автоматическое определение языка для переносов
mincrossrefs=1, % Минимальное количество перекрестных ссылок
backend=biber] % Используемый бэкенд (требует запуск biber)
{biblatex}

% >>> КОММЕНТАРИЙ: Настройка формата пост-заметок в цитатах (например, номера страниц)
\DeclareFieldFormat{postnote}{#1}
\DeclareFieldFormat{multipostnote}{#1}
\DeclareAutoCiteCommand{footnote}[f]{\footcite}{\footcites} % Настройка команды \autocite

% >>> КОММЕНТАРИЙ: Подключение файла с библиографическими записями
\addbibresource{literature.bib}

% >>> КОММЕНТАРИЙ: Подключение библиотеки tcolorbox для создания цветных блоков
\usepackage{tcolorbox}
\tcbuselibrary{most, listingsutf8, minted} % Загрузка необходимых библиотек tcolorbox (most включает breakable)

% >>> КОММЕНТАРИЙ: Глобальные настройки для маленьких tcbox (inline блоки)
\tcbset{tcbox width=auto,left=1mm,top=1mm,bottom=1mm,
right=1mm,boxsep=1mm,middle=1pt}

% >>> КОММЕНТАРИЙ: Определение базового окружения для цветных блоков с заголовком
% breakable: позволяет блоку разрываться между колонками/страницами.
% ВАЖНО: В multicols автоматический разрыв breakable может работать не всегда идеально.
% Если блок разрывается некрасиво, попробуйте разбить контент на несколько блоков
% или используйте \needspace{<длина>} перед блоком.
\newenvironment{mycolorbox}[2]{
\begin{tcolorbox}[capture=minipage,fonttitle=\large\bfseries, enhanced,boxsep=1mm,colback=#1!30!white,
width=\linewidth, % Ширина блока равна ширине текущей колонки
arc=2pt,outer arc=2pt, toptitle=0mm,colframe=#1,opacityback=0.7,nobeforeafter,
breakable, % Разрешить разрыв блока
title=#2] % Текст заголовка блока
}{\end{tcolorbox}}

% >>> КОММЕНТАРИЙ: Окружение subbox (не используется в текущем коде)
\newenvironment{subbox}[2]{
\begin{tcolorbox}[capture=minipage,fonttitle=\normalsize\bfseries, enhanced,boxsep=1mm,colback=#1!30!white,on line,tcbox width=auto,left=0.3em,top=1mm, toptitle=0mm,colframe=#1,opacityback=0.7,nobeforeafter,
breakable,
title=#2]\footnotesize
}{\normalsize\end{tcolorbox}\vspace{0.1em}}

% >>> КОММЕНТАРИЙ: Окружение для размещения tcolorbox'ов в несколько колонок внутри блока (не используется)
\newenvironment{multibox}[1]{
\begin{tcbraster}[raster columns=#1,raster equal height,nobeforeafter,raster column skip=1em,raster left skip=1em,raster right skip=1em]}{\end{tcbraster}}

% >>> КОММЕНТАРИЙ: Окружения-обертки для mycolorbox с предопределенными цветами и заголовками
\newenvironment{textbox}[1]{\begin{mycolorbox}{customcolor}{#1}}{\end{mycolorbox}} % Обычный текстовый блок
\newenvironment{alerttextbox}[1]{\begin{mycolorbox}{alertTextBox}{#1}}{\end{mycolorbox}} % Блок для предупреждений/важной информации

% >>> КОММЕНТАРИЙ: Определение окружения для блоков с кодом с подсветкой minted
\newtcblisting{codebox}[3][]{ % [опции minted], {язык}, {заголовок}
colback=black!10,colframe=black!20, % Цвета фона и рамки
listing only, % Показывать только листинг кода
minted options={ % Опции, передаваемые в minted
    numbers=left, % Нумерация строк слева
    style=default, % Стиль подсветки (можно выбрать другие, напр., 'friendly')
    fontsize=\scriptsize, % Размер шрифта кода
    breaklines, % Автоматический перенос длинных строк
    breaksymbolleft=, % Символ для обозначения переноса (пусто)
    autogobble, % Удаление лишних отступов слева
    linenos, % Показывать номера строк
    numbersep=0.7em, % Отступ номеров строк от кода
    #1 % Дополнительные опции minted, переданные в [1]
    },
enhanced,top=1mm, toptitle=0mm,
left=5mm, % Отступ слева внутри блока (для номеров строк)
arc=0pt,outer arc=0pt, % Прямые углы
title={#3}, % Заголовок блока кода
fonttitle=\small\bfseries\color{mDarkTeal}, % Стиль заголовка блока кода
listing engine=minted, % Движок для листинга - minted
minted language=#2, % Язык программирования для подсветки
breakable % Разрешить разрыв блока кода
}

\newcommand{\punkti}{~\lbrack\dots\rbrack~} % Команда для [...] (не используется)

% >>> КОММЕНТАРИЙ: Переопределение окружения quote для добавления иконок цитат
\renewenvironment{quote}
               {\list{\faQuoteLeft\phantom{ }}{\rightmargin\leftmargin}
                \item\relax\scriptsize\ignorespaces}
               {\unskip\unskip\phantom{xx}\faQuoteRight\endlist}

% >>> КОММЕНТАРИЙ: Вспомогательные команды для создания цветных фонов под текстом (не используются)
\newcommand{\bgupper}[3]{\colorbox{#1}{\color{#2}\huge\bfseries\MakeUppercase{#3}}}
\newcommand{\bg}[3]{\colorbox{#1}{\bfseries\color{#2}#3}}

% >>> КОММЕНТАРИЙ: Команда для форматирования описания команды/функции
% #1: Сама команда (используется \detokenize для корректного отображения спецсимволов)
% #2: Описание команды
\newcommand{\mycommand}[2]{
  {\par\noindent\ttfamily\detokenize{#1}\par} % Вывод команды моноширинным шрифтом
  \nopagebreak % Стараемся не разрывать страницу после команды
  \hangindent=1.5em \hangafter=1 \noindent % Делаем отступ для описания
  \small\textit{#2}\par\vspace{0.5ex} % Вывод описания и небольшой отступ после
}

% >>> КОММЕНТАРИЙ: Команда для форматирования ссылок на ресурсы
% #1: URL ссылки
% #2: Текст ссылки
% #3: Описание ресурса
\newcommand{\resourcelink}[3]{
  {\par\noindent\href{#1}{\ttfamily #2}\par} % Вывод кликабельной ссылки
  \nopagebreak
  \hangindent=1.5em \hangafter=1 \noindent
  \small\textit{#3}\par\vspace{0.5ex} % Вывод описания
}

% >>> КОММЕНТАРИЙ: Вспомогательные команды для стилизации
\newcommand{\sep}{{\scriptsize~\faCircle{ }~}} % Разделитель-кружок
\newcommand{\bggreen}[1]{\medskip\bgupper{w3schools}{black}{#1}\\[0.5em]} % Зеленый заголовок (не используется)
\newcommand{\green}[1]{\smallskip\bg{w3schools}{white}{#1}\\} % Текст на зеленом фоне
\newcommand{\red}[1]{\smallskip\bg{alert}{white}{#1}\\} % Текст на красном фоне
\newcommand{\alertcmd}[1]{\red{#1}\\} % Псевдоним для red (переименовано чтобы не конфликтовать с цветом alert)

\usepackage{multicol} % Пакет для создания нескольких колонок
\setlength{\columnsep}{15pt} % Расстояние между колонками

\setlength{\parindent}{0pt} % Убираем абзацный отступ по умолчанию
\usepackage{csquotes} % Для правильного отображения кавычек в разных языках (используется biblatex)
\newcommand{\loremipsum}{Lorem ipsum dolor sit amet.} % Пример текста (не используется)

% >>> КОММЕНТАРИЙ: Окружения для блоков с разными цветами заголовков (пример, предупреждение, обычный)
% Основаны на tcolorbox, похожи на mycolorbox, но с другими цветами заголовков и рамки.
% Также используют breakable. Применяются для выделения примеров, предупреждений и т.д.
% ВАЖНО: Лог может содержать предупреждения "Underfull \hbox (badness 10000)".
% Это часто случается в узких колонках multicol при наличии кода, URL или текста,
% который плохо переносится. Обычно это лишь косметическая проблема, строки выглядят
% немного не до конца заполненными. Можно игнорировать или попробовать перефразировать/
% использовать \sloppy перед проблемным абзацем.
\newenvironment{myexampleblock}[1]{ % Блок для примеров (зеленый заголовок)
    \tcolorbox[capture=minipage,fonttitle=\small\bfseries\color{mLightGreen}, enhanced,boxsep=1mm,colback=black!10,breakable,noparskip,
    on line,tcbox width=auto,left=0.3em,top=1mm, toptitle=0mm,
    colframe=black!20,arc=0pt,outer arc=0pt,
    opacityback=0.7,nobeforeafter,title=#1]}
    {\endtcolorbox}

\newenvironment{myalertblock}[1]{ % Блок для предупреждений (оранжевый заголовок)
    \tcolorbox[capture=minipage,fonttitle=\small\bfseries\color{mLightBrown}, enhanced,boxsep=1mm,colback=black!10,breakable,noparskip,
    on line,tcbox width=auto,left=0.3em,top=1mm, toptitle=0mm,
    colframe=black!20,arc=0pt,outer arc=0pt,
    opacityback=0.7,nobeforeafter,title=#1]}
    {\endtcolorbox}

\newenvironment{myblock}[1]{ % Обычный информационный блок (бирюзовый заголовок)
    \tcolorbox[capture=minipage,fonttitle=\small\bfseries\color{mDarkTeal}, enhanced,boxsep=1mm,colback=black!10,breakable,noparskip,
    on line,tcbox width=auto,left=0.3em,top=1mm, toptitle=0mm,
    colframe=black!20, arc=0pt,outer arc=0pt,
    opacityback=0.7,nobeforeafter,title=#1]}
    {\endtcolorbox}

% >>> КОММЕНТАРИЙ: Команда для вставки изображения в рамке tcolorbox с подписью
% #1: Опции (не используются)
% #2: Путь к файлу изображения (используется как подпись под изображением)
\newcommand{\mygraphics}[2][]{
\tcbox[enhanced,boxsep=0pt,top=0pt,bottom=0pt,left=0pt,
right=0pt,boxrule=0.4pt,drop fuzzy shadow,clip upper,
colback=black!75!white,toptitle=2pt,bottomtitle=2pt,nobeforeafter,
center title,fonttitle=\small\sffamily,title=\detokenize{#2}] % Используем путь как заголовок
{\includegraphics[width=\the\dimexpr(\linewidth-4mm)\relax]{#2}} % Вставляем изображение, ширина чуть меньше колонки
}

% >>> КОМЕНТАРИЙ: Команда \needspace{<длина>}
% Используйте эту команду *перед* блоком (tcolorbox, section, figure), чтобы убедиться,
% что в текущей колонке есть как минимум <длина> свободного места. Если места нет,
% LaTeX начнет новую колонку/страницу перед выполнением команды.
% Пример: \needspace{10\baselineskip} % Запросить место для примерно 10 строк текста
% Это полезно для предотвращения "висячих" заголовков или некрасивых разрывов блоков.

% >>>>>> НОВОЕ: Стили для графиков PGFPlots <<<<<<
\pgfplotsset{
    % Базовый стиль для всех графиков в шпаргалке
    cheatsheet plot style/.style={
        width=0.98\linewidth, % Ширина чуть меньше колонки для аккуратности
        height=5cm,          % Высота графика (можно изменить по необходимости)
        grid=major,          % Включаем основную сетку
        grid style={dashed, color=black!20}, % Стиль сетки: пунктирная, светло-серая
        axis lines=left,     % Линии осей только слева и снизу
        axis line style={color=black!60, thick}, % Стиль линий осей: серые, потолще
        tick style={color=black!60, thick}, % Стиль меток на осях
        ticklabel style={font=\scriptsize, color=black}, % Стиль подписей меток: маленький шрифт
        label style={font=\small, color=mDarkTeal}, % Стиль подписей осей: шрифт small, цвет как у заголовков блоков
        title style={font=\small\bfseries, color=mDarkTeal}, % Стиль заголовка графика
        legend style={        % Стиль легенды
            font=\scriptsize, % Маленький шрифт
            draw=black!30,    % Легкая рамка вокруг легенды
            fill=white,       % Белый фон
            legend cell align=left, % Выравнивание текста в ячейке легенды
            anchor=north east, % Якорь для позиционирования
            at={(rel axis cs:0.98,0.98)} % Положение в правом верхнем углу внутри области графика
        },
        cycle list={ % Список стилей для линий \addplot (цвета взяты из шаблона)
            {blue, mark=*, thick},
            {alert, mark=square, thick},
            {w3schools, mark=triangle, thick},
            {mLightBrown, mark=diamond, thick},
            {customcolor, mark=oplus, thick},
            {black!50, mark=pentagon, thick},
        },
        % Убираем рамку вокруг графика по умолчанию
        axis background/.style={fill=none},
    },
    % Дополнительный стиль для ROC-кривой (без сетки, с диагональю)
    roc curve style/.style={
        cheatsheet plot style, % Наследуем базовый стиль
        width=6cm, height=6cm, % Делаем квадратным
        grid=none, % Убираем сетку
        xlabel={False Positive Rate (FPR)},
        ylabel={True Positive Rate (TPR)},
        xmin=0, xmax=1,
        ymin=0, ymax=1,
        xtick={0, 0.5, 1},
        ytick={0, 0.5, 1},
        legend pos=south east, % Легенда внизу справа
        % Добавляем диагональную линию случайного угадывания
        extra x ticks={0.5}, extra y ticks={0.5},
        extra tick style={grid=major, grid style={dashed, color=black!40}},
        after end axis/.code={ % Добавляем линию после отрисовки осей
            \draw[dashed, color=black!40] (axis cs:0,0) -- (axis cs:1,1);
        }
    },
    % Дополнительный стиль для PR-кривой
    pr curve style/.style={
        cheatsheet plot style, % Наследуем базовый стиль
        xlabel={Recall (TPR)},
        ylabel={Precision},
        xmin=0, xmax=1,
        ymin=0, ymax=1,
        legend pos=south west, % Легенда внизу слева
    }
}

% % Добавлено для возможности компиляции этого фрагмента отдельно для проверки
% \begin{document}
% \begin{multicols}{2} % Пример использования в multicols
% \lipsum[1] % Просто текст для заполнения

% \begin{textbox}{Пример графика (Базовый стиль)}
%     \begin{center}
%     \begin{tikzpicture}
%         \begin{axis}[cheatsheet plot style, title={Пример $y=x^2$ и $y=x+1$}]
%         \addplot coordinates {(0, 0) (1, 1) (2, 4) (3, 9) (4, 16)};
%         \addlegendentry{$y=x^2$}
%         \addplot coordinates {(0, 1) (1, 2) (2, 3) (3, 4) (4, 5)};
%         \addlegendentry{$y=x+1$}
%         \end{axis}
%     \end{tikzpicture}
%     \end{center}
% \end{textbox}

% \begin{myexampleblock}{Пример ROC-кривой}
%     \begin{center}
%     \begin{tikzpicture}
%         \begin{axis}[roc curve style, title={Пример ROC Curve}]
%         % Пример данных ROC кривой
%         \addplot coordinates { (0,0) (0.1,0.3) (0.2,0.6) (0.4,0.8) (0.6,0.9) (0.8,0.95) (1,1) };
%         \addlegendentry{Модель A (AUC $\approx$ 0.8)}
%         \addplot coordinates { (0,0) (0.2,0.2) (0.4,0.4) (0.6,0.6) (0.8,0.8) (1,1) }; % Пример хуже
%         \addlegendentry{Модель B (AUC = 0.5)}
%         \end{axis}
%     \end{tikzpicture}
%     \end{center}
% \end{myexampleblock}

% \begin{myblock}{Пример PR-кривой}
%     \begin{center}
%     \begin{tikzpicture}
%         \begin{axis}[pr curve style, title={Пример Precision-Recall Curve}]
%         % Пример данных PR кривой (могут сильно зависеть от порога)
%         \addplot coordinates { (0,0.9) (0.1,0.85) (0.3,0.8) (0.5,0.7) (0.7,0.6) (0.9,0.4) (1,0.3) };
%         \addlegendentry{Модель C}
%         \end{axis}
%     \end{tikzpicture}
%     \end{center}
% \end{myblock}

% \lipsum[2-3] % Просто текст для заполнения
% \end{multicols}
% \end{document}

\begin{document}
\pagestyle{empty} % Убираем номера страниц
\small % Уменьшаем базовый размер шрифта для всего документа

% >>> КОММЕНТАРИЙ: Используем multicols для создания трех колонок.
\begin{multicols}{3}
\raggedcolumns % Не растягивать колонки по вертикали

\noindent
\begin{minipage}{\linewidth}
    \centering
    {\bfseries\huge \cheatsheettitle \par}
    \vspace{1ex}
    {\large \cheatsheetauthor \par}
    \vspace{0.5ex}
    {\normalsize \today \par} % Вставляем текущую дату
\end{minipage}
\vspace{2ex}

\thispagestyle{empty} % Убеждаемся, что и на первой странице нет номера

\scriptsize % Уменьшаем шрифт для оглавления и основного текста
\tableofcontents % Генерируем оглавление

% >>> КОММЕНТАРИЙ: Подключаем основной контент шпаргалки
% ================================================
% Алгоритмические Паттерны - Шпаргалка v2
% ================================================

% ================================================
% Секция 1: HashTable / Словари / Множества (Hash Map / Set)
% ================================================
\section{HashTable / Словари / Множества (Hash Map / Set)}

\subsection{Суть Структуры Данных}
\begin{myblock}{Академическое Описание}
    Структура данных, отображающая ключи (keys) на значения (values) с использованием хеш-функции. Обеспечивает в среднем $O(1)$ время для операций вставки, удаления и поиска.
\end{myblock}
\begin{myblock}{Множества (Sets)}
    Хранят только уникальные ключи, эффективно проверяя наличие элемента ($O(1)$ в среднем).
\end{myblock}

\subsection{Простое Объяснение}
\begin{myblock}{Аналогия}
    Представьте себе картотеку. У каждой карточки есть уникальный номер (ключ). Вы можете почти мгновенно найти карточку по номеру, добавить новую или убрать старую. Множество (Set) похоже на список уникальных посетителей: можно быстро проверить, есть ли человек в списке.
\end{myblock}

\subsection{Распознавание Паттерна}
\begin{myblock}{Сигналы и Ключевые Слова}
    Ищите необходимость быстро проверить \emph{наличие} элемента, \emph{подсчитать частоту}, \emph{сгруппировать} элементы, найти \emph{пары} с определенным свойством, проверить на \emph{дубликаты} или \emph{анаграммы}. \newline
    Ключевые слова: "пара", "сумма", "частота", "группа", "уникальный", "дубликат", "анаграмма", \texttt{contains}, \texttt{exists}.
\end{myblock}

\subsection{Частые Комбинации}
\begin{myblock}{Комбинации с Другими Паттернами (в вашем списке)}
    \begin{itemize}[nosep, leftmargin=*]
        \item \textbf{С массивами/строками:} Подсчет частот (\texttt{Group Anagrams}, \texttt{First Unique Character}), поиск пар (\texttt{Two Sum}), проверка условий (\texttt{Longest Substring...}, \texttt{Continuous Subarray Sum}).
        \item \textbf{С Linked Lists:} В реализации \texttt{LRU Cache}.
        \item \textbf{С Design:} В \texttt{Insert Delete GetRandom O(1)}.
        \item \textbf{С Геометрией:} Хранение координат (\texttt{Line Reflection}).
    \end{itemize}
\end{myblock}

\subsection{Пример Python Сниппета}
\begin{codebox}{python}{Подсчет частоты / Проверка уникальности}
def count_frequency(items):
    counts = {} # или collections.defaultdict(int)
    for item in items:
        counts[item] = counts.get(item, 0) + 1
    return counts
# Пример использования set для проверки уникальности
seen = set()
for item in items:
    if item in seen:
        print(f"Duplicate found: {item}")
    seen.add(item)
\end{codebox}

\subsection{Анализ Сложности}
\begin{myblock}{Время (среднее)}
    $O(1)$ для \texttt{add}, \texttt{remove}, \texttt{get}, \texttt{in}. Достигается за счет хеш-функции.
\end{myblock}
\begin{myblock}{Время (худшее)}
    $O(N)$ при коллизиях (редко с хорошими реализациями).
\end{myblock}
\begin{myblock}{Память}
    $O(K)$, где K - количество хранимых уникальных ключей (или пар ключ-значение).
\end{myblock}

\subsection{Задачи из Списка}
\begin{myblock}{Примеры Задач LeetCode}
    \textbf{Easy:} \texttt{Two Sum} (*), \texttt{Jewels and Stones}, \texttt{Intersection of Two Arrays II}, \texttt{First Unique Character in a String}, \texttt{Missing Number} \sep
    \textbf{Medium:} \texttt{Group Anagrams} (*), \texttt{Subarray Sum Equals K} (*), \texttt{Insert Delete GetRandom O(1)} (*), \texttt{Longest Substring Without Repeating Characters} (*), \texttt{Continuous Subarray Sum}, \texttt{Line Reflection} (*)
\end{myblock}

% ================================================
% Секция 2: Two Pointers / Скользящее Окно (Sliding Window)
% ================================================
\section{Two Pointers / Скользящее Окно (Sliding Window)}

\subsection{Суть Техники}
\begin{myblock}{Академическое Описание}
    Техника использования двух (иногда больше) указателей, которые итерируются по структуре данных (массив, строка). Указатели могут двигаться навстречу, в одном направлении, или определять границы "окна".
\end{myblock}

\subsection{Простое Объяснение}
\begin{myblock}{Аналогия}
    Два "пальца", скользящие по данным. Идут навстречу (часто в отсортированных данных) или в одном направлении, определяя "окно".
\end{myblock}

\subsection{Распознавание Паттерна}
\begin{myblock}{Сигналы и Ключевые Слова}
    Работа с \emph{отсортированными} массивами. Поиск \emph{подмассивов/подстрок} с свойствами (макс/мин, уникальность). \emph{Последовательные} элементы, \emph{диапазоны}. Проверка \emph{палиндромов}. \emph{In-place} модификации. \newline
    Ключевые слова: "sorted array", "subarray", "substring", "consecutive", "palindrome", "window", "longest", "shortest", "at most K distinct", "in-place".
\end{myblock}

\subsection{Частые Комбинации}
\begin{myblock}{Комбинации с Другими Паттернами (в вашем списке)}
    \begin{itemize}[nosep, leftmargin=*]
        \item \textbf{С HashTable/Set:} В Sliding Window для отслеживания элементов окна (\texttt{Longest Substring...}, \texttt{Permutation in String}).
        \item \textbf{С Сортировкой:} Предварительный шаг для Two Pointers (\texttt{Two Sum II}).
        \item \textbf{С Linked Lists:} В задачах типа \texttt{Remove Nth Node From End of List}.
    \end{itemize}
\end{myblock}

\subsection{Пример Python Сниппета}
\begin{codebox}{python}{Two Pointers / Sliding Window}
# Two Pointers (с двух концов отсорт. массива)
def find_pair_sum(arr, target):
    left, right = 0, len(arr) - 1
    while left < right:
        current_sum = arr[left] + arr[right]
        if current_sum == target: return (left, right)
        elif current_sum < target: left += 1
        else: right -= 1
    return None
# Sliding Window (макс. сумма окна размера k)
def max_subarray_sum(arr, k):
    max_sum = float('-inf')
    current_sum = 0
    window_start = 0
    for window_end in range(len(arr)):
        current_sum += arr[window_end]
        if window_end >= k - 1:
            max_sum = max(max_sum, current_sum)
            current_sum -= arr[window_start]
            window_start += 1
    return max_sum
\end{codebox}

\subsection{Анализ Сложности}
\begin{myblock}{Время}
    Обычно $O(N)$. Каждый указатель проходит массив не более раза.
\end{myblock}
\begin{myblock}{Память}
    Часто $O(1)$ (in-place). Может быть $O(K)$ для Sliding Window с хранением состояния окна (K - размер алфавита/уникальных элементов).
\end{myblock}

\subsection{Задачи из Списка}
\begin{myblock}{Примеры Задач LeetCode}
    \textbf{Easy:} \texttt{Valid Palindrome} (*), \texttt{Move Zeroes} (*), \texttt{Merge Sorted Array}, \texttt{Is Subsequence}, \texttt{Squares of a Sorted Array}, \texttt{Remove Duplicates from Sorted Array}, \texttt{Two Sum II – Input Array Is Sorted}, \texttt{Reverse Words in a String III} \sep
    \textbf{Medium:} \texttt{Longest Subarray of 1’s After Deleting One Element} (*), \texttt{One Edit Distance} (*), \texttt{Permutation in String} (*), \texttt{Max Consecutive Ones II} (*), \texttt{Maximize Distance to Closest Person} (*), \texttt{Find All Anagrams in a String} (*), \texttt{Interval List Intersections}, \texttt{Max Consecutive Ones III}, \texttt{Longest Palindromic Substring} (Expand Around Center), \texttt{Longest Substring with At Most Two Distinct Characters} (*), \texttt{Partition Labels} (*), \texttt{Product of Array Except Self} (*), \texttt{Remove Nth Node From End of List} (*) \sep
    \textbf{Hard:} \texttt{Trapping Rain Water} (*) (Оптимальное O(N))
\end{myblock}

% ================================================
% Секция 3: Linked Lists / Связанные Списки
% ================================================
\section{Linked Lists / Связанные Списки}

\subsection{Суть Структуры Данных}
\begin{myblock}{Академическое Описание}
    Линейная структура: узлы хранят данные и ссылку \texttt{next}. В двусвязных - и ссылку \texttt{prev}. Доступ по индексу $O(N)$, вставка/удаление при известном узле $O(1)$.
\end{myblock}

\subsection{Простое Объяснение}
\begin{myblock}{Аналогия}
    Цепочка вагонов, каждый знает следующий. Легко вставить/расцепить, если стоишь у нужного вагона, но долго идти до 5-го вагона.
\end{myblock}

\subsection{Распознавание Паттерна}
\begin{myblock}{Сигналы и Ключевые Слова}
    Явное упоминание "Linked List", "Node", "next". Задачи на \emph{реверс}, \emph{циклы}, \emph{слияние}, \emph{удаление} узлов.
\end{myblock}

\subsection{Частые Комбинации}
\begin{myblock}{Комбинации с Другими Паттернами (в вашем списке)}
    \begin{itemize}[nosep, leftmargin=*]
        \item \textbf{С Two Pointers:} Поиск середины, N-го с конца (\texttt{Remove Nth Node...}), палиндром (\texttt{Palindrome Linked List}), цикл.
        \item \textbf{С HashTable:} В \texttt{LRU Cache}.
        \item \textbf{С Рекурсией:} Реверс, слияние, обход.
    \end{itemize}
\end{myblock}

\subsection{Пример Python Сниппета}
\begin{codebox}{python}{Обход / Реверс Списка}
class ListNode:
    def __init__(self, val=0, next=None):
        self.val = val
        self.next = next
def traverse_list(head):
    current = head
    while current:
        print(current.val)
        current = current.next
# Идея реверса (итеративно)
def reverse_list(head):
    prev = None
    current = head
    while current:
        next_node = current.next
        current.next = prev
        prev = current
        current = next_node
    return prev
\end{codebox}

\subsection{Анализ Сложности}
\begin{myblock}{Время}
    $O(N)$ для поиска/доступа по индексу/обхода. $O(1)$ для вставки/удаления в начале или при известном узле.
\end{myblock}
\begin{myblock}{Память}
    $O(N)$ для хранения узлов. Рекурсия может добавить $O(N)$ на стек. Итерация часто $O(1)$ доп. памяти.
\end{myblock}

\subsection{Задачи из Списка}
\begin{myblock}{Примеры Задач LeetCode}
    \textbf{Easy:} \texttt{Reverse Linked List} (*), \texttt{Merge Two Sorted Lists}, \texttt{Palindrome Linked List} \sep
    \textbf{Medium:} \texttt{Add Two Numbers}, \texttt{Remove Nth Node From End of List} (*), \texttt{LRU Cache} (*)
\end{myblock}

% ================================================
% Секция 4: Stack / Стек
% ================================================
\section{Stack / Стек}

\subsection{Суть Структуры Данных}
\begin{myblock}{Академическое Описание}
    Линейная структура данных, работающая по принципу LIFO (Last-In, First-Out). Операции: \texttt{push} (добавить), \texttt{pop} (удалить сверху).
\end{myblock}

\subsection{Простое Объяснение}
\begin{myblock}{Аналогия}
    Стопка книг: кладешь и берешь сверху. Последняя положенная берется первой.
\end{myblock}

\subsection{Распознавание Паттерна}
\begin{myblock}{Сигналы и Ключевые Слова}
    Обработка в \emph{обратном порядке}. Проверка \emph{сбалансированности} (\texttt{()[]\{\}}). Вычисление \emph{RPN}. Итеративный \emph{DFS}. \emph{Отмена} действия (\emph{backtracking}). Поиск \emph{след. большего/меньшего}. \newline
    Ключевые слова: "parentheses", "balanced", "RPN", "backtrack", "next greater element", "simplify path".
\end{myblock}

\subsection{Частые Комбинации}
\begin{myblock}{Комбинации с Другими Паттернами (в вашем списке)}
    \begin{itemize}[nosep, leftmargin=*]
        \item \textbf{С Массивами/Строками:} Обработка символов/чисел (\texttt{Valid Parentheses}, \texttt{Eval RPN}).
        \item \textbf{С Design:} Реализация очередей/стеков (\texttt{Implement Queue...}, \texttt{Max Stack}).
        \item \textbf{С Итераторами:} В \texttt{Flatten Nested List Iterator}.
        \item \textbf{С DP/Greedy:} Оптимизация (гистограммы в \texttt{Maximal Rectangle}).
    \end{itemize}
\end{myblock}

\subsection{Пример Python Сниппета}
\begin{codebox}{python}{Проверка Сбалансированности Скобок}
def is_valid_parentheses(s):
    stack = []
    mapping = {")": "(", "}": "{", "]": "["}
    for char in s:
        if char in mapping.values():
            stack.append(char)
        elif char in mapping.keys():
            if not stack or mapping[char] != stack.pop():
                return False
    return not stack
\end{codebox}

\subsection{Анализ Сложности}
\begin{myblock}{Время}
    $O(1)$ для \texttt{push}, \texttt{pop}, \texttt{peek}. Общее время алгоритма часто $O(N)$.
\end{myblock}
\begin{myblock}{Память}
    $O(N)$ в худшем случае (все элементы в стеке).
\end{myblock}

\subsection{Задачи из Списка}
\begin{myblock}{Примеры Задач LeetCode}
    \textbf{Easy:} \texttt{Valid Parentheses} (*), \texttt{Implement Queue using Stacks}, \texttt{Max Stack} (*) \sep
    \textbf{Medium:} \texttt{Evaluate Reverse Polish Notation}, \texttt{Simplify Path}, \texttt{Flatten Nested List Iterator} (*) \sep
    \textbf{Hard:} \texttt{Maximal Rectangle} (*)
\end{myblock}

% ================================================
% Секция 5: Trees / Деревья (BST, Binary Tree)
% ================================================
\section{Trees / Деревья (BST, Binary Tree)}

\subsection{Суть Структуры Данных}
\begin{myblock}{Академическое Описание}
    Иерархическая структура (узлы, ребра). Бинарное дерево: <= 2 детей. BST: левое < узел < правое. Обходы: BFS (по уровням), DFS (вглубь).
\end{myblock}

\subsection{Простое Объяснение}
\begin{myblock}{Аналогия}
    Семейное древо / структура папок. BST - отсортированный справочник (налево - меньше, направо - больше).
\end{myblock}

\subsection{Распознавание Паттерна}
\begin{myblock}{Сигналы и Ключевые Слова}
    Иерархия. Упоминание "Tree", "Node", "BST". \emph{Обход} (BFS, DFS). \emph{Поиск пути}. \emph{LCA}. \emph{Валидация} (BST, симметрия). \emph{Макс. глубина/сумма пути}.
\end{myblock}

\subsection{Частые Комбинации}
\begin{myblock}{Комбинации с Другими Паттернами (в вашем списке)}
    \begin{itemize}[nosep, leftmargin=*]
        \item \textbf{С Рекурсией:} DFS и многие задачи (\texttt{Validate BST}, \texttt{LCA}, \texttt{Max Path Sum}).
        \item \textbf{С Очередью (Queue):} Для BFS.
        \item \textbf{С Стеком (Stack):} Для итеративного DFS.
    \end{itemize}
\end{myblock}

\subsection{Пример Python Сниппета}
\begin{codebox}{python}{DFS (In-order) / BFS (Level-order)}
class TreeNode:
    def __init__(self, val=0, left=None, right=None):
        self.val = val; self.left = left; self.right = right
# DFS (In-order - рекурсивный)
def inorder_traversal(root):
    res = []; dfs(root, res); return res
def dfs(node, res):
    if not node: return
    dfs(node.left, res); res.append(node.val); dfs(node.right, res)
# BFS (Level-order - итеративный)
from collections import deque
def level_order(root):
    if not root: return []
    res, q = [], deque([root])
    while q:
        level = []
        # Используем \_ для экранирования в LaTeX
        for \_ in range(len(q)):
            node = q.popleft()
            level.append(node.val)
            if node.left: q.append(node.left)
            if node.right: q.append(node.right)
        res.append(level)
    return res
\end{codebox}

\subsection{Анализ Сложности}
\begin{myblock}{Время {(N узлов, H высота)}}
    $O(N)$ для обходов. В сбалансированном BST поиск/вставка/удаление $O(H) = O(\log N)$. В несбалансированном $O(N)$.
\end{myblock}
\begin{myblock}{Память {(N узлов, H высота, W макс. ширина)}}
    $O(N)$ для хранения дерева. BFS $O(W)$ (до $O(N)$). DFS (рекурсия) $O(H)$ (до $O(N)$). DFS (итерация) $O(H)$.
\end{myblock}

\subsection{Задачи из Списка}
\begin{myblock}{Примеры Задач LeetCode}
    \textbf{Easy:} \texttt{Symmetric Tree} (*), \texttt{Range Sum of BST} \sep
    \textbf{Medium:} \texttt{Validate Binary Search Tree} (*), \texttt{Lowest Common Ancestor of a Binary Tree} (*), \texttt{Lowest Common Ancestor of a Binary Tree III} (*) \sep
    \textbf{Hard:} \texttt{Binary Tree Maximum Path Sum} (*)
\end{myblock}

% ================================================
% Секция 6: Heap / Priority Queue / Куча
% ================================================
\section{Heap / Priority Queue / Куча}

\subsection{Суть Структуры Данных}
\begin{myblock}{Академическое Описание}
    Древовидная структура (бинарная куча) со свойством кучи: min-heap (узел <= потомки), max-heap (узел >= потомки).
\end{myblock}
\begin{myblock}{Операции}
    Эффективное добавление ($O(\log N)$) и извлечение min/max ($O(\log N)$). Просмотр min/max $O(1)$. Priority Queue часто реализуется кучей.
\end{myblock}

\subsection{Простое Объяснение}
\begin{myblock}{Аналогия}
    "Умная" очередь, всегда держит самый "важный" (min/max) элемент сверху. Быстро добавить или забрать самый важный.
\end{myblock}

\subsection{Распознавание Паттерна}
\begin{myblock}{Сигналы и Ключевые Слова}
    Постоянный доступ к \emph{наименьшему/наибольшему}. "Найти \emph{K-ый}" или "топ \emph{K}". \emph{Слияние K} списков. Задачи с \emph{приоритетами}. \newline
    Ключевые слова: "top K", "smallest/largest", "K-th element", "median" (иногда), "priority", "merge K sorted lists", "meeting rooms II".
\end{myblock}

\subsection{Частые Комбинации}
\begin{myblock}{Комбинации с Другими Паттернами (в вашем списке)}
    \begin{itemize}[nosep, leftmargin=*]
        \item \textbf{С Сортировкой:} В задачах на интервалы (\texttt{Meeting Rooms II}).
        \item \textbf{С Массивами/Списками:} Загрузка данных, слияние (\texttt{Merge k Sorted Lists}).
    \end{itemize}
\end{myblock}

\subsection{Пример Python Сниппета}
\begin{codebox}{python}{Min-Heap / Поиск K наибольших}
import heapq
# Min-Heap операции
min_heap = []
heapq.heappush(min_heap, 5)
heapq.heappush(min_heap, 1)
smallest = heapq.heappop(min_heap) # 1
# Найти K наибольших (min-heap размера K)
def find_k_largest(nums, k):
    min_heap = []
    for num in nums:
        if len(min_heap) < k:
            heapq.heappush(min_heap, num)
        elif num > min_heap[0]:
            heapq.heapreplace(min_heap, num)
    return list(min_heap)
\end{codebox}

\subsection{Анализ Сложности}
\begin{myblock}{Время (N элементов)}
    Вставка (\texttt{push}) $O(\log N)$. Извлечение (\texttt{pop}) $O(\log N)$. Просмотр (\texttt{heap[0]}) $O(1)$. Построение (\texttt{heapify}) $O(N)$.
\end{myblock}
\begin{myblock}{Память}
    $O(N)$ для хранения всех элементов. $O(K)$ если храним только K элементов.
\end{myblock}

\subsection{Задачи из Списка}
\begin{myblock}{Примеры Задач LeetCode}
    \textbf{Medium:} \texttt{Meeting Rooms II} (*) \sep
    \textbf{Hard:} \texttt{Merge k Sorted Lists} (*)
\end{myblock}

% ================================================
% Секция 7: Binary Search / Бинарный Поиск
% ================================================
\section{Binary Search / Бинарный Поиск}

\subsection{Суть Алгоритма}
\begin{myblock}{Академическое Описание}
    Алгоритм эффективного поиска элемента в \emph{отсортированном} массиве (или другой упорядоченной структуре). Работает путем многократного деления интервала поиска пополам.
\end{myblock}

\subsection{Простое Объяснение}
\begin{myblock}{Аналогия}
    Игра "угадай число" в отсортированном списке. На каждой попытке отбрасываешь половину оставшихся вариантов.
\end{myblock}

\subsection{Распознавание Паттерна}
\begin{myblock}{Сигналы и Ключевые Слова}
    Дан \emph{отсортированный} массив. Поиск элемента, \emph{границы}, первой/последней позиции. Поиск в \emph{повернутом (rotated)} отсортированном массиве. Задача сводится к поиску ответа \texttt{x} в монотонном пространстве решений (существует \texttt{check(x)}, монотонно меняющая T/F) - можно бинарно искать границу. \newline
    Ключевые слова: "sorted array", "find element", "search", "rotated array", "minimum/maximum in rotated", "median of sorted arrays".
\end{myblock}

\subsection{Частые Комбинации}
\begin{myblock}{Комбинации с Другими Паттернами (в вашем списке)}
    \begin{itemize}[nosep, leftmargin=*]
        \item \textbf{С Массивами:} Поиск в отсортированных/повернутых массивах (\texttt{Search in Rotated...}, \texttt{Find Minimum...}, \texttt{Median...}).
        \item \textbf{С Сортировкой:} Часто требует предварительной сортировки.
    \end{itemize}
\end{myblock}

\subsection{Пример Python Сниппета}
\begin{codebox}{python}{Классический Бинарный Поиск}
def binary_search(arr, target):
    left, right = 0, len(arr) - 1
    while left <= right:
        mid = left + (right - left) // 2
        if arr[mid] == target:
            return mid
        elif arr[mid] < target:
            left = mid + 1
        else:
            right = mid - 1
    return -1
\end{codebox}

\subsection{Анализ Сложности}
\begin{myblock}{Время}
    $O(\log N)$. На каждом шаге пространство поиска уменьшается вдвое.
\end{myblock}
\begin{myblock}{Память}
    $O(1)$ для итеративной реализации. $O(\log N)$ для рекурсивной (стек вызовов).
\end{myblock}

\subsection{Задачи из Списка}
\begin{myblock}{Примеры Задач LeetCode}
    \textbf{Medium:} \texttt{Search in Rotated Sorted Array} (*), \texttt{Find Minimum in Rotated Sorted Array} (*) \sep
    \textbf{Hard:} \texttt{Median of Two Sorted Arrays} (*)
\end{myblock}

% ================================================
% Секция 8: Graph / Графы (BFS/DFS)
% ================================================
\section{Graph / Графы (BFS/DFS)}

\subsection{Суть Структуры Данных и Обходов}
\begin{myblock}{Академическое Описание}
    Структура из вершин (nodes) и ребер (edges). Обход в ширину (BFS) исследует граф по уровням. Обход в глубину (DFS) идет вглубь по ветке до упора, затем возвращается.
\end{myblock}

\subsection{Простое Объяснение}
\begin{myblock}{Аналогия}
    Карта дорог (граф): города - вершины, дороги - ребра. BFS - волны на воде: исследуем сначала ближайших. DFS - блуждание по лабиринту: идем до тупика, возвращаемся.
\end{myblock}

\subsection{Распознавание Паттерна}
\begin{myblock}{Сигналы и Ключевые Слова}
    Задачи на \emph{связи}. Поиск \emph{пути}. \emph{Кратчайший путь} в \emph{невзвешенном} графе (BFS). Обход \emph{матрицы/сетки}. \emph{Циклы}. \emph{Связанные компоненты}. Топологическая сортировка (DFS). \newline
    Ключевые слова: "graph", "grid", "matrix", "connected", "path", "shortest path (unweighted)", "cycle", "neighbors", "dependencies", "islands".
\end{myblock}

\subsection{Частые Комбинации}
\begin{myblock}{Комбинации с Другими Паттернами (в вашем списке)}
    \begin{itemize}[nosep, leftmargin=*]
        \item \textbf{С Матрицами:} Граф часто представлен сеткой (\texttt{Number of Islands}).
        \item \textbf{С Очередью (Queue):} Для BFS.
        \item \textbf{С Стеком (Stack) / Рекурсией:} Для DFS.
        \item \textbf{С Множествами (Set):} Для отслеживания посещенных вершин.
    \end{itemize}
\end{myblock}

\subsection{Пример Python Сниппета}
\begin{codebox}{python}{BFS / DFS на Графе (представлен dict)}
from collections import deque
# BFS (поиск пути)
def bfs(graph, start, target):
    q = deque([(start, [start])]); visited = {start}
    while q:
        (v, path) = q.popleft()
        for neighbor in graph.get(v, []):
            if neighbor == target: return path + [neighbor]
            if neighbor not in visited:
                visited.add(neighbor)
                q.append((neighbor, path + [neighbor]))
    return None
# DFS (рекурсивный обход)
def dfs(graph, node, visited):
    visited.add(node); print(node) # Обработка
    for neighbor in graph.get(node, []):
        if neighbor not in visited:
            dfs(graph, neighbor, visited)
# visited_set = set(); dfs(my_graph, start_node, visited_set)
\end{codebox}

\subsection{Анализ Сложности}
\begin{myblock}{Время {(V вершин, E ребер)}}
    $O(V + E)$. Посещаем каждую вершину и ребро константное число раз.
\end{myblock}
\begin{myblock}{Память {(V вершин)}}
    $O(V)$ в худшем случае. Для BFS/DFS нужна очередь/стек и \texttt{visited}, хранящие до $O(V)$ элементов.
\end{myblock}

\subsection{Задачи из Списка}
\begin{myblock}{Примеры Задач LeetCode}
    \textbf{Medium:} \texttt{Number of Islands} (*) (BFS/DFS на сетке), \texttt{Perfect Squares} (BFS на графе состояний)
\end{myblock}

% ================================================
% Секция 9: Design / Проектирование Систем
% ================================================
\section{Design / Проектирование Систем}

\subsection{Суть Задач}
\begin{myblock}{Академическое Описание}
    Задачи, требующие спроектировать и реализовать класс/структуру данных с определенным API и ограничениями по сложности операций. Часто - комбинация стандартных структур.
\end{myblock}

\subsection{Простое Объяснение}
\begin{myblock}{Аналогия}
    Собрать механизм (кэш, итератор) из стандартных "деталей" (словари, списки) так, чтобы он работал быстро ($O(1)$ или $O(\log N)$).
\end{myblock}

\subsection{Распознавание Паттерна}
\begin{myblock}{Сигналы и Ключевые Слова}
    Прямая постановка: "Design...", "Implement...". Требования к \emph{сложности операций} (особенно $O(1)$). Реализация \emph{итераторов}, \emph{кэшей}, \emph{счетчиков}.
\end{myblock}

\subsection{Частые Комбинации}
\begin{myblock}{Комбинации с Другими Паттернами (в вашем списке)}
    Почти всегда комбинация других паттернов:
    \begin{itemize}[nosep, leftmargin=*]
        \item \textbf{С HashTable:} Для $O(1)$ доступа/поиска (\texttt{LRU Cache}, \texttt{InsertDeleteGetRandom}).
        \item \textbf{С Linked Lists:} Для $O(1)$ вставки/удаления/порядка (\texttt{LRU Cache}).
        \item \textbf{С Массивами:} Хранение, $O(1)$ доступ по индексу (\texttt{InsertDeleteGetRandom}).
        \item \textbf{С Стеком/Очередью:} Итераторы, счетчики, спец. стеки.
    \end{itemize}
\end{myblock}

\subsection{Пример Python Сниппета}
\begin{codebox}{python}{Пример: LRU Cache (Идея с OrderedDict)}
from collections import OrderedDict
class LRUCache:
    def __init__(self, capacity: int):
        self.cache = OrderedDict(); self.capacity = capacity
    def get(self, key: int) -> int:
        if key not in self.cache: return -1
        self.cache.move_to_end(key); return self.cache[key]
    def put(self, key: int, value: int) -> None:
        if key in self.cache:
            self.cache[key] = value; self.cache.move_to_end(key)
        else:
            if len(self.cache) >= self.capacity:
                self.cache.popitem(last=False)
            self.cache[key] = value
\end{codebox}

\subsection{Анализ Сложности}
\begin{myblock}{Временная и Пространственная Сложность}
    Анализируется для каждой операции (\texttt{get}, \texttt{put}, etc.) отдельно. Цель — удовлетворить ограничениям. Определяется сложностью базовых структур.
\end{myblock}

\subsection{Задачи из Списка}
\begin{myblock}{Примеры Задач LeetCode}
    \textbf{Easy:} \texttt{Number of Recent Calls} (*), \texttt{Max Stack} (*) \sep
    \textbf{Medium:} \texttt{Zigzag Iterator} (*), \texttt{Insert Delete GetRandom O(1)} (*), \texttt{LRU Cache} (*), \texttt{Design Hit Counter} (*), \texttt{Flatten Nested List Iterator} (*)
\end{myblock}

% ================================================
% Секция 10: Recursion / Backtracking / Динамическое Программирование (DP)
% ================================================
\section{Recursion / Backtracking / DP}

\subsection{Суть Подходов}
\begin{myblock}{Рекурсия}
    Функция вызывает сама себя для решения подзадачи меньшего размера. Требует базового случая.
\end{myblock}
\begin{myblock}{Backtracking}
    Систематический перебор кандидатов. Строит решение пошагово; если шаг неудачен, "откатывается" (backtracks).
\end{myblock}
\begin{myblock}{Динамическое Программирование (DP)}
    Решение через разбиение на \emph{перекрывающиеся подзадачи}. Решения подзадач сохраняются (мемоизация или табуляция).
\end{myblock}

\subsection{Простое Объяснение}
\begin{myblock}{Аналогии}
    \textbf{Рекурсия:} Матрешка. \sep
    \textbf{Backtracking:} Идти по лабиринту. \sep
    \textbf{DP:} Строить из Lego, повторно используя мелкие блоки.
\end{myblock}

\subsection{Распознавание Паттерна}
\begin{myblock}{Сигналы: Рекурсия/Backtracking}
    Структура задачи рекурсивна (деревья). Генерация \emph{всех} перестановок, комбинаций, подмножеств. Поиск \emph{всех} путей. Головоломки. \newline
    Ключевые слова: "generate all", "find all combinations/permutations/subsets".
\end{myblock}
\begin{myblock}{Сигналы: Динамическое Программирование}
    Найти \emph{оптимальное} (min/max, longest/shortest). Подсчитать \emph{количество способов}. Задача разбивается на \emph{пересекающиеся} подзадачи. \newline
    Ключевые слова: "minimum/maximum cost/path/value", "longest/shortest", "number of ways".
\end{myblock}

\subsection{Частые Комбинации}
\begin{myblock}{Комбинации с Другими Паттернами (в вашем списке)}
    \begin{itemize}[nosep, leftmargin=*]
        \item \textbf{С Деревьями:} Рекурсия - основной способ обхода (\texttt{Max Path Sum}).
        \item \textbf{С Массивами/Строками:} DP для последовательностей. Backtracking для комбинаций.
        \item \textbf{С Стеком:} Итеративный Backtracking.
        \item \textbf{С HashTable/Массивом:} Мемоизация в DP.
    \end{itemize}
\end{myblock}

\subsection{Пример Python Сниппета}
\begin{codebox}{python}{Рекурсия / Backtracking / DP (Мемоизация)}
# Рекурсия (Факториал)
def factorial(n):
    if n == 0: return 1; return n * factorial(n - 1)
# Backtracking (Подмножества)
def generate_subsets(nums):
    res, subset = [], []
    def backtrack(start):
        res.append(subset[:])
        for i in range(start, len(nums)):
            subset.append(nums[i]); backtrack(i + 1); subset.pop()
    backtrack(0); return res
# DP (Фибоначчи с мемоизацией)
memo = {}
def fib(n):
    if n in memo: return memo[n]
    if n <= 1: return n
    memo[n] = fib(n - 1) + fib(n - 2); return memo[n]
\end{codebox}

\subsection{Анализ Сложности}
\begin{myblock}{Время: Рекурсия/Backtracking}
    Часто экспоненциальное $O(c^N)$ или факториальное $O(N!)$. Зависит от ветвления и глубины.
\end{myblock}
\begin{myblock}{Время: DP}
    Обычно полиномиальное ($O(N)$, $O(N^2)$, $O(N \cdot M)$). (Кол-во подзадач) * (Время решения одной).
\end{myblock}
\begin{myblock}{Память: Рекурсия/Backtracking}
    $O(H)$, где H - макс. глубина рекурсии (часто $O(N)$ для стека вызовов).
\end{myblock}
\begin{myblock}{Память: DP}
    $O(S)$, где S - кол-во состояний (размер таблицы/мемо, часто $O(N)$ или $O(N \cdot M)$).
\end{myblock}

\subsection{Задачи из Списка}
\begin{myblock}{Примеры Задач LeetCode}
    \textbf{Medium:} \texttt{Generate Parentheses} (*) (Backtracking), \texttt{Perfect Squares} (DP или BFS) \sep
    \textbf{Hard:} \texttt{Binary Tree Maximum Path Sum} (*) (Рекурсия), \texttt{Maximal Rectangle} (*) (DP)
\end{myblock}

% ================================================
% Секция 11: Math / Geometry / Bit Manipulation / Other
% ================================================
\section{Math / Geometry / Bit Manipulation / Other}

\subsection{Суть Категории}
\begin{myblock}{Академическое Описание}
    Задачи, решение которых опирается на математические теоремы, формулы, геометрию, побитовые операции или специфические трюки, не укладывающиеся в стандартные паттерны.
\end{myblock}

\subsection{Простое Объяснение}
\begin{myblock}{Аналогия}
    Иногда нужна школьная математика, геометрия или хитрые трюки с числами/битами.
\end{myblock}

\subsection{Распознавание Паттерна}
\begin{myblock}{Сигналы и Ключевые Слова}
    Работа с \emph{координатами}, точками, линиями. \emph{Делимость}, простые числа. Манипуляции с \emph{битами}. \emph{Вероятность}. Прямая \emph{симуляция}. \emph{Сжатие строк}. \newline
    Ключевые слова: "coordinates", "geometry", "prime", "bits", "random", "probability", "simulate", "compress".
\end{myblock}

\subsection{Частые Комбинации}
\begin{myblock}{Комбинации с Другими Паттернами (в вашем списке)}
    \begin{itemize}[nosep, leftmargin=*]
        \item \textbf{С HashTable:} Хранение геом. объектов/промежут. результатов (\texttt{Line Reflection}).
        \item \textbf{С Массивами/Строками:} Основа для мат. манипуляций/симуляций (\texttt{String Compression}).
    \end{itemize}
\end{myblock}

\subsection{Пример Python Сниппета}
\begin{codebox}{python}{Проверка на степень двойки / Rand10() Идея}
# Проверка на степень двойки (битовая магия)
def is_power_of_two(n):
    return n > 0 and (n & (n - 1) == 0)
# Идея генерации Rand10 из Rand7 (Rejection Sampling)
# def rand10():
#     while True:
#         num = (rand7() - 1) * 7 + rand7() # 1..49
#         if num <= 40: return (num - 1) % 10 + 1
\end{codebox}

\subsection{Анализ Сложности}
\begin{myblock}{Временная и Пространственная Сложность}
    Сильно варьируется: от $O(1)$, $O(\log N)$ (мат./бит.) до $O(N)$, $O(N^2)$ (симуляции, геометрия). Анализируется индивидуально.
\end{myblock}

\subsection{Задачи из Списка}
\begin{myblock}{Примеры Задач LeetCode}
    \textbf{Easy:} \texttt{Summary Ranges} (*), \texttt{Consecutive Characters}, \texttt{Add Strings} \sep
    \textbf{Medium:} \texttt{Line Reflection} (*), \texttt{String Compression} (*), \texttt{Implement Rand10() Using Rand7()} (*)
\end{myblock}

% ================================================
% Секция 12: Sorting / Сортировка
% ================================================
\section{Sorting / Сортировка}

\subsection{Суть Процесса}
\begin{myblock}{Академическое Описание}
    Процесс упорядочивания элементов коллекции. Часто - \textbf{предварительный шаг} для других алгоритмов. Эффективные алгоритмы сравнения: $O(N \log N)$.
\end{myblock}

\subsection{Простое Объяснение}
\begin{myblock}{Аналогия}
    Привести данные в порядок (как слова в словаре), чтобы легче работать дальше.
\end{myblock}

\subsection{Распознавание Паттерна}
\begin{myblock}{Сигналы и Ключевые Слова}
    Явно требуется \emph{отсортированный} вывод. Применение алгоритма для \emph{отсортированных} данных (Binary Search, Two Pointers). Задачи с \emph{интервалами} (слияние, пересечения). \emph{Анаграммы} (сортировка строк).
\end{myblock}

\subsection{Частые Комбинации}
\begin{myblock}{Комбинации с Другими Паттернами (в вашем списке)}
    Является \textbf{предварительным шагом} для:
    \begin{itemize}[nosep, leftmargin=*]
        \item \textbf{Two Pointers:} Поиск пар, обработка.
        \item \textbf{Binary Search:} Обязательное условие.
        \item \textbf{Heap:} Иногда используется вместе (\texttt{Meeting Rooms II}).
        \item \textbf{Greedy:} Часто требует сортировки (\texttt{Merge Intervals}).
    \end{itemize}
\end{myblock}

\subsection{Пример Python Сниппета}
\begin{codebox}{python}{Встроенная Сортировка / Сортировка по Ключу}
# Сортировка на месте
my_list = [3, 1, 4, 1, 5, 9, 2, 6]
my_list.sort() # -> [1, 1, 2, 3, 4, 5, 6, 9]
# Создание нового отсортированного списка
my_tuple = (3, 1, 4, 1, 5)
sorted_list = sorted(my_tuple) # -> [1, 1, 3, 4, 5]
# Сортировка по ключу (по второму элементу)
data = [(1, 5), (3, 2), (2, 8)]
data.sort(key=lambda x: x[1]) # -> [(3, 2), (1, 5), (2, 8)]
\end{codebox}

\subsection{Анализ Сложности}
\begin{myblock}{Время}
    $O(N \log N)$ для эффективных алгоритмов сравнения (Merge Sort, Heap Sort, Quick Sort avg, Timsort). Counting/Radix Sort могут быть $O(N)$ при ограничениях.
\end{myblock}
\begin{myblock}{Память}
    $O(1)$ (in-place Heap Sort) до $O(N)$ (Merge Sort, Timsort). \texttt{sort()} стремится к O(N) в худшем, \texttt{sorted()} всегда $O(N)$ доп. памяти.
\end{myblock}

\subsection{Задачи из Списка}
\begin{myblock}{Примеры Задач LeetCode}
    \textbf{Medium:} \texttt{Merge Intervals} (*) (как первый шаг), \texttt{Meeting Rooms II} (*) (как первый шаг)
\end{myblock}

% --- End of Document ---

% >>> КОММЕНТАРИЙ: Печатаем библиографию (если используется \cite)
\AtNextBibliography{\footnotesize} % Уменьшаем шрифт для библиографии
%\printbibliography % Раскомментируйте, если добавляли цитаты через \cite или \footcite

\end{multicols}

\end{document}