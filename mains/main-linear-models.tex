\documentclass[10pt,a4paper]{article}

\usepackage[landscape,margin=0.7cm]{geometry}
\usepackage[russian,english]{babel} % Поддержка русского и английского

% >>> КОММЕНТАРИЙ: Определяем переменные для заголовка и автора (можно изменить)
\newcommand{\cheatsheettitle}{\color{w3schools}Шпаргалка по линейным моделям / {\color{alert} Концепции} {\color{black} Cheatsheet (XeLaTeX)}}
\newcommand{\cheatsheetauthor}{Краткий справочник}

% >>> КОММЕНТАРИЙ: Подключаем основной файл шаблона с настройками стилей и пакетов
% \documentclass{article} % ЗАКОММЕНТИРОВАНО: Добавлено для возможности компиляции этого фрагмента отдельно для проверки
\usepackage{fontspec}

% >>> КОММЕНТАРИЙ: Установка основных шрифтов документа (требует установленных шрифтов в системе)
\setmainfont{PT Sans} % Основной шрифт для текста
\setsansfont{PT Sans} % Шрифт без засечек (если используется отдельно)
\setmonofont{Liberation Mono} % Моноширинный шрифт (для кода)

\usepackage{fontawesome} % Для иконок (например, \faQuoteLeft)
\usepackage{hyperref} % Для создания кликабельных ссылок (в т.ч. в \resourcelink)
\usepackage{enumitem} % Для настройки списков
\usepackage{lipsum} % Для генерации текста-рыбы (не используется в финальной версии)
\usepackage{xcolor} % Для определения и использования цветов
\usepackage{minted} % Для подсветки синтаксиса кода (требует Python и Pygments, компиляция с -shell-escape)

% >>>>>> НОВОЕ: Пакет для создания графиков <<<<<<
\usepackage{pgfplots}
\pgfplotsset{compat=newest} % Устанавливаем последнюю версию совместимости
\usetikzlibrary{arrows.meta, shapes.geometric} % Загружаем полезные библиотеки TikZ

\usepackage{titlesec} % Для настройки заголовков секций
% >>> КОММЕНТАРИЙ: Настройка отступов для заголовков секций (\section)
\titlespacing*{\section}{0pt}{*1.5}{*0.8} % {отступ слева}{отступ сверху}{отступ снизу}

% >>> КОММЕНТАРИЙ: Форматирование заголовков секций (\section)
\titleformat{\section}
  {\normalfont\large\bfseries\color{mDarkTeal}} % Стиль заголовка (цвет, размер, жирность)
  {\thesection} % Номер секции
  {1em} % Отступ после номера
  {} % Код перед заголовком (здесь пусто)

% >>> КОММЕНТАРИЙ: Определение пользовательских цветов через HTML-коды
\definecolor{customcolor}{HTML}{696EA8} % Основной цвет для блоков textbox
\definecolor{alert}{HTML}{CD5C5C} % Цвет для выделения (красный)
\definecolor{w3schools}{HTML}{4CAF50} % Цвет для выделения (зеленый)
\definecolor{subbox}{gray}{0.60} % Цвет для под-блоков (не используется?)
\definecolor{codecolor}{HTML}{FFC300} % Цвет для кода (не используется?)
\definecolor{mDarkTeal}{HTML}{23373b} % Темно-бирюзовый (для заголовков секций, названий блоков)
\colorlet{beamerboxbg}{black!2} % Фон для beamerbox (не используется?)
\colorlet{beamerboxfg}{mDarkTeal} % Цвет текста для beamerbox (не используется?)
\definecolor{mDarkBrown}{HTML}{604c38} % Темно-коричневый (не используется?)
\definecolor{mLightBrown}{HTML}{EB811B} % Светло-коричневый (для заголовков myalertblock)
\definecolor{mLightGreen}{HTML}{14B03D} % Светло-зеленый (для заголовков myexampleblock)
\definecolor{alertTextBox}{HTML}{A8696E} % Цвет для блоков alerttextbox

% >>> КОММЕНТАРИЙ: Настройка пакета biblatex для библиографии
\usepackage
[citestyle=authoryear, % Стиль цитирования (Автор-Год)
sorting=nty, % Сортировка по Имени, Году, Названию
autocite=footnote, % Команда \autocite создает сноски
autolang=hyphen, % Автоматическое определение языка для переносов
mincrossrefs=1, % Минимальное количество перекрестных ссылок
backend=biber] % Используемый бэкенд (требует запуск biber)
{biblatex}

% >>> КОММЕНТАРИЙ: Настройка формата пост-заметок в цитатах (например, номера страниц)
\DeclareFieldFormat{postnote}{#1}
\DeclareFieldFormat{multipostnote}{#1}
\DeclareAutoCiteCommand{footnote}[f]{\footcite}{\footcites} % Настройка команды \autocite

% >>> КОММЕНТАРИЙ: Подключение файла с библиографическими записями
\addbibresource{literature.bib}

% >>> КОММЕНТАРИЙ: Подключение библиотеки tcolorbox для создания цветных блоков
\usepackage{tcolorbox}
\tcbuselibrary{most, listingsutf8, minted} % Загрузка необходимых библиотек tcolorbox (most включает breakable)

% >>> КОММЕНТАРИЙ: Глобальные настройки для маленьких tcbox (inline блоки)
\tcbset{tcbox width=auto,left=1mm,top=1mm,bottom=1mm,
right=1mm,boxsep=1mm,middle=1pt}

% >>> КОММЕНТАРИЙ: Определение базового окружения для цветных блоков с заголовком
% breakable: позволяет блоку разрываться между колонками/страницами.
% ВАЖНО: В multicols автоматический разрыв breakable может работать не всегда идеально.
% Если блок разрывается некрасиво, попробуйте разбить контент на несколько блоков
% или используйте \needspace{<длина>} перед блоком.
\newenvironment{mycolorbox}[2]{
\begin{tcolorbox}[capture=minipage,fonttitle=\large\bfseries, enhanced,boxsep=1mm,colback=#1!30!white,
width=\linewidth, % Ширина блока равна ширине текущей колонки
arc=2pt,outer arc=2pt, toptitle=0mm,colframe=#1,opacityback=0.7,nobeforeafter,
breakable, % Разрешить разрыв блока
title=#2] % Текст заголовка блока
}{\end{tcolorbox}}

% >>> КОММЕНТАРИЙ: Окружение subbox (не используется в текущем коде)
\newenvironment{subbox}[2]{
\begin{tcolorbox}[capture=minipage,fonttitle=\normalsize\bfseries, enhanced,boxsep=1mm,colback=#1!30!white,on line,tcbox width=auto,left=0.3em,top=1mm, toptitle=0mm,colframe=#1,opacityback=0.7,nobeforeafter,
breakable,
title=#2]\footnotesize
}{\normalsize\end{tcolorbox}\vspace{0.1em}}

% >>> КОММЕНТАРИЙ: Окружение для размещения tcolorbox'ов в несколько колонок внутри блока (не используется)
\newenvironment{multibox}[1]{
\begin{tcbraster}[raster columns=#1,raster equal height,nobeforeafter,raster column skip=1em,raster left skip=1em,raster right skip=1em]}{\end{tcbraster}}

% >>> КОММЕНТАРИЙ: Окружения-обертки для mycolorbox с предопределенными цветами и заголовками
\newenvironment{textbox}[1]{\begin{mycolorbox}{customcolor}{#1}}{\end{mycolorbox}} % Обычный текстовый блок
\newenvironment{alerttextbox}[1]{\begin{mycolorbox}{alertTextBox}{#1}}{\end{mycolorbox}} % Блок для предупреждений/важной информации

% >>> КОММЕНТАРИЙ: Определение окружения для блоков с кодом с подсветкой minted
\newtcblisting{codebox}[3][]{ % [опции minted], {язык}, {заголовок}
colback=black!10,colframe=black!20, % Цвета фона и рамки
listing only, % Показывать только листинг кода
minted options={ % Опции, передаваемые в minted
    numbers=left, % Нумерация строк слева
    style=default, % Стиль подсветки (можно выбрать другие, напр., 'friendly')
    fontsize=\scriptsize, % Размер шрифта кода
    breaklines, % Автоматический перенос длинных строк
    breaksymbolleft=, % Символ для обозначения переноса (пусто)
    autogobble, % Удаление лишних отступов слева
    linenos, % Показывать номера строк
    numbersep=0.7em, % Отступ номеров строк от кода
    #1 % Дополнительные опции minted, переданные в [1]
    },
enhanced,top=1mm, toptitle=0mm,
left=5mm, % Отступ слева внутри блока (для номеров строк)
arc=0pt,outer arc=0pt, % Прямые углы
title={#3}, % Заголовок блока кода
fonttitle=\small\bfseries\color{mDarkTeal}, % Стиль заголовка блока кода
listing engine=minted, % Движок для листинга - minted
minted language=#2, % Язык программирования для подсветки
breakable % Разрешить разрыв блока кода
}

\newcommand{\punkti}{~\lbrack\dots\rbrack~} % Команда для [...] (не используется)

% >>> КОММЕНТАРИЙ: Переопределение окружения quote для добавления иконок цитат
\renewenvironment{quote}
               {\list{\faQuoteLeft\phantom{ }}{\rightmargin\leftmargin}
                \item\relax\scriptsize\ignorespaces}
               {\unskip\unskip\phantom{xx}\faQuoteRight\endlist}

% >>> КОММЕНТАРИЙ: Вспомогательные команды для создания цветных фонов под текстом (не используются)
\newcommand{\bgupper}[3]{\colorbox{#1}{\color{#2}\huge\bfseries\MakeUppercase{#3}}}
\newcommand{\bg}[3]{\colorbox{#1}{\bfseries\color{#2}#3}}

% >>> КОММЕНТАРИЙ: Команда для форматирования описания команды/функции
% #1: Сама команда (используется \detokenize для корректного отображения спецсимволов)
% #2: Описание команды
\newcommand{\mycommand}[2]{
  {\par\noindent\ttfamily\detokenize{#1}\par} % Вывод команды моноширинным шрифтом
  \nopagebreak % Стараемся не разрывать страницу после команды
  \hangindent=1.5em \hangafter=1 \noindent % Делаем отступ для описания
  \small\textit{#2}\par\vspace{0.5ex} % Вывод описания и небольшой отступ после
}

% >>> КОММЕНТАРИЙ: Команда для форматирования ссылок на ресурсы
% #1: URL ссылки
% #2: Текст ссылки
% #3: Описание ресурса
\newcommand{\resourcelink}[3]{
  {\par\noindent\href{#1}{\ttfamily #2}\par} % Вывод кликабельной ссылки
  \nopagebreak
  \hangindent=1.5em \hangafter=1 \noindent
  \small\textit{#3}\par\vspace{0.5ex} % Вывод описания
}

% >>> КОММЕНТАРИЙ: Вспомогательные команды для стилизации
\newcommand{\sep}{{\scriptsize~\faCircle{ }~}} % Разделитель-кружок
\newcommand{\bggreen}[1]{\medskip\bgupper{w3schools}{black}{#1}\\[0.5em]} % Зеленый заголовок (не используется)
\newcommand{\green}[1]{\smallskip\bg{w3schools}{white}{#1}\\} % Текст на зеленом фоне
\newcommand{\red}[1]{\smallskip\bg{alert}{white}{#1}\\} % Текст на красном фоне
\newcommand{\alertcmd}[1]{\red{#1}\\} % Псевдоним для red (переименовано чтобы не конфликтовать с цветом alert)

\usepackage{multicol} % Пакет для создания нескольких колонок
\setlength{\columnsep}{15pt} % Расстояние между колонками

\setlength{\parindent}{0pt} % Убираем абзацный отступ по умолчанию
\usepackage{csquotes} % Для правильного отображения кавычек в разных языках (используется biblatex)
\newcommand{\loremipsum}{Lorem ipsum dolor sit amet.} % Пример текста (не используется)

% >>> КОММЕНТАРИЙ: Окружения для блоков с разными цветами заголовков (пример, предупреждение, обычный)
% Основаны на tcolorbox, похожи на mycolorbox, но с другими цветами заголовков и рамки.
% Также используют breakable. Применяются для выделения примеров, предупреждений и т.д.
% ВАЖНО: Лог может содержать предупреждения "Underfull \hbox (badness 10000)".
% Это часто случается в узких колонках multicol при наличии кода, URL или текста,
% который плохо переносится. Обычно это лишь косметическая проблема, строки выглядят
% немного не до конца заполненными. Можно игнорировать или попробовать перефразировать/
% использовать \sloppy перед проблемным абзацем.
\newenvironment{myexampleblock}[1]{ % Блок для примеров (зеленый заголовок)
    \tcolorbox[capture=minipage,fonttitle=\small\bfseries\color{mLightGreen}, enhanced,boxsep=1mm,colback=black!10,breakable,noparskip,
    on line,tcbox width=auto,left=0.3em,top=1mm, toptitle=0mm,
    colframe=black!20,arc=0pt,outer arc=0pt,
    opacityback=0.7,nobeforeafter,title=#1]}
    {\endtcolorbox}

\newenvironment{myalertblock}[1]{ % Блок для предупреждений (оранжевый заголовок)
    \tcolorbox[capture=minipage,fonttitle=\small\bfseries\color{mLightBrown}, enhanced,boxsep=1mm,colback=black!10,breakable,noparskip,
    on line,tcbox width=auto,left=0.3em,top=1mm, toptitle=0mm,
    colframe=black!20,arc=0pt,outer arc=0pt,
    opacityback=0.7,nobeforeafter,title=#1]}
    {\endtcolorbox}

\newenvironment{myblock}[1]{ % Обычный информационный блок (бирюзовый заголовок)
    \tcolorbox[capture=minipage,fonttitle=\small\bfseries\color{mDarkTeal}, enhanced,boxsep=1mm,colback=black!10,breakable,noparskip,
    on line,tcbox width=auto,left=0.3em,top=1mm, toptitle=0mm,
    colframe=black!20, arc=0pt,outer arc=0pt,
    opacityback=0.7,nobeforeafter,title=#1]}
    {\endtcolorbox}

% >>> КОММЕНТАРИЙ: Команда для вставки изображения в рамке tcolorbox с подписью
% #1: Опции (не используются)
% #2: Путь к файлу изображения (используется как подпись под изображением)
\newcommand{\mygraphics}[2][]{
\tcbox[enhanced,boxsep=0pt,top=0pt,bottom=0pt,left=0pt,
right=0pt,boxrule=0.4pt,drop fuzzy shadow,clip upper,
colback=black!75!white,toptitle=2pt,bottomtitle=2pt,nobeforeafter,
center title,fonttitle=\small\sffamily,title=\detokenize{#2}] % Используем путь как заголовок
{\includegraphics[width=\the\dimexpr(\linewidth-4mm)\relax]{#2}} % Вставляем изображение, ширина чуть меньше колонки
}

% >>> КОМЕНТАРИЙ: Команда \needspace{<длина>}
% Используйте эту команду *перед* блоком (tcolorbox, section, figure), чтобы убедиться,
% что в текущей колонке есть как минимум <длина> свободного места. Если места нет,
% LaTeX начнет новую колонку/страницу перед выполнением команды.
% Пример: \needspace{10\baselineskip} % Запросить место для примерно 10 строк текста
% Это полезно для предотвращения "висячих" заголовков или некрасивых разрывов блоков.

% >>>>>> НОВОЕ: Стили для графиков PGFPlots <<<<<<
\pgfplotsset{
    % Базовый стиль для всех графиков в шпаргалке
    cheatsheet plot style/.style={
        width=0.98\linewidth, % Ширина чуть меньше колонки для аккуратности
        height=5cm,          % Высота графика (можно изменить по необходимости)
        grid=major,          % Включаем основную сетку
        grid style={dashed, color=black!20}, % Стиль сетки: пунктирная, светло-серая
        axis lines=left,     % Линии осей только слева и снизу
        axis line style={color=black!60, thick}, % Стиль линий осей: серые, потолще
        tick style={color=black!60, thick}, % Стиль меток на осях
        ticklabel style={font=\scriptsize, color=black}, % Стиль подписей меток: маленький шрифт
        label style={font=\small, color=mDarkTeal}, % Стиль подписей осей: шрифт small, цвет как у заголовков блоков
        title style={font=\small\bfseries, color=mDarkTeal}, % Стиль заголовка графика
        legend style={        % Стиль легенды
            font=\scriptsize, % Маленький шрифт
            draw=black!30,    % Легкая рамка вокруг легенды
            fill=white,       % Белый фон
            legend cell align=left, % Выравнивание текста в ячейке легенды
            anchor=north east, % Якорь для позиционирования
            at={(rel axis cs:0.98,0.98)} % Положение в правом верхнем углу внутри области графика
        },
        cycle list={ % Список стилей для линий \addplot (цвета взяты из шаблона)
            {blue, mark=*, thick},
            {alert, mark=square, thick},
            {w3schools, mark=triangle, thick},
            {mLightBrown, mark=diamond, thick},
            {customcolor, mark=oplus, thick},
            {black!50, mark=pentagon, thick},
        },
        % Убираем рамку вокруг графика по умолчанию
        axis background/.style={fill=none},
    },
    % Дополнительный стиль для ROC-кривой (без сетки, с диагональю)
    roc curve style/.style={
        cheatsheet plot style, % Наследуем базовый стиль
        width=6cm, height=6cm, % Делаем квадратным
        grid=none, % Убираем сетку
        xlabel={False Positive Rate (FPR)},
        ylabel={True Positive Rate (TPR)},
        xmin=0, xmax=1,
        ymin=0, ymax=1,
        xtick={0, 0.5, 1},
        ytick={0, 0.5, 1},
        legend pos=south east, % Легенда внизу справа
        % Добавляем диагональную линию случайного угадывания
        extra x ticks={0.5}, extra y ticks={0.5},
        extra tick style={grid=major, grid style={dashed, color=black!40}},
        after end axis/.code={ % Добавляем линию после отрисовки осей
            \draw[dashed, color=black!40] (axis cs:0,0) -- (axis cs:1,1);
        }
    },
    % Дополнительный стиль для PR-кривой
    pr curve style/.style={
        cheatsheet plot style, % Наследуем базовый стиль
        xlabel={Recall (TPR)},
        ylabel={Precision},
        xmin=0, xmax=1,
        ymin=0, ymax=1,
        legend pos=south west, % Легенда внизу слева
    }
}

% % Добавлено для возможности компиляции этого фрагмента отдельно для проверки
% \begin{document}
% \begin{multicols}{2} % Пример использования в multicols
% \lipsum[1] % Просто текст для заполнения

% \begin{textbox}{Пример графика (Базовый стиль)}
%     \begin{center}
%     \begin{tikzpicture}
%         \begin{axis}[cheatsheet plot style, title={Пример $y=x^2$ и $y=x+1$}]
%         \addplot coordinates {(0, 0) (1, 1) (2, 4) (3, 9) (4, 16)};
%         \addlegendentry{$y=x^2$}
%         \addplot coordinates {(0, 1) (1, 2) (2, 3) (3, 4) (4, 5)};
%         \addlegendentry{$y=x+1$}
%         \end{axis}
%     \end{tikzpicture}
%     \end{center}
% \end{textbox}

% \begin{myexampleblock}{Пример ROC-кривой}
%     \begin{center}
%     \begin{tikzpicture}
%         \begin{axis}[roc curve style, title={Пример ROC Curve}]
%         % Пример данных ROC кривой
%         \addplot coordinates { (0,0) (0.1,0.3) (0.2,0.6) (0.4,0.8) (0.6,0.9) (0.8,0.95) (1,1) };
%         \addlegendentry{Модель A (AUC $\approx$ 0.8)}
%         \addplot coordinates { (0,0) (0.2,0.2) (0.4,0.4) (0.6,0.6) (0.8,0.8) (1,1) }; % Пример хуже
%         \addlegendentry{Модель B (AUC = 0.5)}
%         \end{axis}
%     \end{tikzpicture}
%     \end{center}
% \end{myexampleblock}

% \begin{myblock}{Пример PR-кривой}
%     \begin{center}
%     \begin{tikzpicture}
%         \begin{axis}[pr curve style, title={Пример Precision-Recall Curve}]
%         % Пример данных PR кривой (могут сильно зависеть от порога)
%         \addplot coordinates { (0,0.9) (0.1,0.85) (0.3,0.8) (0.5,0.7) (0.7,0.6) (0.9,0.4) (1,0.3) };
%         \addlegendentry{Модель C}
%         \end{axis}
%     \end{tikzpicture}
%     \end{center}
% \end{myblock}

% \lipsum[2-3] % Просто текст для заполнения
% \end{multicols}
% \end{document}

\begin{document}
\pagestyle{empty} % Убираем номера страниц
\small % Уменьшаем базовый размер шрифта для всего документа

% >>> КОММЕНТАРИЙ: Используем multicols для создания трех колонок.
\begin{multicols}{3}
\raggedcolumns % Не растягивать колонки по вертикали

\noindent
\begin{minipage}{\linewidth}
    \centering
    {\bfseries\huge \cheatsheettitle \par}
    \vspace{1ex}
    {\large \cheatsheetauthor \par}
    \vspace{0.5ex}
    {\normalsize \today \par} % Вставляем текущую дату
\end{minipage}
\vspace{2ex}

\thispagestyle{empty} % Убеждаемся, что и на первой странице нет номера

\scriptsize % Уменьшаем шрифт для оглавления и основного текста
\tableofcontents % Генерируем оглавление

% >>> КОММЕНТАРИЙ: Подключаем основной контент шпаргалки
% >>> Контент для шпаргалки 3: Линейные Модели

% --- Раздел III.A: Линейная Регрессия ---
\section{Линейная Регрессия (\texttt{Linear Regression})}

\begin{textbox}{Основная Идея}
Простейшая модель для предсказания \textbf{непрерывного} значения (например, цены дома, температуры). Мы пытаемся найти наилучшую прямую (или гиперплоскость в многомерном случае), которая описывает зависимость между признаками ($X$) и целевой переменной ($y$).

Модель: $y \approx \hat{y} = w_0 + w_1 x_1 + w_2 x_2 + \dots + w_n x_n = \mathbf{w}^T \mathbf{x}$
Где $\mathbf{w}$ - вектор весов (параметров) модели, включая свободный член $w_0$ (bias term), $\mathbf{x}$ - вектор признаков объекта (с добавленным $x_0=1$).
\textbf{Интерпретация весов:} Коэффициент $w_j$ показывает, на сколько в среднем изменится предсказание $\hat{y}$, если признак $x_j$ увеличить на 1, при условии, что все остальные признаки остаются неизменными.
\end{textbox}

\begin{myblock}{Функция Потерь: MSE (\texttt{Mean Squared Error})}
Чтобы понять, насколько хорошо наша прямая подходит к данным, мы измеряем ошибку. Самый частый способ - \textbf{Среднеквадратичная Ошибка (MSE)}. Мы суммируем квадраты разностей между реальными значениями ($y_i$) и предсказаниями модели ($\hat{y}_i$) и делим на количество примеров ($m$).
\[
MSE = \frac{1}{m} \sum_{i=1}^{m} (y_i - \hat{y}_i)^2 = \frac{1}{m} \sum_{i=1}^{m} (y_i - \mathbf{w}^T \mathbf{x}_i)^2
\]
\textbf{Почему квадрат?} Он штрафует большие ошибки сильнее и делает функцию потерь дифференцируемой.
\end{myblock}

\begin{myexampleblock}{Обучение: Идея Градиентного Спуска (GD)}
Представьте, что MSE - это холмистая местность, а мы стоим где-то на склоне. Наша цель - найти самую низкую точку (минимум MSE). \textbf{Градиентный спуск (GD)} - это как катиться с горы:
\begin{itemize}
    \item Смотрим, в каком направлении уклон самый крутой (\textbf{градиент} $\nabla MSE$).
    \item Делаем небольшой шаг в \textbf{противоположном} направлении (анти-градиент). Длина шага контролируется \textbf{скоростью обучения} (\textit{learning rate}, $\alpha$).
    \item Повторяем, пока не дойдем до дна (или почти).
\end{itemize}
Формула обновления весов: $\mathbf{w} := \mathbf{w} - \alpha \nabla_{\mathbf{w}} MSE(\mathbf{w})$

\textbf{Варианты GD:}
\begin{itemize}
    \item \textbf{Batch GD}: Градиент считается по \textbf{всей} обучающей выборке на каждом шаге. Точно, но медленно на больших данных.
    \item \textbf{Stochastic GD (SGD)}: Градиент считается по \textbf{одному} случайно выбранному примеру на каждом шаге. Быстро, шумно (может "прыгать" вокруг минимума), хорошо для очень больших данных и онлайн-обучения.
    \item \textbf{Mini-batch GD}: Компромисс. Градиент считается по небольшой \textbf{случайной подвыборке} (\textit{batch}) на каждом шаге. Сочетает преимущества Batch и SGD. Самый популярный вариант.
\end{itemize}
\end{myexampleblock}

\begin{textbox}{Аналитическое Решение (Normal Equation)}
Для линейной регрессии с MSE существует \textbf{точное аналитическое решение}, позволяющее найти оптимальные веса $\mathbf{w}$ без итераций GD.
\[
\mathbf{w}^* = (\mathbf{X}^T \mathbf{X})^{-1} \mathbf{X}^T \mathbf{y}
\]
Где $\mathbf{X}$ - матрица признаков (объекты по строкам, признаки по столбцам, с добавленным столбцом единиц), $\mathbf{y}$ - вектор целевых значений.

\textbf{Плюсы:} Точно, не требует подбора learning rate.
\textbf{Минусы:} Требует вычисления обратной матрицы $(\mathbf{X}^T \mathbf{X})^{-1}$, что вычислительно сложно ($O(n^3)$, где $n$ - число признаков) и может быть невозможно, если матрица вырождена (признаки линейно зависимы). Непрактично при очень большом числе признаков.
\end{textbox}

\begin{alerttextbox}{Основные Предположения Линейной Регрессии}
Модель работает лучше всего, когда выполняются некоторые предположения (на собеседовании важно знать хотя бы названия):
\begin{itemize}
    \item \textbf{Линейность:} Средняя зависимость $y$ от $X$ является линейной.
    \item \textbf{Независимость ошибок:} Ошибки предсказаний для разных наблюдений независимы.
    \item \textbf{Гомоскедастичность:} Разброс (варианция) ошибок одинаков для всех значений $X$. \textit{Аналогия: толщина "облака" точек вокруг линии регрессии примерно одинакова по всей длине.}
    \item \textbf{Нормальность ошибок:} Ошибки распределены нормально (важно для построения доверительных интервалов).
\end{itemize}
Нарушение предположений не всегда делает модель бесполезной, но может влиять на надежность выводов.
\end{alerttextbox}

\begin{myblock}{Полиномиальная Регрессия}
Что если зависимость нелинейная? Можно добавить \textbf{полиномиальные признаки} - степени существующих признаков ($x^2, x^3$) или их взаимодействия ($x_1 x_2$).
Пример: $y \approx w_0 + w_1 x + w_2 x^2$.
\textbf{Важно:} Модель все еще остается \textbf{линейной по параметрам} $\mathbf{w}$! Мы просто применяем линейную регрессию к расширенному набору признаков ($1, x, x^2$). Легко переобучается, требует регуляризации.
\end{myblock}

% --- Раздел III.C: Логистическая Регрессия ---
\section{Логистическая Регрессия (\texttt{Logistic Regression})}

\begin{textbox}{Основная Идея}
Используется для задач \textbf{бинарной классификации} (ответ 0 или 1, "да" или "нет"). Вместо прямого предсказания класса, она моделирует \textbf{вероятность} принадлежности объекта к классу 1.
\textit{Аналогия: предсказать не "сдал/не сдал" экзамен, а вероятность сдачи в зависимости от часов подготовки.}
\end{textbox}

\begin{myblock}{Сигмоида (Логистическая Функция)}
Линейная комбинация признаков ($\mathbf{w}^T \mathbf{x}$) может дать любое вещественное значение. Чтобы получить вероятность (от 0 до 1), результат пропускают через \textbf{сигмоидную функцию} $\sigma(z)$:
\[
\sigma(z) = \frac{1}{1 + e^{-z}} \quad \text{где } z = \mathbf{w}^T \mathbf{x}
\]
Предсказание модели: $\hat{p} = P(y=1 | \mathbf{x}) = \sigma(\mathbf{w}^T \mathbf{x})$.
Решение о классе принимается по порогу (обычно 0.5): если $\hat{p} \ge 0.5$, то класс 1, иначе класс 0.
\end{myblock}

\begin{myblock}{Функция Потерь: LogLoss (Логарифмическая Функция Потерь)}
MSE плохо подходит для вероятностей. Используется \textbf{LogLoss} (или Бинарная Кросс-Энтропия).
\[
LogLoss = -\frac{1}{m} \sum_{i=1}^{m} [ y_i \log(\hat{p}_i) + (1 - y_i) \log(1 - \hat{p}_i) ]
\]
\textbf{Идея:} Сильно штрафует модель, если она уверенно предсказывает неверный класс (например, дает $\hat{p}=0.99$, когда реальный класс $y=0$). Когда предсказание правильное, штраф маленький. Обучается также с помощью GD.
\end{myblock}

\begin{textbox}{Softmax (для Мультиклассовой Классификации)}
Если классов больше двух, используется обобщение логистической регрессии - \textbf{Softmax Regression}. Для каждого класса $k$ вычисляется своя "оценка" $z_k = \mathbf{w}_k^T \mathbf{x}$. Затем эти оценки преобразуются в вероятности с помощью \textbf{функции Softmax}:
\[
P(y=k | \mathbf{x}) = \hat{p}_k = \frac{e^{z_k}}{\sum_{j=1}^{K} e^{z_j}} \quad \text{для } k = 1, \dots, K
\]
Softmax гарантирует, что все $\hat{p}_k$ будут от 0 до 1 и их сумма будет равна 1. Функция потерь - \textbf{Cross-Entropy Loss}.
\end{textbox}

% --- Раздел III.B: Регуляризация ---
\section{Регуляризация L1 и L2}

\begin{alerttextbox}{Что и Зачем?}
\textbf{Регуляризация} - это техника борьбы с \textbf{переобучением} (\textit{overfitting}) линейных (и не только) моделей. Переобучение происходит, когда модель слишком сложная и "запоминает" обучающие данные вместо того, чтобы выучить общие закономерности.
\textbf{Идея:} Добавить к основной функции потерь (MSE или LogLoss) \textbf{штраф} за большие значения весов $\mathbf{w}$.
\textit{Аналогия: Мы не просто просим модель хорошо описать данные (минимизировать MSE/LogLoss), но и говорим ей: "Будь проще! Не усложняй без необходимости!" (штрафуем за большие веса).}
\[
J_{reg}(\mathbf{w}) = J_{original}(\mathbf{w}) + \lambda \cdot R(\mathbf{w})
\]
Где $J_{original}$ - MSE или LogLoss, $R(\mathbf{w})$ - регуляризационный член, $\lambda$ (\textit{lambda}) - \textbf{коэффициент регуляризации} (гиперпараметр), контролирующий силу штрафа.
\textbf{Важно:} Перед применением регуляризации признаки обычно \textbf{масштабируют} (стандартизируют или нормализуют), чтобы штраф не зависел от исходного масштаба признаков.
\end{alerttextbox}

\begin{myblock}{L2 Регуляризация (Ridge / Гребневая)}
Штраф пропорционален \textbf{сумме квадратов} весов.
\[
R_{L2}(\mathbf{w}) = \sum_{j=1}^{n} w_j^2
\]
\textbf{Примечание:} Свободный член $w_0$ обычно \textbf{не регуляризуют}, т.к. он отвечает за общее смещение модели, а не за ее сложность взаимодействия с признаками.
\textbf{Эффект:} Заставляет веса быть \textbf{маленькими}, но редко обнуляет их полностью. Уменьшает веса пропорционально их величине. Хорошо работает почти всегда.
\textit{Аналогия: Родитель, который немного урезает карманные расходы всем детям (весам), особенно тем, кто тратит больше.}
\end{myblock}

\begin{myblock}{L1 Регуляризация (Lasso)}
Штраф пропорционален \textbf{сумме модулей} весов.
\[
R_{L1}(\mathbf{w}) = \sum_{j=1}^{n} |w_j|
\]
\textbf{Примечание:} Свободный член $w_0$ обычно \textbf{не регуляризуют} по той же причине, что и в L2.
\textbf{Эффект:} Может \textbf{обнулять} некоторые веса, эффективно производя \textbf{отбор признаков} (\textit{feature selection}). Полезна, когда есть подозрение, что многие признаки неинформативны.
\textit{Аналогия: Строгий родитель, который лишает карманных денег (обнуляет вес) некоторых детей (признаков) за провинности, а остальным может тоже немного урезать.}
\end{myblock}

\begin{myexampleblock}{Связь с Bias-Variance Trade-off}
Регуляризация - это инструмент управления компромиссом между смещением (bias) и разбросом (variance):
\begin{itemize}
    \item \textbf{Без регуляризации ($\lambda=0$):} Модель может иметь низкое смещение (хорошо подходит к обучающим данным), но высокий разброс (сильно меняется при изменении данных, переобучается).
    \item \textbf{С регуляризацией ($\lambda > 0$):} Добавляя штраф, мы \textbf{увеличиваем смещение} (модель становится "проще", может хуже подходить к обучающим данным), но \textbf{уменьшаем разброс} (модель становится стабильнее, лучше обобщается на новые данные).
\end{itemize}
Подбор оптимального $\lambda$ (через кросс-валидацию) позволяет найти баланс.
\end{myexampleblock}

% --- Конец контента ---

% >>> КОММЕНТАРИЙ: Печатаем библиографию (если используется \cite)
\AtNextBibliography{\footnotesize} % Уменьшаем шрифт для библиографии
%\printbibliography % Раскомментируйте, если добавляли цитаты через \cite или \footcite

\end{multicols}

\end{document}