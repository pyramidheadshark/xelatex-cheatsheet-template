\documentclass[10pt,a4paper]{article}

\usepackage[landscape,margin=0.7cm]{geometry}
\usepackage[russian,english]{babel} % Поддержка русского и английского

% >>> КОММЕНТАРИЙ: Определяем переменные для заголовка и автора (можно изменить)
\newcommand{\cheatsheettitle}{\color{w3schools}Шпаргалка по нейронным сетям / {\color{alert} Концепции} {\color{black} Cheatsheet (XeLaTeX)}}
\newcommand{\cheatsheetauthor}{Краткий справочник}

% >>> КОММЕНТАРИЙ: Подключаем основной файл шаблона с настройками стилей и пакетов
% \documentclass{article} % ЗАКОММЕНТИРОВАНО: Добавлено для возможности компиляции этого фрагмента отдельно для проверки
\usepackage{fontspec}

% >>> КОММЕНТАРИЙ: Установка основных шрифтов документа (требует установленных шрифтов в системе)
\setmainfont{PT Sans} % Основной шрифт для текста
\setsansfont{PT Sans} % Шрифт без засечек (если используется отдельно)
\setmonofont{Liberation Mono} % Моноширинный шрифт (для кода)

\usepackage{fontawesome} % Для иконок (например, \faQuoteLeft)
\usepackage{hyperref} % Для создания кликабельных ссылок (в т.ч. в \resourcelink)
\usepackage{enumitem} % Для настройки списков
\usepackage{lipsum} % Для генерации текста-рыбы (не используется в финальной версии)
\usepackage{xcolor} % Для определения и использования цветов
\usepackage{minted} % Для подсветки синтаксиса кода (требует Python и Pygments, компиляция с -shell-escape)

% >>>>>> НОВОЕ: Пакет для создания графиков <<<<<<
\usepackage{pgfplots}
\pgfplotsset{compat=newest} % Устанавливаем последнюю версию совместимости
\usetikzlibrary{arrows.meta, shapes.geometric} % Загружаем полезные библиотеки TikZ

\usepackage{titlesec} % Для настройки заголовков секций
% >>> КОММЕНТАРИЙ: Настройка отступов для заголовков секций (\section)
\titlespacing*{\section}{0pt}{*1.5}{*0.8} % {отступ слева}{отступ сверху}{отступ снизу}

% >>> КОММЕНТАРИЙ: Форматирование заголовков секций (\section)
\titleformat{\section}
  {\normalfont\large\bfseries\color{mDarkTeal}} % Стиль заголовка (цвет, размер, жирность)
  {\thesection} % Номер секции
  {1em} % Отступ после номера
  {} % Код перед заголовком (здесь пусто)

% >>> КОММЕНТАРИЙ: Определение пользовательских цветов через HTML-коды
\definecolor{customcolor}{HTML}{696EA8} % Основной цвет для блоков textbox
\definecolor{alert}{HTML}{CD5C5C} % Цвет для выделения (красный)
\definecolor{w3schools}{HTML}{4CAF50} % Цвет для выделения (зеленый)
\definecolor{subbox}{gray}{0.60} % Цвет для под-блоков (не используется?)
\definecolor{codecolor}{HTML}{FFC300} % Цвет для кода (не используется?)
\definecolor{mDarkTeal}{HTML}{23373b} % Темно-бирюзовый (для заголовков секций, названий блоков)
\colorlet{beamerboxbg}{black!2} % Фон для beamerbox (не используется?)
\colorlet{beamerboxfg}{mDarkTeal} % Цвет текста для beamerbox (не используется?)
\definecolor{mDarkBrown}{HTML}{604c38} % Темно-коричневый (не используется?)
\definecolor{mLightBrown}{HTML}{EB811B} % Светло-коричневый (для заголовков myalertblock)
\definecolor{mLightGreen}{HTML}{14B03D} % Светло-зеленый (для заголовков myexampleblock)
\definecolor{alertTextBox}{HTML}{A8696E} % Цвет для блоков alerttextbox

% >>> КОММЕНТАРИЙ: Настройка пакета biblatex для библиографии
\usepackage
[citestyle=authoryear, % Стиль цитирования (Автор-Год)
sorting=nty, % Сортировка по Имени, Году, Названию
autocite=footnote, % Команда \autocite создает сноски
autolang=hyphen, % Автоматическое определение языка для переносов
mincrossrefs=1, % Минимальное количество перекрестных ссылок
backend=biber] % Используемый бэкенд (требует запуск biber)
{biblatex}

% >>> КОММЕНТАРИЙ: Настройка формата пост-заметок в цитатах (например, номера страниц)
\DeclareFieldFormat{postnote}{#1}
\DeclareFieldFormat{multipostnote}{#1}
\DeclareAutoCiteCommand{footnote}[f]{\footcite}{\footcites} % Настройка команды \autocite

% >>> КОММЕНТАРИЙ: Подключение файла с библиографическими записями
\addbibresource{literature.bib}

% >>> КОММЕНТАРИЙ: Подключение библиотеки tcolorbox для создания цветных блоков
\usepackage{tcolorbox}
\tcbuselibrary{most, listingsutf8, minted} % Загрузка необходимых библиотек tcolorbox (most включает breakable)

% >>> КОММЕНТАРИЙ: Глобальные настройки для маленьких tcbox (inline блоки)
\tcbset{tcbox width=auto,left=1mm,top=1mm,bottom=1mm,
right=1mm,boxsep=1mm,middle=1pt}

% >>> КОММЕНТАРИЙ: Определение базового окружения для цветных блоков с заголовком
% breakable: позволяет блоку разрываться между колонками/страницами.
% ВАЖНО: В multicols автоматический разрыв breakable может работать не всегда идеально.
% Если блок разрывается некрасиво, попробуйте разбить контент на несколько блоков
% или используйте \needspace{<длина>} перед блоком.
\newenvironment{mycolorbox}[2]{
\begin{tcolorbox}[capture=minipage,fonttitle=\large\bfseries, enhanced,boxsep=1mm,colback=#1!30!white,
width=\linewidth, % Ширина блока равна ширине текущей колонки
arc=2pt,outer arc=2pt, toptitle=0mm,colframe=#1,opacityback=0.7,nobeforeafter,
breakable, % Разрешить разрыв блока
title=#2] % Текст заголовка блока
}{\end{tcolorbox}}

% >>> КОММЕНТАРИЙ: Окружение subbox (не используется в текущем коде)
\newenvironment{subbox}[2]{
\begin{tcolorbox}[capture=minipage,fonttitle=\normalsize\bfseries, enhanced,boxsep=1mm,colback=#1!30!white,on line,tcbox width=auto,left=0.3em,top=1mm, toptitle=0mm,colframe=#1,opacityback=0.7,nobeforeafter,
breakable,
title=#2]\footnotesize
}{\normalsize\end{tcolorbox}\vspace{0.1em}}

% >>> КОММЕНТАРИЙ: Окружение для размещения tcolorbox'ов в несколько колонок внутри блока (не используется)
\newenvironment{multibox}[1]{
\begin{tcbraster}[raster columns=#1,raster equal height,nobeforeafter,raster column skip=1em,raster left skip=1em,raster right skip=1em]}{\end{tcbraster}}

% >>> КОММЕНТАРИЙ: Окружения-обертки для mycolorbox с предопределенными цветами и заголовками
\newenvironment{textbox}[1]{\begin{mycolorbox}{customcolor}{#1}}{\end{mycolorbox}} % Обычный текстовый блок
\newenvironment{alerttextbox}[1]{\begin{mycolorbox}{alertTextBox}{#1}}{\end{mycolorbox}} % Блок для предупреждений/важной информации

% >>> КОММЕНТАРИЙ: Определение окружения для блоков с кодом с подсветкой minted
\newtcblisting{codebox}[3][]{ % [опции minted], {язык}, {заголовок}
colback=black!10,colframe=black!20, % Цвета фона и рамки
listing only, % Показывать только листинг кода
minted options={ % Опции, передаваемые в minted
    numbers=left, % Нумерация строк слева
    style=default, % Стиль подсветки (можно выбрать другие, напр., 'friendly')
    fontsize=\scriptsize, % Размер шрифта кода
    breaklines, % Автоматический перенос длинных строк
    breaksymbolleft=, % Символ для обозначения переноса (пусто)
    autogobble, % Удаление лишних отступов слева
    linenos, % Показывать номера строк
    numbersep=0.7em, % Отступ номеров строк от кода
    #1 % Дополнительные опции minted, переданные в [1]
    },
enhanced,top=1mm, toptitle=0mm,
left=5mm, % Отступ слева внутри блока (для номеров строк)
arc=0pt,outer arc=0pt, % Прямые углы
title={#3}, % Заголовок блока кода
fonttitle=\small\bfseries\color{mDarkTeal}, % Стиль заголовка блока кода
listing engine=minted, % Движок для листинга - minted
minted language=#2, % Язык программирования для подсветки
breakable % Разрешить разрыв блока кода
}

\newcommand{\punkti}{~\lbrack\dots\rbrack~} % Команда для [...] (не используется)

% >>> КОММЕНТАРИЙ: Переопределение окружения quote для добавления иконок цитат
\renewenvironment{quote}
               {\list{\faQuoteLeft\phantom{ }}{\rightmargin\leftmargin}
                \item\relax\scriptsize\ignorespaces}
               {\unskip\unskip\phantom{xx}\faQuoteRight\endlist}

% >>> КОММЕНТАРИЙ: Вспомогательные команды для создания цветных фонов под текстом (не используются)
\newcommand{\bgupper}[3]{\colorbox{#1}{\color{#2}\huge\bfseries\MakeUppercase{#3}}}
\newcommand{\bg}[3]{\colorbox{#1}{\bfseries\color{#2}#3}}

% >>> КОММЕНТАРИЙ: Команда для форматирования описания команды/функции
% #1: Сама команда (используется \detokenize для корректного отображения спецсимволов)
% #2: Описание команды
\newcommand{\mycommand}[2]{
  {\par\noindent\ttfamily\detokenize{#1}\par} % Вывод команды моноширинным шрифтом
  \nopagebreak % Стараемся не разрывать страницу после команды
  \hangindent=1.5em \hangafter=1 \noindent % Делаем отступ для описания
  \small\textit{#2}\par\vspace{0.5ex} % Вывод описания и небольшой отступ после
}

% >>> КОММЕНТАРИЙ: Команда для форматирования ссылок на ресурсы
% #1: URL ссылки
% #2: Текст ссылки
% #3: Описание ресурса
\newcommand{\resourcelink}[3]{
  {\par\noindent\href{#1}{\ttfamily #2}\par} % Вывод кликабельной ссылки
  \nopagebreak
  \hangindent=1.5em \hangafter=1 \noindent
  \small\textit{#3}\par\vspace{0.5ex} % Вывод описания
}

% >>> КОММЕНТАРИЙ: Вспомогательные команды для стилизации
\newcommand{\sep}{{\scriptsize~\faCircle{ }~}} % Разделитель-кружок
\newcommand{\bggreen}[1]{\medskip\bgupper{w3schools}{black}{#1}\\[0.5em]} % Зеленый заголовок (не используется)
\newcommand{\green}[1]{\smallskip\bg{w3schools}{white}{#1}\\} % Текст на зеленом фоне
\newcommand{\red}[1]{\smallskip\bg{alert}{white}{#1}\\} % Текст на красном фоне
\newcommand{\alertcmd}[1]{\red{#1}\\} % Псевдоним для red (переименовано чтобы не конфликтовать с цветом alert)

\usepackage{multicol} % Пакет для создания нескольких колонок
\setlength{\columnsep}{15pt} % Расстояние между колонками

\setlength{\parindent}{0pt} % Убираем абзацный отступ по умолчанию
\usepackage{csquotes} % Для правильного отображения кавычек в разных языках (используется biblatex)
\newcommand{\loremipsum}{Lorem ipsum dolor sit amet.} % Пример текста (не используется)

% >>> КОММЕНТАРИЙ: Окружения для блоков с разными цветами заголовков (пример, предупреждение, обычный)
% Основаны на tcolorbox, похожи на mycolorbox, но с другими цветами заголовков и рамки.
% Также используют breakable. Применяются для выделения примеров, предупреждений и т.д.
% ВАЖНО: Лог может содержать предупреждения "Underfull \hbox (badness 10000)".
% Это часто случается в узких колонках multicol при наличии кода, URL или текста,
% который плохо переносится. Обычно это лишь косметическая проблема, строки выглядят
% немного не до конца заполненными. Можно игнорировать или попробовать перефразировать/
% использовать \sloppy перед проблемным абзацем.
\newenvironment{myexampleblock}[1]{ % Блок для примеров (зеленый заголовок)
    \tcolorbox[capture=minipage,fonttitle=\small\bfseries\color{mLightGreen}, enhanced,boxsep=1mm,colback=black!10,breakable,noparskip,
    on line,tcbox width=auto,left=0.3em,top=1mm, toptitle=0mm,
    colframe=black!20,arc=0pt,outer arc=0pt,
    opacityback=0.7,nobeforeafter,title=#1]}
    {\endtcolorbox}

\newenvironment{myalertblock}[1]{ % Блок для предупреждений (оранжевый заголовок)
    \tcolorbox[capture=minipage,fonttitle=\small\bfseries\color{mLightBrown}, enhanced,boxsep=1mm,colback=black!10,breakable,noparskip,
    on line,tcbox width=auto,left=0.3em,top=1mm, toptitle=0mm,
    colframe=black!20,arc=0pt,outer arc=0pt,
    opacityback=0.7,nobeforeafter,title=#1]}
    {\endtcolorbox}

\newenvironment{myblock}[1]{ % Обычный информационный блок (бирюзовый заголовок)
    \tcolorbox[capture=minipage,fonttitle=\small\bfseries\color{mDarkTeal}, enhanced,boxsep=1mm,colback=black!10,breakable,noparskip,
    on line,tcbox width=auto,left=0.3em,top=1mm, toptitle=0mm,
    colframe=black!20, arc=0pt,outer arc=0pt,
    opacityback=0.7,nobeforeafter,title=#1]}
    {\endtcolorbox}

% >>> КОММЕНТАРИЙ: Команда для вставки изображения в рамке tcolorbox с подписью
% #1: Опции (не используются)
% #2: Путь к файлу изображения (используется как подпись под изображением)
\newcommand{\mygraphics}[2][]{
\tcbox[enhanced,boxsep=0pt,top=0pt,bottom=0pt,left=0pt,
right=0pt,boxrule=0.4pt,drop fuzzy shadow,clip upper,
colback=black!75!white,toptitle=2pt,bottomtitle=2pt,nobeforeafter,
center title,fonttitle=\small\sffamily,title=\detokenize{#2}] % Используем путь как заголовок
{\includegraphics[width=\the\dimexpr(\linewidth-4mm)\relax]{#2}} % Вставляем изображение, ширина чуть меньше колонки
}

% >>> КОМЕНТАРИЙ: Команда \needspace{<длина>}
% Используйте эту команду *перед* блоком (tcolorbox, section, figure), чтобы убедиться,
% что в текущей колонке есть как минимум <длина> свободного места. Если места нет,
% LaTeX начнет новую колонку/страницу перед выполнением команды.
% Пример: \needspace{10\baselineskip} % Запросить место для примерно 10 строк текста
% Это полезно для предотвращения "висячих" заголовков или некрасивых разрывов блоков.

% >>>>>> НОВОЕ: Стили для графиков PGFPlots <<<<<<
\pgfplotsset{
    % Базовый стиль для всех графиков в шпаргалке
    cheatsheet plot style/.style={
        width=0.98\linewidth, % Ширина чуть меньше колонки для аккуратности
        height=5cm,          % Высота графика (можно изменить по необходимости)
        grid=major,          % Включаем основную сетку
        grid style={dashed, color=black!20}, % Стиль сетки: пунктирная, светло-серая
        axis lines=left,     % Линии осей только слева и снизу
        axis line style={color=black!60, thick}, % Стиль линий осей: серые, потолще
        tick style={color=black!60, thick}, % Стиль меток на осях
        ticklabel style={font=\scriptsize, color=black}, % Стиль подписей меток: маленький шрифт
        label style={font=\small, color=mDarkTeal}, % Стиль подписей осей: шрифт small, цвет как у заголовков блоков
        title style={font=\small\bfseries, color=mDarkTeal}, % Стиль заголовка графика
        legend style={        % Стиль легенды
            font=\scriptsize, % Маленький шрифт
            draw=black!30,    % Легкая рамка вокруг легенды
            fill=white,       % Белый фон
            legend cell align=left, % Выравнивание текста в ячейке легенды
            anchor=north east, % Якорь для позиционирования
            at={(rel axis cs:0.98,0.98)} % Положение в правом верхнем углу внутри области графика
        },
        cycle list={ % Список стилей для линий \addplot (цвета взяты из шаблона)
            {blue, mark=*, thick},
            {alert, mark=square, thick},
            {w3schools, mark=triangle, thick},
            {mLightBrown, mark=diamond, thick},
            {customcolor, mark=oplus, thick},
            {black!50, mark=pentagon, thick},
        },
        % Убираем рамку вокруг графика по умолчанию
        axis background/.style={fill=none},
    },
    % Дополнительный стиль для ROC-кривой (без сетки, с диагональю)
    roc curve style/.style={
        cheatsheet plot style, % Наследуем базовый стиль
        width=6cm, height=6cm, % Делаем квадратным
        grid=none, % Убираем сетку
        xlabel={False Positive Rate (FPR)},
        ylabel={True Positive Rate (TPR)},
        xmin=0, xmax=1,
        ymin=0, ymax=1,
        xtick={0, 0.5, 1},
        ytick={0, 0.5, 1},
        legend pos=south east, % Легенда внизу справа
        % Добавляем диагональную линию случайного угадывания
        extra x ticks={0.5}, extra y ticks={0.5},
        extra tick style={grid=major, grid style={dashed, color=black!40}},
        after end axis/.code={ % Добавляем линию после отрисовки осей
            \draw[dashed, color=black!40] (axis cs:0,0) -- (axis cs:1,1);
        }
    },
    % Дополнительный стиль для PR-кривой
    pr curve style/.style={
        cheatsheet plot style, % Наследуем базовый стиль
        xlabel={Recall (TPR)},
        ylabel={Precision},
        xmin=0, xmax=1,
        ymin=0, ymax=1,
        legend pos=south west, % Легенда внизу слева
    }
}

% % Добавлено для возможности компиляции этого фрагмента отдельно для проверки
% \begin{document}
% \begin{multicols}{2} % Пример использования в multicols
% \lipsum[1] % Просто текст для заполнения

% \begin{textbox}{Пример графика (Базовый стиль)}
%     \begin{center}
%     \begin{tikzpicture}
%         \begin{axis}[cheatsheet plot style, title={Пример $y=x^2$ и $y=x+1$}]
%         \addplot coordinates {(0, 0) (1, 1) (2, 4) (3, 9) (4, 16)};
%         \addlegendentry{$y=x^2$}
%         \addplot coordinates {(0, 1) (1, 2) (2, 3) (3, 4) (4, 5)};
%         \addlegendentry{$y=x+1$}
%         \end{axis}
%     \end{tikzpicture}
%     \end{center}
% \end{textbox}

% \begin{myexampleblock}{Пример ROC-кривой}
%     \begin{center}
%     \begin{tikzpicture}
%         \begin{axis}[roc curve style, title={Пример ROC Curve}]
%         % Пример данных ROC кривой
%         \addplot coordinates { (0,0) (0.1,0.3) (0.2,0.6) (0.4,0.8) (0.6,0.9) (0.8,0.95) (1,1) };
%         \addlegendentry{Модель A (AUC $\approx$ 0.8)}
%         \addplot coordinates { (0,0) (0.2,0.2) (0.4,0.4) (0.6,0.6) (0.8,0.8) (1,1) }; % Пример хуже
%         \addlegendentry{Модель B (AUC = 0.5)}
%         \end{axis}
%     \end{tikzpicture}
%     \end{center}
% \end{myexampleblock}

% \begin{myblock}{Пример PR-кривой}
%     \begin{center}
%     \begin{tikzpicture}
%         \begin{axis}[pr curve style, title={Пример Precision-Recall Curve}]
%         % Пример данных PR кривой (могут сильно зависеть от порога)
%         \addplot coordinates { (0,0.9) (0.1,0.85) (0.3,0.8) (0.5,0.7) (0.7,0.6) (0.9,0.4) (1,0.3) };
%         \addlegendentry{Модель C}
%         \end{axis}
%     \end{tikzpicture}
%     \end{center}
% \end{myblock}

% \lipsum[2-3] % Просто текст для заполнения
% \end{multicols}
% \end{document}

\begin{document}
\pagestyle{empty} % Убираем номера страниц
\small % Уменьшаем базовый размер шрифта для всего документа

% >>> КОММЕНТАРИЙ: Используем multicols для создания трех колонок.
\begin{multicols}{3}
\raggedcolumns % Не растягивать колонки по вертикали

\noindent
\begin{minipage}{\linewidth}
    \centering
    {\bfseries\huge \cheatsheettitle \par}
    \vspace{1ex}
    {\large \cheatsheetauthor \par}
    \vspace{0.5ex}
    {\normalsize \today \par} % Вставляем текущую дату
\end{minipage}
\vspace{2ex}

\thispagestyle{empty} % Убеждаемся, что и на первой странице нет номера

\scriptsize % Уменьшаем шрифт для оглавления и основного текста
\tableofcontents % Генерируем оглавление

% >>> КОММЕНТАРИЙ: Подключаем основной контент шпаргалки
% >>> Обновленный Контент для шпаргалки "Введение в Нейронные Сети (NN)" v3

\section{VI. Введение в Нейронные Сети (NN)}

\begin{textbox}{Цель раздела}
    Понять базовые компоненты нейронных сетей (нейроны, слои, функции активации), основной механизм обучения (Backpropagation) и методы его улучшения (оптимизаторы, регуляризация). Заложить основу для понимания сверточных и рекуррентных сетей.
\end{textbox}

% --- VI.A: Базовые Структуры ---
\subsection{VI.A Базовые Структуры: Нейроны и Слои}

\begin{myblock}{Искусственный Нейрон: Вычислительный Элемент}
    \textbf{Что это:} Математическая модель, имитирующая работу биологического нейрона.
    \textbf{Как работает:}
    \begin{enumerate}
        \item Принимает входы ($x_i$).
        \item Умножает каждый вход на его \textbf{вес} ($w_i$) $\rightarrow$ $w_i x_i$.
        \item Суммирует взвешенные входы $\rightarrow$ $z_{sum} = \sum_{i} w_i x_i$.
        \item Добавляет \textbf{смещение} ($b$) $\rightarrow$ $z = z_{sum} + b$.
        \item Пропускает результат $z$ через \textbf{функцию активации} $f(\cdot)$ $\rightarrow$ $y = f(z)$ (выход нейрона).
    \end{enumerate}
    \textbf{Обучаемые параметры:} Веса $w_i$ и смещение $b$.
\end{myblock}

\begin{myblock}{Многослойный Перцептрон (MLP): Архитектура}
    \textbf{Что это:} Классическая нейросеть из нескольких слоев нейронов.
    \textbf{Слои:}
    \begin{itemize}
        \item \textbf{Входной (Input):} Принимает признаки $X$. Не содержит вычислительных нейронов.
        \item \textbf{Скрытые (Hidden):} Один или более. Здесь происходит основная обработка, извлечение паттернов.
        \item \textbf{Выходной (Output):} Формирует результат. Структура зависит от задачи (1 нейрон/линейная для регрессии, 1 нейрон/сигмоида для бинарной клас., N нейронов/Softmax для многоклассовой).
    \end{itemize}
    \textbf{Связи:} Обычно \textbf{полносвязные} (Dense) — каждый нейрон слоя связан с каждым нейроном следующего.
\end{myblock}

% --- VI.B: Функции активации ---
\subsection{VI.B Функции Активации: Нелинейность и Свойства}

\begin{alerttextbox}{Зачем нужна Нелинейность?}
    Без нелинейных функций активации в скрытых слоях вся сеть была бы эквивалентна простой линейной модели. Нелинейность позволяет изучать сложные зависимости.
\end{alerttextbox}

\begin{myexampleblock}{ReLU (Rectified Linear Unit)}
    \[ f(x) = \max(0, x) \]
    \textbf{Свойства:} Вычислительно проста. Не насыщается для $x>0$ (помогает с затуханием градиента).
    \textbf{Недостаток:} "Умирающие ReLU" (нейрон перестает активироваться и обучаться, если $x$ всегда $\le 0$).
    \textbf{Использование:} Стандартный выбор для скрытых слоев.
\end{myexampleblock}

\begin{myexampleblock}{Leaky ReLU}
    \[ f(x) = \begin{cases} x & \text{if } x > 0 \\ \alpha x & \text{if } x \le 0 \end{cases} \quad (\alpha \approx 0.01-0.2) \]
    \textbf{Свойства:} Решает проблему "умирающих ReLU", давая малый ненулевой градиент при $x \le 0$.
\end{myexampleblock}

\begin{myexampleblock}{ELU (Exponential Linear Unit)}
     \[ f(x) = \begin{cases} x & \text{if } x > 0 \\ \alpha (e^x - 1) & \text{if } x \le 0 \end{cases} \quad (\alpha > 0) \]
    \textbf{Свойства:} Похожа на Leaky ReLU, но использует экспоненту. Может давать лучшие результаты, чем ReLU/Leaky ReLU. Выход для $x<0$ отрицательный.
\end{myexampleblock}

\begin{myexampleblock}{Sigmoid (Сигмоида)}
    \[ f(x) = \frac{1}{1 + e^{-x}} \]
    \textbf{Свойства:} Выход [0, 1], удобен для вероятностей.
    \textbf{Недостатки:} Затухание градиентов. Выход не центрирован около нуля.
    \textbf{Использование:} Выходной слой бинарной классификации. Редко в скрытых слоях современных сетей.
\end{myexampleblock}

\begin{myexampleblock}{Tanh (Гиперболический тангенс)}
    \[ f(x) = \tanh(x) \]
    \textbf{Свойства:} Выход [-1, 1], центрирован около нуля (лучше Sigmoid для скрытых слоев).
    \textbf{Недостатки:} Затухание градиентов (хотя меньше, чем у Sigmoid).
    \textbf{Использование:} Иногда в скрытых слоях, часто в RNN/LSTM.
\end{myexampleblock}

\begin{myexampleblock}{Softmax}
    \[ f(x_i) = \frac{e^{x_i}}{\sum_{j} e^{x_j}} \]
    \textbf{Свойства:} Преобразует вектор логитов в распределение вероятностей (сумма=1).
    \textbf{Использование:} \textbf{Только} в выходном слое для \textbf{многоклассовой классификации}.
\end{myexampleblock}

% Обновленный график функций активации
\begin{myblock}{}
    {\noindent\small\bfseries\color{mDarkTeal}Графики популярных функций активации\par\vspace{1ex}}
    \begin{center}
    \begin{tikzpicture}
        \begin{axis}[
            cheatsheet plot style,
            title={Функции активации},
            xlabel={$x$},
            ylabel={$f(x)$},
            xmin=-4, xmax=4, % Уменьшим диапазон для лучшей видимости
            ymin=-1.5, ymax=2.5, % Расширим немного вверх
            axis lines=middle,
            legend style={at={(1.05,1)}, anchor=north west}, % Легенда справа вверху
            samples=100
        ]
        \addplot [blue, thick, domain=-4:4] {max(0, x)}; \addlegendentry{ReLU}
        \addplot [alert, thick, domain=-4:4] {1 / (1 + exp(-x))}; \addlegendentry{Sigmoid}
        \addplot [w3schools, thick, domain=-4:4] {tanh(x)}; \addlegendentry{Tanh}
        % Leaky ReLU с alpha=0.1
        \addplot [mDarkTeal, dashed, thick, domain=-4:4] { (x > 0) * x + (x <= 0) * 0.1 * x }; \addlegendentry{Leaky ReLU ($\alpha=0.1$)}
        % ELU с alpha=1
        \addplot [orange, dotted, thick, domain=-4:4] { (x > 0) * x + (x <= 0) * 1 * (exp(x) - 1) }; \addlegendentry{ELU ($\alpha=1$)}
        \end{axis}
    \end{tikzpicture}
    \end{center}
\end{myblock}

\begin{alerttextbox}{Проблема Затухания/Взрыва Градиентов}
    \textbf{Проблема:} При обучении глубоких сетей градиенты могут стать исчезающе малыми (\textbf{затухание}) или аномально большими (\textbf{взрыв}).
    \textbf{Последствия:} Замедление или остановка обучения (затухание), нестабильность (взрыв).
    \textbf{Решения:} Выбор активаций (ReLU и др.), правильная инициализация весов, Batch Normalization, обрезание градиентов (для взрыва).
\end{alerttextbox}

% --- VI.C: Backpropagation ---
\subsection{VI.C Backpropagation: Как Сеть Учится}

\begin{alerttextbox}{Backpropagation: Ключевой Алгоритм Обучения}
    \textbf{Цель:} Эффективно вычислить \textbf{градиенты} функции потерь $J$ по всем обучаемым параметрам ($w, b$). Градиент $\partial J / \partial w$ показывает, как сильно изменение веса $w$ повлияет на итоговую ошибку $J$.
\end{alerttextbox}

\begin{myblock}{Этап 1: Прямой Проход (Forward Pass)}
    \textbf{Что происходит:} Данные $X$ проходят через сеть слой за слоем от входа к выходу. На каждом слое вычисляются взвешенные суммы ($z$) и активации ($a$). Получаем итоговые предсказания $\hat{y}$.
    \textbf{Результат:} Предсказания $\hat{y}$ и значения активаций $a$ на всех слоях (они понадобятся для обратного прохода).
    \textbf{Затем:} Вычисляется \textbf{функция потерь} $J(\hat{y}, y)$, измеряющая ошибку предсказания.
\end{myblock}

\begin{myblock}{Этап 2: Обратный Проход (Backward Pass)}
    \textbf{Что происходит:} "Ошибка" $J$ распространяется обратно от выхода ко входу. На каждом слое вычисляются градиенты по параметрам этого слоя и по его входам (активациям предыдущего слоя).
    \textbf{Шаги (идем от слоя L к слою 1):}
    \begin{enumerate}
        \item \textbf{Слой L (Выходной):}
            \begin{itemize}
                \item Вычисляем $\partial J / \partial a_L$ (как ошибка зависит от выхода сети).
                \item Вычисляем $\partial J / \partial z_L = (\partial J / \partial a_L) \odot f'_L(z_L)$. (\textit{Пояснение: Насколько ошибка зависит от пред-активационного значения $z_L$? Зависит от того, как она зависит от $a_L$ и как $a_L$ меняется с $z_L$ (это $f'_L$). $\odot$ - поэлементное умножение.}).
                \item Вычисляем $\partial J / \partial W_L = (\partial J / \partial z_L) \cdot a_{L-1}^T$ и $\partial J / \partial b_L = \sum (\partial J / \partial z_L)$. (\textit{Пояснение: Зная, как $z_L$ влияет на ошибку, и зная, как $W_L, b_L$ влияют на $z_L$ (через вход $a_{L-1}$), находим градиенты для параметров.}).
            \end{itemize}
        \item \textbf{Слой l (Скрытый):}
             \begin{itemize}
                \item Вычисляем $\partial J / \partial a_l = W_{l+1}^T \cdot (\partial J / \partial z_{l+1})$. (\textit{Пояснение: Ошибка "приходит" из следующего слоя $l+1$. Насколько она зависит от выхода $a_l$ этого слоя? Зависит от того, как ошибка зависит от $z_{l+1}$ и как $z_{l+1}$ зависит от $a_l$ (через веса $W_{l+1}$)}).
                \item Вычисляем $\partial J / \partial z_l = (\partial J / \partial a_l) \odot f'_l(z_l)$. (\textit{Аналогично выходному слою}).
                \item Вычисляем $\partial J / \partial W_l = (\partial J / \partial z_l) \cdot a_{l-1}^T$ и $\partial J / \partial b_l = \sum (\partial J / \partial z_l)$. (\textit{Аналогично выходному слою}).
            \end{itemize}
        \item \textbf{Повторение:} Шаги для скрытого слоя повторяются до слоя 1.
    \end{enumerate}
    \textbf{Результат:} Градиенты $\partial J / \partial W_l$ и $\partial J / \partial b_l$ для всех слоев $l$.
    \textbf{Механизм:} Эффективное применение \textbf{цепного правила (chain rule)} дифференцирования.
\end{myblock}

% --- VI.D: Оптимизаторы ---
\subsection{VI.D Оптимизаторы: Обновление Весов}

\begin{alerttextbox}{Роль Оптимизатора}
    Использует градиенты, полученные от Backpropagation, для вычисления и применения обновлений к весам $\mathbf{w}$ и смещениям $\mathbf{b}$, чтобы минимизировать функцию потерь $J$.
\end{alerttextbox}

\begin{myblock}{SGD (Stochastic Gradient Descent)}
    \textbf{Идея:} Простой шаг в направлении анти-градиента, вычисленного по батчу.
    \textbf{Формула:} $\mathbf{w} := \mathbf{w} - \alpha \cdot \nabla J(\mathbf{w})$.
    \textbf{Параметр:} Learning rate $\alpha$.
\end{myblock}

\begin{myblock}{Momentum}
    \textbf{Идея:} Добавить "инерцию" к SGD. Учитывает предыдущий шаг обновления $v$.
    \textbf{Формула:} $v_t = \beta v_{t-1} + \alpha \nabla J(\mathbf{w})$; $\mathbf{w} := \mathbf{w} - v_t$.
    \textbf{Параметры:} $\alpha$, $\beta$ (момент, обычно ~0.9).
    \textbf{Польза:} Ускоряет сходимость, помогает преодолевать плато.
\end{myblock}

\begin{myblock}{AdaGrad (Adaptive Gradient)}
    \textbf{Идея:} Адаптивный learning rate для каждого параметра. Уменьшает шаг для часто обновляемых параметров.
    \textbf{Формула:} Накапливает квадрат градиента $G$; $\Delta w_i = \frac{\alpha}{\sqrt{G_{ii} + \epsilon}} \nabla J_i(w)$.
    \textbf{Польза:} Хорош для разреженных данных.
    \textbf{Недостаток:} Learning rate может слишком быстро затухнуть.
\end{myblock}

\begin{myblock}{RMSProp}
    \textbf{Идея:} Исправить проблему AdaGrad с затуханием шага. Использует скользящее среднее квадратов градиентов $E[g^2]$.
    \textbf{Формула:} $E[g^2]_t = \gamma E[g^2]_{t-1} + (1-\gamma)(\nabla J)^2$; $\Delta w = \frac{\alpha}{\sqrt{E[g^2]_t + \epsilon}} \nabla J(w)$.
    \textbf{Параметры:} $\alpha$, $\gamma$ (коэфф. затухания, ~0.9).
\end{myblock}

\begin{myblock}{Adam (Adaptive Moment Estimation)}
    \textbf{Идея:} Сочетает Momentum (скользящее среднее градиентов $m$) и RMSProp (скользящее среднее квадратов градиентов $v$).
    \textbf{Формула:} Использует $m$ и $v$ для вычисления адаптивного шага. Включает коррекцию смещения.
    \textbf{Польза:} Часто эффективен по умолчанию, хорошо работает на широком круге задач.
    \textbf{Параметры:} $\alpha$, $\beta_1$ (~0.9), $\beta_2$ (~0.999).
\end{myblock}

% --- VI.E: Dropout, BatchNorm ---
\subsection{VI.E Стабилизация и Регуляризация Обучения}

\begin{myexampleblock}{Dropout (Прореживание)}
    \textbf{Что это:} Метод \textbf{регуляризации} для борьбы с переобучением.
    \textbf{Как работает (на обучении):} Случайным образом обнуляет выходы части нейронов слоя с вероятностью $p$.
    \textbf{Как работает (на предсказании):} Использует все нейроны, но масштабирует их выходы на $(1-p)$.
    \textbf{Эффект:} Заставляет сеть учить более робастные и распределенные представления.
\end{myexampleblock}

\begin{myexampleblock}{Batch Normalization (BatchNorm)}
    \textbf{Что это:} Техника для \textbf{стабилизации и ускорения} обучения.
    \textbf{Как работает (на обучении):}
    \begin{enumerate}
        \item Нормализует входы $z$ слоя по батчу (среднее 0, дисперсия 1).
        \item Масштабирует и сдвигает результат с помощью обучаемых $\gamma$ и $\beta$.
        \item Обновляет скользящие средние $\mu_{run}, \sigma^2_{run}$.
    \end{enumerate}
    \textbf{Как работает (на предсказании):} Использует $\mu_{run}, \sigma^2_{run}$ и обученные $\gamma, \beta$.
    \textbf{Эффект:} Борется с internal covariate shift, позволяет использовать больший learning rate, имеет легкий регуляризующий эффект. Обычно вставляется \textit{до} функции активации.
\end{myexampleblock}

% --- VI.F: CNN и RNN ---
\subsection{VI.F Специализированные Архитектуры}

\begin{textbox}{Зачем нужны специализированные сети?}
    MLP универсальны, но для данных с внутренней структурой (пространственной или временной) CNN и RNN часто более эффективны.
\end{textbox}

\begin{myblock}{CNN (Convolutional Neural Networks)}
    \textbf{Применение:} Изображения, видео, данные с сетчатой структурой.
    \textbf{Ключевые Идеи:}
    \begin{itemize}
        \item \textbf{Сверточный слой:} Применяет \textbf{фильтры (ядра)} для обнаружения локальных паттернов (грани, текстуры). Использует \textit{локальные связи} и \textit{разделяемые веса}. Выход - \textbf{карты признаков}.
        \item \textbf{Пулинг слой:} Уменьшает пространственный размер карт признаков (Max Pooling, Average Pooling), обеспечивая инвариантность к малым сдвигам.
    \end{itemize}
    \textbf{Архитектура:} Чередование [Conv -> Activation -> Pooling]. Затем полносвязные слои для классификации/регрессии.
\end{myblock}

\begin{myblock}{RNN (Recurrent Neural Networks)}
    \textbf{Применение:} Последовательные данные (текст, временные ряды, речь).
    \textbf{Ключевая Идея:} \textbf{Рекуррентная связь} позволяет сети иметь "память" (\textbf{скрытое состояние} $h_t$), передаваемую от шага к шагу.
    \textbf{Простая RNN:} $h_t = f(W_{xh} x_t + W_{hh} h_{t-1} + b_h)$. Веса $W$ разделяемые по времени.
    \textbf{Проблема:} Затухание/взрыв градиентов на длинных последовательностях (BPTT).
\end{myblock}

\begin{myblock}{LSTM (Long Short-Term Memory) и GRU (Gated Recurrent Unit)}
    \textbf{Что это:} Продвинутые RNN-ячейки для решения проблемы градиентов.
    \textbf{Как работают:} Используют \textbf{гейты} (механизмы управления информацией с Sigmoid/Tanh), чтобы контролировать, что запоминать, что забывать, и что передавать дальше.
    \textbf{LSTM:} Имеет 3 гейта (input, forget, output) и состояние ячейки (cell state).
    \textbf{GRU:} Упрощенная версия с 2 гейтами (reset, update).
    \textbf{Использование:} Стандарт де-факто для задач с последовательностями вместо простых RNN.
\end{myblock}

% --- Конец обновленного контента ---

% >>> КОММЕНТАРИЙ: Печатаем библиографию (если используется \cite)
\AtNextBibliography{\footnotesize} % Уменьшаем шрифт для библиографии
%\printbibliography % Раскомментируйте, если добавляли цитаты через \cite или \footcite

\end{multicols}

\end{document}