\documentclass[10pt,a4paper]{article}

\usepackage[landscape,margin=0.7cm]{geometry}
\usepackage[russian,english]{babel} % Поддержка русского и английского

% >>> КОММЕНТАРИЙ: Определяем переменные для заголовка и автора (можно изменить)
\newcommand{\cheatsheettitle}{\color{w3schools}Шпаргалка по Pandas, Numpy, Matplotlib / {\color{alert} Полезные методы} {\color{black} Cheatsheet (XeLaTeX)}}
\newcommand{\cheatsheetauthor}{Краткий справочник по основным операциям}

% >>> КОММЕНТАРИЙ: Подключаем основной файл шаблона с настройками стилей и пакетов
% \documentclass{article} % ЗАКОММЕНТИРОВАНО: Добавлено для возможности компиляции этого фрагмента отдельно для проверки
\usepackage{fontspec}

% >>> КОММЕНТАРИЙ: Установка основных шрифтов документа (требует установленных шрифтов в системе)
\setmainfont{PT Sans} % Основной шрифт для текста
\setsansfont{PT Sans} % Шрифт без засечек (если используется отдельно)
\setmonofont{Liberation Mono} % Моноширинный шрифт (для кода)

\usepackage{fontawesome} % Для иконок (например, \faQuoteLeft)
\usepackage{hyperref} % Для создания кликабельных ссылок (в т.ч. в \resourcelink)
\usepackage{enumitem} % Для настройки списков
\usepackage{lipsum} % Для генерации текста-рыбы (не используется в финальной версии)
\usepackage{xcolor} % Для определения и использования цветов
\usepackage{minted} % Для подсветки синтаксиса кода (требует Python и Pygments, компиляция с -shell-escape)

% >>>>>> НОВОЕ: Пакет для создания графиков <<<<<<
\usepackage{pgfplots}
\pgfplotsset{compat=newest} % Устанавливаем последнюю версию совместимости
\usetikzlibrary{arrows.meta, shapes.geometric} % Загружаем полезные библиотеки TikZ

\usepackage{titlesec} % Для настройки заголовков секций
% >>> КОММЕНТАРИЙ: Настройка отступов для заголовков секций (\section)
\titlespacing*{\section}{0pt}{*1.5}{*0.8} % {отступ слева}{отступ сверху}{отступ снизу}

% >>> КОММЕНТАРИЙ: Форматирование заголовков секций (\section)
\titleformat{\section}
  {\normalfont\large\bfseries\color{mDarkTeal}} % Стиль заголовка (цвет, размер, жирность)
  {\thesection} % Номер секции
  {1em} % Отступ после номера
  {} % Код перед заголовком (здесь пусто)

% >>> КОММЕНТАРИЙ: Определение пользовательских цветов через HTML-коды
\definecolor{customcolor}{HTML}{696EA8} % Основной цвет для блоков textbox
\definecolor{alert}{HTML}{CD5C5C} % Цвет для выделения (красный)
\definecolor{w3schools}{HTML}{4CAF50} % Цвет для выделения (зеленый)
\definecolor{subbox}{gray}{0.60} % Цвет для под-блоков (не используется?)
\definecolor{codecolor}{HTML}{FFC300} % Цвет для кода (не используется?)
\definecolor{mDarkTeal}{HTML}{23373b} % Темно-бирюзовый (для заголовков секций, названий блоков)
\colorlet{beamerboxbg}{black!2} % Фон для beamerbox (не используется?)
\colorlet{beamerboxfg}{mDarkTeal} % Цвет текста для beamerbox (не используется?)
\definecolor{mDarkBrown}{HTML}{604c38} % Темно-коричневый (не используется?)
\definecolor{mLightBrown}{HTML}{EB811B} % Светло-коричневый (для заголовков myalertblock)
\definecolor{mLightGreen}{HTML}{14B03D} % Светло-зеленый (для заголовков myexampleblock)
\definecolor{alertTextBox}{HTML}{A8696E} % Цвет для блоков alerttextbox

% >>> КОММЕНТАРИЙ: Настройка пакета biblatex для библиографии
\usepackage
[citestyle=authoryear, % Стиль цитирования (Автор-Год)
sorting=nty, % Сортировка по Имени, Году, Названию
autocite=footnote, % Команда \autocite создает сноски
autolang=hyphen, % Автоматическое определение языка для переносов
mincrossrefs=1, % Минимальное количество перекрестных ссылок
backend=biber] % Используемый бэкенд (требует запуск biber)
{biblatex}

% >>> КОММЕНТАРИЙ: Настройка формата пост-заметок в цитатах (например, номера страниц)
\DeclareFieldFormat{postnote}{#1}
\DeclareFieldFormat{multipostnote}{#1}
\DeclareAutoCiteCommand{footnote}[f]{\footcite}{\footcites} % Настройка команды \autocite

% >>> КОММЕНТАРИЙ: Подключение файла с библиографическими записями
\addbibresource{literature.bib}

% >>> КОММЕНТАРИЙ: Подключение библиотеки tcolorbox для создания цветных блоков
\usepackage{tcolorbox}
\tcbuselibrary{most, listingsutf8, minted} % Загрузка необходимых библиотек tcolorbox (most включает breakable)

% >>> КОММЕНТАРИЙ: Глобальные настройки для маленьких tcbox (inline блоки)
\tcbset{tcbox width=auto,left=1mm,top=1mm,bottom=1mm,
right=1mm,boxsep=1mm,middle=1pt}

% >>> КОММЕНТАРИЙ: Определение базового окружения для цветных блоков с заголовком
% breakable: позволяет блоку разрываться между колонками/страницами.
% ВАЖНО: В multicols автоматический разрыв breakable может работать не всегда идеально.
% Если блок разрывается некрасиво, попробуйте разбить контент на несколько блоков
% или используйте \needspace{<длина>} перед блоком.
\newenvironment{mycolorbox}[2]{
\begin{tcolorbox}[capture=minipage,fonttitle=\large\bfseries, enhanced,boxsep=1mm,colback=#1!30!white,
width=\linewidth, % Ширина блока равна ширине текущей колонки
arc=2pt,outer arc=2pt, toptitle=0mm,colframe=#1,opacityback=0.7,nobeforeafter,
breakable, % Разрешить разрыв блока
title=#2] % Текст заголовка блока
}{\end{tcolorbox}}

% >>> КОММЕНТАРИЙ: Окружение subbox (не используется в текущем коде)
\newenvironment{subbox}[2]{
\begin{tcolorbox}[capture=minipage,fonttitle=\normalsize\bfseries, enhanced,boxsep=1mm,colback=#1!30!white,on line,tcbox width=auto,left=0.3em,top=1mm, toptitle=0mm,colframe=#1,opacityback=0.7,nobeforeafter,
breakable,
title=#2]\footnotesize
}{\normalsize\end{tcolorbox}\vspace{0.1em}}

% >>> КОММЕНТАРИЙ: Окружение для размещения tcolorbox'ов в несколько колонок внутри блока (не используется)
\newenvironment{multibox}[1]{
\begin{tcbraster}[raster columns=#1,raster equal height,nobeforeafter,raster column skip=1em,raster left skip=1em,raster right skip=1em]}{\end{tcbraster}}

% >>> КОММЕНТАРИЙ: Окружения-обертки для mycolorbox с предопределенными цветами и заголовками
\newenvironment{textbox}[1]{\begin{mycolorbox}{customcolor}{#1}}{\end{mycolorbox}} % Обычный текстовый блок
\newenvironment{alerttextbox}[1]{\begin{mycolorbox}{alertTextBox}{#1}}{\end{mycolorbox}} % Блок для предупреждений/важной информации

% >>> КОММЕНТАРИЙ: Определение окружения для блоков с кодом с подсветкой minted
\newtcblisting{codebox}[3][]{ % [опции minted], {язык}, {заголовок}
colback=black!10,colframe=black!20, % Цвета фона и рамки
listing only, % Показывать только листинг кода
minted options={ % Опции, передаваемые в minted
    numbers=left, % Нумерация строк слева
    style=default, % Стиль подсветки (можно выбрать другие, напр., 'friendly')
    fontsize=\scriptsize, % Размер шрифта кода
    breaklines, % Автоматический перенос длинных строк
    breaksymbolleft=, % Символ для обозначения переноса (пусто)
    autogobble, % Удаление лишних отступов слева
    linenos, % Показывать номера строк
    numbersep=0.7em, % Отступ номеров строк от кода
    #1 % Дополнительные опции minted, переданные в [1]
    },
enhanced,top=1mm, toptitle=0mm,
left=5mm, % Отступ слева внутри блока (для номеров строк)
arc=0pt,outer arc=0pt, % Прямые углы
title={#3}, % Заголовок блока кода
fonttitle=\small\bfseries\color{mDarkTeal}, % Стиль заголовка блока кода
listing engine=minted, % Движок для листинга - minted
minted language=#2, % Язык программирования для подсветки
breakable % Разрешить разрыв блока кода
}

\newcommand{\punkti}{~\lbrack\dots\rbrack~} % Команда для [...] (не используется)

% >>> КОММЕНТАРИЙ: Переопределение окружения quote для добавления иконок цитат
\renewenvironment{quote}
               {\list{\faQuoteLeft\phantom{ }}{\rightmargin\leftmargin}
                \item\relax\scriptsize\ignorespaces}
               {\unskip\unskip\phantom{xx}\faQuoteRight\endlist}

% >>> КОММЕНТАРИЙ: Вспомогательные команды для создания цветных фонов под текстом (не используются)
\newcommand{\bgupper}[3]{\colorbox{#1}{\color{#2}\huge\bfseries\MakeUppercase{#3}}}
\newcommand{\bg}[3]{\colorbox{#1}{\bfseries\color{#2}#3}}

% >>> КОММЕНТАРИЙ: Команда для форматирования описания команды/функции
% #1: Сама команда (используется \detokenize для корректного отображения спецсимволов)
% #2: Описание команды
\newcommand{\mycommand}[2]{
  {\par\noindent\ttfamily\detokenize{#1}\par} % Вывод команды моноширинным шрифтом
  \nopagebreak % Стараемся не разрывать страницу после команды
  \hangindent=1.5em \hangafter=1 \noindent % Делаем отступ для описания
  \small\textit{#2}\par\vspace{0.5ex} % Вывод описания и небольшой отступ после
}

% >>> КОММЕНТАРИЙ: Команда для форматирования ссылок на ресурсы
% #1: URL ссылки
% #2: Текст ссылки
% #3: Описание ресурса
\newcommand{\resourcelink}[3]{
  {\par\noindent\href{#1}{\ttfamily #2}\par} % Вывод кликабельной ссылки
  \nopagebreak
  \hangindent=1.5em \hangafter=1 \noindent
  \small\textit{#3}\par\vspace{0.5ex} % Вывод описания
}

% >>> КОММЕНТАРИЙ: Вспомогательные команды для стилизации
\newcommand{\sep}{{\scriptsize~\faCircle{ }~}} % Разделитель-кружок
\newcommand{\bggreen}[1]{\medskip\bgupper{w3schools}{black}{#1}\\[0.5em]} % Зеленый заголовок (не используется)
\newcommand{\green}[1]{\smallskip\bg{w3schools}{white}{#1}\\} % Текст на зеленом фоне
\newcommand{\red}[1]{\smallskip\bg{alert}{white}{#1}\\} % Текст на красном фоне
\newcommand{\alertcmd}[1]{\red{#1}\\} % Псевдоним для red (переименовано чтобы не конфликтовать с цветом alert)

\usepackage{multicol} % Пакет для создания нескольких колонок
\setlength{\columnsep}{15pt} % Расстояние между колонками

\setlength{\parindent}{0pt} % Убираем абзацный отступ по умолчанию
\usepackage{csquotes} % Для правильного отображения кавычек в разных языках (используется biblatex)
\newcommand{\loremipsum}{Lorem ipsum dolor sit amet.} % Пример текста (не используется)

% >>> КОММЕНТАРИЙ: Окружения для блоков с разными цветами заголовков (пример, предупреждение, обычный)
% Основаны на tcolorbox, похожи на mycolorbox, но с другими цветами заголовков и рамки.
% Также используют breakable. Применяются для выделения примеров, предупреждений и т.д.
% ВАЖНО: Лог может содержать предупреждения "Underfull \hbox (badness 10000)".
% Это часто случается в узких колонках multicol при наличии кода, URL или текста,
% который плохо переносится. Обычно это лишь косметическая проблема, строки выглядят
% немного не до конца заполненными. Можно игнорировать или попробовать перефразировать/
% использовать \sloppy перед проблемным абзацем.
\newenvironment{myexampleblock}[1]{ % Блок для примеров (зеленый заголовок)
    \tcolorbox[capture=minipage,fonttitle=\small\bfseries\color{mLightGreen}, enhanced,boxsep=1mm,colback=black!10,breakable,noparskip,
    on line,tcbox width=auto,left=0.3em,top=1mm, toptitle=0mm,
    colframe=black!20,arc=0pt,outer arc=0pt,
    opacityback=0.7,nobeforeafter,title=#1]}
    {\endtcolorbox}

\newenvironment{myalertblock}[1]{ % Блок для предупреждений (оранжевый заголовок)
    \tcolorbox[capture=minipage,fonttitle=\small\bfseries\color{mLightBrown}, enhanced,boxsep=1mm,colback=black!10,breakable,noparskip,
    on line,tcbox width=auto,left=0.3em,top=1mm, toptitle=0mm,
    colframe=black!20,arc=0pt,outer arc=0pt,
    opacityback=0.7,nobeforeafter,title=#1]}
    {\endtcolorbox}

\newenvironment{myblock}[1]{ % Обычный информационный блок (бирюзовый заголовок)
    \tcolorbox[capture=minipage,fonttitle=\small\bfseries\color{mDarkTeal}, enhanced,boxsep=1mm,colback=black!10,breakable,noparskip,
    on line,tcbox width=auto,left=0.3em,top=1mm, toptitle=0mm,
    colframe=black!20, arc=0pt,outer arc=0pt,
    opacityback=0.7,nobeforeafter,title=#1]}
    {\endtcolorbox}

% >>> КОММЕНТАРИЙ: Команда для вставки изображения в рамке tcolorbox с подписью
% #1: Опции (не используются)
% #2: Путь к файлу изображения (используется как подпись под изображением)
\newcommand{\mygraphics}[2][]{
\tcbox[enhanced,boxsep=0pt,top=0pt,bottom=0pt,left=0pt,
right=0pt,boxrule=0.4pt,drop fuzzy shadow,clip upper,
colback=black!75!white,toptitle=2pt,bottomtitle=2pt,nobeforeafter,
center title,fonttitle=\small\sffamily,title=\detokenize{#2}] % Используем путь как заголовок
{\includegraphics[width=\the\dimexpr(\linewidth-4mm)\relax]{#2}} % Вставляем изображение, ширина чуть меньше колонки
}

% >>> КОМЕНТАРИЙ: Команда \needspace{<длина>}
% Используйте эту команду *перед* блоком (tcolorbox, section, figure), чтобы убедиться,
% что в текущей колонке есть как минимум <длина> свободного места. Если места нет,
% LaTeX начнет новую колонку/страницу перед выполнением команды.
% Пример: \needspace{10\baselineskip} % Запросить место для примерно 10 строк текста
% Это полезно для предотвращения "висячих" заголовков или некрасивых разрывов блоков.

% >>>>>> НОВОЕ: Стили для графиков PGFPlots <<<<<<
\pgfplotsset{
    % Базовый стиль для всех графиков в шпаргалке
    cheatsheet plot style/.style={
        width=0.98\linewidth, % Ширина чуть меньше колонки для аккуратности
        height=5cm,          % Высота графика (можно изменить по необходимости)
        grid=major,          % Включаем основную сетку
        grid style={dashed, color=black!20}, % Стиль сетки: пунктирная, светло-серая
        axis lines=left,     % Линии осей только слева и снизу
        axis line style={color=black!60, thick}, % Стиль линий осей: серые, потолще
        tick style={color=black!60, thick}, % Стиль меток на осях
        ticklabel style={font=\scriptsize, color=black}, % Стиль подписей меток: маленький шрифт
        label style={font=\small, color=mDarkTeal}, % Стиль подписей осей: шрифт small, цвет как у заголовков блоков
        title style={font=\small\bfseries, color=mDarkTeal}, % Стиль заголовка графика
        legend style={        % Стиль легенды
            font=\scriptsize, % Маленький шрифт
            draw=black!30,    % Легкая рамка вокруг легенды
            fill=white,       % Белый фон
            legend cell align=left, % Выравнивание текста в ячейке легенды
            anchor=north east, % Якорь для позиционирования
            at={(rel axis cs:0.98,0.98)} % Положение в правом верхнем углу внутри области графика
        },
        cycle list={ % Список стилей для линий \addplot (цвета взяты из шаблона)
            {blue, mark=*, thick},
            {alert, mark=square, thick},
            {w3schools, mark=triangle, thick},
            {mLightBrown, mark=diamond, thick},
            {customcolor, mark=oplus, thick},
            {black!50, mark=pentagon, thick},
        },
        % Убираем рамку вокруг графика по умолчанию
        axis background/.style={fill=none},
    },
    % Дополнительный стиль для ROC-кривой (без сетки, с диагональю)
    roc curve style/.style={
        cheatsheet plot style, % Наследуем базовый стиль
        width=6cm, height=6cm, % Делаем квадратным
        grid=none, % Убираем сетку
        xlabel={False Positive Rate (FPR)},
        ylabel={True Positive Rate (TPR)},
        xmin=0, xmax=1,
        ymin=0, ymax=1,
        xtick={0, 0.5, 1},
        ytick={0, 0.5, 1},
        legend pos=south east, % Легенда внизу справа
        % Добавляем диагональную линию случайного угадывания
        extra x ticks={0.5}, extra y ticks={0.5},
        extra tick style={grid=major, grid style={dashed, color=black!40}},
        after end axis/.code={ % Добавляем линию после отрисовки осей
            \draw[dashed, color=black!40] (axis cs:0,0) -- (axis cs:1,1);
        }
    },
    % Дополнительный стиль для PR-кривой
    pr curve style/.style={
        cheatsheet plot style, % Наследуем базовый стиль
        xlabel={Recall (TPR)},
        ylabel={Precision},
        xmin=0, xmax=1,
        ymin=0, ymax=1,
        legend pos=south west, % Легенда внизу слева
    }
}

% % Добавлено для возможности компиляции этого фрагмента отдельно для проверки
% \begin{document}
% \begin{multicols}{2} % Пример использования в multicols
% \lipsum[1] % Просто текст для заполнения

% \begin{textbox}{Пример графика (Базовый стиль)}
%     \begin{center}
%     \begin{tikzpicture}
%         \begin{axis}[cheatsheet plot style, title={Пример $y=x^2$ и $y=x+1$}]
%         \addplot coordinates {(0, 0) (1, 1) (2, 4) (3, 9) (4, 16)};
%         \addlegendentry{$y=x^2$}
%         \addplot coordinates {(0, 1) (1, 2) (2, 3) (3, 4) (4, 5)};
%         \addlegendentry{$y=x+1$}
%         \end{axis}
%     \end{tikzpicture}
%     \end{center}
% \end{textbox}

% \begin{myexampleblock}{Пример ROC-кривой}
%     \begin{center}
%     \begin{tikzpicture}
%         \begin{axis}[roc curve style, title={Пример ROC Curve}]
%         % Пример данных ROC кривой
%         \addplot coordinates { (0,0) (0.1,0.3) (0.2,0.6) (0.4,0.8) (0.6,0.9) (0.8,0.95) (1,1) };
%         \addlegendentry{Модель A (AUC $\approx$ 0.8)}
%         \addplot coordinates { (0,0) (0.2,0.2) (0.4,0.4) (0.6,0.6) (0.8,0.8) (1,1) }; % Пример хуже
%         \addlegendentry{Модель B (AUC = 0.5)}
%         \end{axis}
%     \end{tikzpicture}
%     \end{center}
% \end{myexampleblock}

% \begin{myblock}{Пример PR-кривой}
%     \begin{center}
%     \begin{tikzpicture}
%         \begin{axis}[pr curve style, title={Пример Precision-Recall Curve}]
%         % Пример данных PR кривой (могут сильно зависеть от порога)
%         \addplot coordinates { (0,0.9) (0.1,0.85) (0.3,0.8) (0.5,0.7) (0.7,0.6) (0.9,0.4) (1,0.3) };
%         \addlegendentry{Модель C}
%         \end{axis}
%     \end{tikzpicture}
%     \end{center}
% \end{myblock}

% \lipsum[2-3] % Просто текст для заполнения
% \end{multicols}
% \end{document}

\begin{document}
\pagestyle{empty} % Убираем номера страниц
\small % Уменьшаем базовый размер шрифта для всего документа

% >>> КОММЕНТАРИЙ: Используем multicols для создания трех колонок.
\begin{multicols}{3}
\raggedcolumns % Не растягивать колонки по вертикали

\noindent
\begin{minipage}{\linewidth}
    \centering
    {\bfseries\huge \cheatsheettitle \par}
    \vspace{1ex}
    {\large \cheatsheetauthor \par}
    \vspace{0.5ex}
    {\normalsize \today \par} % Вставляем текущую дату
\end{minipage}
\vspace{2ex}

\thispagestyle{empty} % Убеждаемся, что и на первой странице нет номера

\scriptsize % Уменьшаем шрифт для оглавления и основного текста
\tableofcontents % Генерируем оглавление

% >>> КОММЕНТАРИЙ: Подключаем основной контент шпаргалки
% >>> Контент для шпаргалки "Python для DS - Ключевые Библиотеки" (v2)

% --- Раздел 1: NumPy ---
\section{NumPy: Математика и Массивы (\texttt{np})}

\begin{textbox}{Зачем NumPy?}
    Это фундамент для числовых вычислений в Python. Представь его как **супер-оптимизированную таблицу Excel** для работы с числами, особенно с большими наборами данных. Все операции выполняются очень быстро. Основной объект - \textbf{ndarray} (n-мерный массив).
\end{textbox}

\begin{myblock}{{Создание Массивов}}
    Основные способы "завести" себе массив.
    \begin{codebox}{python}{Создание NumPy массивов}
    import numpy as np

    # Из Python списка
    arr1 = np.array([1, 2, 3, 4, 5])

    # Массив нулей (форма: 2 строки, 3 столбца)
    zeros = np.zeros((2, 3))

    # Массив единиц
    ones = np.ones((3, 2))

    # Последовательность чисел (как range)
    seq1 = np.arange(0, 10, 2) # Старт, стоп (не вкл.), шаг

    # Заданное количество чисел в интервале
    seq2 = np.linspace(0, 1, 5) # Старт, стоп (вкл.), кол-во
    \end{codebox}
\end{myblock}

\begin{myblock}{{Базовые Операции}}
    NumPy позволяет делать математику сразу со всем массивом.
    \begin{codebox}{python}{Операции с массивами}
    a = np.array([1, 2, 3])
    b = np.array([4, 5, 6])

    # Поэлементные операции
    c = a + b  # -> array([5, 7, 9])
    d = a * 2  # -> array([2, 4, 6])
    e = a ** 2 # -> array([1, 4, 9])

    # Математические функции
    f = np.sin(a)
    g = np.exp(a)

    # Матричное умножение (для 1D - скалярное)
    dot_product = np.dot(a, b) # 1*4 + 2*5 + 3*6 = 32

    # Сравнения
    bool_arr = a > 1 # -> array([False, True, True])
    \end{codebox}
    \textit{Аналогия с широковещанием (broadcasting):} NumPy "растягивает" массивы меньшей размерности (например, число `2` в `a * 2`), чтобы их формы совпали для операции. Как если бы ты красил стену валиком (операция), а краска (число) сама распределялась по всей ширине.
\end{myblock}

\begin{myblock}{{Форма и Размер (shape, reshape)}}
    Важно понимать "габариты" массива.
    \begin{codebox}{python}{Атрибуты и методы формы}
    data = np.array([[1, 2, 3], [4, 5, 6]])

    # Форма (кортеж: строки, столбцы, ...)
    print(data.shape)  # -> (2, 3)

    # Количество измерений
    print(data.ndim)   # -> 2

    # Общее количество элементов
    print(data.size)   # -> 6

    # Изменение формы (кол-во элементов должно совпадать!)
    reshaped = data.reshape((3, 2))
    # [[1, 2],
    #  [3, 4],
    #  [5, 6]]

    # "Вытягивание" в 1D массив
    flattened = data.flatten() # или data.reshape(-1)
    # [1, 2, 3, 4, 5, 6]
    \end{codebox}
    \textbf{Важно:} \texttt{shape} - это "чертеж" массива (например, 2 строки на 3 столбца), а \texttt{size} - общее число "кирпичиков" (элементов).
\end{myblock}

\begin{myblock}{{Индексирование и Срезы}} % <<< --- НОВЫЙ БЛОК --- <<<
    Доступ к элементам массива.
    \begin{codebox}{python}{Доступ к элементам NumPy}
    arr = np.array([[10, 20, 30], [40, 50, 60]])

    # Первый элемент (первая строка, первый столбец)
    print(arr[0, 0]) # -> 10
    # или
    print(arr[0][0]) # -> 10

    # Вся первая строка
    print(arr[0, :]) # -> array([10, 20, 30])
    # или
    print(arr[0])    # -> array([10, 20, 30])

    # Весь второй столбец
    print(arr[:, 1]) # -> array([20, 50])

    # Срез: первые две колонки первой строки
    print(arr[0, 0:2]) # -> array([10, 20])

    # Boolean индексирование
    print(arr[arr > 30]) # -> array([40, 50, 60])
    \end{codebox}
\end{myblock}

% --- Раздел 2: Pandas ---
\section{Pandas: Таблицы Данных для Анализа (\texttt{pd})}

\begin{textbox}{Зачем Pandas?}
    Твой основной инструмент для работы с **табличными данными** (как в Excel или SQL). Строит свою работу поверх NumPy. Два главных объекта: \textbf{Series} (один столбец) и \textbf{DataFrame} (таблица, коллекция Series).
    \textit{Аналогия:} Если NumPy - это супер-массив чисел, то Pandas - это **умная электронная таблица**, которую можно программировать.
\end{textbox}

\begin{myblock}{{Чтение Данных (CSV)}}
    Самый частый способ загрузить данные.
    \begin{codebox}{python}{Чтение CSV}
    import pandas as pd

    # Предположим, есть файл 'your_data.csv'
    # df = pd.read_csv('your_data.csv')

    # Частые параметры:
    # sep=';' - если разделитель точка с запятой
    # header=None - если в файле нет заголовков
    # names=['col1', 'col2'] - задать имена столбцов
    # usecols=['col1', 'col3'] - прочитать только нужные

    # Создадим пример DataFrame для демонстрации
    df_data = {'col_A': [1, 2, 3, 4], 'col_B': ['x', 'y', 'x', 'z'], 'col_C': [10, 20, 30, 40]}
    df = pd.DataFrame(df_data) # Используем этот df далее

    # Посмотреть первые/последние строки
    print(df.head())    # Первые 5 строк
    print(df.tail(3))   # Последние 3 строки

    # Информация о DataFrame (типы, пропуски)
    df.info()

    # Базовые статистики для числовых столбцов
    print(df.describe())
    \end{codebox}
\end{myblock}

\begin{myblock}{{DataFrame и Series}}
    Основные структуры данных Pandas.
    \begin{codebox}{python}{Создание и доступ}
    # Создание Series (один столбец)
    s = pd.Series([10, 20, 30], index=['a', 'b', 'c'], name='MySeries')

    # Создание DataFrame (таблица) из словаря
    data = {'col_A': [1, 2, 3], 'col_B': ['x', 'y', 'z']}
    df_example = pd.DataFrame(data)

    # Доступ к столбцу (возвращает Series)
    col_a_series = df_example['col_A']
    # или (если имя без пробелов/спецсимволов)
    col_b_series = df_example.col_B

    # Доступ к нескольким столбцам (возвращает DataFrame)
    subset_df = df_example[['col_A', 'col_B']]
    \end{codebox}
\end{myblock}

\begin{alerttextbox}{{Выборка: \texttt{.loc} vs \texttt{.iloc} (Важно!)}}
    Частая путаница у новичков! Это два основных способа выбрать строки/столбцы.
    \begin{itemize}
        \item \texttt{.loc[]}: Выбирает по \textbf{МЕТКАМ} (именам) индекса и столбцов. \textbf{Включает} правую границу среза.
        \item \texttt{.iloc[]}: Выбирает по \textbf{ЦЕЛОЧИСЛЕННЫМ ПОЗИЦИЯМ} (номерам, начиная с 0). \textbf{НЕ включает} правую границу среза (как в Python).
    \end{itemize}
    \textit{Аналогия:} Представь библиотеку. \texttt{.loc} - это поиск книги по названию и автору (метки). \texttt{.iloc} - это взять "пятую книгу с третьей полки" (позиции).
    \begin{codebox}{python}{Примеры .loc и .iloc (используем df из блока "Чтение")}
    # df имеет стандартный числовой индекс [0, 1, 2, 3]

    # --- .loc (по МЕТКАМ индекса и столбцов) ---
    # Строка по метке индекса (здесь метка=число)
    print(df.loc[0])
    # Срез строк по меткам (включительно!)
    print(df.loc[0:2]) # Строки с индексами 0, 1, 2
    # Строки и столбцы по меткам
    print(df.loc[[0, 3], ['col_A', 'col_C']])
    # Конкретная ячейка
    print(df.loc[1, 'col_B'])

    # --- .iloc (по ПОЗИЦИЯМ) ---
    # Строка по номеру (первая)
    print(df.iloc[0])
    # Срез строк по номерам (НЕ включительно!)
    print(df.iloc[0:2]) # Строки 0 и 1 (т.е. с индексами 0 и 1)
    # Строки и столбцы по номерам
    print(df.iloc[[0, 3], [0, 2]]) # 1я и 4я строки, 1й и 3й столбцы
    # Конкретная ячейка
    print(df.iloc[1, 1]) # Элемент на пересечении 2й строки, 2го столбца
    \end{codebox}
\end{alerttextbox}

\begin{myblock}{{Фильтрация Данных}}
    Отбор строк по условиям (используем `df` из блока "Чтение").
    \begin{codebox}{python}{Способы фильтрации}
    # 1. Boolean Indexing (основной способ)
    # Условие: выбрать строки, где значение в 'col_A' > 2
    filter1 = df[df['col_A'] > 2]
    print("Filter 1:\n", filter1)

    # Несколько условий: & (И), | (ИЛИ), ~ (НЕ)
    # Обязательно скобки вокруг каждого условия!
    filter2 = df[(df['col_A'] > 1) & (df['col_B'] == 'x')]
    print("Filter 2:\n", filter2)

    # Использование .isin()
    filter3 = df[df['col_B'].isin(['x', 'y'])]
    print("Filter 3:\n", filter3)

    # 2. Метод .query() (удобно для сложных условий)
    # Строка запроса похожа на SQL WHERE
    filter4 = df.query('col_A > 2 and col_B == "z"')
    print("Filter 4:\n", filter4)
    # Можно использовать переменные с @
    threshold = 15
    filter5 = df.query('col_C > @threshold')
    print("Filter 5:\n", filter5)
    \end{codebox}
\end{myblock}

\begin{myblock}{{Группировка и Агрегация (\texttt{groupby().agg()})}}
    Разделяй (по группам), Властвуй (применяй функцию), Объединяй (результаты).
    \textit{Аналогия:} Разложить все фрукты по корзинам (группировка по типу фрукта), затем посчитать вес яблок, средний размер апельсинов и т.д. (агрегация), и записать результаты для каждой корзины.
    \begin{codebox}{python}{Пример GroupBy (используем df из блока "Чтение")}
    # Сгруппировать по 'col_B' (категориальный столбец),
    # посчитать сумму 'col_C' (числовой)
    # и среднее/количество для 'col_A' (числовой)
    agg_results = df.groupby('col_B').agg(
        total_C=('col_C', 'sum'),       # Новый столбец = ('старый', 'функция')
        average_A=('col_A', 'mean'),
        count_A=('col_A', 'count')
    )
    print("Aggregation Results:\n", agg_results)
    # agg_results будет DataFrame с 'col_B' в индексе

    # Можно группировать по нескольким столбцам
    # multi_group = df.groupby(['cat1', 'cat2']).size() # размер групп
    \end{codebox}
\end{myblock}

\begin{myblock}{{Объединение Таблиц (\texttt{merge}, \texttt{join}, \texttt{concat})}}
    Склеивание данных из разных источников.
    \begin{itemize}
        \item \textbf{`pd.merge(df1, df2, on='key', how='...')`}: Похоже на \textbf{SQL JOIN}. Объединяет по общим столбцам (`on='key'`) или индексам. `how` определяет тип: `'inner'` (только общие ключи), `'outer'` (все ключи), `'left'`, `'right'`.
        \item \textbf{`pd.concat([df1, df2], axis=...)`}: Простое \textbf{склеивание} таблиц. `axis=0` (по умолчанию) - добавить строки df2 под df1. `axis=1` - добавить столбцы df2 справа от df1 (требует совпадения индексов или аккуратности).
        \item \textbf{`df1.join(df2)`}: Удобный метод для объединения по \textbf{индексу} (похож на `merge` с `left\_index=True`, `right\_index=True`).
    \end{itemize}
    \textit{Аналогия:} `merge` - как найти общих друзей (ключи) в двух списках контактов. `concat` - как приклеить один список контактов под другим (`axis=0`) или положить их рядом (`axis=1`).
    \begin{codebox}{python}{Примеры Merge и Concat}
    # Создадим два DataFrame для примеров
    df1 = pd.DataFrame({'user_id': ['u1', 'u2', 'u3'], 'value1': [10, 20, 30]})
    df2 = pd.DataFrame({'user_id': ['u2', 'u3', 'u4'], 'value2': [55, 66, 77]})
    df3 = pd.DataFrame({'user_id': ['u5', 'u6'], 'value1': [40, 50]}) # Для concat

    # Merge (Inner Join по 'user_id')
    df_merged = pd.merge(df1, df2, on='user_id', how='inner')
    print("Merged DF:\n", df_merged)

    # Concat (склеить строки df1 и df3)
    df_concatenated = pd.concat([df1, df3], axis=0, ignore_index=True)
    print("Concatenated DF (rows):\n", df_concatenated)
    # ignore_index=True сбрасывает исходные индексы
    \end{codebox}
\end{myblock}

\begin{myblock}{{Применение Функций (\texttt{.apply()})}}
    Для сложных операций, которые нельзя сделать стандартными методами (используем `df` из блока "Чтение").
    \begin{codebox}{python}{Примеры .apply()}
    # Применить функцию к каждому элементу столбца (Series)
    df['C_percent'] = df['col_C'].apply(lambda x: x / df['col_C'].sum() * 100)
    print("DF with C_percent:\n", df)

    # Применить функцию к каждой строке (axis=1)
    def custom_logic(row):
        # row - это Series, представляющий строку
        return row['col_A'] * 10 + row['col_C'] if row['col_B'] == 'x' else row['col_C']

    df['result'] = df.apply(custom_logic, axis=1)
    print("DF with result:\n", df)
    \end{codebox}
    \textbf{Осторожно:} `.apply()` может быть медленным на больших данных. По возможности используй встроенные векторизованные операции NumPy/Pandas.
\end{myblock}

% --- Раздел 3: Чтение/Запись Файлов ---
\section{Чтение и Запись Файлов (Базовый Python)}

\begin{textbox}{Зачем?}
    Хотя Pandas отлично работает с CSV/Excel, иногда нужно работать с обычными текстовыми файлами (логи, конфиги, .txt). Важно делать это безопасно.
    \textit{Аналогия:} Файл - это коробка. Открытие/закрытие - это работа с крышкой. Конструкция `with open(...)` гарантирует, что ты \textbf{всегда закроешь коробку}, даже если что-то пойдет не так при упаковке/распаковке.
\end{textbox}

\begin{myblock}{{Стандартный способ Python (\texttt{with open})}}
    Рекомендуемый подход для работы с файлами.
    \begin{codebox}{python}{Чтение и запись текстовых файлов}
    # Запись в файл (перезапишет, если существует)
    lines_to_write = ["Строка 1\n", "Строка 2\n"]
    try: # Добавим try-except для случая, если файл недоступен
        with open('my_output.txt', 'w', encoding='utf-8') as f:
            f.write("Одна строка\n")
            f.writelines(lines_to_write)
        print("Файл 'my_output.txt' успешно записан.")
    except IOError as e:
        print(f"Ошибка записи файла: {e}")

    # Чтение из файла
    try:
        with open('my_output.txt', 'r', encoding='utf-8') as f:
            print("\nЧитаем файл 'my_output.txt':")
            # content_str = f.read()      # Прочитать весь файл в строку
            # content_list = f.readlines() # Прочитать все строки в список
            for line in f:             # Читать построчно (эффективно)
                print(line.strip())
    except FileNotFoundError:
        print("Файл 'my_output.txt' не найден.")
    except IOError as e:
         print(f"Ошибка чтения файла: {e}")


    # Режимы: 'r' - чтение, 'w' - запись (перезапись), 'a' - дозапись в конец
    # encoding='utf-8' - важно для русского языка!
    \end{codebox}
\end{myblock}

\begin{myblock}{{Pandas для Структурированных Файлов}}
    Для CSV, Excel, JSON, SQL и др. используй встроенные функции Pandas (используем `df` из блока "Применение Функций").
    \begin{codebox}{python}{Запись в CSV/Excel с Pandas}
    try:
        # Сохранить DataFrame в CSV без индекса
        df.to_csv('output_data.csv', index=False, encoding='utf-8')
        print("\nDataFrame успешно сохранен в 'output_data.csv'")

        # Сохранить DataFrame в Excel (раскомментируйте, если нужен Excel)
        # df.to_excel('output_data.xlsx', index=False, sheet_name='MyData')
        # print("DataFrame успешно сохранен в 'output_data.xlsx'")
    except Exception as e:
        print(f"Ошибка сохранения DataFrame: {e}")
    \end{codebox}
\end{myblock}

% --- Раздел 4: Визуализация ---
\section{Визуализация: Matplotlib и Seaborn}

\begin{textbox}{Зачем?}
    "Лучше один раз увидеть, чем сто раз услышать". Графики помогают понять данные, найти паттерны, выбросы и представить результаты.
    \begin{itemize}
        \item \textbf{Matplotlib (`plt`):} Низкоуровневая библиотека, дает полный контроль. \textit{Аналогия:} Набор инструментов художника (холст, кисти, краски).
        \item \textbf{Seaborn (`sns`):} Высокоуровневая, построена на Matplotlib. Упрощает создание красивых статистических графиков, хорошо интегрируется с Pandas. \textit{Аналогия:} Готовые шаблоны или трафареты для рисования стандартных фигур.
    \end{itemize}
\end{textbox}

\begin{myblock}{{Основы Matplotlib (\texttt{plt})}}
    Базовые команды для создания простых графиков.
    \begin{codebox}{python}{Примеры Matplotlib}
    import matplotlib.pyplot as plt
    import numpy as np
    import pandas as pd # Нужен для примера scatter

    # Данные для графиков
    x_lin = np.linspace(0, 10, 100)
    y1_lin = np.sin(x_lin)
    y2_lin = np.cos(x_lin)
    data_hist = np.random.randn(1000)
    data_box = [np.random.normal(0, std, 100) for std in range(1, 4)]
    # DataFrame для scatter
    df_scatter = pd.DataFrame({'feature1': np.random.rand(50) * 10,
                               'feature2': np.random.rand(50) * 10})

    # --- Создание фигур ---
    # Линейный график
    plt.figure(figsize=(8, 4)) # Размер фигуры (опционально)
    plt.plot(x_lin, y1_lin, label='sin(x)')
    plt.plot(x_lin, y2_lin, label='cos(x)', linestyle='--')
    plt.title('Линейный график (Matplotlib)')
    plt.xlabel('Ось X')
    plt.ylabel('Ось Y')
    plt.legend() # Показать легенду
    plt.grid(True) # Добавить сетку

    # Диаграмма рассеяния (Scatter plot)
    plt.figure()
    plt.scatter(df_scatter['feature1'], df_scatter['feature2'], alpha=0.5) # alpha - прозрачность
    plt.title('Диаграмма рассеяния (Matplotlib)')
    plt.xlabel('Feature 1')
    plt.ylabel('Feature 2')

    # Гистограмма
    plt.figure()
    plt.hist(data_hist, bins=30, color='skyblue', edgecolor='black')
    plt.title('Гистограмма (Matplotlib)')
    plt.xlabel('Значение')
    plt.ylabel('Частота')

    # Ящик с усами (Box plot)
    plt.figure()
    plt.boxplot(data_box, labels=['Gr1', 'Gr2', 'Gr3'])
    plt.title('Ящик с усами (Matplotlib)')
    plt.ylabel('Значение')

    # ВАЖНО: Отображение графиков plt.show() будет в конце секции
    \end{codebox}
\end{myblock}

\begin{myblock}{{Seaborn (\texttt{sns}) для Статистики}}
    Более красивые и статистически ориентированные графики, часто одной строкой.
    \begin{codebox}{python}{Примеры Seaborn}
    import seaborn as sns
    import matplotlib.pyplot as plt # Часто используется для донастройки sns графиков

    # Загрузка примера данных из seaborn
    tips = sns.load_dataset("tips") # DataFrame с данными о чаевых

    # --- Создание фигур ---
    # Линейный график (с доверительным интервалом по умолчанию)
    plt.figure() # Можно управлять размером через plt
    sns.lineplot(x="total_bill", y="tip", data=tips)
    plt.title('Линейный график (Seaborn)')

    # Диаграмма рассеяния (с возможностью раскраски по категории)
    plt.figure()
    sns.scatterplot(x="total_bill", y="tip", hue="time", data=tips)
    plt.title('Диаграмма рассеяния (Seaborn)')

    # Гистограмма (с оценкой плотности KDE)
    plt.figure()
    sns.histplot(data=tips, x="total_bill", kde=True)
    plt.title('Гистограмма (Seaborn)')

    # Ящик с усами (удобно для сравнения по категориям)
    plt.figure()
    sns.boxplot(x="day", y="total_bill", data=tips)
    plt.title('Ящик с усами (Seaborn)')

    # ВАЖНО: Отображение графиков plt.show() будет в конце секции
    \end{codebox}
    Seaborn часто автоматически подписывает оси и создает легенды, используя имена столбцов DataFrame.
\end{myblock}

\begin{myexampleblock}{Когда Какой График Использовать?}
    Краткий гид по выбору типа визуализации:
    \begin{itemize}
        \item \textbf{Линейный график (\texttt{plot}/\texttt{lineplot}):} Показать \textbf{тренд} или изменение показателя во времени (или по другой непрерывной оси). Сравнение трендов нескольких групп.
        \item \textbf{Диаграмма рассеяния (\texttt{scatter}/\texttt{scatterplot}):} Посмотреть \textbf{взаимосвязь} между двумя \textit{числовыми} переменными. Помогает найти корреляции, кластеры, выбросы.
        \item \textbf{Гистограмма (\texttt{hist}/\texttt{histplot}):} Понять \textbf{распределение} одной \textit{числовой} переменной. Как часто встречаются те или иные значения? Есть ли пики? Симметрично ли распределение?
        \item \textbf{Ящик с усами (\texttt{boxplot}):} Сравнить распределения \textit{числовой} переменной по нескольким \textit{категориям}. Показывает медиану, квартили, разброс и потенциальные выбросы в каждой группе.
    \end{itemize}
\end{myexampleblock}

% --- Команда для отображения всех графиков ---
% Поместите эту команду после всех блоков с графиками
\begin{center}
\textit{Для отображения всех созданных выше графиков Matplotlib/Seaborn, выполните:}
\begin{codebox}{python}{Показать все графики}
# Эта команда должна быть вызвана один раз после всех команд plt.figure()/sns.*plot()
# В средах типа Jupyter Notebook/Lab графики могут отображаться автоматически.
# В обычных Python скриптах plt.show() обязателен.
plt.show()
\end{codebox}
\end{center}


% --- Конец контента ---

% >>> КОММЕНТАРИЙ: Печатаем библиографию (если используется \cite)
\AtNextBibliography{\footnotesize} % Уменьшаем шрифт для библиографии
%\printbibliography % Раскомментируйте, если добавляли цитаты через \cite или \footcite

\end{multicols}

\end{document}