\documentclass[10pt,a4paper]{article}

\usepackage[landscape,margin=0.7cm]{geometry}
\usepackage[russian,english]{babel} % Поддержка русского и английского

% >>> КОММЕНТАРИЙ: Определяем переменные для заголовка и автора (можно изменить)
\newcommand{\cheatsheettitle}{\color{w3schools}Шпаргалка по Random forest / {\color{alert} Концепции} {\color{black} Cheatsheet (XeLaTeX)}}
\newcommand{\cheatsheetauthor}{Краткий справочник}

% >>> КОММЕНТАРИЙ: Подключаем основной файл шаблона с настройками стилей и пакетов
% \documentclass{article} % ЗАКОММЕНТИРОВАНО: Добавлено для возможности компиляции этого фрагмента отдельно для проверки
\usepackage{fontspec}

% >>> КОММЕНТАРИЙ: Установка основных шрифтов документа (требует установленных шрифтов в системе)
\setmainfont{PT Sans} % Основной шрифт для текста
\setsansfont{PT Sans} % Шрифт без засечек (если используется отдельно)
\setmonofont{Liberation Mono} % Моноширинный шрифт (для кода)

\usepackage{fontawesome} % Для иконок (например, \faQuoteLeft)
\usepackage{hyperref} % Для создания кликабельных ссылок (в т.ч. в \resourcelink)
\usepackage{enumitem} % Для настройки списков
\usepackage{lipsum} % Для генерации текста-рыбы (не используется в финальной версии)
\usepackage{xcolor} % Для определения и использования цветов
\usepackage{minted} % Для подсветки синтаксиса кода (требует Python и Pygments, компиляция с -shell-escape)

% >>>>>> НОВОЕ: Пакет для создания графиков <<<<<<
\usepackage{pgfplots}
\pgfplotsset{compat=newest} % Устанавливаем последнюю версию совместимости
\usetikzlibrary{arrows.meta, shapes.geometric} % Загружаем полезные библиотеки TikZ

\usepackage{titlesec} % Для настройки заголовков секций
% >>> КОММЕНТАРИЙ: Настройка отступов для заголовков секций (\section)
\titlespacing*{\section}{0pt}{*1.5}{*0.8} % {отступ слева}{отступ сверху}{отступ снизу}

% >>> КОММЕНТАРИЙ: Форматирование заголовков секций (\section)
\titleformat{\section}
  {\normalfont\large\bfseries\color{mDarkTeal}} % Стиль заголовка (цвет, размер, жирность)
  {\thesection} % Номер секции
  {1em} % Отступ после номера
  {} % Код перед заголовком (здесь пусто)

% >>> КОММЕНТАРИЙ: Определение пользовательских цветов через HTML-коды
\definecolor{customcolor}{HTML}{696EA8} % Основной цвет для блоков textbox
\definecolor{alert}{HTML}{CD5C5C} % Цвет для выделения (красный)
\definecolor{w3schools}{HTML}{4CAF50} % Цвет для выделения (зеленый)
\definecolor{subbox}{gray}{0.60} % Цвет для под-блоков (не используется?)
\definecolor{codecolor}{HTML}{FFC300} % Цвет для кода (не используется?)
\definecolor{mDarkTeal}{HTML}{23373b} % Темно-бирюзовый (для заголовков секций, названий блоков)
\colorlet{beamerboxbg}{black!2} % Фон для beamerbox (не используется?)
\colorlet{beamerboxfg}{mDarkTeal} % Цвет текста для beamerbox (не используется?)
\definecolor{mDarkBrown}{HTML}{604c38} % Темно-коричневый (не используется?)
\definecolor{mLightBrown}{HTML}{EB811B} % Светло-коричневый (для заголовков myalertblock)
\definecolor{mLightGreen}{HTML}{14B03D} % Светло-зеленый (для заголовков myexampleblock)
\definecolor{alertTextBox}{HTML}{A8696E} % Цвет для блоков alerttextbox

% >>> КОММЕНТАРИЙ: Настройка пакета biblatex для библиографии
\usepackage
[citestyle=authoryear, % Стиль цитирования (Автор-Год)
sorting=nty, % Сортировка по Имени, Году, Названию
autocite=footnote, % Команда \autocite создает сноски
autolang=hyphen, % Автоматическое определение языка для переносов
mincrossrefs=1, % Минимальное количество перекрестных ссылок
backend=biber] % Используемый бэкенд (требует запуск biber)
{biblatex}

% >>> КОММЕНТАРИЙ: Настройка формата пост-заметок в цитатах (например, номера страниц)
\DeclareFieldFormat{postnote}{#1}
\DeclareFieldFormat{multipostnote}{#1}
\DeclareAutoCiteCommand{footnote}[f]{\footcite}{\footcites} % Настройка команды \autocite

% >>> КОММЕНТАРИЙ: Подключение файла с библиографическими записями
\addbibresource{literature.bib}

% >>> КОММЕНТАРИЙ: Подключение библиотеки tcolorbox для создания цветных блоков
\usepackage{tcolorbox}
\tcbuselibrary{most, listingsutf8, minted} % Загрузка необходимых библиотек tcolorbox (most включает breakable)

% >>> КОММЕНТАРИЙ: Глобальные настройки для маленьких tcbox (inline блоки)
\tcbset{tcbox width=auto,left=1mm,top=1mm,bottom=1mm,
right=1mm,boxsep=1mm,middle=1pt}

% >>> КОММЕНТАРИЙ: Определение базового окружения для цветных блоков с заголовком
% breakable: позволяет блоку разрываться между колонками/страницами.
% ВАЖНО: В multicols автоматический разрыв breakable может работать не всегда идеально.
% Если блок разрывается некрасиво, попробуйте разбить контент на несколько блоков
% или используйте \needspace{<длина>} перед блоком.
\newenvironment{mycolorbox}[2]{
\begin{tcolorbox}[capture=minipage,fonttitle=\large\bfseries, enhanced,boxsep=1mm,colback=#1!30!white,
width=\linewidth, % Ширина блока равна ширине текущей колонки
arc=2pt,outer arc=2pt, toptitle=0mm,colframe=#1,opacityback=0.7,nobeforeafter,
breakable, % Разрешить разрыв блока
title=#2] % Текст заголовка блока
}{\end{tcolorbox}}

% >>> КОММЕНТАРИЙ: Окружение subbox (не используется в текущем коде)
\newenvironment{subbox}[2]{
\begin{tcolorbox}[capture=minipage,fonttitle=\normalsize\bfseries, enhanced,boxsep=1mm,colback=#1!30!white,on line,tcbox width=auto,left=0.3em,top=1mm, toptitle=0mm,colframe=#1,opacityback=0.7,nobeforeafter,
breakable,
title=#2]\footnotesize
}{\normalsize\end{tcolorbox}\vspace{0.1em}}

% >>> КОММЕНТАРИЙ: Окружение для размещения tcolorbox'ов в несколько колонок внутри блока (не используется)
\newenvironment{multibox}[1]{
\begin{tcbraster}[raster columns=#1,raster equal height,nobeforeafter,raster column skip=1em,raster left skip=1em,raster right skip=1em]}{\end{tcbraster}}

% >>> КОММЕНТАРИЙ: Окружения-обертки для mycolorbox с предопределенными цветами и заголовками
\newenvironment{textbox}[1]{\begin{mycolorbox}{customcolor}{#1}}{\end{mycolorbox}} % Обычный текстовый блок
\newenvironment{alerttextbox}[1]{\begin{mycolorbox}{alertTextBox}{#1}}{\end{mycolorbox}} % Блок для предупреждений/важной информации

% >>> КОММЕНТАРИЙ: Определение окружения для блоков с кодом с подсветкой minted
\newtcblisting{codebox}[3][]{ % [опции minted], {язык}, {заголовок}
colback=black!10,colframe=black!20, % Цвета фона и рамки
listing only, % Показывать только листинг кода
minted options={ % Опции, передаваемые в minted
    numbers=left, % Нумерация строк слева
    style=default, % Стиль подсветки (можно выбрать другие, напр., 'friendly')
    fontsize=\scriptsize, % Размер шрифта кода
    breaklines, % Автоматический перенос длинных строк
    breaksymbolleft=, % Символ для обозначения переноса (пусто)
    autogobble, % Удаление лишних отступов слева
    linenos, % Показывать номера строк
    numbersep=0.7em, % Отступ номеров строк от кода
    #1 % Дополнительные опции minted, переданные в [1]
    },
enhanced,top=1mm, toptitle=0mm,
left=5mm, % Отступ слева внутри блока (для номеров строк)
arc=0pt,outer arc=0pt, % Прямые углы
title={#3}, % Заголовок блока кода
fonttitle=\small\bfseries\color{mDarkTeal}, % Стиль заголовка блока кода
listing engine=minted, % Движок для листинга - minted
minted language=#2, % Язык программирования для подсветки
breakable % Разрешить разрыв блока кода
}

\newcommand{\punkti}{~\lbrack\dots\rbrack~} % Команда для [...] (не используется)

% >>> КОММЕНТАРИЙ: Переопределение окружения quote для добавления иконок цитат
\renewenvironment{quote}
               {\list{\faQuoteLeft\phantom{ }}{\rightmargin\leftmargin}
                \item\relax\scriptsize\ignorespaces}
               {\unskip\unskip\phantom{xx}\faQuoteRight\endlist}

% >>> КОММЕНТАРИЙ: Вспомогательные команды для создания цветных фонов под текстом (не используются)
\newcommand{\bgupper}[3]{\colorbox{#1}{\color{#2}\huge\bfseries\MakeUppercase{#3}}}
\newcommand{\bg}[3]{\colorbox{#1}{\bfseries\color{#2}#3}}

% >>> КОММЕНТАРИЙ: Команда для форматирования описания команды/функции
% #1: Сама команда (используется \detokenize для корректного отображения спецсимволов)
% #2: Описание команды
\newcommand{\mycommand}[2]{
  {\par\noindent\ttfamily\detokenize{#1}\par} % Вывод команды моноширинным шрифтом
  \nopagebreak % Стараемся не разрывать страницу после команды
  \hangindent=1.5em \hangafter=1 \noindent % Делаем отступ для описания
  \small\textit{#2}\par\vspace{0.5ex} % Вывод описания и небольшой отступ после
}

% >>> КОММЕНТАРИЙ: Команда для форматирования ссылок на ресурсы
% #1: URL ссылки
% #2: Текст ссылки
% #3: Описание ресурса
\newcommand{\resourcelink}[3]{
  {\par\noindent\href{#1}{\ttfamily #2}\par} % Вывод кликабельной ссылки
  \nopagebreak
  \hangindent=1.5em \hangafter=1 \noindent
  \small\textit{#3}\par\vspace{0.5ex} % Вывод описания
}

% >>> КОММЕНТАРИЙ: Вспомогательные команды для стилизации
\newcommand{\sep}{{\scriptsize~\faCircle{ }~}} % Разделитель-кружок
\newcommand{\bggreen}[1]{\medskip\bgupper{w3schools}{black}{#1}\\[0.5em]} % Зеленый заголовок (не используется)
\newcommand{\green}[1]{\smallskip\bg{w3schools}{white}{#1}\\} % Текст на зеленом фоне
\newcommand{\red}[1]{\smallskip\bg{alert}{white}{#1}\\} % Текст на красном фоне
\newcommand{\alertcmd}[1]{\red{#1}\\} % Псевдоним для red (переименовано чтобы не конфликтовать с цветом alert)

\usepackage{multicol} % Пакет для создания нескольких колонок
\setlength{\columnsep}{15pt} % Расстояние между колонками

\setlength{\parindent}{0pt} % Убираем абзацный отступ по умолчанию
\usepackage{csquotes} % Для правильного отображения кавычек в разных языках (используется biblatex)
\newcommand{\loremipsum}{Lorem ipsum dolor sit amet.} % Пример текста (не используется)

% >>> КОММЕНТАРИЙ: Окружения для блоков с разными цветами заголовков (пример, предупреждение, обычный)
% Основаны на tcolorbox, похожи на mycolorbox, но с другими цветами заголовков и рамки.
% Также используют breakable. Применяются для выделения примеров, предупреждений и т.д.
% ВАЖНО: Лог может содержать предупреждения "Underfull \hbox (badness 10000)".
% Это часто случается в узких колонках multicol при наличии кода, URL или текста,
% который плохо переносится. Обычно это лишь косметическая проблема, строки выглядят
% немного не до конца заполненными. Можно игнорировать или попробовать перефразировать/
% использовать \sloppy перед проблемным абзацем.
\newenvironment{myexampleblock}[1]{ % Блок для примеров (зеленый заголовок)
    \tcolorbox[capture=minipage,fonttitle=\small\bfseries\color{mLightGreen}, enhanced,boxsep=1mm,colback=black!10,breakable,noparskip,
    on line,tcbox width=auto,left=0.3em,top=1mm, toptitle=0mm,
    colframe=black!20,arc=0pt,outer arc=0pt,
    opacityback=0.7,nobeforeafter,title=#1]}
    {\endtcolorbox}

\newenvironment{myalertblock}[1]{ % Блок для предупреждений (оранжевый заголовок)
    \tcolorbox[capture=minipage,fonttitle=\small\bfseries\color{mLightBrown}, enhanced,boxsep=1mm,colback=black!10,breakable,noparskip,
    on line,tcbox width=auto,left=0.3em,top=1mm, toptitle=0mm,
    colframe=black!20,arc=0pt,outer arc=0pt,
    opacityback=0.7,nobeforeafter,title=#1]}
    {\endtcolorbox}

\newenvironment{myblock}[1]{ % Обычный информационный блок (бирюзовый заголовок)
    \tcolorbox[capture=minipage,fonttitle=\small\bfseries\color{mDarkTeal}, enhanced,boxsep=1mm,colback=black!10,breakable,noparskip,
    on line,tcbox width=auto,left=0.3em,top=1mm, toptitle=0mm,
    colframe=black!20, arc=0pt,outer arc=0pt,
    opacityback=0.7,nobeforeafter,title=#1]}
    {\endtcolorbox}

% >>> КОММЕНТАРИЙ: Команда для вставки изображения в рамке tcolorbox с подписью
% #1: Опции (не используются)
% #2: Путь к файлу изображения (используется как подпись под изображением)
\newcommand{\mygraphics}[2][]{
\tcbox[enhanced,boxsep=0pt,top=0pt,bottom=0pt,left=0pt,
right=0pt,boxrule=0.4pt,drop fuzzy shadow,clip upper,
colback=black!75!white,toptitle=2pt,bottomtitle=2pt,nobeforeafter,
center title,fonttitle=\small\sffamily,title=\detokenize{#2}] % Используем путь как заголовок
{\includegraphics[width=\the\dimexpr(\linewidth-4mm)\relax]{#2}} % Вставляем изображение, ширина чуть меньше колонки
}

% >>> КОМЕНТАРИЙ: Команда \needspace{<длина>}
% Используйте эту команду *перед* блоком (tcolorbox, section, figure), чтобы убедиться,
% что в текущей колонке есть как минимум <длина> свободного места. Если места нет,
% LaTeX начнет новую колонку/страницу перед выполнением команды.
% Пример: \needspace{10\baselineskip} % Запросить место для примерно 10 строк текста
% Это полезно для предотвращения "висячих" заголовков или некрасивых разрывов блоков.

% >>>>>> НОВОЕ: Стили для графиков PGFPlots <<<<<<
\pgfplotsset{
    % Базовый стиль для всех графиков в шпаргалке
    cheatsheet plot style/.style={
        width=0.98\linewidth, % Ширина чуть меньше колонки для аккуратности
        height=5cm,          % Высота графика (можно изменить по необходимости)
        grid=major,          % Включаем основную сетку
        grid style={dashed, color=black!20}, % Стиль сетки: пунктирная, светло-серая
        axis lines=left,     % Линии осей только слева и снизу
        axis line style={color=black!60, thick}, % Стиль линий осей: серые, потолще
        tick style={color=black!60, thick}, % Стиль меток на осях
        ticklabel style={font=\scriptsize, color=black}, % Стиль подписей меток: маленький шрифт
        label style={font=\small, color=mDarkTeal}, % Стиль подписей осей: шрифт small, цвет как у заголовков блоков
        title style={font=\small\bfseries, color=mDarkTeal}, % Стиль заголовка графика
        legend style={        % Стиль легенды
            font=\scriptsize, % Маленький шрифт
            draw=black!30,    % Легкая рамка вокруг легенды
            fill=white,       % Белый фон
            legend cell align=left, % Выравнивание текста в ячейке легенды
            anchor=north east, % Якорь для позиционирования
            at={(rel axis cs:0.98,0.98)} % Положение в правом верхнем углу внутри области графика
        },
        cycle list={ % Список стилей для линий \addplot (цвета взяты из шаблона)
            {blue, mark=*, thick},
            {alert, mark=square, thick},
            {w3schools, mark=triangle, thick},
            {mLightBrown, mark=diamond, thick},
            {customcolor, mark=oplus, thick},
            {black!50, mark=pentagon, thick},
        },
        % Убираем рамку вокруг графика по умолчанию
        axis background/.style={fill=none},
    },
    % Дополнительный стиль для ROC-кривой (без сетки, с диагональю)
    roc curve style/.style={
        cheatsheet plot style, % Наследуем базовый стиль
        width=6cm, height=6cm, % Делаем квадратным
        grid=none, % Убираем сетку
        xlabel={False Positive Rate (FPR)},
        ylabel={True Positive Rate (TPR)},
        xmin=0, xmax=1,
        ymin=0, ymax=1,
        xtick={0, 0.5, 1},
        ytick={0, 0.5, 1},
        legend pos=south east, % Легенда внизу справа
        % Добавляем диагональную линию случайного угадывания
        extra x ticks={0.5}, extra y ticks={0.5},
        extra tick style={grid=major, grid style={dashed, color=black!40}},
        after end axis/.code={ % Добавляем линию после отрисовки осей
            \draw[dashed, color=black!40] (axis cs:0,0) -- (axis cs:1,1);
        }
    },
    % Дополнительный стиль для PR-кривой
    pr curve style/.style={
        cheatsheet plot style, % Наследуем базовый стиль
        xlabel={Recall (TPR)},
        ylabel={Precision},
        xmin=0, xmax=1,
        ymin=0, ymax=1,
        legend pos=south west, % Легенда внизу слева
    }
}

% % Добавлено для возможности компиляции этого фрагмента отдельно для проверки
% \begin{document}
% \begin{multicols}{2} % Пример использования в multicols
% \lipsum[1] % Просто текст для заполнения

% \begin{textbox}{Пример графика (Базовый стиль)}
%     \begin{center}
%     \begin{tikzpicture}
%         \begin{axis}[cheatsheet plot style, title={Пример $y=x^2$ и $y=x+1$}]
%         \addplot coordinates {(0, 0) (1, 1) (2, 4) (3, 9) (4, 16)};
%         \addlegendentry{$y=x^2$}
%         \addplot coordinates {(0, 1) (1, 2) (2, 3) (3, 4) (4, 5)};
%         \addlegendentry{$y=x+1$}
%         \end{axis}
%     \end{tikzpicture}
%     \end{center}
% \end{textbox}

% \begin{myexampleblock}{Пример ROC-кривой}
%     \begin{center}
%     \begin{tikzpicture}
%         \begin{axis}[roc curve style, title={Пример ROC Curve}]
%         % Пример данных ROC кривой
%         \addplot coordinates { (0,0) (0.1,0.3) (0.2,0.6) (0.4,0.8) (0.6,0.9) (0.8,0.95) (1,1) };
%         \addlegendentry{Модель A (AUC $\approx$ 0.8)}
%         \addplot coordinates { (0,0) (0.2,0.2) (0.4,0.4) (0.6,0.6) (0.8,0.8) (1,1) }; % Пример хуже
%         \addlegendentry{Модель B (AUC = 0.5)}
%         \end{axis}
%     \end{tikzpicture}
%     \end{center}
% \end{myexampleblock}

% \begin{myblock}{Пример PR-кривой}
%     \begin{center}
%     \begin{tikzpicture}
%         \begin{axis}[pr curve style, title={Пример Precision-Recall Curve}]
%         % Пример данных PR кривой (могут сильно зависеть от порога)
%         \addplot coordinates { (0,0.9) (0.1,0.85) (0.3,0.8) (0.5,0.7) (0.7,0.6) (0.9,0.4) (1,0.3) };
%         \addlegendentry{Модель C}
%         \end{axis}
%     \end{tikzpicture}
%     \end{center}
% \end{myblock}

% \lipsum[2-3] % Просто текст для заполнения
% \end{multicols}
% \end{document}

\begin{document}
\pagestyle{empty} % Убираем номера страниц
\small % Уменьшаем базовый размер шрифта для всего документа

% >>> КОММЕНТАРИЙ: Используем multicols для создания трех колонок.
\begin{multicols}{3}
\raggedcolumns % Не растягивать колонки по вертикали

\noindent
\begin{minipage}{\linewidth}
    \centering
    {\bfseries\huge \cheatsheettitle \par}
    \vspace{1ex}
    {\large \cheatsheetauthor \par}
    \vspace{0.5ex}
    {\normalsize \today \par} % Вставляем текущую дату
\end{minipage}
\vspace{2ex}

\thispagestyle{empty} % Убеждаемся, что и на первой странице нет номера

\scriptsize % Уменьшаем шрифт для оглавления и основного текста
\tableofcontents % Генерируем оглавление

% >>> КОММЕНТАРИЙ: Подключаем основной контент шпаргалки
% >>> Контент для шпаргалки "Случайный Лес (Random Forest)" - v2 (Разбивка)

% ================================================
% Введение в Случайный Лес
% ================================================
\begin{myblock}{Случайный Лес: Определение}
    \textbf{Случайный Лес (Random Forest, RF)} — это ансамблевый метод машинного обучения, который строит множество деревьев решений во время обучения и выводит класс, который является модой классов (классификация) или средним предсказанием (регрессия) отдельных деревьев. Это один из самых популярных и эффективных "из коробки" алгоритмов.
\end{myblock}

\begin{myexampleblock}{Аналогия: Комитет Экспертов}
    Представьте, что для принятия важного решения вы собираете \textbf{комитет разных экспертов} (много деревьев). Каждый эксперт смотрит на проблему немного под своим углом. Вы принимаете решение на основе их коллективного мнения. Случайный лес применяет тот же принцип, но с деревьями решений.
\end{myexampleblock}

% ================================================
% Секция I: Идея Бэггинга (Bagging)
% ================================================
\section{Идея Бэггинга (Bagging)}

\begin{textbox}{Определение: Bagging = Bootstrap Aggregating}
    \textbf{Бэггинг} — это основной принцип, лежащий в основе Случайного Леса. Он сочетает две ключевые идеи: Bootstrap и Aggregating.
\end{textbox}

\begin{myblock}{Шаг 1: Bootstrap (Бутстрэп)}
    Создается множество (\(N\)) подвыборок из исходного обучающего датасета. Каждая подвыборка формируется путем случайного выбора объектов \textbf{с возвращением}.
    \begin{itemize}[nosep, leftmargin=*]
        \item Некоторые объекты могут попасть в одну подвыборку несколько раз.
        \item Некоторые объекты могут не попасть ни разу.
        \item Размер каждой подвыборки обычно равен размеру исходного датасета.
    \end{itemize}
    Объекты, не попавшие в конкретную бутстрэп-выборку (\(\approx 37\%\)), называются \textbf{Out-of-Bag (OOB)}.
\end{myblock}

\begin{myblock}{Шаг 2: Aggregating (Агрегация)}
    На каждой Bootstrap-подвыборке независимо обучается своя модель (в случае RF — дерево решений). Затем предсказания всех \(N\) моделей агрегируются:
    \begin{itemize}[nosep, leftmargin=*]
        \item \textbf{Регрессия:} Усреднение предсказаний.
        \item \textbf{Классификация:} Голосование большинством (выбор самого популярного класса).
    \end{itemize}
\end{myblock}

\begin{textbox}{Цель Бэггинга и OOB-оценка}
    \textbf{Цель бэггинга:} Снизить \textbf{дисперсию (variance)} ансамблевой модели. Индивидуальные деревья могут иметь высокую дисперсию (переобучаться), но усреднение их предсказаний сглаживает ошибки и повышает устойчивость. \newline
    \textbf{OOB-оценка:} Объекты Out-of-Bag могут использоваться для оценки качества модели (OOB-оценка) без необходимости выделения отдельной валидационной выборки. Для каждого OOB-объекта предсказание делается ансамблем деревьев, которые \textit{не обучались} на этом объекте.
\end{textbox}

% ================================================
% Секция II: Случайный Выбор Признаков (Feature Subsampling)
% ================================================
\section{Случайный Выбор Признаков (Feature Subsampling)}

\begin{myblock}{Дополнительная Случайность в RF}
    В отличие от простого бэггинга деревьев, Случайный Лес вносит \textbf{дополнительный элемент случайности} при построении каждого дерева:
    \begin{itemize}[nosep, leftmargin=*]
        \item При поиске лучшего разбиения (split) в каждом узле дерева, алгоритм рассматривает не все доступные признаки, а только их \textbf{случайное подмножество}.
        \item Размер этого подмножества (\texttt{max\_features}) является важным гиперпараметром.
        \item Типичные значения \texttt{max\_features}: \(\sqrt{p}\) для классификации, \(p/3\) для регрессии (где \(p\) — общее число признаков).
    \end{itemize}
\end{myblock}

\begin{textbox}{Цель Случайного Выбора Признаков: Декорреляция Деревьев}
    \textit{Зачем это нужно?} Это делается для \textbf{декорреляции} деревьев в ансамбле.
    \begin{itemize}[nosep, leftmargin=*]
        \item Если бы все деревья видели все признаки, и существовал бы один очень сильный признак, большинство деревьев использовали бы его для первого (и, возможно, последующих) разбиений.
        \item В результате деревья были бы очень похожими (скоррелированными).
        \item Усреднение предсказаний сильно скоррелированных моделей не дает значительного снижения дисперсии.
    \end{itemize}
    Случайный выбор признаков заставляет разные деревья фокусироваться на разных наборах признаков, делая их более \textbf{разнообразными} и независимыми.
\end{textbox}

\begin{myexampleblock}{Аналогия: Разные Аспекты для Экспертов}
    Возвращаясь к комитету экспертов: чтобы они не пришли к одному выводу, опираясь на самый очевидный факт, вы просите каждого эксперта при анализе сосредоточиться только на \textbf{случайном наборе аспектов} проблемы. Это побуждает их исследовать разные стороны вопроса и дает более надежный коллективный результат.
\end{myexampleblock}

% ================================================
% Секция III: Как RF Уменьшает Дисперсию
% ================================================
\section{Как Уменьшает Дисперсию}

\begin{myblock}{Борьба с Переобучением через Усреднение}
    Ключевая сила Случайного Леса — в его способности значительно \textbf{уменьшать дисперсию} по сравнению с одним деревом решений, при этом не сильно увеличивая (или даже немного уменьшая) \textbf{смещение (bias)}.
\end{myblock}

\begin{myblock}{Сравнение Bias/Variance: Одно Дерево vs RF}
    \begin{itemize}[nosep, leftmargin=*]
        \item \textbf{Одно дерево решений:}
            \begin{itemize}[label=\textbullet, nosep, leftmargin=*]
                \item \textit{Низкое смещение:} Может хорошо подогнаться под обучающие данные, уловить сложные зависимости.
                \item \textit{Высокая дисперсия:} Сильно меняется при небольшом изменении данных, легко переобучается.
            \end{itemize}
        \item \textbf{Случайный Лес (RF):}
            \begin{itemize}[label=\textbullet, nosep, leftmargin=*]
                \item \textit{Относительно низкое смещение:} Наследует гибкость от деревьев.
                \item \textit{Значительно сниженная дисперсия:} Благодаря усреднению и декорреляции.
            \end{itemize}
    \end{itemize}
\end{myblock}

\begin{textbox}{Механизмы Снижения Дисперсии в RF}
    \begin{itemize}[nosep, leftmargin=*]
        \item \textbf{Бэггинг (усреднение):} Усреднение предсказаний \(N\) моделей, ошибки которых не полностью скоррелированы, приводит к снижению общей дисперсии ансамбля. Чем больше деревьев (\(N\)), тем ниже дисперсия (до определенного предела).
        \item \textbf{Случайный выбор признаков (декорреляция):} Уменьшает корреляцию между деревьями, что делает усреднение еще более эффективным для снижения дисперсии.
    \end{itemize}
    В итоге, RF достигает хорошего \textbf{компромисса смещения-дисперсии (Bias-Variance Tradeoff)}, создавая модель, которая одновременно гибкая и устойчивая к переобучению.
\end{textbox}

\begin{myexampleblock}{Аналогия: Мудрость Толпы}
    Один человек может сильно ошибаться в оценке (высокая дисперсия). Но если усреднить оценки большой группы людей, где ошибки случайны и не связаны, итоговая оценка будет гораздо ближе к истине (низкая дисперсия). RF использует похожий принцип.
\end{myexampleblock}

% ================================================
% Секция IV: Важность Признаков (Feature Importance)
% ================================================
\section{Важность Признаков (Feature Importance)}

\begin{textbox}{Зачем Оценивать Важность?}
    Хотя Случайный Лес — это ансамбль, что усложняет прямую интерпретацию, он предоставляет методы для оценки вклада каждого признака. Это помогает:
    \begin{itemize}[nosep, leftmargin=*]
        \item Понять, какие данные действительно влияют на модель.
        \item Упростить модель через отбор признаков (Feature Selection).
        \item Получить инсайты о предметной области.
    \end{itemize}
    Существует два основных подхода: MDI (Gini Importance) и Permutation Importance (MDA).
\end{textbox}

% --- IV.A: Mean Decrease in Impurity (MDI) / Gini Importance ---
\subsection{A Mean Decrease in Impurity (MDI) / Gini Importance}

\begin{myblock}{Идея MDI: Вклад в Чистоту Узлов}
    Этот метод оценивает важность признака на основе того, насколько сильно его использование для разделений в деревьях \textbf{уменьшает нечистоту (Impurity)} узлов (например, Gini Impurity для классификации или MSE для регрессии). Признак считается важным, если он часто выбирается для разделения и эти разделения значительно "очищают" данные. Расчет происходит \textbf{на обучающей выборке} во время построения леса.
\end{myblock}

\begin{myexampleblock}{Как Считается MDI (Детально)}
    \begin{enumerate}[nosep, wide, labelindent=0pt]
        \item Обучаем Random Forest.
        \item Для \textbf{каждого дерева} в лесу:
            \begin{itemize}[nosep, leftmargin=*]
                \item Для \textbf{каждого внутреннего узла}, где произошло разделение по признаку $F$:
                    \begin{itemize}[label=\textbullet, nosep, leftmargin=*]
                    \item Рассчитываем \textbf{уменьшение нечистоты (Information Gain / Variance Reduction)} в этом узле: $\Delta Impurity_{node} = Impurity(parent) - WeightedImpurity(children)$.
                    \item Рассчитываем \textbf{взвешенное уменьшение нечистоты}: $WeightedDecrease_{node} = N_{node} \times \Delta Impurity_{node}$, где $N_{node}$ — количество объектов в узле.
                    \end{itemize}
            \end{itemize}
        \item Для \textbf{каждого признака $F$}:
            \begin{itemize}[nosep, leftmargin=*]
                \item Суммируем взвешенные уменьшения нечистоты ($WeightedDecrease_{node}$) по \textbf{всем узлам всех деревьев}, где признак $F$ использовался для разделения. Это дает "общую важность" $TotalImportance(F)$.
            \end{itemize}
        \item \textbf{Нормализация:} Общую важность каждого признака делят на сумму важностей всех признаков: $Importance(F) = \frac{TotalImportance(F)}{\sum_{j} TotalImportance(F_j)}$.
    \end{enumerate}
    \textbf{Формула (концептуально):}
    \[
    Importance_{MDI}(F) \propto \sum_{\text{trees}} \sum_{\substack{\text{nodes split} \\ \text{on } F}} N_{\text{node}} \cdot \Delta Impurity_{\text{node}}
    \]
\end{myexampleblock}

\begin{myblock}{Плюсы MDI}
    \begin{itemize}[nosep, leftmargin=*]
        \item \textbf{Быстро} считается (информация доступна сразу после обучения).
        \item Обычно предоставляется по умолчанию в библиотеках (например, \texttt{feature\_importances\_} в scikit-learn).
    \end{itemize}
\end{myblock}

\begin{alerttextbox}{Минусы и Предостережения по MDI}
    \begin{itemize}[nosep, leftmargin=*]
        \item Склонен \textbf{завышать важность} числовых признаков и категориальных признаков с большим количеством уникальных значений (высокой кардинальностью).
        \item Может давать \textbf{неадекватные результаты для скоррелированных признаков} (важность может "делиться" между ними или присваиваться только одному).
        \item Показывает, насколько признак был \textit{полезен для построения деревьев} на обучающих данных, но не обязательно, насколько он важен для \textit{предсказаний} на новых данных. \textbf{Использовать с осторожностью!}
    \end{itemize}
\end{alerttextbox}

% --- IV.B: Mean Decrease in Accuracy (MDA) / Permutation Importance ---
\subsection{B Mean Decrease in Accuracy (MDA) / Permutation Importance}

\begin{myblock}{Идея MDA: Влияние "Поломки" Признака на Качество}
    Этот метод оценивает важность признака, измеряя, насколько \textbf{ухудшится качество предсказания} модели (например, Accuracy, F1, R², MSE), если "сломать" связь между этим признаком и целевой переменной путем случайного перемешивания его значений. Расчет происходит \textbf{на отложенной (не обучающей!) выборке}.
\end{myblock}

\begin{myexampleblock}{Как Считается Permutation Importance (Детально)}
    \begin{enumerate}[nosep, wide, labelindent=0pt]
        \item Обучаем Random Forest.
        \item Выбираем \textbf{отложенную выборку} (OOB, валидационную или тестовую).
        \item Рассчитываем \textbf{базовую метрику качества} $Score_{base}$ модели на этой выборке.
        \item Для \textbf{каждого признака $F$}:
            \begin{itemize}[nosep, leftmargin=*]
                \item Создаем копию отложенной выборки.
                \item В этой копии \textbf{случайно перемешиваем значения} только в столбце признака $F$.
                \item Делаем предсказания модели на \textbf{модифицированной} выборке.
                \item Рассчитываем метрику качества $Score_{permuted}(F)$ на этих предсказаниях.
                \item \textbf{Важность признака $F$} = $Score_{base} - Score_{permuted}(F)$.
            \end{itemize}
        \item (Опционально, для стабильности) Повторяем шаг 4 несколько раз с разными случайными перемешиваниями для каждого признака и усредняем полученные значения важности.
    \end{enumerate}
    \textbf{Формула (концептуально):}
    \[
    Importance_{Permutation}(F) = Score_{base} - \mathbb{E}[Score_{permuted}(F)]
    \]
    где $\mathbb{E}[\cdot]$ означает ожидаемое значение по разным перемешиваниям.
\end{myexampleblock}

\begin{myblock}{Плюсы Permutation Importance}
    \begin{itemize}[nosep, leftmargin=*]
        \item Более \textbf{надежен}, чем MDI, как показатель реального влияния на предсказания.
        \item Напрямую измеряет влияние признака на \textbf{предсказательную способность} модели на новых данных.
        \item Идея метода \textbf{модель-агностична} (можно применять к любой обученной модели).
    \end{itemize}
\end{myblock}

\begin{alerttextbox}{Минусы и Предостережения по Permutation Importance}
    \begin{itemize}[nosep, leftmargin=*]
        \item \textbf{Вычислительно затратен} (требует многократных предсказаний модели).
        \item Результат может зависеть от конкретной отложенной выборки и случайности перемешивания (рекомендуется усреднять по нескольким запускам).
        \item Интерпретация при \textbf{сильно скоррелированных признаках} требует осторожности: перемешивание одного признака может не сильно ухудшить метрику, если модель может использовать его коррелированный "заменитель". Это может привести к занижению важности обоих признаков.
    \end{itemize}
\end{alerttextbox}

% --- IV.C: Сравнение MDI и Permutation Importance ---
\subsection{C Сравнение MDI и Permutation Importance}

\begin{alerttextbox}{Ключевые Различия (Частый Вопрос на Собеседованиях)}
    \begin{center}
    \begin{tabular}{|l|p{5.5cm}|p{5.5cm}|}
        \hline
        \textbf{Характеристика} & \textbf{MDI (Gini Importance)} & \textbf{Permutation Importance (MDA)} \\
        \hline
        \textbf{Что измеряет?} & Насколько признак \textbf{использовался} для уменьшения нечистоты узлов при \textbf{обучении}. & Насколько "поломка" признака \textbf{влияет} на \textbf{качество предсказания} на \textbf{новых} данных. \\
        \hline
        \textbf{На каких данных?} & Обучающая выборка & Отложенная выборка (OOB, validation, test) \\
        \hline
        \textbf{Скорость} & Быстро & Медленно \\
        \hline
        \textbf{Надежность} & Менее надежен, предвзят к типу признаков, обучающей выборке & Более надежен как показатель предсказательной силы \\
        \hline
        \textbf{Скоррел. признаки} & Может "делить", завышать/занижать важность & Может занижать важность обоих \\
        \hline
        \textbf{Модель-агностичность} & Специфичен для деревьев & Идея применима ко многим моделям \\
        \hline
        \textbf{Основное Применение} & Быстрый анализ, оценка по умолчанию & Надежная оценка влияния, отбор признаков \\
        \hline
    \end{tabular}
    \end{center}
    \textbf{Ключевой вывод:} Permutation Importance обычно считается более надежным показателем реальной важности признака для \textbf{производительности} модели. MDI показывает "популярность" признака при построении модели.
\end{alerttextbox}

% ================================================
% Секция V: Ключевые Гиперпараметры
% ================================================
\section{Ключевые Гиперпараметры}

\begin{textbox}{Основные Параметры для Настройки RF}
    Хотя RF часто хорошо работает "из коробки", тюнинг гиперпараметров может улучшить результат. Важнейшие из них:
\end{textbox}

\begin{myblock}{\texttt{n\_estimators}}
    Количество деревьев в лесу.
    \begin{itemize}[nosep, leftmargin=*]
        \item \textbf{Влияние:} Больше деревьев -> ниже дисперсия ансамбля (до некоторого предела), стабильнее результат, но дольше обучение и предсказание.
        \item \textbf{Типичные значения:} 100, 500, 1000 и более. Обычно выбирают достаточно большим значением, пока производительность на валидации не перестанет расти или время обучения не станет чрезмерным.
    \end{itemize}
\end{myblock}

\begin{myblock}{\texttt{max\_features}}
    Количество признаков, случайно выбираемых для рассмотрения при поиске лучшего сплита в каждом узле.
    \begin{itemize}[nosep, leftmargin=*]
        \item \textbf{Влияние:} Ключевой параметр для контроля корреляции между деревьями. Меньшее значение -> более декоррелированные деревья -> большее снижение дисперсии, но потенциально большее смещение (каждое дерево слабее). Большее значение -> деревья более похожи -> меньшее снижение дисперсии, но потенциально меньшее смещение.
        \item \textbf{Типичные значения (и отправные точки):} \texttt{sqrt(p)} (классификация), \texttt{p/3} или \texttt{log2(p)} (регрессия). Часто требует подбора через кросс-валидацию.
    \end{itemize}
\end{myblock}

\begin{myblock}{Параметры Отдельных Деревьев}
    Гиперпараметры базовых деревьев решений также влияют на лес и могут использоваться для контроля сложности и предотвращения переобучения, хотя RF менее чувствителен к ним, чем одно дерево.
    \begin{itemize}[nosep, leftmargin=*]
        \item \textbf{\texttt{max\_depth}:} Максимальная глубина деревьев. Ограничение глубины уменьшает сложность и дисперсию отдельных деревьев.
        \item \textbf{\texttt{min\_samples\_split}:} Минимальное количество объектов в узле для его разделения.
        \item \textbf{\texttt{min\_samples\_leaf}:} Минимальное количество объектов в листовом узле.
    \end{itemize}
    \textit{Стратегия:} Часто в RF деревья строят почти до максимальной глубины (e.g., \texttt{max\_depth=None}), полагаясь на усреднение и \texttt{max\_features} для контроля переобучения. Однако ограничение глубины или увеличение \texttt{min\_samples\_leaf}/\texttt{min\_samples\_split} может быть полезно для уменьшения размера модели и времени обучения, иногда даже улучшая качество.
\end{myblock}

% ================================================
% Секция VI: Сравнение с Конкурентами
% ================================================
\section{Сравнение с Конкурентами}

% --- VI.A: RF vs. Одно Дерево Решений ---
\subsection{A RF vs. Одно Дерево Решений}

\begin{myblock}{Преимущества RF перед Одним Деревом}
    \begin{itemize}[nosep, leftmargin=*]
        \item Значительно \textbf{меньше переобучается} благодаря усреднению и декорреляции.
        \item Гораздо \textbf{более устойчив} к изменениям в данных (низкая дисперсия).
        \item Обычно показывает \textbf{более высокую точность} и обобщающую способность на практике.
    \end{itemize}
\end{myblock}

\begin{myblock}{Недостатки RF перед Одним Деревом}
    \begin{itemize}[nosep, leftmargin=*]
        \item \textbf{Менее интерпретируем.} Сложно понять логику принятия решения ансамбля по сравнению с одним деревом, которое можно визуализировать.
        \item Требует \textbf{больше вычислительных ресурсов} (память для хранения деревьев, время для обучения и предсказания).
    \end{itemize}
\end{myblock}

% --- VI.B: RF vs. Линейные Модели ---
\subsection{B RF vs. Линейные Модели (Логистическая/Линейная Регрессия)}

\begin{myblock}{Преимущества RF перед Линейными Моделями}
    \begin{itemize}[nosep, leftmargin=*]
        \item Легко улавливает \textbf{нелинейные зависимости} между признаками и целью.
        \item Автоматически обрабатывает \textbf{взаимодействия} между признаками.
        \item \textbf{Не требует масштабирования} признаков (решения в узлах основаны на порогах).
        \item Менее чувствителен к \textbf{выбросам} в признаках.
        \item Часто дает хорошее качество \textbf{"из коробки"} с минимальной предобработкой данных и настройкой.
    \end{itemize}
\end{myblock}

\begin{myblock}{Недостатки RF перед Линейными Моделями}
    \begin{itemize}[nosep, leftmargin=*]
        \item \textbf{Менее интерпретируем}, чем линейные модели, где веса имеют ясный смысл (при условии корректной подготовки данных).
        \item Может быть \textbf{медленнее} в обучении и особенно в предсказании на очень больших датасетах или при большом количестве деревьев.
        \item Плохо \textbf{экстраполирует}. Предсказания RF ограничены диапазоном значений целевой переменной, виденных в обучающих данных (по сути, среднее по листьям). Линейные модели могут экстраполировать.
        \item Может требовать \textbf{значительно больше памяти}.
        \item На \textbf{очень разреженных данных} (много нулей, как в тексте) линейные модели часто работают лучше и быстрее.
    \end{itemize}
\end{myblock}

% --- Конец контента ---

% >>> КОММЕНТАРИЙ: Печатаем библиографию (если используется \cite)
\AtNextBibliography{\footnotesize} % Уменьшаем шрифт для библиографии
%\printbibliography % Раскомментируйте, если добавляли цитаты через \cite или \footcite

\end{multicols}

\end{document}