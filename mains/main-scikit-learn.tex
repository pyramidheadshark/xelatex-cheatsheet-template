\documentclass[10pt,a4paper]{article}

\usepackage[landscape,margin=0.7cm]{geometry}
\usepackage[russian,english]{babel} % Поддержка русского и английского

% >>> КОММЕНТАРИЙ: Определяем переменные для заголовка и автора (можно изменить)
\newcommand{\cheatsheettitle}{\color{w3schools}Шпаргалка по Scikit-learn / {\color{alert} Sklearn} {\color{black} Cheatsheet (XeLaTeX)}}
\newcommand{\cheatsheetauthor}{Краткий справочник по основным операциям}

% >>> КОММЕНТАРИЙ: Подключаем основной файл шаблона с настройками стилей и пакетов
% \documentclass{article} % ЗАКОММЕНТИРОВАНО: Добавлено для возможности компиляции этого фрагмента отдельно для проверки
\usepackage{fontspec}

% >>> КОММЕНТАРИЙ: Установка основных шрифтов документа (требует установленных шрифтов в системе)
\setmainfont{PT Sans} % Основной шрифт для текста
\setsansfont{PT Sans} % Шрифт без засечек (если используется отдельно)
\setmonofont{Liberation Mono} % Моноширинный шрифт (для кода)

\usepackage{fontawesome} % Для иконок (например, \faQuoteLeft)
\usepackage{hyperref} % Для создания кликабельных ссылок (в т.ч. в \resourcelink)
\usepackage{enumitem} % Для настройки списков
\usepackage{lipsum} % Для генерации текста-рыбы (не используется в финальной версии)
\usepackage{xcolor} % Для определения и использования цветов
\usepackage{minted} % Для подсветки синтаксиса кода (требует Python и Pygments, компиляция с -shell-escape)

% >>>>>> НОВОЕ: Пакет для создания графиков <<<<<<
\usepackage{pgfplots}
\pgfplotsset{compat=newest} % Устанавливаем последнюю версию совместимости
\usetikzlibrary{arrows.meta, shapes.geometric} % Загружаем полезные библиотеки TikZ

\usepackage{titlesec} % Для настройки заголовков секций
% >>> КОММЕНТАРИЙ: Настройка отступов для заголовков секций (\section)
\titlespacing*{\section}{0pt}{*1.5}{*0.8} % {отступ слева}{отступ сверху}{отступ снизу}

% >>> КОММЕНТАРИЙ: Форматирование заголовков секций (\section)
\titleformat{\section}
  {\normalfont\large\bfseries\color{mDarkTeal}} % Стиль заголовка (цвет, размер, жирность)
  {\thesection} % Номер секции
  {1em} % Отступ после номера
  {} % Код перед заголовком (здесь пусто)

% >>> КОММЕНТАРИЙ: Определение пользовательских цветов через HTML-коды
\definecolor{customcolor}{HTML}{696EA8} % Основной цвет для блоков textbox
\definecolor{alert}{HTML}{CD5C5C} % Цвет для выделения (красный)
\definecolor{w3schools}{HTML}{4CAF50} % Цвет для выделения (зеленый)
\definecolor{subbox}{gray}{0.60} % Цвет для под-блоков (не используется?)
\definecolor{codecolor}{HTML}{FFC300} % Цвет для кода (не используется?)
\definecolor{mDarkTeal}{HTML}{23373b} % Темно-бирюзовый (для заголовков секций, названий блоков)
\colorlet{beamerboxbg}{black!2} % Фон для beamerbox (не используется?)
\colorlet{beamerboxfg}{mDarkTeal} % Цвет текста для beamerbox (не используется?)
\definecolor{mDarkBrown}{HTML}{604c38} % Темно-коричневый (не используется?)
\definecolor{mLightBrown}{HTML}{EB811B} % Светло-коричневый (для заголовков myalertblock)
\definecolor{mLightGreen}{HTML}{14B03D} % Светло-зеленый (для заголовков myexampleblock)
\definecolor{alertTextBox}{HTML}{A8696E} % Цвет для блоков alerttextbox

% >>> КОММЕНТАРИЙ: Настройка пакета biblatex для библиографии
\usepackage
[citestyle=authoryear, % Стиль цитирования (Автор-Год)
sorting=nty, % Сортировка по Имени, Году, Названию
autocite=footnote, % Команда \autocite создает сноски
autolang=hyphen, % Автоматическое определение языка для переносов
mincrossrefs=1, % Минимальное количество перекрестных ссылок
backend=biber] % Используемый бэкенд (требует запуск biber)
{biblatex}

% >>> КОММЕНТАРИЙ: Настройка формата пост-заметок в цитатах (например, номера страниц)
\DeclareFieldFormat{postnote}{#1}
\DeclareFieldFormat{multipostnote}{#1}
\DeclareAutoCiteCommand{footnote}[f]{\footcite}{\footcites} % Настройка команды \autocite

% >>> КОММЕНТАРИЙ: Подключение файла с библиографическими записями
\addbibresource{literature.bib}

% >>> КОММЕНТАРИЙ: Подключение библиотеки tcolorbox для создания цветных блоков
\usepackage{tcolorbox}
\tcbuselibrary{most, listingsutf8, minted} % Загрузка необходимых библиотек tcolorbox (most включает breakable)

% >>> КОММЕНТАРИЙ: Глобальные настройки для маленьких tcbox (inline блоки)
\tcbset{tcbox width=auto,left=1mm,top=1mm,bottom=1mm,
right=1mm,boxsep=1mm,middle=1pt}

% >>> КОММЕНТАРИЙ: Определение базового окружения для цветных блоков с заголовком
% breakable: позволяет блоку разрываться между колонками/страницами.
% ВАЖНО: В multicols автоматический разрыв breakable может работать не всегда идеально.
% Если блок разрывается некрасиво, попробуйте разбить контент на несколько блоков
% или используйте \needspace{<длина>} перед блоком.
\newenvironment{mycolorbox}[2]{
\begin{tcolorbox}[capture=minipage,fonttitle=\large\bfseries, enhanced,boxsep=1mm,colback=#1!30!white,
width=\linewidth, % Ширина блока равна ширине текущей колонки
arc=2pt,outer arc=2pt, toptitle=0mm,colframe=#1,opacityback=0.7,nobeforeafter,
breakable, % Разрешить разрыв блока
title=#2] % Текст заголовка блока
}{\end{tcolorbox}}

% >>> КОММЕНТАРИЙ: Окружение subbox (не используется в текущем коде)
\newenvironment{subbox}[2]{
\begin{tcolorbox}[capture=minipage,fonttitle=\normalsize\bfseries, enhanced,boxsep=1mm,colback=#1!30!white,on line,tcbox width=auto,left=0.3em,top=1mm, toptitle=0mm,colframe=#1,opacityback=0.7,nobeforeafter,
breakable,
title=#2]\footnotesize
}{\normalsize\end{tcolorbox}\vspace{0.1em}}

% >>> КОММЕНТАРИЙ: Окружение для размещения tcolorbox'ов в несколько колонок внутри блока (не используется)
\newenvironment{multibox}[1]{
\begin{tcbraster}[raster columns=#1,raster equal height,nobeforeafter,raster column skip=1em,raster left skip=1em,raster right skip=1em]}{\end{tcbraster}}

% >>> КОММЕНТАРИЙ: Окружения-обертки для mycolorbox с предопределенными цветами и заголовками
\newenvironment{textbox}[1]{\begin{mycolorbox}{customcolor}{#1}}{\end{mycolorbox}} % Обычный текстовый блок
\newenvironment{alerttextbox}[1]{\begin{mycolorbox}{alertTextBox}{#1}}{\end{mycolorbox}} % Блок для предупреждений/важной информации

% >>> КОММЕНТАРИЙ: Определение окружения для блоков с кодом с подсветкой minted
\newtcblisting{codebox}[3][]{ % [опции minted], {язык}, {заголовок}
colback=black!10,colframe=black!20, % Цвета фона и рамки
listing only, % Показывать только листинг кода
minted options={ % Опции, передаваемые в minted
    numbers=left, % Нумерация строк слева
    style=default, % Стиль подсветки (можно выбрать другие, напр., 'friendly')
    fontsize=\scriptsize, % Размер шрифта кода
    breaklines, % Автоматический перенос длинных строк
    breaksymbolleft=, % Символ для обозначения переноса (пусто)
    autogobble, % Удаление лишних отступов слева
    linenos, % Показывать номера строк
    numbersep=0.7em, % Отступ номеров строк от кода
    #1 % Дополнительные опции minted, переданные в [1]
    },
enhanced,top=1mm, toptitle=0mm,
left=5mm, % Отступ слева внутри блока (для номеров строк)
arc=0pt,outer arc=0pt, % Прямые углы
title={#3}, % Заголовок блока кода
fonttitle=\small\bfseries\color{mDarkTeal}, % Стиль заголовка блока кода
listing engine=minted, % Движок для листинга - minted
minted language=#2, % Язык программирования для подсветки
breakable % Разрешить разрыв блока кода
}

\newcommand{\punkti}{~\lbrack\dots\rbrack~} % Команда для [...] (не используется)

% >>> КОММЕНТАРИЙ: Переопределение окружения quote для добавления иконок цитат
\renewenvironment{quote}
               {\list{\faQuoteLeft\phantom{ }}{\rightmargin\leftmargin}
                \item\relax\scriptsize\ignorespaces}
               {\unskip\unskip\phantom{xx}\faQuoteRight\endlist}

% >>> КОММЕНТАРИЙ: Вспомогательные команды для создания цветных фонов под текстом (не используются)
\newcommand{\bgupper}[3]{\colorbox{#1}{\color{#2}\huge\bfseries\MakeUppercase{#3}}}
\newcommand{\bg}[3]{\colorbox{#1}{\bfseries\color{#2}#3}}

% >>> КОММЕНТАРИЙ: Команда для форматирования описания команды/функции
% #1: Сама команда (используется \detokenize для корректного отображения спецсимволов)
% #2: Описание команды
\newcommand{\mycommand}[2]{
  {\par\noindent\ttfamily\detokenize{#1}\par} % Вывод команды моноширинным шрифтом
  \nopagebreak % Стараемся не разрывать страницу после команды
  \hangindent=1.5em \hangafter=1 \noindent % Делаем отступ для описания
  \small\textit{#2}\par\vspace{0.5ex} % Вывод описания и небольшой отступ после
}

% >>> КОММЕНТАРИЙ: Команда для форматирования ссылок на ресурсы
% #1: URL ссылки
% #2: Текст ссылки
% #3: Описание ресурса
\newcommand{\resourcelink}[3]{
  {\par\noindent\href{#1}{\ttfamily #2}\par} % Вывод кликабельной ссылки
  \nopagebreak
  \hangindent=1.5em \hangafter=1 \noindent
  \small\textit{#3}\par\vspace{0.5ex} % Вывод описания
}

% >>> КОММЕНТАРИЙ: Вспомогательные команды для стилизации
\newcommand{\sep}{{\scriptsize~\faCircle{ }~}} % Разделитель-кружок
\newcommand{\bggreen}[1]{\medskip\bgupper{w3schools}{black}{#1}\\[0.5em]} % Зеленый заголовок (не используется)
\newcommand{\green}[1]{\smallskip\bg{w3schools}{white}{#1}\\} % Текст на зеленом фоне
\newcommand{\red}[1]{\smallskip\bg{alert}{white}{#1}\\} % Текст на красном фоне
\newcommand{\alertcmd}[1]{\red{#1}\\} % Псевдоним для red (переименовано чтобы не конфликтовать с цветом alert)

\usepackage{multicol} % Пакет для создания нескольких колонок
\setlength{\columnsep}{15pt} % Расстояние между колонками

\setlength{\parindent}{0pt} % Убираем абзацный отступ по умолчанию
\usepackage{csquotes} % Для правильного отображения кавычек в разных языках (используется biblatex)
\newcommand{\loremipsum}{Lorem ipsum dolor sit amet.} % Пример текста (не используется)

% >>> КОММЕНТАРИЙ: Окружения для блоков с разными цветами заголовков (пример, предупреждение, обычный)
% Основаны на tcolorbox, похожи на mycolorbox, но с другими цветами заголовков и рамки.
% Также используют breakable. Применяются для выделения примеров, предупреждений и т.д.
% ВАЖНО: Лог может содержать предупреждения "Underfull \hbox (badness 10000)".
% Это часто случается в узких колонках multicol при наличии кода, URL или текста,
% который плохо переносится. Обычно это лишь косметическая проблема, строки выглядят
% немного не до конца заполненными. Можно игнорировать или попробовать перефразировать/
% использовать \sloppy перед проблемным абзацем.
\newenvironment{myexampleblock}[1]{ % Блок для примеров (зеленый заголовок)
    \tcolorbox[capture=minipage,fonttitle=\small\bfseries\color{mLightGreen}, enhanced,boxsep=1mm,colback=black!10,breakable,noparskip,
    on line,tcbox width=auto,left=0.3em,top=1mm, toptitle=0mm,
    colframe=black!20,arc=0pt,outer arc=0pt,
    opacityback=0.7,nobeforeafter,title=#1]}
    {\endtcolorbox}

\newenvironment{myalertblock}[1]{ % Блок для предупреждений (оранжевый заголовок)
    \tcolorbox[capture=minipage,fonttitle=\small\bfseries\color{mLightBrown}, enhanced,boxsep=1mm,colback=black!10,breakable,noparskip,
    on line,tcbox width=auto,left=0.3em,top=1mm, toptitle=0mm,
    colframe=black!20,arc=0pt,outer arc=0pt,
    opacityback=0.7,nobeforeafter,title=#1]}
    {\endtcolorbox}

\newenvironment{myblock}[1]{ % Обычный информационный блок (бирюзовый заголовок)
    \tcolorbox[capture=minipage,fonttitle=\small\bfseries\color{mDarkTeal}, enhanced,boxsep=1mm,colback=black!10,breakable,noparskip,
    on line,tcbox width=auto,left=0.3em,top=1mm, toptitle=0mm,
    colframe=black!20, arc=0pt,outer arc=0pt,
    opacityback=0.7,nobeforeafter,title=#1]}
    {\endtcolorbox}

% >>> КОММЕНТАРИЙ: Команда для вставки изображения в рамке tcolorbox с подписью
% #1: Опции (не используются)
% #2: Путь к файлу изображения (используется как подпись под изображением)
\newcommand{\mygraphics}[2][]{
\tcbox[enhanced,boxsep=0pt,top=0pt,bottom=0pt,left=0pt,
right=0pt,boxrule=0.4pt,drop fuzzy shadow,clip upper,
colback=black!75!white,toptitle=2pt,bottomtitle=2pt,nobeforeafter,
center title,fonttitle=\small\sffamily,title=\detokenize{#2}] % Используем путь как заголовок
{\includegraphics[width=\the\dimexpr(\linewidth-4mm)\relax]{#2}} % Вставляем изображение, ширина чуть меньше колонки
}

% >>> КОМЕНТАРИЙ: Команда \needspace{<длина>}
% Используйте эту команду *перед* блоком (tcolorbox, section, figure), чтобы убедиться,
% что в текущей колонке есть как минимум <длина> свободного места. Если места нет,
% LaTeX начнет новую колонку/страницу перед выполнением команды.
% Пример: \needspace{10\baselineskip} % Запросить место для примерно 10 строк текста
% Это полезно для предотвращения "висячих" заголовков или некрасивых разрывов блоков.

% >>>>>> НОВОЕ: Стили для графиков PGFPlots <<<<<<
\pgfplotsset{
    % Базовый стиль для всех графиков в шпаргалке
    cheatsheet plot style/.style={
        width=0.98\linewidth, % Ширина чуть меньше колонки для аккуратности
        height=5cm,          % Высота графика (можно изменить по необходимости)
        grid=major,          % Включаем основную сетку
        grid style={dashed, color=black!20}, % Стиль сетки: пунктирная, светло-серая
        axis lines=left,     % Линии осей только слева и снизу
        axis line style={color=black!60, thick}, % Стиль линий осей: серые, потолще
        tick style={color=black!60, thick}, % Стиль меток на осях
        ticklabel style={font=\scriptsize, color=black}, % Стиль подписей меток: маленький шрифт
        label style={font=\small, color=mDarkTeal}, % Стиль подписей осей: шрифт small, цвет как у заголовков блоков
        title style={font=\small\bfseries, color=mDarkTeal}, % Стиль заголовка графика
        legend style={        % Стиль легенды
            font=\scriptsize, % Маленький шрифт
            draw=black!30,    % Легкая рамка вокруг легенды
            fill=white,       % Белый фон
            legend cell align=left, % Выравнивание текста в ячейке легенды
            anchor=north east, % Якорь для позиционирования
            at={(rel axis cs:0.98,0.98)} % Положение в правом верхнем углу внутри области графика
        },
        cycle list={ % Список стилей для линий \addplot (цвета взяты из шаблона)
            {blue, mark=*, thick},
            {alert, mark=square, thick},
            {w3schools, mark=triangle, thick},
            {mLightBrown, mark=diamond, thick},
            {customcolor, mark=oplus, thick},
            {black!50, mark=pentagon, thick},
        },
        % Убираем рамку вокруг графика по умолчанию
        axis background/.style={fill=none},
    },
    % Дополнительный стиль для ROC-кривой (без сетки, с диагональю)
    roc curve style/.style={
        cheatsheet plot style, % Наследуем базовый стиль
        width=6cm, height=6cm, % Делаем квадратным
        grid=none, % Убираем сетку
        xlabel={False Positive Rate (FPR)},
        ylabel={True Positive Rate (TPR)},
        xmin=0, xmax=1,
        ymin=0, ymax=1,
        xtick={0, 0.5, 1},
        ytick={0, 0.5, 1},
        legend pos=south east, % Легенда внизу справа
        % Добавляем диагональную линию случайного угадывания
        extra x ticks={0.5}, extra y ticks={0.5},
        extra tick style={grid=major, grid style={dashed, color=black!40}},
        after end axis/.code={ % Добавляем линию после отрисовки осей
            \draw[dashed, color=black!40] (axis cs:0,0) -- (axis cs:1,1);
        }
    },
    % Дополнительный стиль для PR-кривой
    pr curve style/.style={
        cheatsheet plot style, % Наследуем базовый стиль
        xlabel={Recall (TPR)},
        ylabel={Precision},
        xmin=0, xmax=1,
        ymin=0, ymax=1,
        legend pos=south west, % Легенда внизу слева
    }
}

% % Добавлено для возможности компиляции этого фрагмента отдельно для проверки
% \begin{document}
% \begin{multicols}{2} % Пример использования в multicols
% \lipsum[1] % Просто текст для заполнения

% \begin{textbox}{Пример графика (Базовый стиль)}
%     \begin{center}
%     \begin{tikzpicture}
%         \begin{axis}[cheatsheet plot style, title={Пример $y=x^2$ и $y=x+1$}]
%         \addplot coordinates {(0, 0) (1, 1) (2, 4) (3, 9) (4, 16)};
%         \addlegendentry{$y=x^2$}
%         \addplot coordinates {(0, 1) (1, 2) (2, 3) (3, 4) (4, 5)};
%         \addlegendentry{$y=x+1$}
%         \end{axis}
%     \end{tikzpicture}
%     \end{center}
% \end{textbox}

% \begin{myexampleblock}{Пример ROC-кривой}
%     \begin{center}
%     \begin{tikzpicture}
%         \begin{axis}[roc curve style, title={Пример ROC Curve}]
%         % Пример данных ROC кривой
%         \addplot coordinates { (0,0) (0.1,0.3) (0.2,0.6) (0.4,0.8) (0.6,0.9) (0.8,0.95) (1,1) };
%         \addlegendentry{Модель A (AUC $\approx$ 0.8)}
%         \addplot coordinates { (0,0) (0.2,0.2) (0.4,0.4) (0.6,0.6) (0.8,0.8) (1,1) }; % Пример хуже
%         \addlegendentry{Модель B (AUC = 0.5)}
%         \end{axis}
%     \end{tikzpicture}
%     \end{center}
% \end{myexampleblock}

% \begin{myblock}{Пример PR-кривой}
%     \begin{center}
%     \begin{tikzpicture}
%         \begin{axis}[pr curve style, title={Пример Precision-Recall Curve}]
%         % Пример данных PR кривой (могут сильно зависеть от порога)
%         \addplot coordinates { (0,0.9) (0.1,0.85) (0.3,0.8) (0.5,0.7) (0.7,0.6) (0.9,0.4) (1,0.3) };
%         \addlegendentry{Модель C}
%         \end{axis}
%     \end{tikzpicture}
%     \end{center}
% \end{myblock}

% \lipsum[2-3] % Просто текст для заполнения
% \end{multicols}
% \end{document}

\begin{document}
\pagestyle{empty} % Убираем номера страниц
\small % Уменьшаем базовый размер шрифта для всего документа

% >>> КОММЕНТАРИЙ: Используем multicols для создания трех колонок.
\begin{multicols}{3}
\raggedcolumns % Не растягивать колонки по вертикали

\noindent
\begin{minipage}{\linewidth}
    \centering
    {\bfseries\huge \cheatsheettitle \par}
    \vspace{1ex}
    {\large \cheatsheetauthor \par}
    \vspace{0.5ex}
    {\normalsize \today \par} % Вставляем текущую дату
\end{minipage}
\vspace{2ex}

\thispagestyle{empty} % Убеждаемся, что и на первой странице нет номера

\scriptsize % Уменьшаем шрифт для оглавления и основного текста
\tableofcontents % Генерируем оглавление

% >>> КОММЕНТАРИЙ: Подключаем основной контент шпаргалки
% >>> Контент для шпаргалки по Основам Scikit-learn

% --- Введение ---
\begin{myblock}{Scikit-learn: Ваш Швейцарский Нож в ML}
    \textbf{Scikit-learn (sklearn)} — это фундаментальная библиотека Python для классического машинного обучения. Она предоставляет единообразный интерфейс (\textbf{API}) для большинства алгоритмов и инструментов предобработки, оценки и выбора моделей. Знание её основ — \textbf{must-have}. \textbf{Ключевая сила — единообразный интерфейс}, позволяющий легко пробовать разные алгоритмы без необходимости изучать новый синтаксис для каждой модели.
    \textbf{Аналогия:} Представь себе ящик с инструментами. Sklearn — это такой ящик, где все инструменты (модели, препроцессоры) имеют похожие ручки (методы `fit`, `predict`, `transform`), что сильно упрощает работу.
\end{myblock}

% --- Раздел 1 ---
\section{Основной API: Глаголы ML}

\begin{textbox}{Ключевые Методы Оценщиков (Estimators)}
    Все объекты sklearn, которые учатся на данных (модели, препроцессоры), называются \textbf{оценщиками (estimators)}. У них есть стандартные методы:
    \begin{itemize}
        \item \texttt{fit(X, y)}: "Обучись". Главный метод для тренировки модели. Принимает обучающие данные (`X`) и целевую переменную (`y`, если модель с учителем). `y` **не требуется для большинства препроцессоров** (например, Scaler, Encoder, Imputer): \texttt{fit(X)}. \textbf{Аналогия:} Это как показать собаке команды (`X`) и правильные реакции (`y`), чтобы она научилась. Или как настроить инструмент по эталону (`X`).
        \item \texttt{predict(X)}: "Предскажи". Используется \textit{после} \texttt{fit}. Генерирует предсказания для новых данных `X`. \textbf{Аналогия:} Попросить обученную собаку выполнить команду на новых данных.
        \item \texttt{predict\_proba(X)}: "Оцени вероятности". Используется \textit{после} \texttt{fit} для \textbf{классификаторов}. Возвращает вероятности принадлежности к каждому классу (массив \texttt{[n\_samples, n\_classes]}). Очень полезно для оценки уверенности модели и для построения ROC-кривых. \textbf{Аналогия:} Спросить у собаки, насколько она уверена, что это именно та команда (например, 80% "сидеть", 20% "лежать").
        \item \texttt{transform(X)}: "Преобразуй". Используется \textit{после} \texttt{fit} для \textbf{препроцессоров} (например, `StandardScaler`, `PCA`, `OneHotEncoder`). Применяет выученное преобразование к данным `X`. \textbf{Аналогия:} Использовать настроенный инструмент (например, линейку, откалиброванную по \texttt{fit}) для измерения новых объектов.
        \item \texttt{fit\_transform(X, y=None)}: "Обучись и преобразуй". Оптимизированная комбинация \texttt{fit(X)} и \texttt{transform(X)}. Часто используется для обучающей выборки, чтобы избежать повторных вычислений. \textbf{Важно:} Применять только к \textbf{обучающей} выборке, чтобы избежать **утечки информации из тестовой выборки** в процесс обучения препроцессора! Для тестовой выборки используется только \texttt{transform}. \textbf{Аналогия:} Найти среднее и стандартное отклонение (fit) и сразу же стандартизировать данные (transform) за один проход.
        \item \texttt{score(X, y)}: "Оцени качество". Используется \textit{после} \texttt{fit}. Возвращает метрику по умолчанию (accuracy для классификаторов, R² для регрессоров) на данных `X`, `y`. Удобно для быстрой проверки.
    \end{itemize}
\end{textbox}

\begin{codebox}{python}{Базовый пример API}
from sklearn.linear_model import LogisticRegression
from sklearn.preprocessing import StandardScaler
from sklearn.model_selection import train_test_split
import numpy as np

# Допустим, есть данные X (фичи) и y (цель)
# X = np.array(...)
# y = np.array(...)
# X_train, X_test, y_train, y_test = train_test_split(X, y, test_size=0.2)

# 1. Препроцессор (Scaler)
# Масштабирование часто улучшает сходимость и производительность моделей, чувствительных к масштабу признаков (линейные модели, SVM, нейросети, kNN).
scaler = StandardScaler()
# Обучаем на ТРЕНИРОВОЧНЫХ данных
scaler.fit(X_train)
# Преобразуем ТРЕНИРОВОЧНЫЕ и ТЕСТОВЫЕ данные
X_train_scaled = scaler.transform(X_train)
X_test_scaled = scaler.transform(X_test)
# Альтернатива для трейна: X_train_scaled = scaler.fit_transform(X_train)

# 2. Модель (Классификатор)
model = LogisticRegression()
# Обучаем модель на масштабированных ТРЕНИРОВОЧНЫХ данных
model.fit(X_train_scaled, y_train)

# 3. Предсказания
# Предсказания классов для теста
y_pred = model.predict(X_test_scaled)
# Предсказания вероятностей для теста
y_pred_proba = model.predict_proba(X_test_scaled)

print(f"Предсказанные классы: {y_pred[:5]}")
print(f"Вероятности классов: \n{y_pred_proba[:5]}")
# Оценка качества (accuracy по умолчанию для классификатора)
# print(f"Точность на тесте: {model.score(X_test_scaled, y_test)}")
\end{codebox}

% --- Раздел 2 ---
\section{Pipeline: Конвейер Обработки}

\begin{myblock}{Зачем нужен Pipeline?}
    Часто рабочий процесс ML включает несколько шагов предобработки (масштабирование, кодирование категорий, извлечение признаков) перед обучением модели. \textbf{Pipeline} позволяет объединить эти шаги в единый объект (оценщик).
    \textbf{Преимущества:}
    \begin{itemize}
        \item \textbf{Удобство:} Все шаги выполняются одной командой \texttt{fit} и \texttt{predict}/\texttt{transform}.
        \item \textbf{Предотвращение утечки данных (Data Leakage):} Гарантирует, что \texttt{fit} препроцессоров вызывается *только* на обучающих фолдах при кросс-валидации, а \texttt{transform} — на обучающих и валидационных/тестовых. Это \textbf{критически важно} для получения несмещенной оценки качества модели. % ИЗМЕНЕНИЕ ЗДЕСЬ
        \item \textbf{Легкость настройки:} Гиперпараметры шагов пайплайна можно подбирать с помощью `GridSearchCV`/`RandomizedSearchCV`.
    \end{itemize}
    \textbf{Аналогия:} Пайплайн — это как сборочный конвейер на заводе. Сырье (данные) проходит через несколько станций обработки (препроцессоры) и на выходе получается готовый продукт (предсказание модели).
\end{myblock}

\begin{codebox}{python}{Пример Pipeline}
from sklearn.pipeline import Pipeline
from sklearn.impute import SimpleImputer # Для заполнения пропусков
from sklearn.preprocessing import StandardScaler, OneHotEncoder
from sklearn.linear_model import LogisticRegression
# Допустим, есть X_train, y_train

# Создаем пайплайн:
# Шаг 1: Заполнение пропусков средним
# Шаг 2: Масштабирование числовых признаков
# Шаг 3: Обучение модели
pipe = Pipeline([
    ('imputer', SimpleImputer(strategy='mean')), # Имя шага, объект
    ('scaler', StandardScaler()),
    ('classifier', LogisticRegression())
])

# Обучаем весь пайплайн как единое целое
pipe.fit(X_train, y_train)

# Делаем предсказания (данные пройдут через imputer -> scaler -> predict)
# y_pred_pipe = pipe.predict(X_test)
# print(f"Точность пайплайна: {pipe.score(X_test, y_test)}")
\end{codebox}

% --- Раздел 3 ---
\section{ColumnTransformer: Работа с Разными Типами Признаков}

\begin{myexampleblock}{Обработка Гетерогенных Данных}
    Реальные данные часто содержат столбцы разных типов: числовые, категориальные. К ним нужно применять разную предобработку (например, масштабирование к числовым, One-Hot Encoding к категориальным). \textbf{ColumnTransformer} позволяет применять разные трансформаторы к разным подмножествам столбцов. Сам `ColumnTransformer` является таким же \textit{оценщиком (estimator)} sklearn и может использоваться как самостоятельно, так и (что чаще всего) в качестве шага внутри `Pipeline`.
    \textbf{Аналогия:} Представь, что в больницу пришел пациент (строка данных). У него есть разные проблемы (признаки): перелом (числовой признак), аллергия (категориальный). `ColumnTransformer` — это регистратура, которая направляет пациента к нужным специалистам: травматологу (масштабирование) и аллергологу (One-Hot Encoding).
\end{myexampleblock}

\begin{codebox}{python}{Пример ColumnTransformer в Pipeline}
from sklearn.compose import ColumnTransformer
from sklearn.pipeline import Pipeline
from sklearn.impute import SimpleImputer
from sklearn.preprocessing import StandardScaler, OneHotEncoder
from sklearn.linear_model import LogisticRegression
from sklearn.model_selection import train_test_split
import pandas as pd

# Допустим, есть DataFrame df с числовыми и категориальными фичами
# df = pd.DataFrame(...)
# X = df.drop('target', axis=1)
# y = df['target']
# X_train, X_test, y_train, y_test = train_test_split(X, y, test_size=0.2)

# Определяем числовые и категориальные столбцы
numeric_features = X_train.select_dtypes(include=['int64', 'float64']).columns
categorical_features = X_train.select_dtypes(include=['object', 'category']).columns

# Создаем пайплайны для каждого типа признаков
numeric_transformer = Pipeline(steps=[
    ('imputer', SimpleImputer(strategy='median')),
    ('scaler', StandardScaler())])

categorical_transformer = Pipeline(steps=[
    ('imputer', SimpleImputer(strategy='most_frequent')),
    ('onehot', OneHotEncoder(handle_unknown='ignore'))]) # handle_unknown='ignore' важен!

# Создаем ColumnTransformer
preprocessor = ColumnTransformer(
    transformers=[
        ('num', numeric_transformer, numeric_features), # Имя, трансформер, колонки
        ('cat', categorical_transformer, categorical_features)])

# Создаем основной пайплайн, включающий препроцессор и модель
model_pipe = Pipeline(steps=[('preprocessor', preprocessor),
                           ('classifier', LogisticRegression())])

# Обучаем и используем как обычно
model_pipe.fit(X_train, y_train)
# y_pred = model_pipe.predict(X_test)
# print(f"Точность модели с ColumnTransformer: {model_pipe.score(X_test, y_test)}")
\end{codebox}

% --- Раздел 4 ---
\section{Подбор Гиперпараметров}

\begin{alerttextbox}{GridSearchCV vs RandomizedSearchCV}
    Большинство моделей имеют \textbf{гиперпараметры} — настройки, которые не выучиваются из данных, а задаются до обучения (например, глубина дерева, коэффициент регуляризации). Их нужно подбирать.
    \begin{itemize}
        \item \textbf{GridSearchCV}: "Поиск по сетке". Перебирает \textbf{все возможные комбинации} заданных гиперпараметров. Тщательно, но медленно, особенно если параметров много. \textbf{Аналогия:} Пекарь пробует испечь хлеб при всех комбинациях температуры (180, 200, 220) и времени (30, 40, 50 минут) — всего 3*3=9 попыток.
        \item \textbf{RandomizedSearchCV}: "Случайный поиск". Выбирает \textbf{случайные комбинации} гиперпараметров из заданных диапазонов или распределений. Работает быстрее GridSearchCV, часто находит достаточно хорошие параметры за меньшее число итераций (\texttt{n\_iter}). Особенно эффективен, когда параметров много или не все параметры одинаково важны (позволяет быстрее найти "достаточно хорошую" область). \textbf{Аналогия:} Пекарь пробует случайные 5 комбинаций температуры и времени из возможных диапазонов. Может не найти идеал, но быстро найдет хороший вариант.
    \end{itemize}
    Оба инструмента используют \textbf{кросс-валидацию (CV)} для оценки качества каждой комбинации параметров на обучающей выборке, чтобы избежать переобучения на конкретном разбиении.
\end{alerttextbox}

\begin{codebox}{python}{Пример GridSearchCV (с Pipeline)}
from sklearn.model_selection import GridSearchCV
from sklearn.ensemble import RandomForestClassifier
# Используем model_pipe из примера с ColumnTransformer, но заменим LogReg на RF
# model_pipe = Pipeline(steps=[('preprocessor', preprocessor),
#                            ('classifier', RandomForestClassifier(random_state=42))])

# Задаем сетку параметров для поиска
# Обратите внимание на синтаксис для доступа к параметрам шага пайплайна:
# 'имя_шага__имя_параметра'
param_grid = {
    'preprocessor__num__imputer__strategy': ['mean', 'median'], # Параметр вложенного пайплайна
    'classifier__n_estimators': [100, 200], # Параметр модели
    'classifier__max_depth': [None, 10, 20],
    'classifier__min_samples_split': [2, 5]
}

# Создаем объект GridSearchCV
# cv=5 означает 5-кратную кросс-валидацию
# n_jobs=-1 использует все доступные ядра процессора
# Важно выбрать метрику (scoring), соответствующую задаче! Не всегда accuracy - лучший выбор.
grid_search = GridSearchCV(model_pipe, param_grid, cv=5, scoring='accuracy', n_jobs=-1, verbose=1)

# Запускаем поиск (может занять время!)
grid_search.fit(X_train, y_train)

# Лучшие параметры и лучший скор
print(f"Лучшие параметры: {grid_search.best_params_}")
print(f"Лучший скор на CV: {grid_search.best_score_:.4f}")

# Лучшая модель уже обучена на всех трейн данных с лучшими параметрами
best_model = grid_search.best_estimator_
# y_pred_best = best_model.predict(X_test)
# print(f"Точность лучшей модели на тесте: {best_model.score(X_test, y_test)}")

# Для RandomizedSearchCV синтаксис похож, но вместо param_grid
# используется param_distributions (словари с распределениями scipy.stats)
# и добавляется параметр n_iter.
\end{codebox}

% --- Раздел 5 ---
\section{Часто Используемые Импорты}

\begin{myblock}{Джентльменский Набор для Старта}
    Этот список поможет быстро начать работу над типичной задачей.
\end{myblock}
\begin{codebox}{python}{Основные импорты моделей и метрик}
# --- Модели ---
# Линейные модели
from sklearn.linear_model import LinearRegression # Регрессия
from sklearn.linear_model import LogisticRegression # Классификация
from sklearn.linear_model import Ridge, Lasso # Регрессия с регуляризацией

# Метод опорных векторов (SVM)
from sklearn.svm import SVC # Классификация
from sklearn.svm import SVR # Регрессия

# Деревья решений
from sklearn.tree import DecisionTreeClassifier
from sklearn.tree import DecisionTreeRegressor

# Ансамбли: Бэггинг (случайный лес)
from sklearn.ensemble import RandomForestClassifier
from sklearn.ensemble import RandomForestRegressor

# Ансамбли: Бустинг
from sklearn.ensemble import GradientBoostingClassifier
from sklearn.ensemble import GradientBoostingRegressor
# Популярные библиотеки бустинга (устанавливаются отдельно, часто производительнее)
# import xgboost as xgb # pip install xgboost
# import lightgbm as lgb # pip install lightgbm
# import catboost as cb # pip install catboost

# --- Предобработка ---
from sklearn.preprocessing import StandardScaler, MinMaxScaler, RobustScaler
from sklearn.preprocessing import OneHotEncoder, OrdinalEncoder
from sklearn.impute import SimpleImputer
from sklearn.compose import ColumnTransformer
from sklearn.pipeline import Pipeline, make_pipeline # make_pipeline - упрощенное создание Pipeline без явного именования шагов

# --- Выборка и Оценка Моделей ---
from sklearn.model_selection import train_test_split
from sklearn.model_selection import cross_val_score, KFold, StratifiedKFold
from sklearn.model_selection import GridSearchCV, RandomizedSearchCV

# --- Метрики ---
# Классификация
from sklearn.metrics import accuracy_score, precision_score, recall_score, f1_score
from sklearn.metrics import roc_auc_score, roc_curve
from sklearn.metrics import confusion_matrix, classification_report
# Регрессия
from sklearn.metrics import mean_squared_error, mean_absolute_error, r2_score

# --- Другое ---
import numpy as np
import pandas as pd
import matplotlib.pyplot as plt
import seaborn as sns
\end{codebox}

% --- Конец контента ---

% >>> КОММЕНТАРИЙ: Печатаем библиографию (если используется \cite)
\AtNextBibliography{\footnotesize} % Уменьшаем шрифт для библиографии
%\printbibliography % Раскомментируйте, если добавляли цитаты через \cite или \footcite

\end{multicols}

\end{document}