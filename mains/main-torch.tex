\documentclass[10pt,a4paper]{article}

\usepackage[landscape,margin=0.7cm]{geometry}
\usepackage[russian,english]{babel} % Поддержка русского и английского

% >>> КОММЕНТАРИЙ: Определяем переменные для заголовка и автора (можно изменить)
\newcommand{\cheatsheettitle}{\color{w3schools}Шпаргалка по PyTorch и TensorFlow / {\color{alert} Torch} {\color{black} Cheatsheet (XeLaTeX)}}
\newcommand{\cheatsheetauthor}{Краткий справочник по основным операциям}

% >>> КОММЕНТАРИЙ: Подключаем основной файл шаблона с настройками стилей и пакетов
% \documentclass{article} % ЗАКОММЕНТИРОВАНО: Добавлено для возможности компиляции этого фрагмента отдельно для проверки
\usepackage{fontspec}

% >>> КОММЕНТАРИЙ: Установка основных шрифтов документа (требует установленных шрифтов в системе)
\setmainfont{PT Sans} % Основной шрифт для текста
\setsansfont{PT Sans} % Шрифт без засечек (если используется отдельно)
\setmonofont{Liberation Mono} % Моноширинный шрифт (для кода)

\usepackage{fontawesome} % Для иконок (например, \faQuoteLeft)
\usepackage{hyperref} % Для создания кликабельных ссылок (в т.ч. в \resourcelink)
\usepackage{enumitem} % Для настройки списков
\usepackage{lipsum} % Для генерации текста-рыбы (не используется в финальной версии)
\usepackage{xcolor} % Для определения и использования цветов
\usepackage{minted} % Для подсветки синтаксиса кода (требует Python и Pygments, компиляция с -shell-escape)

% >>>>>> НОВОЕ: Пакет для создания графиков <<<<<<
\usepackage{pgfplots}
\pgfplotsset{compat=newest} % Устанавливаем последнюю версию совместимости
\usetikzlibrary{arrows.meta, shapes.geometric} % Загружаем полезные библиотеки TikZ

\usepackage{titlesec} % Для настройки заголовков секций
% >>> КОММЕНТАРИЙ: Настройка отступов для заголовков секций (\section)
\titlespacing*{\section}{0pt}{*1.5}{*0.8} % {отступ слева}{отступ сверху}{отступ снизу}

% >>> КОММЕНТАРИЙ: Форматирование заголовков секций (\section)
\titleformat{\section}
  {\normalfont\large\bfseries\color{mDarkTeal}} % Стиль заголовка (цвет, размер, жирность)
  {\thesection} % Номер секции
  {1em} % Отступ после номера
  {} % Код перед заголовком (здесь пусто)

% >>> КОММЕНТАРИЙ: Определение пользовательских цветов через HTML-коды
\definecolor{customcolor}{HTML}{696EA8} % Основной цвет для блоков textbox
\definecolor{alert}{HTML}{CD5C5C} % Цвет для выделения (красный)
\definecolor{w3schools}{HTML}{4CAF50} % Цвет для выделения (зеленый)
\definecolor{subbox}{gray}{0.60} % Цвет для под-блоков (не используется?)
\definecolor{codecolor}{HTML}{FFC300} % Цвет для кода (не используется?)
\definecolor{mDarkTeal}{HTML}{23373b} % Темно-бирюзовый (для заголовков секций, названий блоков)
\colorlet{beamerboxbg}{black!2} % Фон для beamerbox (не используется?)
\colorlet{beamerboxfg}{mDarkTeal} % Цвет текста для beamerbox (не используется?)
\definecolor{mDarkBrown}{HTML}{604c38} % Темно-коричневый (не используется?)
\definecolor{mLightBrown}{HTML}{EB811B} % Светло-коричневый (для заголовков myalertblock)
\definecolor{mLightGreen}{HTML}{14B03D} % Светло-зеленый (для заголовков myexampleblock)
\definecolor{alertTextBox}{HTML}{A8696E} % Цвет для блоков alerttextbox

% >>> КОММЕНТАРИЙ: Настройка пакета biblatex для библиографии
\usepackage
[citestyle=authoryear, % Стиль цитирования (Автор-Год)
sorting=nty, % Сортировка по Имени, Году, Названию
autocite=footnote, % Команда \autocite создает сноски
autolang=hyphen, % Автоматическое определение языка для переносов
mincrossrefs=1, % Минимальное количество перекрестных ссылок
backend=biber] % Используемый бэкенд (требует запуск biber)
{biblatex}

% >>> КОММЕНТАРИЙ: Настройка формата пост-заметок в цитатах (например, номера страниц)
\DeclareFieldFormat{postnote}{#1}
\DeclareFieldFormat{multipostnote}{#1}
\DeclareAutoCiteCommand{footnote}[f]{\footcite}{\footcites} % Настройка команды \autocite

% >>> КОММЕНТАРИЙ: Подключение файла с библиографическими записями
\addbibresource{literature.bib}

% >>> КОММЕНТАРИЙ: Подключение библиотеки tcolorbox для создания цветных блоков
\usepackage{tcolorbox}
\tcbuselibrary{most, listingsutf8, minted} % Загрузка необходимых библиотек tcolorbox (most включает breakable)

% >>> КОММЕНТАРИЙ: Глобальные настройки для маленьких tcbox (inline блоки)
\tcbset{tcbox width=auto,left=1mm,top=1mm,bottom=1mm,
right=1mm,boxsep=1mm,middle=1pt}

% >>> КОММЕНТАРИЙ: Определение базового окружения для цветных блоков с заголовком
% breakable: позволяет блоку разрываться между колонками/страницами.
% ВАЖНО: В multicols автоматический разрыв breakable может работать не всегда идеально.
% Если блок разрывается некрасиво, попробуйте разбить контент на несколько блоков
% или используйте \needspace{<длина>} перед блоком.
\newenvironment{mycolorbox}[2]{
\begin{tcolorbox}[capture=minipage,fonttitle=\large\bfseries, enhanced,boxsep=1mm,colback=#1!30!white,
width=\linewidth, % Ширина блока равна ширине текущей колонки
arc=2pt,outer arc=2pt, toptitle=0mm,colframe=#1,opacityback=0.7,nobeforeafter,
breakable, % Разрешить разрыв блока
title=#2] % Текст заголовка блока
}{\end{tcolorbox}}

% >>> КОММЕНТАРИЙ: Окружение subbox (не используется в текущем коде)
\newenvironment{subbox}[2]{
\begin{tcolorbox}[capture=minipage,fonttitle=\normalsize\bfseries, enhanced,boxsep=1mm,colback=#1!30!white,on line,tcbox width=auto,left=0.3em,top=1mm, toptitle=0mm,colframe=#1,opacityback=0.7,nobeforeafter,
breakable,
title=#2]\footnotesize
}{\normalsize\end{tcolorbox}\vspace{0.1em}}

% >>> КОММЕНТАРИЙ: Окружение для размещения tcolorbox'ов в несколько колонок внутри блока (не используется)
\newenvironment{multibox}[1]{
\begin{tcbraster}[raster columns=#1,raster equal height,nobeforeafter,raster column skip=1em,raster left skip=1em,raster right skip=1em]}{\end{tcbraster}}

% >>> КОММЕНТАРИЙ: Окружения-обертки для mycolorbox с предопределенными цветами и заголовками
\newenvironment{textbox}[1]{\begin{mycolorbox}{customcolor}{#1}}{\end{mycolorbox}} % Обычный текстовый блок
\newenvironment{alerttextbox}[1]{\begin{mycolorbox}{alertTextBox}{#1}}{\end{mycolorbox}} % Блок для предупреждений/важной информации

% >>> КОММЕНТАРИЙ: Определение окружения для блоков с кодом с подсветкой minted
\newtcblisting{codebox}[3][]{ % [опции minted], {язык}, {заголовок}
colback=black!10,colframe=black!20, % Цвета фона и рамки
listing only, % Показывать только листинг кода
minted options={ % Опции, передаваемые в minted
    numbers=left, % Нумерация строк слева
    style=default, % Стиль подсветки (можно выбрать другие, напр., 'friendly')
    fontsize=\scriptsize, % Размер шрифта кода
    breaklines, % Автоматический перенос длинных строк
    breaksymbolleft=, % Символ для обозначения переноса (пусто)
    autogobble, % Удаление лишних отступов слева
    linenos, % Показывать номера строк
    numbersep=0.7em, % Отступ номеров строк от кода
    #1 % Дополнительные опции minted, переданные в [1]
    },
enhanced,top=1mm, toptitle=0mm,
left=5mm, % Отступ слева внутри блока (для номеров строк)
arc=0pt,outer arc=0pt, % Прямые углы
title={#3}, % Заголовок блока кода
fonttitle=\small\bfseries\color{mDarkTeal}, % Стиль заголовка блока кода
listing engine=minted, % Движок для листинга - minted
minted language=#2, % Язык программирования для подсветки
breakable % Разрешить разрыв блока кода
}

\newcommand{\punkti}{~\lbrack\dots\rbrack~} % Команда для [...] (не используется)

% >>> КОММЕНТАРИЙ: Переопределение окружения quote для добавления иконок цитат
\renewenvironment{quote}
               {\list{\faQuoteLeft\phantom{ }}{\rightmargin\leftmargin}
                \item\relax\scriptsize\ignorespaces}
               {\unskip\unskip\phantom{xx}\faQuoteRight\endlist}

% >>> КОММЕНТАРИЙ: Вспомогательные команды для создания цветных фонов под текстом (не используются)
\newcommand{\bgupper}[3]{\colorbox{#1}{\color{#2}\huge\bfseries\MakeUppercase{#3}}}
\newcommand{\bg}[3]{\colorbox{#1}{\bfseries\color{#2}#3}}

% >>> КОММЕНТАРИЙ: Команда для форматирования описания команды/функции
% #1: Сама команда (используется \detokenize для корректного отображения спецсимволов)
% #2: Описание команды
\newcommand{\mycommand}[2]{
  {\par\noindent\ttfamily\detokenize{#1}\par} % Вывод команды моноширинным шрифтом
  \nopagebreak % Стараемся не разрывать страницу после команды
  \hangindent=1.5em \hangafter=1 \noindent % Делаем отступ для описания
  \small\textit{#2}\par\vspace{0.5ex} % Вывод описания и небольшой отступ после
}

% >>> КОММЕНТАРИЙ: Команда для форматирования ссылок на ресурсы
% #1: URL ссылки
% #2: Текст ссылки
% #3: Описание ресурса
\newcommand{\resourcelink}[3]{
  {\par\noindent\href{#1}{\ttfamily #2}\par} % Вывод кликабельной ссылки
  \nopagebreak
  \hangindent=1.5em \hangafter=1 \noindent
  \small\textit{#3}\par\vspace{0.5ex} % Вывод описания
}

% >>> КОММЕНТАРИЙ: Вспомогательные команды для стилизации
\newcommand{\sep}{{\scriptsize~\faCircle{ }~}} % Разделитель-кружок
\newcommand{\bggreen}[1]{\medskip\bgupper{w3schools}{black}{#1}\\[0.5em]} % Зеленый заголовок (не используется)
\newcommand{\green}[1]{\smallskip\bg{w3schools}{white}{#1}\\} % Текст на зеленом фоне
\newcommand{\red}[1]{\smallskip\bg{alert}{white}{#1}\\} % Текст на красном фоне
\newcommand{\alertcmd}[1]{\red{#1}\\} % Псевдоним для red (переименовано чтобы не конфликтовать с цветом alert)

\usepackage{multicol} % Пакет для создания нескольких колонок
\setlength{\columnsep}{15pt} % Расстояние между колонками

\setlength{\parindent}{0pt} % Убираем абзацный отступ по умолчанию
\usepackage{csquotes} % Для правильного отображения кавычек в разных языках (используется biblatex)
\newcommand{\loremipsum}{Lorem ipsum dolor sit amet.} % Пример текста (не используется)

% >>> КОММЕНТАРИЙ: Окружения для блоков с разными цветами заголовков (пример, предупреждение, обычный)
% Основаны на tcolorbox, похожи на mycolorbox, но с другими цветами заголовков и рамки.
% Также используют breakable. Применяются для выделения примеров, предупреждений и т.д.
% ВАЖНО: Лог может содержать предупреждения "Underfull \hbox (badness 10000)".
% Это часто случается в узких колонках multicol при наличии кода, URL или текста,
% который плохо переносится. Обычно это лишь косметическая проблема, строки выглядят
% немного не до конца заполненными. Можно игнорировать или попробовать перефразировать/
% использовать \sloppy перед проблемным абзацем.
\newenvironment{myexampleblock}[1]{ % Блок для примеров (зеленый заголовок)
    \tcolorbox[capture=minipage,fonttitle=\small\bfseries\color{mLightGreen}, enhanced,boxsep=1mm,colback=black!10,breakable,noparskip,
    on line,tcbox width=auto,left=0.3em,top=1mm, toptitle=0mm,
    colframe=black!20,arc=0pt,outer arc=0pt,
    opacityback=0.7,nobeforeafter,title=#1]}
    {\endtcolorbox}

\newenvironment{myalertblock}[1]{ % Блок для предупреждений (оранжевый заголовок)
    \tcolorbox[capture=minipage,fonttitle=\small\bfseries\color{mLightBrown}, enhanced,boxsep=1mm,colback=black!10,breakable,noparskip,
    on line,tcbox width=auto,left=0.3em,top=1mm, toptitle=0mm,
    colframe=black!20,arc=0pt,outer arc=0pt,
    opacityback=0.7,nobeforeafter,title=#1]}
    {\endtcolorbox}

\newenvironment{myblock}[1]{ % Обычный информационный блок (бирюзовый заголовок)
    \tcolorbox[capture=minipage,fonttitle=\small\bfseries\color{mDarkTeal}, enhanced,boxsep=1mm,colback=black!10,breakable,noparskip,
    on line,tcbox width=auto,left=0.3em,top=1mm, toptitle=0mm,
    colframe=black!20, arc=0pt,outer arc=0pt,
    opacityback=0.7,nobeforeafter,title=#1]}
    {\endtcolorbox}

% >>> КОММЕНТАРИЙ: Команда для вставки изображения в рамке tcolorbox с подписью
% #1: Опции (не используются)
% #2: Путь к файлу изображения (используется как подпись под изображением)
\newcommand{\mygraphics}[2][]{
\tcbox[enhanced,boxsep=0pt,top=0pt,bottom=0pt,left=0pt,
right=0pt,boxrule=0.4pt,drop fuzzy shadow,clip upper,
colback=black!75!white,toptitle=2pt,bottomtitle=2pt,nobeforeafter,
center title,fonttitle=\small\sffamily,title=\detokenize{#2}] % Используем путь как заголовок
{\includegraphics[width=\the\dimexpr(\linewidth-4mm)\relax]{#2}} % Вставляем изображение, ширина чуть меньше колонки
}

% >>> КОМЕНТАРИЙ: Команда \needspace{<длина>}
% Используйте эту команду *перед* блоком (tcolorbox, section, figure), чтобы убедиться,
% что в текущей колонке есть как минимум <длина> свободного места. Если места нет,
% LaTeX начнет новую колонку/страницу перед выполнением команды.
% Пример: \needspace{10\baselineskip} % Запросить место для примерно 10 строк текста
% Это полезно для предотвращения "висячих" заголовков или некрасивых разрывов блоков.

% >>>>>> НОВОЕ: Стили для графиков PGFPlots <<<<<<
\pgfplotsset{
    % Базовый стиль для всех графиков в шпаргалке
    cheatsheet plot style/.style={
        width=0.98\linewidth, % Ширина чуть меньше колонки для аккуратности
        height=5cm,          % Высота графика (можно изменить по необходимости)
        grid=major,          % Включаем основную сетку
        grid style={dashed, color=black!20}, % Стиль сетки: пунктирная, светло-серая
        axis lines=left,     % Линии осей только слева и снизу
        axis line style={color=black!60, thick}, % Стиль линий осей: серые, потолще
        tick style={color=black!60, thick}, % Стиль меток на осях
        ticklabel style={font=\scriptsize, color=black}, % Стиль подписей меток: маленький шрифт
        label style={font=\small, color=mDarkTeal}, % Стиль подписей осей: шрифт small, цвет как у заголовков блоков
        title style={font=\small\bfseries, color=mDarkTeal}, % Стиль заголовка графика
        legend style={        % Стиль легенды
            font=\scriptsize, % Маленький шрифт
            draw=black!30,    % Легкая рамка вокруг легенды
            fill=white,       % Белый фон
            legend cell align=left, % Выравнивание текста в ячейке легенды
            anchor=north east, % Якорь для позиционирования
            at={(rel axis cs:0.98,0.98)} % Положение в правом верхнем углу внутри области графика
        },
        cycle list={ % Список стилей для линий \addplot (цвета взяты из шаблона)
            {blue, mark=*, thick},
            {alert, mark=square, thick},
            {w3schools, mark=triangle, thick},
            {mLightBrown, mark=diamond, thick},
            {customcolor, mark=oplus, thick},
            {black!50, mark=pentagon, thick},
        },
        % Убираем рамку вокруг графика по умолчанию
        axis background/.style={fill=none},
    },
    % Дополнительный стиль для ROC-кривой (без сетки, с диагональю)
    roc curve style/.style={
        cheatsheet plot style, % Наследуем базовый стиль
        width=6cm, height=6cm, % Делаем квадратным
        grid=none, % Убираем сетку
        xlabel={False Positive Rate (FPR)},
        ylabel={True Positive Rate (TPR)},
        xmin=0, xmax=1,
        ymin=0, ymax=1,
        xtick={0, 0.5, 1},
        ytick={0, 0.5, 1},
        legend pos=south east, % Легенда внизу справа
        % Добавляем диагональную линию случайного угадывания
        extra x ticks={0.5}, extra y ticks={0.5},
        extra tick style={grid=major, grid style={dashed, color=black!40}},
        after end axis/.code={ % Добавляем линию после отрисовки осей
            \draw[dashed, color=black!40] (axis cs:0,0) -- (axis cs:1,1);
        }
    },
    % Дополнительный стиль для PR-кривой
    pr curve style/.style={
        cheatsheet plot style, % Наследуем базовый стиль
        xlabel={Recall (TPR)},
        ylabel={Precision},
        xmin=0, xmax=1,
        ymin=0, ymax=1,
        legend pos=south west, % Легенда внизу слева
    }
}

% % Добавлено для возможности компиляции этого фрагмента отдельно для проверки
% \begin{document}
% \begin{multicols}{2} % Пример использования в multicols
% \lipsum[1] % Просто текст для заполнения

% \begin{textbox}{Пример графика (Базовый стиль)}
%     \begin{center}
%     \begin{tikzpicture}
%         \begin{axis}[cheatsheet plot style, title={Пример $y=x^2$ и $y=x+1$}]
%         \addplot coordinates {(0, 0) (1, 1) (2, 4) (3, 9) (4, 16)};
%         \addlegendentry{$y=x^2$}
%         \addplot coordinates {(0, 1) (1, 2) (2, 3) (3, 4) (4, 5)};
%         \addlegendentry{$y=x+1$}
%         \end{axis}
%     \end{tikzpicture}
%     \end{center}
% \end{textbox}

% \begin{myexampleblock}{Пример ROC-кривой}
%     \begin{center}
%     \begin{tikzpicture}
%         \begin{axis}[roc curve style, title={Пример ROC Curve}]
%         % Пример данных ROC кривой
%         \addplot coordinates { (0,0) (0.1,0.3) (0.2,0.6) (0.4,0.8) (0.6,0.9) (0.8,0.95) (1,1) };
%         \addlegendentry{Модель A (AUC $\approx$ 0.8)}
%         \addplot coordinates { (0,0) (0.2,0.2) (0.4,0.4) (0.6,0.6) (0.8,0.8) (1,1) }; % Пример хуже
%         \addlegendentry{Модель B (AUC = 0.5)}
%         \end{axis}
%     \end{tikzpicture}
%     \end{center}
% \end{myexampleblock}

% \begin{myblock}{Пример PR-кривой}
%     \begin{center}
%     \begin{tikzpicture}
%         \begin{axis}[pr curve style, title={Пример Precision-Recall Curve}]
%         % Пример данных PR кривой (могут сильно зависеть от порога)
%         \addplot coordinates { (0,0.9) (0.1,0.85) (0.3,0.8) (0.5,0.7) (0.7,0.6) (0.9,0.4) (1,0.3) };
%         \addlegendentry{Модель C}
%         \end{axis}
%     \end{tikzpicture}
%     \end{center}
% \end{myblock}

% \lipsum[2-3] % Просто текст для заполнения
% \end{multicols}
% \end{document}

\begin{document}
\pagestyle{empty} % Убираем номера страниц
\small % Уменьшаем базовый размер шрифта для всего документа

% >>> КОММЕНТАРИЙ: Используем multicols для создания трех колонок.
\begin{multicols}{3}
\raggedcolumns % Не растягивать колонки по вертикали

\noindent
\begin{minipage}{\linewidth}
    \centering
    {\bfseries\huge \cheatsheettitle \par}
    \vspace{1ex}
    {\large \cheatsheetauthor \par}
    \vspace{0.5ex}
    {\normalsize \today \par} % Вставляем текущую дату
\end{minipage}
\vspace{2ex}

\thispagestyle{empty} % Убеждаемся, что и на первой странице нет номера

\scriptsize % Уменьшаем шрифт для оглавления и основного текста
\tableofcontents % Генерируем оглавление

% >>> КОММЕНТАРИЙ: Подключаем основной контент шпаргалки
% >>> Контент для шпаргалки Cheatsheet 15: PyTorch/TensorFlow Quick Start

% --- Введение: Снимаем Страх ---
\begin{textbox}{Зачем нужны DL Фреймворки?}
    PyTorch и TensorFlow --- это мощные инструменты, которые сильно упрощают создание и обучение глубоких нейронных сетей (Deep Learning, DL). Они берут на себя сложную математику и оптимизацию, позволяя тебе сосредоточиться на архитектуре модели и данных.
    \begin{itemize}
        \item \textbf{Аналогия:} Представь, что строишь сложный механизм. Вместо того чтобы вытачивать каждую шестеренку вручную, ты используешь готовые стандартные блоки (слои, функции активации) и инструменты (оптимизаторы, расчет градиентов), которые предоставляют фреймворки.
        \item \textbf{Цель этой шпаргалки:} Понять самые базовые "строительные блоки" и процесс "сборки" (обучения) в PyTorch и TensorFlow. Не бойся синтаксиса, концепции очень похожи!
    \end{itemize}
\end{textbox}

% --- Раздел 1: Главная Сущность - Тензор ---
\section{Тензор: Основа Всего}

\begin{myblock}{Что такое Тензор?}
    \textbf{Тензор (Tensor)} --- это фундаментальная структура данных в DL фреймворках. По сути, это многомерный массив чисел.
    \begin{itemize}
        \item 0-мерный тензор: скаляр (просто число).
        \item 1-мерный тензор: вектор (массив).
        \item 2-мерный тензор: матрица (таблица).
        \item 3-мерный тензор: куб чисел (например, цветное изображение RGB).
        \item и так далее...
    \end{itemize}
    \textbf{Аналогия:} Думай о тензоре как о супер-продвинутом NumPy массиве, который умеет работать на GPU и "помнит" вычисления для градиентов.
    В тензорах хранятся входные данные, веса модели, выходы слоев и т.д.
    \textit{Микро-уточнение для TensorFlow:} В TF основной неизменяемый тип — \texttt{tf.Tensor} (результат операций, создается через \texttt{tf.constant}, \texttt{tf.zeros} и т.д.). Для изменяемых параметров модели, которые нужно отслеживать для градиентов, используется \texttt{tf.Variable}.
\end{myblock}

\begin{codebox}{python}{Создание Тензоров}
# === PyTorch ===
import torch

# Создание из списка Python
t_list_pt = torch.tensor([[1., 2.], [3., 4.]])

# Создание тензора из нулей
t_zeros_pt = torch.zeros((2, 3)) # Форма (2 строки, 3 столбца)

# Создание тензора из случайных чисел
t_rand_pt = torch.rand((2, 2))

# === TensorFlow ===
import tensorflow as tf

# Создание из списка Python (создает tf.Tensor)
t_list_tf = tf.constant([[1., 2.], [3., 4.]])

# Создание тензора из нулей (создает tf.Tensor)
t_zeros_tf = tf.zeros((2, 3)) # Форма (2 строки, 3 столбца)

# Создание тензора из случайных чисел (создает tf.Tensor)
t_rand_tf = tf.random.uniform((2, 2))
# tf.Variable обычно создается отдельно для весов модели
\end{codebox}


% --- Раздел 2: Магия Автоматического Дифференцирования ---
\section{Автодифференцирование: Как Сеть Учится}

\begin{myblock}{Зачем это нужно?}
    Обучение нейросети — это подбор таких весов, чтобы ошибка (loss) была минимальной. Для этого используется \textbf{градиентный спуск}. Нам нужно знать, как небольшое изменение каждого веса влияет на итоговую ошибку. Это и есть \textbf{градиент}. Вычислять его вручную для сложных сетей нереально.
    \textbf{Автоматическое дифференцирование (Autograd)} — это механизм, который автоматически вычисляет градиенты функции потерь по всем параметрам модели.
    \textbf{Аналогия:} Представь "волшебного бухгалтера", который следит за каждой математической операцией в сети. Когда ты получаешь итоговую ошибку, ты можешь спросить у него: "Эй, как изменение вот этого конкретного веса в самом начале повлияло на эту ошибку?". И он мгновенно даст тебе ответ (градиент).
\end{myblock}

\begin{codebox}{python}{Autograd/GradientTape на практике}
# === PyTorch (Autograd) ===
import torch

# Тензор, для которого нужно считать градиенты
x_pt = torch.tensor(2.0, requires_grad=True)
w_pt = torch.tensor(3.0, requires_grad=True)
b_pt = torch.tensor(1.0, requires_grad=True)

# Операции
y_pt = w_pt * x_pt + b_pt # y = 3 * 2 + 1 = 7

# Вычисляем градиенты d(y)/d(w), d(y)/d(x), d(y)/d(b)
y_pt.backward()

# Градиенты сохраняются в атрибуте .grad
print(f"PyTorch dy/dw: {w_pt.grad}") # Ожидаем x = 2
print(f"PyTorch dy/dx: {x_pt.grad}") # Ожидаем w = 3
print(f"PyTorch dy/db: {b_pt.grad}") # Ожидаем 1

# === TensorFlow (GradientTape) ===
import tensorflow as tf

# Используем tf.Variable, чтобы TF отслеживал их
x_tf = tf.Variable(2.0)
w_tf = tf.Variable(3.0)
b_tf = tf.Variable(1.0)

# Операции должны быть внутри контекста GradientTape
with tf.GradientTape() as tape:
    y_tf = w_tf * x_tf + b_tf # y = 3 * 2 + 1 = 7

# Вычисляем градиенты
# tape.gradient(целевая_переменная, список_переменных_по_которым_считаем)
gradients = tape.gradient(y_tf, {'w': w_tf, 'x': x_tf, 'b': b_tf})

print(f"TensorFlow dy/dw: {gradients['w']}") # Ожидаем x = 2
print(f"TensorFlow dy/dx: {gradients['x']}") # Ожидаем w = 3
print(f"TensorFlow dy/db: {gradients['b']}") # Ожидаем 1
\end{codebox}
\begin{alerttextbox}{Важно!}
    В PyTorch нужно явно указывать \texttt{requires\_grad=True} для тензоров, по которым будут считаться градиенты (обычно это параметры модели). В TensorFlow для \texttt{tf.Variable} отслеживание включено по умолчанию, а операции нужно помещать внутрь \texttt{tf.GradientTape()}.
\end{alerttextbox}

% --- Раздел 3: Строим Простую Модель ---
\section{Модель: Рецепт Преобразования Данных}

\begin{myblock}{Что такое модель?}
    \textbf{Модель (Model)} — это, по сути, функция (часто очень сложная), которая преобразует входные данные в выходные (предсказание). В DL она обычно состоит из последовательности \textbf{слоев (Layers)}. Каждый слой выполняет определенное преобразование над данными, используя свои \textbf{параметры (веса)}.
    \textbf{Аналогия:} Модель — это как кулинарный рецепт. Входные данные — ингредиенты. Слои — шаги рецепта (смешать, запечь, нарезать). Веса слоя — это параметры шага (сколько муки, температура духовки). Выход модели — готовое блюдо (предсказание).
\end{myblock}

\begin{codebox}{python}{Пример простой линейной модели}
# === PyTorch (nn.Module) ===
import torch
import torch.nn as nn

class SimpleLinearPT(nn.Module):
    def __init__(self, input_dim, output_dim):
        super(SimpleLinearPT, self).__init__()
        # Определяем слои в конструкторе
        self.linear = nn.Linear(input_dim, output_dim)

    # Метод forward определяет прямой проход
    def forward(self, x):
        # Определяем порядок вычислений (прямой проход)
        out = self.linear(x)
        return out

# Создаем модель: входная размерность 5, выходная 1
model_pt = SimpleLinearPT(input_dim=5, output_dim=1)
print("PyTorch Model:", model_pt)

# === TensorFlow (tf.keras.Model / tf.keras.Sequential) ===
import tensorflow as tf
from tensorflow.keras.layers import Dense
from tensorflow.keras import Model, Sequential

# Простой способ через Sequential (для линейной последовательности слоев)
model_tf_seq = Sequential([
    Dense(units=1, input_shape=(5,)) # units - выходная размерность
])
print("\nTensorFlow Sequential Model:")
model_tf_seq.build(input_shape=(None, 5)) # Явно строим модель (None - размер батча)
model_tf_seq.summary()

# Более гибкий способ через наследование Model
class SimpleLinearTF(Model):
    def __init__(self, output_dim):
        super(SimpleLinearTF, self).__init__()
        # Определяем слои
        self.dense = Dense(units=output_dim)

    # Метод call определяет прямой проход (аналог forward в PyTorch)
    def call(self, x):
        # Определяем прямой проход
        return self.dense(x)

# Создаем модель: выходная размерность 1
model_tf_custom = SimpleLinearTF(output_dim=1)
# Чтобы увидеть summary, нужно "вызвать" модель на данных нужной формы
_ = model_tf_custom(tf.zeros((1, 5))) # Прогоняем "пустой" тензор для построения
print("\nTensorFlow Custom Model:")
model_tf_custom.summary()
\end{codebox}

% --- Раздел 4: Цикл Обучения - Сердце Процесса ---
\section{Цикл Обучения: Шаг за Шагом к Результату}

\begin{myblock}{Как происходит обучение?}
    Обучение — это итеративный процесс, где модель настраивает свои веса, чтобы минимизировать ошибку на обучающих данных. Каждый проход по данным называется \textbf{эпохой}. Внутри эпохи данные обычно делятся на \textbf{батчи (batches)}.
    \textbf{Аналогия:} Студент готовится к экзамену (обучение). У него есть учебник (данные). Он читает главу за главой (эпохи). Каждую главу он прорабатывает по частям (батчи). Для каждой части:
    1.  Пытается ответить на вопросы (делает предсказание - \textbf{forward pass}).
    2.  Сравнивает свои ответы с правильными (считает ошибку - \textbf{loss}).
    3.  Анализирует, где ошибся и почему (считает градиенты - \textbf{backward pass}).
    4.  Корректирует свое понимание материала (обновляет веса - \textbf{optimizer step}).
    5.  Перед следующей частью "очищает" голову от предыдущих размышлений (обнуляет градиенты - \textbf{zero grad}).
\end{myblock}

\begin{codebox}{python}{Концептуальный цикл обучения (Псевдокод)}
# --- Общие компоненты ---
# model = ... (ваша модель PyTorch или TensorFlow)
# criterion = ... (функция потерь, напр., nn.MSELoss() или tf.keras.losses.MeanSquaredError())
# optimizer = ... (оптимизатор, напр., torch.optim.SGD() или tf.keras.optimizers.SGD())
# train_loader = ... (загрузчик данных, который выдает батчи X, y)
# num_epochs = ... (количество эпох)

# === PyTorch ===
# for epoch in range(num_epochs):
#     for i, (inputs, labels) in enumerate(train_loader):
#         # 1. Forward pass: получить предсказания
#         outputs = model(inputs)
#
#         # 2. Calculate loss: сравнить с реальными метками
#         loss = criterion(outputs, labels)
#
#         # 3. Backward pass: вычислить градиенты
#         optimizer.zero_grad() # !!! ОБНУЛИТЬ градиенты перед backward() !!!
#         loss.backward()
#
#         # 4. Optimizer step: обновить веса модели
#         optimizer.step()
#
#     # (Опционально) Логирование потерь за эпоху...

# === TensorFlow (с использованием model.fit - проще для старта) ===
# model.compile(optimizer=optimizer, loss=criterion, metrics=['accuracy']) # Опционально метрики
#
# history = model.fit(train_loader, # Или просто X_train, y_train
#                     epochs=num_epochs,
#                     # validation_data=validation_loader # Опционально
#                    )
# print("Обучение завершено!")

# === TensorFlow (ручной цикл - для понимания) ===
# for epoch in range(num_epochs):
#     for step, (x_batch_train, y_batch_train) in enumerate(train_loader):
#         with tf.GradientTape() as tape:
#             # 1. Forward pass
#             logits = model(x_batch_train, training=True) # Указать training=True для некоторых слоев
#             # 2. Calculate loss
#             loss_value = criterion(y_batch_train, logits)
#
#         # 3. Backward pass (вычисление градиентов)
#         grads = tape.gradient(loss_value, model.trainable_weights)
#
#         # 4. Optimizer step (применение градиентов)
#         optimizer.apply_gradients(zip(grads, model.trainable_weights))
#
#     # (Опционально) Логирование потерь за эпоху...
\end{codebox}
\begin{alerttextbox}{Ключевые моменты цикла:}
    *   \textbf{Forward Pass:} Прогон данных через модель для получения предсказания.
    *   \textbf{Loss Calculation:} Вычисление функции потерь (насколько предсказание отличается от истины).
    *   \textbf{Backward Pass:} Вычисление градиентов функции потерь по параметрам модели (используя \texttt{autograd} / \texttt{GradientTape}).
    *   \textbf{Optimizer Step:} Обновление параметров модели с использованием вычисленных градиентов и выбранного алгоритма оптимизации (SGD, Adam и т.д.).
    *   \textbf{Zero Gradients (PyTorch):} **Критически важно** обнулять градиенты перед каждым \texttt{backward()} вызовом в PyTorch (\texttt{optimizer.zero\_grad()}), иначе градиенты будут накапливаться от предыдущих батчей. В TensorFlow при использовании \texttt{GradientTape} это происходит автоматически для каждого вызова \texttt{tape.gradient}.
\end{alerttextbox}


% --- Конец контента ---

% >>> КОММЕНТАРИЙ: Печатаем библиографию (если используется \cite)
\AtNextBibliography{\footnotesize} % Уменьшаем шрифт для библиографии
%\printbibliography % Раскомментируйте, если добавляли цитаты через \cite или \footcite

\end{multicols}

\end{document}